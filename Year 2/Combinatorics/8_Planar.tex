\section{Planar Graphs}

\subsection{Arcs}

An arc is a subset of $\mathbb{R}^2$ of the type $\sigma : [0, 1] \to \mathbb{R}^2$ 
where $\sigma$ is an injective, continuous map and $\sigma(0), \sigma(1)$
are the endpoints of the arc. Injectivity here ensures the arc does not cross itself.

\subsection{Drawings}

For a graph $G = (V, E)$, drawing it is equivalent to assigning: \begin{itemize}
  \item A point $p$ in $\mathbb{R}^2$ for each $v$ in $V$ (such that the map
  from vertices to points is injective),
  \item An arc $\sigma$ for each $e = \{x, y\}$ in $E$ (such that $\sigma$ intersects
  exactly two points, the points corresponding to $x$ and $y$).
\end{itemize}

\subsection{Planar Drawings and Graphs}

A drawing with a set of arcs $A$ is planar if for each $\sigma_1, \sigma_2$ in $A$, 
we have that $\sigma_1, \sigma_2$ either intersect at their endpoints or not at all.
A graph is planar if it admits at least one planar drawing.

\subsubsection{Examples of Non-planar Subgraphs}

We have that $K_5$ and $K_{3, 3}$ are not planar.

\subsubsection{Consequences of Non-planar Subgraphs}

If a graph has a non-planar subgraph, it can't be planar.

\subsection{Jordan Curves}

An arc in the plane whose endpoints are equal is called a Jordan curve.

\subsubsection{Jordan Curve Theorem}

For any Jordan curve $C$, $C$ divides the plane into exactly two connected regions
called the 'interior' and 'exterior'. The curve is the boundary of these regions. 

\subsection{Faces}

For a planar graph $G$, a face of a drawing of $G$ is a connected region
bound by the Jordan curves formed by the arcs of the drawing. 
The region going off to infinity is the outer face and the rest are inner faces.

\subsubsection{Euler's Formula}

For a connected graph $G = (V, E)$ where $F$ is the set of faces of a given 
drawing of $G$, we have that: \begin{gather*}
  |V| - |E| + |F| = 2.
\end{gather*}
\begin{proof}
    Suppose $|E| = 0$, then $|V| = 1$ and the number of faces is clearly one.
    Thus: \begin{gather*}
        |V| - |E| + |F| = 1 - 0 + 1 = 2,
    \end{gather*} as required.
    Suppose $|E| \geq 1$ and that Euler's formula holds for all graphs
    on strictly fewer edges. If $G$ contains no cycles, it is a tree so
    we know that $|V| = |E| - 1$. As there are no cycles, there are no
    Jordan curves in our drawing. Thus, we know that $|F| = 1$: \begin{gather*}
        |V| - |E| + |F| = |V| - (|V| - 1) + 1 = 2.
    \end{gather*} If we now consider the case where $G$ contains at least one cycle,
    we fix some $e$ in $E$ in the edge set of the cycle and consider 
    $G' = (V, E \setminus \{e\})$. As $e$ was part of a cycle, $G'$ is connected.
    So, we apply our inductive hypothesis to see that where $F'$ is the set
    of faces of the drawing of $G'$ (the drawing for $G$ with the arc
    corresponding to $e$ removed): \begin{align*}
        |V| - |E \setminus \{e\}| + |F'| &=
        |V| - (|E| - 1) + |F'| \\
        &= 2.
    \end{align*} We can see that $e$ was adjacent to two distinct faces of the drawing
    of $G$. Thus, upon its removal, these two faces became one. So, $|F'| = |F| - 1$
    and thus: \begin{align*}
        |V| - |E| + |F| &= |V| - |E| + |F'| + 1 \\
        &= |V| - (|E| - 1) + |F'| \\
        &= 2.
    \end{align*} as required.
\end{proof}

\subsection{Edge Bound on Planar Graphs}

For $G = (V, E)$ a planar graph on at least three vertices: \begin{gather*}
  |E| \leq 3(|V| - 2).
\end{gather*}
\begin{proof}
    Consider a planar drawing on $G$ with a set of faces $F$. By Euler's formula,
    we know that: \begin{gather*}
        |V| - |E| + |F| = 2.
    \end{gather*} We consider the number $N$ of pairs $(e, f)$ where
    $e$ is an edge and $f$ is a face bordered by the arc corresponding to $e$.
    Since each $e$ in $E$ borders on two faces, $N \leq 2|E|$. Since each face is
    formed by a cycle which has at least 3 edges, $N \geq 3|F|$. Thus, 
    $2|E| \geq 3|F|$. Combining this with Euler's formula we get that: \begin{gather*}
        6 = 3|V| - 3|E| + 3|F| \leq 3|V| - |E|.
    \end{gather*} The result follows.
\end{proof}