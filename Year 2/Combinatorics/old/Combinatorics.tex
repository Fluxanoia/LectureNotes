\documentclass[a4paper, 12pt, twoside]{article}
\usepackage[left = 3cm, right = 3cm]{geometry}
\usepackage[english]{babel}
\usepackage[utf8]{inputenc}
\usepackage{mathtools}
\usepackage{amssymb}
\usepackage{amsmath}
\usepackage{multicol}

\begin{document}

\title{Combinatorics Notes}
\date{}
\author{\textit{paraphrased by} Tyler Wright}
\maketitle

\vfill

\textit{An important note, these notes are absolutely \textbf{NOT}
  guaranteed to be correct, representative of the course, or rigorous.
  Any result of this is not the author's fault.}

\newpage

\section{Bipartite Graphs}

\subsection{Definition of a Bipartite Graph}

A graph $G = (V, E)$ is bipartite if $V$ can be partitioned into
two vertex sets $V_1, V_2$ such that each edge connects a vertex
from $V_1$ to a vertex in $V_2$.

\subsection{Characterisation of Bipartite Graphs}

A graph is bipartite if and only if it contains no odd cycle.

\subsection{The Handshake Lemma for Bipartite Graphs}

We have that for $G = (V, E)$ a bipartite graph with bipartition
$V_1, V_2$: \begin{gather*}
  \sum_{v \in V_1} \text{deg}(v) = \sum_{v \in V_2} \text{deg}(v). \\
\end{gather*}

\subsection{Hall's Marriage Problem}

\subsubsection{Definition of a Matching}

For $G = (V, E)$ a bipartite graph with bipartition $X, Y$,
a matching from $X$ to $Y$ is a set of edges: \begin{gather*}
  M = \{(x, y) : x \in X, y \in Y\},
\end{gather*} such that $f : X \to Y$ defined by: \begin{gather*}
  f(x) := y \qquad \text{ where } (x, y) \in M,
\end{gather*} is injective.
\\[\baselineskip]
\textit{In other words, $|M| = |X|$ and each $y$ in $Y$ appears 
in at most one edge in $M$.}

\subsubsection{Hall's Marriage Theorem}

For $G = (V, E)$ a bipartite graph with bipartition $X, Y$: \begin{gather*}
  G \text{ has a matching from } X \text{ to } Y \\
  \Longleftrightarrow \\
  \text{For all } S \subseteq X, |N(S)| \geq |S|.
\end{gather*} We also have that if: \begin{gather*}
  \text{min}_{x \in X}\big[\text{deg}(x)\big] \geq
  \text{max}_{y \in Y}\big[\text{deg}(y)\big],
\end{gather*} then $G$ has a matching from $X$ to $Y$.

\vfill

\section{Trees and Forests}

\subsection{Definition of a Forest}

A graph $F = (V, E)$ is a forest if it has no cycles
(is \textbf{acyclic}).

\subsection{Definition of a Tree}

A graph is a tree if it is a forest and is connected.

\subsection{Definition of a Leaf}

For a vertex $v$ in a tree, $v$ is a leaf if it has degree one.

\subsection{Existence of Leaves}

For a tree $T$ of order $2$ or more, we have that $T$ has a leaf.

\subsection{Characterisation of Trees}

We have that for a graph $G = (V, E)$, the following is equivalent:
\begin{itemize}
  \item $G$ is a tree
  \item $G$ is maximally acyclic ($G$ is acyclic and the addition
  of any edge forms a cycle)
  \item $G$ is minimally connected ($G$ is connected and the removal
  of any edge disconnects it)
  \item $G$ is connected and $|E| = |V| - 1$
  \item $G$ is acyclic and $|E| = |V| - 1$
  \item Any two vertices in $G$ are connected by a unique path.
\end{itemize}

\subsection{Minimum Spanning Trees}

In a connected, undirected graph $G = (V, E)$, we have that a
spanning tree $T = (V, E')$ of $G$ is a subgraph of $G$ where
$T$ is a tree and $E' \subseteq E$.
\\[\baselineskip]
A spanning tree on $G$ is minimal if there is no other spanning tree
on $G$ with a lower weight.

\subsubsection{Existence of spanning trees}

We have that there is a spanning tree in a graph $G$ if and only if
$G$ is connected.

\newpage

\subsubsection{Kruskal's algorithm}

For a graph $G = (V, E)$, we have 
the following steps to the algorithm: \begin{enumerate}
  \item Generate a graph $T = (V, \emptyset)$
  \item Sort the edges by weight
  \item For each edge $(u, v)$ (in increasing order): \begin{itemize}
    \item If $u$ or $v$ are not connected in $T$, add $(u, v)$ to $T$
    \item Stop if there are $|V| - 1$ edges in $T$ or if we have run
    out of edges.
  \end{itemize}
\end{enumerate} When this terminates, if the order of $T$ is $|V| - 1$
then $T$ is a minimum spanning tree. Otherwise, $T$ is an acyclic
graph with $n - k$ components.

\section{Cliques and Independent Sets}

\subsection{Definition of a Triangle}

We often call $K_3$ (the complete graph on three vertices) a triangle. A graph $G$
contains a triangle if a subgraph of $G$ is isomorphic to $K_3$.

\subsection{Mantel's Theorem}

For $G = (V, E)$ a graph of order $n$ that contains no triangles, we have that: \begin{gather*}
  |E| \leq \left\lfloor \frac{n^2}{4} \right\rfloor = \left\lfloor \left(\frac{n}{2}\right)^2 \right\rfloor.
\end{gather*} We also have that: \begin{gather*}
  \left[ |E| = \left\lfloor \frac{n^2}{4} \right\rfloor \right]
  \Rightarrow
  \left[ G \cong K_{k, n - k} \text{ where } k = \left\lfloor \frac{n}{2} \right\rfloor \right],
\end{gather*} there always exists a graph where the equality above holds.

\section{Planar Graphs}

The motivator for understanding planar graphs is the problem of drawing graphs in
the plane without intersecting edges.

\subsection{Definition of an Arc}

An arc is a subset of $\mathbb{R}^2$ of the type $\sigma : [0, 1] \to \mathbb{R}^2$ 
where $\sigma$ is an injective, continuous map and $\sigma(0), \sigma(1)$
are the endpoints of the arc. Injectivity here ensures the arc does not cross itself.

\subsection{Definition of a Drawing}

For a graph $G = (V, E)$, drawing it is equivalent to assigning: \begin{itemize}
  \item A point $p$ in $\mathbb{R}^2$ for each $v$ in $V$ (such that the map
  from vertices to points is injective)
  \item An arc $\sigma$ for each $e = (x, y)$ in $E$ (such that $\sigma$ intersects
  exactly two points, the points corresponding to $x$ and $y$).
\end{itemize}

\subsection{Definition of a Planar Drawings and Graphs}

A drawing with a set of arcs $A$ is planar if for each $\sigma_1, \sigma_2$ in $A$, 
we have that $\sigma_1, \sigma_2$ either intersect at their endpoints or not at all.
\\[\baselineskip]
A graph is planar if it admits at least one planar drawing. We have that $K_5$ 
and $K_{3, 3}$ are not planar.

\subsubsection{Non-Planar Subgraphs}

For a graph $G$ with $G'$ a subgraph of $G$ that is not planar, we have that
$G$ is not planar.

\subsection{Definition of a Jordan Curve}

An arc in the plane whose endpoints conincide is called a Jordan curve.

\subsection{Jordan Curve Theorem}

For any Jordan curve $C$, $C$ divides the plane into exactly two connected regions
called the 'interior' and 'exterior'. The curve is the boundary of these regions. 

\subsection{Definition of a Face}

For a planar graph $G$, a face of a drawing of $G$ is a connected region
bound by the drawing. The region going off to infinity is the outer face and
the rest are inner faces.

\subsection{Euler's Formula}

For a connected graph $G = (V, E)$ where $F$ is the set of faces of a given 
drawing of $G$, we have that: \begin{gather*}
  |V| - |E| + |F| = 2.
\end{gather*}

\subsection{Edge Bound on Planar Graphs}

For $G = (V, E)$ a planar graph on at least three vertices: \begin{gather*}
  |E| \leq 3(|V| - 2).
\end{gather*}

\section{Graph Colouring}

\subsection{$k$-colouring}

A $k$-colouring of a graph $G = (V, E)$ is an assignment of $[k]$ to $V$ 
performed by $c : V \to [k]$ such that for $u, v$ adjacent vertices in $V$, 
$c(u) \neq c(v)$. A graph is $k$ colourable if a $k$-colouring exists for it.

\subsection{Chromatic Number}

The chromatic number of a graph $G$ denoted by $\chi(G)$ is the smallest 
$k$ such that $G$ is $k$ colourable.

\subsection{Bound on Chromatic Number}

For a graph $G$, we have that for some $k$ in $\mathbb{Z}$: \begin{gather*}
  \Delta(G) \leq k \qquad \Rightarrow \qquad \chi(G) \leq k + 1.
\end{gather*}

\subsection{Definition of a Map}

A map is a graph derived from some traditional map where regions correspond to faces,
points where at least three regions border each other are vertices and, the border between
exactly two regions are edges. We assume these regions are connected and they do not touch
solely at a point (or several points)

\subsection{Five Colour Theorem}

Every map with corresponding graph $G$ can be coloured with five colours, 
that is $\chi(G) \leq 5$.

\subsection{Dual Graphs}

Given a planar graph $G = (V, E)$ and a fixed planar drawing of $G$, the dual graph
$G^* = (V^*, E^*)$ relative to this drawing is a planar graph obtained by assigning
a vertex to each face and connecting these vertices by and edge if their corresponding
faces border.

\subsubsection{$k$-colourability of the dual graph}

We have that for a graph $G$, $G$ is $k$-colourable if and only if $G^*$ is 
$k$-colourable.

\section{Order from Disorder}

\subsection{Definition of the Ramsey Number}

For some $s$ in $\mathbb{Z}_{> 1}$, we let $r(s)$ be the smallest $n$ in $\mathbb{N}$ 
such that whenever the edges of $K_n$ are $2$-coloured, there exists a monochromatic $K_s$.
We have that this exists for all $s$ as chosen above.
\\[\baselineskip]
Equivalently, $r(s)$ is the smallest $n$ such that for any graph $G$ on $n$
vertices satisfies either: \begin{align*}
  K_s &\subseteq G \\
  &\text{or} \\
  K_s &\subseteq \bar{G}.
\end{align*}

\subsection{Definition of the Off-diagonal Ramsey Number}

For some $s, t$ in $\mathbb{Z}_{> 1}$, we let $r(s, t)$ be the least $n$ in $\mathbb{N}$
such that whenever the edges of $K_n$ are $2$-coloured with colour set $\{A, B\}$, 
there exists an $A$-monochromatic $K_s$ or a $B$-monochromatic $K_t$.
We have that this exists for all $s, t$ as chosen above.

\subsubsection{Properties of the Off-diagonal Ramsey Number}

We have for all $s, t$ in $\mathbb{Z}_{> 1}$: \begin{itemize}
  \item $r(s, s) = r(s)$
  \item $r(s, t) = r(t, s)$
  \item $r(2, s) = s$.
\end{itemize}

\subsection{Ramsey's Theorem}

We have that for all $s, t$ in $\mathbb{Z}_{> 2}$: \begin{gather*}
  r(s, t) \leq r(s - 1, t) + r(s, t - 1).
\end{gather*}

\subsection{An Upper Bound on Ramsey Numbers}

For all $s, t$ in $\mathbb{Z}_{> 1}$, we have that: \begin{gather*}
  r(s, t) \leq 2^{s + t},
\end{gather*} an consequence of this is that: \begin{gather*}
  r(s) \leq 4^{s}.
\end{gather*}

\subsection{A Lower Bound on Ramsey Numbers}

For all $s$ in $\mathbb{Z}_{> 1}$, we have that: \begin{gather*}
  r(s) \geq 2^{\frac{s}{2}}.
\end{gather*}

\subsection{The $k$-colour Ramsey Number}

For some $k$ in $\mathbb{Z}_{> 0}$, $s$ in $\mathbb{Z}_{> 1}$, we let $r_k(s)$ 
be the smallest $n$ in $\mathbb{N}$ such that whenever the edges of $K_n$ are
$k$-coloured, there exists a monochromatic $K_s$.
We have that this exists for all $k, s$ as chosen above.

\subsection{Infinite Ramsey}

For a set $A$ and $k$ in $\mathbb{Z}_{> 0}$,  we have that: \begin{gather*}
  A^{(k)} = \{\{a, b\} : a, b \in A, a \neq b\},
\end{gather*} the set of subsets of $A$ of size two not containing duplicates.
\\[\baselineskip]
Let $\mathbb{N}^{(2)}$ be $2$-coloured, we have that there exists an infinite set 
$M \subseteq \mathbb{N}$ such that $M^{(2)}$ in $\mathbb{N}^{(2)}$ is monochromatic.

\end{document}