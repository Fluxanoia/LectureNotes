\section{Graph Theory}

\subsection{Graphs}

A graph $G$ is a set system $(V, E)$ where the elements of $E$ have
size $2$. Some definitions and facts follow from the definition:
\begin{itemize}
  \item The elements of $V$ are \textbf{vertices},
  \item The elements of $E$ are called \textbf{edges},
  \item The size of $V$ is often called the \textbf{order} of $G$,
  \item $G$ is a $2$-uniform set with ground set $V$,
  \item $u, v$ in $V$ are adjacent if $\{u, v\}$ is in $E$.
\end{itemize}  

\subsubsection{Graph Isomorphisms}

For two graphs $G_1 = (V_1, E_1)$, $G_2 = (V_2, E_2)$, we say that
$G_1$ and $G_2$ are isomorphic ($G_1 \cong G_2$) if there exists a
bijection $\phi : V_1 \to V_2$ such that for each pair of vertices
$u, v$ in $V$ we have that: \begin{gather*}
  \{u, v\} \in E_1 \Longleftrightarrow \{\phi(u), \phi(v)\} \in E_2.
\end{gather*}

\subsection{Neighbourhood and Degree}

For a graph $G = (V, E)$ the neighbourhood of $v$ in $V$ 
is the set of all adjacent vertices (denoted by $N_G(v)$). The
neighbourhood of a set $S$ is simply the union of the neighbourhoods
of the elements of $S$ (minus the vertices in $S$).
The degree of $v$ is simply the size of $N_G(v)$ denoted 
by $\Deg(v)$.

\subsubsection{Minimum and Maximum Degree}

For a graph $G = (V, E)$ we have that the following to represent
minimum and maximum degree: \begin{align*}
  \delta(G) &:= \Min\{\text{deg}(v) : v \in V\} \\
  \Delta(G) &:= \Max\{\text{deg}(v) : v \in V\}.
\end{align*}

\subsection{The Handshake Lemma}

For a graph $G = (V, E)$, we have that: \begin{gather*}
  |E| = \frac{\sum_{v \in V} \text{deg}(v)}{2},
\end{gather*} as each edge visits two vertices, contributing
twice to the sum of the degrees of a graph.

\subsection{Subgraphs}

A graph $G' = (V', E')$ is a subgraph of $G = (V, E)$ if
$V' \subseteq V$ and $E' \subseteq E$ such that for all $e$
in $E'$ we have that $e \subseteq V'$.

\subsubsection{Induced Subgraphs}

An induced subgraph generated of $G = (V, E)$
is a subgraph $G' = (V', E')$ where: \begin{gather*} 
  E' = \{\{u, v\} \in E \text{ such that } u, v \in V'\}.
\end{gather*} Induced subgraphs are generated from a subset of 
the vertices of a graph by selecting all the edges that
are subsets of our chosen vertex set.

\subsection{Complements of Graphs}

For a graph $G = (V, E)$, we have that $\bar{G} = (V, \bar{E})$ is
the complement of $G$ where $\bar{E} = \{\{u, v\} : u, v \in V\} \setminus E$.

\subsection{Walks}

A walk of length is a set of vertices connected by edges.
Its length is the number of edges it traverses.
\\[\baselineskip]
A walk is closed if its first and last vertex are identical.

\subsubsection{Types of Walks}

\begin{center}
    \begin{tabular} {| c | c | c | c | c |}
        \hline
        Name & Closed? & Repeats vertices? & Repeats edges? \\
        \hline \hline
        Walk        & Not necessarily & Can    & Can    \\ \hline
        Trail       & Not necessarily & Can    & Cannot \\ \hline
        Paths       & Not necessarily & Cannot & Cannot \\ \hline
        Circuit     & Yes             & Can    & Cannot \\ \hline
        Cycles      & Yes             & Cannot & Cannot \\ \hline
    \end{tabular}
\end{center}

\subsubsection{Walks in Paths and Paths in Walks}

Consider $G = (V, E)$ with $u, v$ in $V$, we have that:
\begin{gather*}
  \text{There's a walk between } u \text{ and } v
  \Longleftrightarrow
  \text{There's a path between } u \text{ and } v.
\end{gather*} Thus, where there's a cycle, there's a circuit
and vice-versa.

\subsubsection{Odd Cycles in Odd Circuits} \label{oddcycles}

If $G$ a graph admits an odd circuit, there's also
an odd cycle in $G$.
\begin{proof}
  Take $C = (x_1, \ldots, x_n)$ to be an odd circuit in $G$.
  Take $i, j$ in $[n]$ such that $i < j$ and $x_i = x_j$ (if no
  such $i, j$ exist then $C$ is an odd cycle). As: \begin{align*}
    C_1 &= (x_i, x_{i + 1}, \ldots, x_j) \\
    C_2 &= (x_j, x_{j + 1}, \ldots, x_n, x_1, \ldots, x_i),
  \end{align*} partition $C$, their lengths must sum to
  the length of $C$, which is odd. Thus, the length of
  $C_1$ or $C_2$ must be odd. Supposing without loss of generality
  that $C_1$ is of odd length, if $C_1$ is a cycle, we are done.
  If not, we repeat the process on $C_1$. The length of the circuit 
  we are considering is strictly decreasing by this process, 
  so this must terminate.
\end{proof}

\subsection{Connected Graphs}

A graph is connected if there exists a path between any two vertices 
in the graph.

\subsubsection{Connected Components}

A component of a graph $G$ is a maximally connected subgraph of $G$.

\subsection{Euler Circuits}

An Euler circuit is a circuit in which each edge in a graph is
traversed exactly once. As a consequence, each vertex is travelled 
at least once. Graphs with Euler circuits are said to be Eulerian.

\newpage

\subsubsection{Partitioning Even Regular Graphs} \label{eulerpart}

For a graph $G = (V, E)$, if each vertex has even degree, we can
partition its edge set into disjoint subsets $E_1, \ldots, E_s$
such that for each $i$ in $[s]$, $E_i$ is the edge set of a cycle.
\begin{proof}
    Supposing each $v$ in $V$ has even degree, if $E = \emptyset$ then
    the statement holds trivially. Suppose $E$ is non-empty and the
    statement holds for all graphs with strictly fewer edges. We
    pick $v$ in $V$ and generate a path $P$ (starting at $v$) 
    by checking if the current end of our path has an edge connecting 
    to some $v'$ in $P$. If it does, we have cycle. If not, there will 
    always be an edge to choose as we entered the vertex and it has even degree 
    (so there must be another edge to leave it). As the edge set is finite,
    this process must end, giving us a circuit (so a cycle). As we
    have generated a cycle $C$, we create $G' = (V, E \setminus E(C))$. 
    But now $|E(G')| < |E|$ so we can split its edge set into
    disjoint subsets $E_1, \ldots, E_s$ satisfying the statement.
    Thus, $E_1, \ldots, E_s, E(C)$ satisfies the statement for $G$.
\end{proof}

\subsubsection{Conditions for an Euler Circuit}

An Euler circuit in a connected graph $G = (V, E)$ exists if and only 
if each vertex in $V$ has even degree.
\begin{proof}
    If $G$ has an Euler circuit, the circuit must enter and exit each $v$
    in $V$ an even number of times. Thus, the degree of each vertex is even.
    If each $v$ in $V$ has even degree, consider (\ref{eulerpart}), partitioning
    $E$ into disjoint subsets $E_1, \ldots, E_s$ all edge sets of cycles.
    Taking $V(E_i)$ to be the vertex set traversed by $E_i$ for all $i$ in $[s]$,
    we have that $V(E_1)$ must share a vertex with some $V(E_i)$ for some $i$ in $[s]$
    as otherwise this would contradict the connectivity of $G$.
    We stitch the edge sets together to form a circuit starting at some
    intersection of $V(E_1)$ and $V(E_i)$ and traversing all of
    $E_1$ then $E_i$. We repeat this until there is only one edge set
    which must be our Euler circuit as its edge set is the union of a
    partition of the edge set.
\end{proof}

\subsection{Hamiltonian Cycles}

A Hamiltonian cycle is a cycle that visits each vertex exactly once.
Graphs with Hamiltonian cycles are said to be Hamiltonian.

\subsubsection{Hamiltonian Paths}

A Hamiltonian path is a path that visits each vertex exactly once.

\newpage

\subsubsection{Dirac's Theorem}

For a graph $G = (V, E)$ where $n = |V| \geq 3$: \begin{gather*}
  \delta(G) \geq \frac{n}{2} \Rightarrow G \text{ is Hamiltonian.}
\end{gather*}
\begin{proof}
    Observe that for some $x, y$ in $V$ if $\{x, y\}$ is not in $E$,
    then we have that as $|V \setminus \{x, y\}| = n - 2$ and
    $|N_G(x)| \geq \frac{n}{2}$, and $|N_G(y)| \geq \frac{n}{2}$: 
    \begin{gather*}
        N_G(x) \cap N_G(y) \neq \emptyset,
    \end{gather*} by the Pigeonhole principle. Take $P = (x_1, \ldots, x_k)$
    to be the longest path in $G$. We have that $k \geq 3$ as $G$ is
    connected on at least 3 vertices. Also, we can assume $G$ has no
    $k$-cycle as: \begin{itemize}
        \item If $k = n$, we have the desired Hamiltonian cycle,
        \item If $k < n$, we have a $k$-cycle in $G$, but as $G$ is connected
        we can take some $x$ in $N_G(P)$ and connect it to $P$ to form
        a path of length $k + 1$ contradicting the maximality of $P$. 
    \end{itemize} Thus, we have that $\{x_1, x_k\}$ is not in $E$.
    Also, we have that for any $i$ in $\{2, \ldots, k - 1\}$, we can't
    have $\{x_1, x_i\}$ and $\{x_{i - 1}, x_k\}$ in $E$ as that would form
    a $k$-cycle $P_k$: \begin{gather} \label{consecneighbours}
        P_k = (x_1, x_i, \ldots, x_k, x_{i - 1}, \ldots, x_2).
    \end{gather} By the maximality of $P$: 
    \begin{equation*}   
        \begin{aligned}
            N_G(x_1) &\subseteq \{x_2, \ldots, x_{k - 1}\} \\
            N_G(x_k) &\subseteq \{x_2, \ldots, x_{k - 1}\},
        \end{aligned}
    \end{equation*}
    as otherwise we could simply connect the element not in our
    path to end of $P$, contradicting the maximality of $P$.
    It follows that: \begin{gather*}
        N_G(x_1) = \{x \in V : \{x_1, x\} \in E\} \qquad
        \text{and} \qquad
        N_G(x_i)^+ = \{x_i : x_{i - 1} \in N_G(x_k)\},
    \end{gather*} are disjoint subsets of $\{x_2, \ldots, x_k\}$
    by the statement describing (\ref{consecneighbours}).
    But, $\{x_2, \ldots, x_k\}$ is of size $k - 1$ and; 
    $N_G(x_1)$ and $N_G(x_1)^+$ both have size at least $\frac{n}{2}$.
    Thus, a contradiction - $G$ has a Hamiltonian cycle.
\end{proof}