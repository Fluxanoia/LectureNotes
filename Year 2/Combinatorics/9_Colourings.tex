\section{Graph Colouring}

\subsection{$k$-colouring}

A $k$-colouring of a graph $G = (V, E)$ is an assignment of $[k]$ to $V$ 
performed by $c : V \to [k]$ such that for $u, v$ adjacent vertices in $V$, 
$c(u) \neq c(v)$. A graph is $k$ colourable if a $k$-colouring exists for it.

\subsection{Chromatic Number}

The chromatic number of a graph $G$ denoted by $\chi(G)$ is the smallest 
$k$ such that $G$ is $k$ colourable.

\subsubsection{Bound on Chromatic Number}

For a graph $G = (V, E)$, we have that for some $k$ in $\mathbb{Z}$, if $\Delta(G) \leq k$
then $\chi(G) \leq k + 1$.
\begin{proof}
    Suppose $|V| = 1$, then $\Delta(G) = 0$ and we can see that $G$ is 1-colourable.
    Suppose $|V| > 1$ and the statement holds for all graphs on strictly fewer vertices.
    Pick some $x$ in $V$, let $G' = (V \setminus \{x\}, E')$ be the induced graph
    on $V \setminus \{x\}$. We have that $\Delta(G') \leq \Delta(G)$ as we only
    removed edges to form $G'$ from $G$. 
    Thus, $G'$ is at least ($k + 1$)-colourable by the inductive hypothesis. 
    We take some $(k + 1)$-colouring of $G'$ and we apply it to $G$, leaving all but 
    $x$ coloured.
    As $\Deg(x) \leq k$, even if each of the vertices connected to $x$ have distinct
    colours, there will still be the $(k + 1)^\text{th}$ colour for $x$ to use.
    Thus, $G$ is ($k + 1$)-colourable as required.
\end{proof}

\subsection{Maps}

A map is a graph derived from some traditional map where regions correspond to faces,
points where at least three regions border each other are vertices and, the border between
exactly two regions are edges. We assume these regions are connected and they do not touch
solely at a point (or several points).

\subsubsection{Dual Graphs}

Given a planar graph $G = (V, E)$ and a fixed planar drawing of $G$, the dual graph
$G^* = (V^*, E^*)$ relative to this drawing is a planar graph obtained by assigning
a vertex to each face and connecting these vertices by and edge if their corresponding
faces border.
\\[\baselineskip]
These are key in considering colourings of maps as we can consider the coloring
of faces of a given graph.

\subsubsection{$k$-colourability of the Dual Graph}

We have that for a graph $G$, $G$ is $k$-colourable if and only if $G^*$ is 
$k$-colourable. 

\subsubsection{Five Colour Theorem}

Every map with corresponding graph $G = (V, E)$ can be coloured with five colours, 
that is $\chi(G) \leq 5$.
\begin{proof}
    Suppose $|V| \leq 5$, then we just assign each vertex a unique colour and 
    we are done.
    Suppose $|V| > 5$ and for all planar graphs on strictly fewer vertices, 
    we have that said graph is 5-colourable. We know that $G$ has a vertex
    of degree at most five as if all $v$ in $V$ were such that $\Deg(v) \geq 6$
    then: \begin{gather*}
        |E| = \frac{1}{2}\sum_{v \in V} \Deg(v) \geq 3|V|,
    \end{gather*} but $|E| \leq 3|V| - 6$ so: \begin{gather*}
        3|V| \leq |E| \leq 3|V| - 6,
    \end{gather*} which is impossible. So, taking $v$ in $V$ to have degree
    at most five, we split our argument into two cases.

    \paragraph{Case 1} Suppose $\Deg(v) < 5$. We consider
    $G' = (V \setminus \{v\}, E')$ the induced graph on $V \setminus \{v\}$.
    By the inductive hypothesis, we have a 5-colouring of $G'$. Applying this
    to $G$, we leave $v$ yet to be coloured. As $v$ has four or fewer neighbours,
    we can choose a remaining colour from our colour set and colour $v$.
    
    \paragraph{Case 2} 
    Suppose $\Deg(v) = 5$. We fix a planar drawing of $G$ and label 
    the neighbours of $v$ with $t, u, x, y, z$ (clockwise about $v$ with respect to
    our fixed drawing). By the inductive hypothesis, we have a 5-colouring on the
    induced subgraph on $V \setminus \{v\}$ (similarly to \textbf{Case 1}). Applying
    this colouring to $G$, we have yet to colour $v$. If $t, u, x, y, z$ use
    at most four colours, we can use the remaining colour in our colour set
    to colour $v$. Otherwise, we consider: \begin{gather*}
        C : N(v) \to \{c_1, c_2, c_3, c_4, c_5\},
    \end{gather*} where $\{c_1, \ldots, c_5\}$ is the colour set and
    for each $v'$ in $N(v)$, $C(v')$ maps to the colour of $v'$ in our
    colouring.

    \paragraph{Case 2a} 
    Suppose there is no path connecting $x$ and $z$ in $G$ without traversing $v$.
    Take $V_{x}$ to be the set of vertices with a path $P$ leading to $x$
    such that: \begin{itemize}
        \item For each vertex in $P$, it is either $C(x)$ or $C(z)$ coloured,
        \item $P$ does not include $v$.
    \end{itemize} We can see that $z$ is not in $V_x$ as we cannot traverse $v$.
    We define a new colouring on $G$ where each $v'$ in $V_x$ has its colour changed
    to $C(x)$ if it was $C(z)$ coloured and vice versa. This must be a valid
    colouring because if there was a conflict, then that would mean that some
    $C(x)$ coloured vertex $v_x$ was connected to a $C(z)$ coloured vertex $v_z$
    but exactly one of $v_x$ and $v_z$ was not in $V_x$. But, this violates the 
    definition of $V_x$ as they are connected so the path to one of them
    defines the path to the other. With this new colouring, $C(x) = C(z)$
    so we have a spare colour for $v$.
        
    \paragraph{Case 2b} 
    Suppose there is a path $P_{xz}$ connecting $x$ and $z$ in $G$ without traversing $v$.
    This path combined with $\{x, v\}$ and $\{v, z\}$ forms a cycle and hence a Jordan curve
    enveloping either $y$ or; $t$ and $u$. Suppose there is a path connecting $y$ and $t$, 
    then it must use a vertex in the cycle formed by $P_{xz}$ as one of $y$ or $t$ are
    on the interior of the Jordan curve formed by it. Thus, taking $V_{y}$ to be the 
    set of vertices with a path $P$ leading to $y$ such that: \begin{itemize}
        \item For each vertex in $P$, it is either $C(y)$ or $C(t)$ coloured,
        \item $P$ does not include $v$.
    \end{itemize} We must have an analogous situation to \textbf{Case 2a} as there
    is no path from $y$ to $t$ solely using vertices in $V_y$. This ends the proof.
\end{proof}