\section{Ramsey Numbers}

For some $s$ in $\mathbb{Z}_{> 1}$, we let $r(s)$ be the smallest $n$ in $\mathbb{N}$ 
such that whenever the edges of $K_n$ are $2$-coloured, there exists a monochromatic $K_s$.
We have that this exists for all $s$ as chosen above (proven in (\ref{ramt})).

\subsection{Off-diagonal Ramsey Numbers}

For some $s, t$ in $\mathbb{Z}_{> 1}$, we let $r(s, t)$ be the least $n$ in $\mathbb{N}$
such that whenever the edges of $K_n$ are $2$-coloured with colour set $\{A, B\}$, 
there exists an $A$-monochromatic $K_s$ or a $B$-monochromatic $K_t$.
We have that this exists for all $s, t$ as chosen above (proven in (\ref{ramt})).

\subsubsection{Properties of the Off-diagonal Ramsey Number}

We have for all $s, t$ in $\mathbb{Z}_{> 1}$: \begin{itemize}
  \item $r(s, s) = r(s)$,
  \item $r(s, t) = r(t, s)$,
  \item $r(2, s) = s$.
\end{itemize}

\subsection{Ramsey's Theorem} \label{ramt}

We have that for all $s, t$ in $\mathbb{Z}_{> 2}$, $r(s, t)$ exists and: 
\begin{gather} \label{ramtg}
  r(s, t) \leq r(s - 1, t) + r(s, t - 1).
\end{gather}
\begin{proof}
    It suffices to show that if $r(s - 1, t)$ and $r(s, t - 1)$ exist then
    (\ref{ramtg}) is true because then we can show the theorem is true by
    induction on $s + t$. We have $r(2, 2) = 2$ so we 
    suppose $r(s - 1, t)$ and $r(s, t - 1)$ exist for each $s, t$ in 
    $\mathbb{Z}_{> 2}$. Set $a = r(s - 1, t)$, $b = r(s, t - 1)$, and
    consider a 2-colouring on $K_{a + b}$. Pick $v$ in $V(K_{a + b})$,
    we can see that it has degree $a + b - 1$ so must have at least
    $a$ 'red' neighbours or at least $b$ 'blue' neighbours. 
    
    \paragraph{Case 1} Suppose $v$ has $a$ 'red' neighbours.
    We consider $K_a$ the induced subgraph on the 'red' neighbourhood of $v$ which
    must contain a 'red' $K_{s - 1}$ or a 'blue' $K_t$. In the former case,
    a 'red' $K_s$ is formed in $K_{a + b}$ as each vertex in the 'red' $K_{s - 1}$
    is attached by a 'red' edge to $v$. In the latter case, the 'blue' $K_t$ is 'blue'
    in $K_{a + b}$ also.

    \paragraph{Case 2} The working is analogous to \textbf{Case 1}.
\end{proof}

\subsection{An Upper Bound on Ramsey Numbers}

For all $s, t$ in $\mathbb{Z}_{> 1}$, we have that: \begin{gather*}
  r(s, t) \leq 2^{s + t},
\end{gather*} an consequence of this is that: \begin{gather*}
  r(s) \leq 4^{s}.
\end{gather*}
\begin{proof}
    Suppose $s = 2$, then as $r(2, t) = t$ we can consider the inequality: 
    \begin{gather*}
        t \leq 2^{2 + t} = 4 \cdot 2^t,
    \end{gather*} which holds for each $t$ as defined in the theorem.
    Suppose $s, t > 2$ and that the theorem holds for each $s', t'$
    in $\mathbb{Z}_{> 1}$ such that $s' + t' < s + t$. We know that:
    \begin{gather*}
        r(s, t) \leq r(s, t - 1) + r(s - 1, t),
    \end{gather*} and by the inductive hypothesis: \begin{gather*}
        r(s, t) \leq 2^{s + t - 1} + 2^{s + t - 1} = 2^{s + t}. \qedhere
    \end{gather*}
\end{proof} 

\subsection{The $k$-colour Ramsey Number}

For some $k$ in $\mathbb{Z}_{> 0}$, $s$ in $\mathbb{Z}_{> 1}$, we let $r_k(s)$ 
be the smallest $n$ in $\mathbb{N}$ such that whenever the edges of $K_n$ are
$k$-coloured, there exists a monochromatic $K_s$.
We have that this exists for all $k, s$ as chosen above.
\begin{proof}
    For $k = 1$, $r_k(s) = s$ as it simply asks which is the smallest $n$ such that
    $K_s$ is isomorphic to a subgraph of $K_n$. We have already proven the case
    for $k = 2$ by our work in this chapter. 
    \\[\baselineskip]
    Take the colour set to be $\{c_1, \ldots, c_k\}$.
    For any $k > 2$, we want to show
    that $r_k(s) \leq r(s, r_{k - 1}(s)) = n$ by considering a two colouring on
    $K_n$ with $c_1$ and $c_2, \ldots, c_k$. By the definition of $n$, 
    in a given colouring of $K_n$, we either
    get a $c_1$ coloured $K_s$ or a $c_2, \ldots, c_k$ coloured $K_{r_{k - 1}(s)}$.
    In the former case, we are done. In the latter case, we consider just the subgraph
    isomorphic to $K_{r_{k - 1}(s)}$ and by the definition of $r_{k - 1}$ we see
    that it must contain a $c_i$ coloured $K_s$ for some $i$ in $\{2, \ldots, k\}$
    as required.
\end{proof}

\subsection{Infinite Ramsey}

For a set $A$ and $k$ in $\mathbb{Z}_{> 0}$,  we have that: \begin{gather*}
  A^{(k)} = \{\{a, b\} : a, b \in A, a \neq b\},
\end{gather*} the set of subsets of $A$ of size two not containing duplicates.
\\[\baselineskip]
Let $\mathbb{N}^{(2)}$ be $2$-coloured, we have that there exists an infinite set 
$M \subseteq \mathbb{N}$ such that $M^{(2)}$ in $\mathbb{N}^{(2)}$ is monochromatic.
\begin{proof}
    Take $a_1$ in $\mathbb{N}$. Since there are infinitely many $a$ in $\mathbb{N}$
    such that $\{a_1, a\}$ is in $\mathbb{N}^{(2)}$, there exists an infinite
    $A_1 \subseteq \mathbb{N}\setminus\{a_1\}$ such that for all $b$ in $A_1$, 
    we have $\{a_1, b\}$ coloured identically with some colour $c_1$. 
    We choose $a_2$ in $A_1$ and obtain an infinite set 
    $A_2 \subseteq A_1 \setminus \{a_2\}$ such that for all $b$ in $A_2$, 
    we have $\{a_2, b\}$ coloured identically with some colour $c_2$.
    We continue this infinitely obtaining $a_1, a_2, \ldots$ in $\mathbb{N}$
    together with the colours $c_1, c_2, \ldots$ such that for $i, j$ in 
    $\mathbb{Z}_{>0}$ with $i < j$, the edge $\{a_i, a_j\}$ has colour $c_i$.
    However, since there are only two colours, infinitely many of these colours
    must agree which gives us our set of values.
\end{proof}