\documentclass[a4paper, 12pt, twoside]{article}
\usepackage[left = 3cm, right = 3cm]{geometry}
\usepackage[english]{babel}
\usepackage[utf8]{inputenc}
\usepackage{mathtools}
\usepackage{amssymb}
\usepackage{amsmath}
\usepackage{multicol}

\begin{document}

\title{Combinatorics Notes}
\date{}
\author{\textit{paraphrased by} Tyler Wright}
\maketitle

\vfill

\textit{An important note, these notes are absolutely \textbf{NOT}
  guaranteed to be correct, representative of the course, or rigorous.
  Any result of this is not the author's fault.}

\newpage

\section{Counting Techniques}

\subsection{The Bijection Rule}

For $n$ in $\mathbb{N}$, we define $[n] := \{1, 2, \ldots, n\}$.
\\[\baselineskip]
For a given set $X$, if there exists a bijective function 
$f : [n] \to X$ for some $n$ in $\mathbb{N}$, $X$ has $n$ elements
(or rather $|X| = n$).
\\[\baselineskip]
This can also be achieved by listing out the elements of 
$X = \{x_1, x_2, \ldots, x_n\}$ as we can use $f : [n] \to X$
where $i$ maps to $x_i$.

\subsection{The Addition Rule}

We can count the amount of elements in a given set $X$ by
splitting $X$ into disjoint sets, counting them, and adding the
results.
\\[\baselineskip]
For $n$ in $\mathbb{N}$, and $X_1, \ldots, X_n$ pairwise disjoint
sets: \begin{gather*}
  \left|\bigcup_{i = 1}^n X_i\right| = \sum_{i = 1}^n |X_i|.
\end{gather*} \textit{For a set of sets $A$, pairwise disjoint means 
for two given sets in $A$, they are either disjoint or equal.}

\subsection{The Multiplication Rule}

If a counting problem can be split into a number of stages, we can
use the product of the number of choices at each stage to find
the total number of outcomes.
\\[\baselineskip]
\textit{For example, if we want to find how many three digit numbers 
there are, we can consider it as choosing three digits. We can choose
$1, 2, \ldots, 9$ for the first digit and $0, 1, \ldots, 9$ for 
the rest so we get $9 \cdot 10^2$ possibilities.}

\subsection{Inclusion-Exclusion Principle}

For $n$ in $\mathbb{N}$, and $X_1, \ldots, X_n$ sets: \begin{align*}
  \left|\bigcup_{i = 1}^n X_i\right| &= \sum_{i = 1}^n |X_i| \\
  &-\sum_{i_1 \neq i_2} |X_{i_1} \cap X_{i_2}| \\
  &+\sum_{i_1 \neq i_2 \neq i_3} |X_{i_1} \cap X_{i_2} \cap X{i_3}| \\
  &\ldots
\end{align*} \textit{Essentially, this says that the size of the
union of some finite number of sets is the sum of their sizes,
minus the sum of their \textbf{paired} intersections,
plus the sum of the intersections of \textbf{trios}, etc.}

\subsection{The Factorial}

For $n$ in $\mathbb{N}$ we can define the factorial $n!$: \begin{align*}
  n! = \begin{cases*}
    1 & n = 0 \\
    \prod_{i = 1}^n(i) & \text{otherwise.}
  \end{cases*}
\end{align*} For $k$ in $\mathbb{N}$ we can further define $(n)_k$: \begin{gather*}
  (n)_k = \frac{n!}{(n-k)!} = n(n-1)(n-2)\cdots(n-k+1).
\end{gather*} \textit{This can be though of as the factorial with $k$
elements (starting at $n$). So, $(n)_n = n!$, $(n)_1 = n$, etc.}

\newpage

\subsection{The Binomial Coefficient}

For $n, k$ in $\mathbb{N}$, we can define the binomial coefficient: \begin{gather*}
 {n \choose k} = \frac{n!}{k!(n-k)!} = \frac{(n)_k}{k!}.
\end{gather*} 
This is the number of ways of choosing $k$-element subsets
from an $n$-element set. Furthermore, we have: \begin{gather*}
  {n \choose k} = {n \choose n - k},
\end{gather*} as choosing $k$ elements is equivalent to choosing
$n - k$ elements to remove.
\\[\baselineskip]
There are some notes to be made on the definition: \begin{itemize}
  \item ${n \choose k} = 0$ if $k > n$
  \item ${n \choose 0} = {n \choose n} = 1$
  \item ${n \choose k} \geq 0$
\end{itemize}

\subsection{Pascal's Identity}

Say we are selecting $k$ elements from an $n$-element set (unordered, 
without repeats). We will see that there are ${n \choose k}$ possibilities.
If we fix an element in the set, we can either include said element in
our selection or exclude it giving ${n - 1 \choose k - 1}$ and 
${n - 1 \choose k}$ possibilities respectively. Thus: \begin{gather*}
  {n \choose k} = {n - 1 \choose k - 1} + {n - 1 \choose k}.
\end{gather*}

\subsection{The Binomial Theorem}

By performing induction on Pascal's identity, we can see that for 
$a, b$ in $\mathbb{C}$ and $n$ in $\mathbb{N}$: \begin{gather*}
  (a + b)^n = \sum_{i = 0}^n {n \choose i}a^ib^{n - i}.
\end{gather*} Setting $a = b = 1$, we get $2^n = \sum_{i = 0}^n 
{n \choose i}$.

\subsection{The Pigeonhole Principle}

For $m, n, k$ in $\mathbb{N}$, if we have $k$ objects being 
distributed into $n$ boxes and $n > mk$ then one box must contain
at least $k + 1$ objects.

\section{Selection}

For this section, we will consider $n, k$ in $\mathbb{N}$.

\subsection{Ordered Selection with Repeats}

As we select, we have $n$ choices, and we select $k$ times. Thus,
by the Multiplication Rule, we get $n^k$ outcomes.

\subsection{Ordered Selection without Repeats}

As we select, the amount of choices we have decreases by one each time.
We start with $n$ choices and select $k$ times. Thus, by the Multiplication
Rule, we get \newline $n(n-1)\cdots(n-k+1) = (n)_k$ outcomes.

\subsection{Unordered Selection with Repeats}

Let the set we are selecting from be $\{x_1, \ldots, x_n\}$. In this case, 
any solution can be aggregated into a list indicating how
many times the $i^{\text{th}}$ element was selected (for some $i$ in $[n]$).
For example, if we select $x_1$ three times and $x_2$ five times, 
the outcome would be of the form $\{3, 5, \ldots\}$.
\\[\baselineskip]
It can be seen that for each of these solutions, the sum of the elements in
the set must equal $k$. We can construct a solution by starting with a set
of all zeroes $\{0, 0, 0, \ldots\}$ and distributing $k$ into the set. For
example, for $n = 4$ and $k = 3$ the following are solutions: \begin{gather*}
  \{1, 1, 1, 0\} \text{ as } 1 + 1 + 1 + 0 = 3 = k, \\
  \{0, 2, 0, 1\} \text{ as } 0 + 2 + 0 + 1 = 3 = k, \\
  \{3, 0, 0, 0\} \text{ as } 3 + 0 + 0 + 0 = 3 = k.
\end{gather*} These solutions correspond to $\{x_1, x_2, x_3\}, 
\{x_2, x_2, x_4\}, \{x_1, x_1, x_1\}$ respectively.
\\[\baselineskip]
This distribution of $k$ can be thought of as seperating $k$ into $n$
groups. For example, the solution $\{1, 1, 0, 1\}$ corresponds to: \begin{gather*}
  \bullet | \bullet | | \bullet.
\end{gather*} The dots and dividers are identical respectively, and we
have a total of $k$ dots plus $n - 1$ dividers equalling $k + n - 1$
elements. We can choose where to place the dividers beforehand and
then fill in the dots, thus we have: \begin{gather*}
  {k + n - 1 \choose n - 1}
\end{gather*} choices.

\subsection{Unordered Selection without Repeats}

This is identical to to the ordered case but we divide by the number
of permutations of the solutions as order does not matter. Thus, we get:
\begin{gather*}
  \frac{(n)_k}{k!} = {n \choose k}.
\end{gather*}

\section{Generating Functions}

\subsection{Definition of a Generating Function}

For a sequence $(a_n)_{n \geq 0}$, we
can associate a \textbf{formal power series}: \begin{gather*}
  f(x) = \sum_{k = 0}^\infty a_nx^n = a_0 + a_1x + a_2x^2 + \cdots.
\end{gather*} We say $f(x)$ is the generating function of $(a_n)$,
or write: \begin{align*}
  a_0, a_1, a_2, \ldots &\leftrightarrows a_0 + a_1x + a_2x^2 + \cdots \\
  (a_n)_{n \geq 0} &\leftrightarrows f(x).
\end{align*} Note, however, that this doesn't imply that the series is convergent.

\subsection{Generating Functions of Finite Sequences}

For finite sequences (or rather, sequences with finitely many
non-zero terms), we have that their generating functions
can be written as polynomials.

\subsection{The Scaling Rule}

For a sequence $(a_n)_{n \geq 0}$ with an
associated generating function $f(x)$ and $c$ in $\mathbb{R}$: \begin{gather*}
  (ca_n)_{n \geq 0} \leftrightarrows cf(x).
\end{gather*}

\subsection{The Addition Rule}

For the sequences $(a_n)_{n \geq 0}$, $(b_m)_{m \geq 0}$ 
with the associated generating functions $f(x), g(x)$ respectively: \begin{gather*}
  (a + b)_{n \geq 0} \leftrightarrows f(x) + g(x).
\end{gather*}

\subsection{The Right-Shift Rule}

For a sequence $(a_n)_{n \geq 0}$ with an associated generating 
function $f(x)$, we can add $k$ in $\mathbb{N}$ leading zeroes by
multiplying the sequence by $x_k$: \begin{gather*}
  0, \ldots, 0, a_0, a_1, \ldots \leftrightarrows x^kf(x).
\end{gather*}

\end{document}