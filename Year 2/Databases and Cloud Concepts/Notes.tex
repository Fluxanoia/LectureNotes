\documentclass[a4paper, 12pt, twoside]{article}
\usepackage[left = 3cm, right = 3cm]{geometry}
\usepackage[english]{babel}
\usepackage[utf8]{inputenc}
\usepackage{mathtools}
\usepackage{amssymb}
\usepackage{amsmath}
\usepackage{amsthm}
\usepackage{multicol}
\usepackage{multirow}
\usepackage{pgfplots}
\usepackage{pgfplotstable}
\usepackage{listings}
\usepackage{xcolor}
\usepackage{color}
\usepackage{graphicx}

\pgfplotsset{compat=1.5.1}

\lstset{frame=none,
  language=html,
  aboveskip=3mm,
  belowskip=3mm,
  showstringspaces=false,
  columns=flexible,
  basicstyle={\small\ttfamily},
  numbers=none,
  numberstyle=\tiny\color{gray},
  keywordstyle=\color{blue},
  commentstyle=\color{gray},
  stringstyle=\color{orange},
  breaklines=true,
  breakatwhitespace=true,
  tabsize=2
}

\begin{document}

\title{Databases and Cloud Concepts Notes}
\date{}
\author{by Tyler Wright \\
  \\
  github.com/Fluxanoia $\qquad$ fluxanoia.co.uk
}
\maketitle

\vfill

\textit{These notes are not necessarily correct,
consistent, representative of the course as it stands today or, 
rigorous. Any result of the above is not the author's fault.}
\\[\baselineskip]
\textbf{Furthermore, these notes are incomplete and will remain so for the 
foreseeable future.}

% \addtocounter{section}{-1}
% \section{Notation}

We commonly deal with the following concepts in 
Language Engineering
which I will abbreviate as follows for brevity:
\begin{center}
    \begin{tabular}{ | r | c | }
        \hline
        Term & Notation \\
        \hline \hline
        \hline
    \end{tabular}
\end{center}

\newpage

\tableofcontents

\section{The Internet}

The internet is a world-wide computer network, connecting computing devices
also known as hosts or end systems. These connections can take many forms,
such as cables and radio waves. Intermediate switching devices inbetween
hosts are known as routers.

\subsection{Clients and Servers}

A program or machine that responds to requests from others is called a server.
A program or machine that sends requests to a server is a client.

\subsection{Internet Layers}

There are four internet layers: \begin{center}
    \begin{tabular}{ | c | c | p{6cm} | }
        \hline
        Layer & Common Protocol & Description \\     
        \hline \hline
        Application & HTTP & 
        Web browsers making requests and parsing responses \\
        \hline
        Transport   & TCP  & 
        Breaks requests down into numbered packets and
        can reassemble messages \\
        \hline
        Network     & IP   & 
        Attaches addresses to packets and groups
        packets based on their incoming addresses \\
        \hline
        Physical    &      & 
        Sends bits to the local router and assembles bits 
        into packets \\
        \hline
    \end{tabular}
\end{center}

\subsection{Protocols}

Protocols are an agreement on how to communicate.

\subsubsection{HTTP - HyperText Transfer Protocol}

There are four main operations that can be carried out on HTTP resources:
\begin{center}
    \begin{tabular}{ | c | c |}
        \hline
        Operation & Performed by... \\
        \hline \hline
        Creation & \texttt{HTTP POST} \\
        \hline
        Reading  & \texttt{HTTP GET} \\
        \hline
        Updating & \texttt{HTTP PUT} \\
        \hline
        Deletion & \texttt{HTTP DELETE} \\
        \hline
    \end{tabular}
\end{center} Requests are formed by an operation as well as
a \texttt{host} and \texttt{content-type} parameter to describe 
the format of information.

\subsubsection{URL - Uniform Resource Locator}

Each URL is formed by a scheme (like \texttt{http} or \texttt{https}),
a host (like \texttt{www.bristol.ac}), a path (like \texttt{.uk/home/maths}).
Paths can have queries attached, preceded by \texttt{?} as parameters.



\section{HTML - HyperText Markup Language}

\subsection{Tags, Attributes and, Values}

Tags form the structure of HTML, with \texttt{html},
\texttt{head} and, \texttt{body} usually forming the top
levels: \begin{lstlisting}
    <html>
        <head>
            <title>Title<\title>
        <\head>
        <body>
            <p>Paragraph<\p>
        <\body>
    <\html>
\end{lstlisting} Attributes form parts of tags and, as expected,
assign attributes to tags. This can describe the width of elements
(\texttt{width}), the hyperlink attached to text (\texttt{href}) and,
more: \begin{lstlisting}
    <a href="www.bristol.ac.uk">Bristol<\a>
\end{lstlisting}

\subsubsection{Common Tags}

Below is a table containing common HTML tags: \begin{center}
    \begin{tabular}{| c | c |}
        \hline
        Tag & Description \\
        \hline \hline
        \texttt{h1}, $\ldots$ \texttt{h6} & Headings \\
        \hline
        \texttt{p} & Paragraph \\
        \hline
        \texttt{br} & New line \\
        \hline
        \texttt{ul} & Unordered list \\
        \hline
        \texttt{ol} & Ordered list \\
        \hline
        \texttt{li} & List item \\
        \hline
        \texttt{em} & Emphasis \\
        \hline
        \texttt{strong} & Importance \\
        \hline
        \texttt{q} & Quote \\
        \hline
        \texttt{cite} & Citation \\
        \hline
        \texttt{var} & Variable \\
        \hline
        \texttt{code} & Source code \\
        \hline
    \end{tabular}
\end{center}

\subsection{Block and Inline Elements}

Block level elements take up the full width of the container
and start on new lines, so stack vertically. \newline
Inline elements don't start on new lines and only take
up as much width as is necessary, so stack horizontally.

\subsubsection{Block Tags}

Below is a table containing some of the block HTML tags: 
\begin{center}
    \begin{tabular}{| c | p{5cm} |}
        \hline
        Tag & Description \\
        \hline \hline
        \texttt{header} & This is the very top of the page \\
        \hline
        \texttt{main} & This fills the space inbetween the 
        header and footer \\
        \hline
        \texttt{section} & This forms subsections of blocks \\
        \hline
        \texttt{div} & No meaning, for layout purposes \\
        \hline
        \texttt{p} & This forms paragraphs of text \\
        \hline
        \texttt{figure} & This forms images \\
        \hline
        \texttt{nav} & This fills the space left of the main block \\
        \hline
        \texttt{aside} & This fills the space right of the main block \\
        \hline
        \texttt{footer} & This is the very bottom of the page \\
        \hline
    \end{tabular}
\end{center}

\subsection{Common Attributes}

Below is a table containing some of the common HTML attributes: 
\begin{center}
    \begin{tabular}{| c | p{5cm} |}
        \hline
        Attribute & Description \\
        \hline \hline
        \texttt{id} & Uniquely identifies the tag with the value \\
        \hline
        \texttt{class} & Marks tags you want to operate as a group \\
        \hline
    \end{tabular}
\end{center}

\newpage

\subsection{Forms}

The form tag, shown in the example: \begin{lstlisting}
    <form method="post" action="/comment/comment.php">
        <p>
            <label for="name">Name:</label>
            <input type="text" id="name" />
        </p>
        <p>
            <button type="submit">OK</button>
        </p>
    </form>
\end{lstlisting} The \texttt{method} attribute takes two values
\texttt{GET} and \texttt{POST}. The former places the
information in the URL parameters and the latter utilises
a HTTP request. \newline
The \texttt{action} attribute defines an action to be
performed when the form is submitted. In this case,
it's sent to a PHP script. \newline
The \texttt{label for} attribute should link to a \texttt{input 
id}. Additionally, the \texttt{input name} attribute is the key
which accompanies the input value in the request. \newline
The \texttt{button} tag has three types, a \texttt{submit} button
that makes the request, a \texttt{reset} button that resets all fields and,
a \texttt{button} type that does nothing by default but can
be configured using Javascript.
\\[\baselineskip]
Types can be used to make field use a specific format or be
required. Additionally, they can be given placeholder text
and autocompletion properties.
\section{CSS - Cascading Stylesheets}

CSS describes how HTML elements should be drawn to the
screen. It can be used: \begin{itemize}
    \item Inline with the \texttt{style} attribute,
    \item Internally with the \texttt{style} tag in
    the \texttt{head} section,
    \item Externally via linking to a \texttt{.css} file. 
\end{itemize}

\subsection{Stylesheet Linking}

We can link to external stylesheets as follows: \begin{lstlisting}
    <link rel="stylesheet" href="styles.css"> 
\end{lstlisting}

\subsection{CSS File Structure}

The parts of CSS files are formed as follows: \begin{lstlisting}
    selector {
        key: value;
    }
\end{lstlisting}

\subsubsection{Selectors}

Selectors can be a: \begin{itemize}
    \item tag, written simply as \texttt{div},
    \item class, written as \texttt{.class},
    \item id, written as \texttt{\#id}.
\end{itemize}
\section{Encoding}

Encoding is about mapping symbols to bytes. There are many
standards, of which we will see a few.

\subsection{ASCII - American Standard Code for Information 
\newline Interchange}

ASCII contains the digits 0 to 9, the lowercase and uppercase
English alphabet, some punctuation and, special characters.
Each of these is represented by a seven bits.

\subsection{UTF - Unicode Transformation Format}

The first 128 characters of UTF-8 correspond to the
characters of ASCII making UTF-8 backwards compatible 
with ASCII. The unicode character set contains around 
136,000 characters. The individual formats (UTF-8, UTF-16, etc.) 
encode these differently and have different memory
requirements. We can choose to use UTF-8 in HTML as follows:
\begin{lstlisting}
    <meta charset="UTF8" />
\end{lstlisting}

\subsection{CSV - Comma Separated Values}

CSV use commas to separate field and \texttt{CR LF} to
separate records. The record at the top is reserved (usually)
for the titles of the columns.

\subsubsection{Streams}

We can read CSV files in as streams. Thinking about stream
operations is important when considering web programming as data
is usually a stream. We cannot perform operations on streams
that require more than one pass (like standard deviation).
\\[\baselineskip]
A few operations we can do are: \begin{itemize}
    \item filter - omitting as we go,
    \item map - mapping as we go,
    \item sum - summing as we go. 
\end{itemize}
\section{Representing Data}

\subsection{Trees}

Representing data as trees requires three separators for the
start and end of an item, and for fields. Additional
separators are needed for quoting and escaping.

\subsection{XML - Extensible Markup Language}

The goal of XML is to create a straight-forward way of 
representing data that is machine and human readable
that also can give context to data.
\\[\baselineskip]
It allows portable, non-proprietary, hierarchical data storage.
Common parsers are XPath and XQuery.

\subsubsection{The Structure}

XML documents are formed of five components: \begin{itemize}
    \item the XML declaration,
    \item the root element,
    \item attributes,
    \item child elements,
    \item text data,
\end{itemize} illustrated below: \begin{lstlisting}[language=XML]
    <?xml version="1.0" encoding="UTF-8"?>
    <labReport patientId="1234567890" specimenID="750853">
        <testResult date="2005-01-25-T12:15:37-09:00">
            <test>
                <testCode scheme="myCodes">42Hxx</testCode>
                <testName>Potassium</testName>
            </test>
        </testResult>
    </labReport>
\end{lstlisting}

\subsubsection{Validation}

There are two validation methods, DTD (Document Type Definition)
and schema. We consider the example: \begin{lstlisting}[language=XML]
    <candidate>
        <name>Catherine Slade</name>
        <party>
            <name>Green</name>
        </party>
        <ward>
            <name>Bedminster</name>
            <electorate>9951</electorate>
        </ward>
    </candidate>
\end{lstlisting} we have the DTD validation format: 
\begin{lstlisting}
    <?xml version="1.0"?>
    <!DOCTYPE candidate [
        <!ELEMENT candidate (name, party, ward)>
        <!ELEMENT name (#PCDATA)>
        <!ELEMENT party (name)>
        <!ELEMENT ward (name, electorate)>
        <!ELEMENT electorate (#PCDATA)>
    ]>
    <candidate> ... </candidate>
\end{lstlisting} where \texttt{PCDATA} is parsed character data.
Also, we have the XML Schema Definition:
\begin{lstlisting}[language=XML]
    <?xml version="1.0"?>
    <xs:schema xmlns:xs="http://www.w3.org/2001/XMLSchema">
        <xs:element name="candidate">
            <xs:complexType><xs:sequence>
        <xs:element name="name" type="xs:string" />
        <xs:element name="party"><xs:complexType><xs:sequence>
            <xs:element name="name" type="xs:string" />
        </xs:sequence></xs:complexType></xs:element>
        <xs:element name="ward"><xs:complexType><xs:sequence>
            <xs:element name="name" type="xs:string" />
            <xs:element name="electorate"
                        type="xs:nonNegativeInteger" />
            </xs:sequence></xs:complexType></xs:element>
        </xs:sequence></xs:complexType></xs:element>
    </xs:schema>
\end{lstlisting}

\subsection{JSON - Javascript Object Notation}

JSON is a machine and human friendly data format.
As it is formed by text, it can be parsed and generated by most
programming languages and can be transmitted easily.

\subsubsection{Comparisions to XML}

Here are some key differences: \begin{itemize}
    \item JSON allows arrays,
    \item JSON tends to be shorter,
    \item JSON is quicker to read and write,
    \item JSON can be parsed by standard functions.
\end{itemize}
\section{Databases}

\subsection{Web Architecture}

A multitier architecture or $n$-tier architecture is a 
client-server
architecture which physically separates presentation,
application processing and data management functions.
\\[\baselineskip]
A common example is the 3-tier architecture which is formed
by presentation, application and, database layers.

\subsection{SQL}

SQL tables are formed by these main components: \begin{itemize}
    \item Columns / Fields / Attributes,
    \item Rows / Records / Tuples.
\end{itemize}

\subsubsection{Super Keys}

A super key is a combination of the fields of a table such that
using just those columns, we can uniquely identify each record.

\subsubsection{Candidate Keys}

A candidate key is a minimal super key.

\subsubsection{The Primary Key}

The primary key is a chosen, 'most important', candidate key.

\newpage

\subsubsection{Useful Commands}

There are many useful (MariaDB) SQL commands:
\begin{center}
    \begin{tabular}{| c | p{5cm} | p{5cm} |}
        \hline
        Command & Description & More information \\
        \hline \hline
        \texttt{CREATE} 
        & Creates a table 
        & \\
        \hline
        \texttt{DROP} 
        & Deletes a table 
        & \\
        \hline 
        \texttt{TRUNCATE} 
        & Deletes all records in a table 
        & \\
        \hline
        \texttt{SELECT}
        & Picks values from a table 
        & Use * to select all \\
        \hline
        \texttt{INSERT} 
        & Inserts a record into a table 
        & \\
        \hline
        \texttt{DELETE} 
        & Deletes all records in a table 
        & Usually used with a \texttt{WHERE} clause \\
        \hline
        \texttt{UPDATE} 
        & Updates values in a table 
        & \\
        \hline
        \texttt{AUTO\_INCREMENT} 
        & Automatically increments and assigns a field
        & \\
        \hline
        \texttt{--} 
        & Initiates a comment 
        & \\
        \hline
        \texttt{''} 
        & Used for strings & \\
        \hline
        \texttt{``} 
        & Used for database values & \\
        \hline
    \end{tabular}
\end{center}

\subsubsection{Exporting and Importing}

Using mysqldump we can export a database using the following
command in the MySQL client command line (with a following 
example): 
\begin{lstlisting}
    mysqldump -u student [options]
        dbname > filename.sql

    mysqldump --skip-lock-tables
        --add-drop-table dbname > filename.sql
\end{lstlisting} We can similarly use the client to import
a database: \begin{lstlisting}
    mysql -u student dbname < filename.sql
\end{lstlisting}

\section{The Cloud}

Cloud computing is the on-demand delivery of IT resources and applications via the
Internet. This allows quick deployment of services, optimisation of resources and,
increased efficiency.

\subsection{Server Scaling}

Servers can scale up or down (vertically); or out or in (horizontally).
Vertical scaling is about the size of the individual server, this allows code to be
portable but cannot be done indefinitely. Horizontal scaling is about the number of
servers which has an incremental cost but comes with the issues of a distributed system.
\\[\baselineskip]
This can be performed in tandem with multiple web instances, load handlers, caches and, 
content distribution networks to increase performance for concurrent users.

\subsection{Services}

Infrastructure, platforms and, software can be provided as a service:
\begin{center}
    \begin{tabular}{ c | c | c | c | }
        \cline{2-4}
                         & Infrastructure            & Platform                  & Software \\
        \cline{2-4}
        \hline
        \multicolumn{1}{|c|}{Data}             & \multirow{3}{*}{provided} & \multirow{2}{*}{provided} & provided \\
        \cline{1-1} \cline{4-4}
        \multicolumn{1}{|c|}{Applications}     &                           &                           & \multirow{3}{*}{rented} \\
        \cline{1-1} \cline{3-3}
        \multicolumn{1}{|c|}{OS and Libraries} &                           & \multirow{2}{*}{rented}   & \\
        \cline{1-1} \cline{2-2}
        \multicolumn{1}{|c|}{Hardware}         & rented                    &                           & \\
        \hline
    \end{tabular}
\end{center} The advantages of these services are reliability, lower overhead and,
scalability.

\subsection{Imaging and Configuration Management}

One issue when scaling out is needing to configure new servers with the same
applications as your current server(s). This can be performed with: \begin{itemize}
    \item System images which are snapshots of a system that can be applied
    to another,
    \item Scripts that perform all the required tasks - this can be somewhat
    automated by using a configuration manager,
\end{itemize} these scripts often use package managers like \texttt{apt} and 
\texttt{maven}.

\subsection{Virtualisation}

Library hell is when two applications that you need to run concurrently require
different versions of the same library. To solve this problem, we can separate
problematic applications by putting them in their own virtual machine using
a hypervisor.



\end{document}