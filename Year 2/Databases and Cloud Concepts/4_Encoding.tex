\section{Encoding}

Encoding is about mapping symbols to bytes. There are many
standards, of which we will see a few.

\subsection{ASCII - American Standard Code for Information 
\newline Interchange}

ASCII contains the digits 0 to 9, the lowercase and uppercase
English alphabet, some punctuation and, special characters.
Each of these is represented by a seven bits.

\subsection{UTF - Unicode Transformation Format}

The first 128 characters of UTF-8 correspond to the
characters of ASCII making UTF-8 backwards compatible 
with ASCII. The unicode character set contains around 
136,000 characters. The individual formats (UTF-8, UTF-16, etc.) 
encode these differently and have different memory
requirements. We can choose to use UTF-8 in HTML as follows:
\begin{lstlisting}
    <meta charset="UTF8" />
\end{lstlisting}

\subsection{CSV - Comma Separated Values}

CSV use commas to separate field and \texttt{CR LF} to
separate records. The record at the top is reserved (usually)
for the titles of the columns.

\subsubsection{Streams}

We can read CSV files in as streams. Thinking about stream
operations is important when considering web programming as data
is usually a stream. We cannot perform operations on streams
that require more than one pass (like standard deviation).
\\[\baselineskip]
A few operations we can do are: \begin{itemize}
    \item filter - omitting as we go,
    \item map - mapping as we go,
    \item sum - summing as we go. 
\end{itemize}