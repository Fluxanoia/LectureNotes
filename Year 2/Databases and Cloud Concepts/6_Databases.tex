\section{Databases}

\subsection{Web Architecture}

A multitier architecture or $n$-tier architecture is a 
client-server
architecture which physically separates presentation,
application processing and data management functions.
\\[\baselineskip]
A common example is the 3-tier architecture which is formed
by presentation, application and, database layers.

\subsection{SQL}

SQL tables are formed by these main components: \begin{itemize}
    \item Columns / Fields / Attributes,
    \item Rows / Records / Tuples.
\end{itemize}

\subsubsection{Super Keys}

A super key is a combination of the fields of a table such that
using just those columns, we can uniquely identify each record.

\subsubsection{Candidate Keys}

A candidate key is a minimal super key.

\subsubsection{The Primary Key}

The primary key is a chosen, 'most important', candidate key.

\newpage

\subsubsection{Useful Commands}

There are many useful (MariaDB) SQL commands:
\begin{center}
    \begin{tabular}{| c | p{5cm} | p{5cm} |}
        \hline
        Command & Description & More information \\
        \hline \hline
        \texttt{CREATE} 
        & Creates a table 
        & \\
        \hline
        \texttt{DROP} 
        & Deletes a table 
        & \\
        \hline 
        \texttt{TRUNCATE} 
        & Deletes all records in a table 
        & \\
        \hline
        \texttt{SELECT}
        & Picks values from a table 
        & Use * to select all \\
        \hline
        \texttt{INSERT} 
        & Inserts a record into a table 
        & \\
        \hline
        \texttt{DELETE} 
        & Deletes all records in a table 
        & Usually used with a \texttt{WHERE} clause \\
        \hline
        \texttt{UPDATE} 
        & Updates values in a table 
        & \\
        \hline
        \texttt{AUTO\_INCREMENT} 
        & Automatically increments and assigns a field
        & \\
        \hline
        \texttt{--} 
        & Initiates a comment 
        & \\
        \hline
        \texttt{''} 
        & Used for strings & \\
        \hline
        \texttt{``} 
        & Used for database values & \\
        \hline
    \end{tabular}
\end{center}

\subsubsection{Exporting and Importing}

Using mysqldump we can export a database using the following
command in the MySQL client command line (with a following 
example): 
\begin{lstlisting}
    mysqldump -u student [options]
        dbname > filename.sql

    mysqldump --skip-lock-tables
        --add-drop-table dbname > filename.sql
\end{lstlisting} We can similarly use the client to import
a database: \begin{lstlisting}
    mysql -u student dbname < filename.sql
\end{lstlisting}

\newpage

\subsection{Relational Databases}

In relational databases we have tables (relations) formed by
rows (tuples) and columns (attributes) containing data.

\subsubsection{Entities}

When forming a database, we have to consider the entities
that are of interest. Entities have attributes (generally,
if an attribute is referred to by multiple entities, it
should be its own entity).

\subsubsection{Keys}

Keys identify instances of our entities. Considering candidate keys
(a minimal key uniquely identifying entities), if it spans multiple
attributes we say it is composite. Additionally, an artifical
attribute generated for the sole purpose of being a primary key
is called a surrogate key.

\subsubsection{Relationships}

Relationships associate types of entities. We use Crow's Foot
notation:
\begin{center}
    \includegraphics[width = 7cm]{crowsfoot.PNG}
\end{center} The example on the right describing the fact that a
unit is taken by at least one student and each student takes
any number of units.
\\[\baselineskip]
We can introduce intermediary entities that describe relationships
between entities. \begin{center}
    \includegraphics[width = 9cm]{associative.PNG}
\end{center} Note how the new entity is unique to the 
pair of entities it describes the relationship between.

\subsubsection{Forming Tables from Relationships}

Entities become tables, attributes become columns, entity instances
become rows. 
\\[\baselineskip]
For unique relationships between entities, we
can use a \texttt{UNIQUE} foreign key relating them. 
If two entities require each other, we merge their tables. 
Otherwise, wherever the relationship is mandatory, we place the key
(with the property \texttt{NOT NULL}) or if it isn't mandatory at all,
we can use the property \texttt{NULL}.
\\[\baselineskip]
For one-to-many relationships, we place the foreign key on the
'many' side of the relationship.
\\[\baselineskip]
For many-to-many relationships, we create a new table (a 'join' table)
containing two foreign keys to the original tables (with its 
primary key being the composite of these two foreign keys).
This table contains pairs of IDs from the original tables to 
describe relationships.

\subsection{Projection and Selection}

Projection is about selecting columns from tables whereas 
Selection is about selecting rows from tables. While
projecting, we can perform operations on our columns where
it makes sense (concatenating first and last names, adding
one, etc.). While selecting, we can provide constraints on
what records are selected.

\subsection{Products and Joining}

\subsubsection{Inner Joining}

To get the cartesian product of two tables we can use the following
command: \begin{lstlisting}
    SELECT * FROM TABLE1, TABLE2 WHERE TABLE1.id = TABLE2.id;
\end{lstlisting} This pairs each record in \texttt{TABLE1} with
a record in \texttt{TABLE2}. We can write this using the 
\texttt{INNER JOIN}: \begin{lstlisting}
    SELECT * FROM TABLE1 INNER JOIN TABLE2 ON TABLE1.id = TABLE2.id;
\end{lstlisting} and can stack this calls too: \begin{lstlisting}
    SELECT * FROM TABLE1
        INNER JOIN TABLE2 ON TABLE1.id = TABLE2.id
        INNER JOIN TABLE3 ON TABLE2.id = TABLE3.id
        ...
    WHERE ...;
\end{lstlisting} 

\newpage
\noindent
If the two tables have a field with the same
name, we can use \texttt{JOIN USING}: \begin{lstlisting}
    SELECT * FROM TABLE1 
        INNER JOIN TABLE2
    USING (id);
\end{lstlisting} In fact, using \texttt{NATURAL JOIN} SQL will
search for an identical column automatically: \begin{lstlisting}
    SELECT * FROM TABLE1 
        NATURAL JOIN TABLE2;
\end{lstlisting}

\subsubsection{Outer Joining}

We may potentially join while referencing rows with \texttt{NULL}
content. This is where we use the \texttt{LEFT}, \texttt{RIGHT}
and, \texttt{FULL OUTER JOIN}s. The \texttt{LEFT} outer join
ensures that each record from the left table appears in the
result and similarly for the \texttt{RIGHT} outer join.
The \texttt{FULL} outer join ensure each record from both tables
appears at least once.

\subsubsection{Self Joining}

We can join a table to itself, for example - we may want to
find all pairs of field subject to a constraint on one
of their attributes. This can be done using the \texttt{INNER JOIN}:
\begin{lstlisting}
    SELECT * FROM TABLE1 T1
        INNER JOIN TABLE2 T2
        ON ...
        WHERE ...;
\end{lstlisting}

\subsection{Set Operations}

The set operation commands have the form: \begin{lstlisting}
    SELECT ...
        [UNION [ALL], INTERSECT, EXCEPT]
    SELECT ...;
\end{lstlisting} They all behave as expected, the extra \texttt{ALL}
condition on the \texttt{UNION} command ensures that duplicates
are not removed.

\subsection{Summarising}

\subsubsection{Unique Records}

When selecting data, we can specify that we want the result
to be \texttt{DISTINCT}: \begin{lstlisting}
    SELECT DISTINCT ... FROM ...;
\end{lstlisting}

\subsubsection{Summing Records}

When selecting data, we can return the number of rows
in the resulting table with \texttt{COUNT}: \begin{lstlisting}
    SELECT COUNT(...) FROM ...;
\end{lstlisting} By using \texttt{COUNT} in conjunction
with \texttt{DISTINCT} can be more informative.
Additionally, we can use * to count all the rows in a table.
\\[\baselineskip]
\texttt{COUNT} always returns a single value (in a row) and
ignores \texttt{NULL} values.

\subsubsection{Grouping}

We can use \texttt{GROUP BY} to associate \texttt{COUNT}s
with other values in our table: \begin{lstlisting}
    SELECT key, COUNT(*) FROM ... GROUP BY key;
\end{lstlisting} In fact, \texttt{COUNT} is what we would
call a 'aggregate' function - taking in a list of values
and returning a single value - and we can replace \texttt{COUNT}
with other aggregate functions like the average, maximum, minimum
and, median.
\\[\baselineskip]
We have that \texttt{WHERE} clauses must appear before
aggregation, so cannot refer to aggregates. However,
we can use \texttt{HAVING} which is capable of refering to
aggregates and column aliases exclusively.

\newpage

\subsection{Subqueries}

We can utilise subqueries by putting queries in
parentheses in place of values in queries. We can
use: \begin{lstlisting}
    WHERE ...
        [
            [NOT] IN,
            [NOT] EXISTS,
            ANY,
            ALL,
        ];
\end{lstlisting} with subqueries to filter quickly.

\subsubsection{Views}

We can use \texttt{CREATE VIEW} to create a reference to a select statement
which will act like a table: \begin{lstlisting}
    CREATE VIEW name AS
        SELECT ... FROM ...;
\end{lstlisting}

\subsection{Execution Order}

The execution order of SQL queries is detailed below:
\begin{lstlisting}
    SELECT ...
    FROM ...
    ... JOIN ... 
    WHERE ...
    GROUP BY ...
    HAVING ...
    ORDER BY ...;
\end{lstlisting}

\subsection{Normalisation}

This eliminates redundancy and dependency which causes anomalies when inserting
and deleting data. This causes the database to become more complicated but
allows operations on it to be performed more easily.
\\[\baselineskip]
The normal forms are numbered and cumulative.

\subsubsection{1NF - The First Normal Form}

A schema is in 1NF if there are no: \begin{itemize}
    \item Collection-valued attributes,
    \item Repeated attribute groups, as in duplicate
    information in different tables,
    \item Duplicate rows.
\end{itemize}

\subsubsection{2NF - The Second Normal Form}

A scheme is in 2NF if it is in 1NF and there are no partial functional
dependencies from a candidate key to a non-key attribute. This means
an attribute should not be uniquely defined by a subset of the candidate key.

\subsubsection{3NF - The Third Normal Form}

A schema is in 3NF if it is in 2NF and has no transitive functional dependencies.
This means there are no set of three attributes $\{A, B, C\}$ where $B$ and $C$
are non-key and: \begin{gather*}
    A \rightarrow B \rightarrow C,
\end{gather*} meaning $A$ is functionally dependent on $B$ which is functionally 
dependent on $C$. As it doesn't matter if $A$ is key, we can just consider functional
dependencies between two non-key attributes and just use our candidate key as $A$.
We can see that the consequence of this is that all non-key attributes depend only on
the candidate keys.
\\[\baselineskip]
When there is such a transitive functional dependency, we can decompose the table
into two tables without losing any information (lossless decomposition).

\subsubsection{BCNF - Boyce-Codd Normal Form}

A scheme is in BCNF if it is in 3NF and the left side of each non-trivial
functional dependency is a superkey.

\subsubsection{4NF - The Fourth Normal Form}

A schema is in 4NF if it is in BCNF and for every multi-valued dependency, 
the dependency is trivial or the left side is a superkey. A multi-valued dependency
is where when we insert a new record, insert multiple to satisfy the structure of the
table.

\subsubsection{5NF - The Fifth Normal Form}

A scheme is in 5NF if it is in 4NF and all decompositions of it are not lossless.

\subsection{JDBC - Java Database Connectivity}

In JDBC, we prepare SQL statements with placeholder parameters which we can
then execute, passing in the necessary parameters.
