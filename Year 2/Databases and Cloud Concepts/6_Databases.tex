\section{Databases}

\subsection{Web Architecture}

A multitier architecture or $n$-tier architecture is a 
client-server
architecture which physically separates presentation,
application processing and data management functions.
\\[\baselineskip]
A common example is the 3-tier architecture which is formed
by presentation, application and, database layers.

\subsection{SQL}

SQL tables are formed by these main components: \begin{itemize}
    \item Columns / Fields / Attributes,
    \item Rows / Records / Tuples.
\end{itemize}

\subsubsection{Super Keys}

A super key is a combination of the fields of a table such that
using just those columns, we can uniquely identify each record.

\subsubsection{Candidate Keys}

A candidate key is a minimal super key.

\subsubsection{The Primary Key}

The primary key is a chosen, 'most important', candidate key.

\newpage

\subsubsection{Useful Commands}

There are many useful (MariaDB) SQL commands:
\begin{center}
    \begin{tabular}{| c | p{5cm} | p{5cm} |}
        \hline
        Command & Description & More information \\
        \hline \hline
        \texttt{CREATE} 
        & Creates a table 
        & \\
        \hline
        \texttt{DROP} 
        & Deletes a table 
        & \\
        \hline 
        \texttt{TRUNCATE} 
        & Deletes all records in a table 
        & \\
        \hline
        \texttt{SELECT}
        & Picks values from a table 
        & Use * to select all \\
        \hline
        \texttt{INSERT} 
        & Inserts a record into a table 
        & \\
        \hline
        \texttt{DELETE} 
        & Deletes all records in a table 
        & Usually used with a \texttt{WHERE} clause \\
        \hline
        \texttt{UPDATE} 
        & Updates values in a table 
        & \\
        \hline
        \texttt{AUTO\_INCREMENT} 
        & Automatically increments and assigns a field
        & \\
        \hline
        \texttt{--} 
        & Initiates a comment 
        & \\
        \hline
        \texttt{''} 
        & Used for strings & \\
        \hline
        \texttt{``} 
        & Used for database values & \\
        \hline
    \end{tabular}
\end{center}

\subsubsection{Exporting and Importing}

Using mysqldump we can export a database using the following
command in the MySQL client command line (with a following 
example): 
\begin{lstlisting}
    mysqldump -u student [options]
        dbname > filename.sql

    mysqldump --skip-lock-tables
        --add-drop-table dbname > filename.sql
\end{lstlisting} We can similarly use the client to import
a database: \begin{lstlisting}
    mysql -u student dbname < filename.sql
\end{lstlisting}
