\section{The Cloud}

Cloud computing is the on-demand delivery of IT resources and applications via the
Internet. This allows quick deployment of services, optimisation of resources and,
increased efficiency.

\subsection{Server Scaling}

Servers can scale up or down (vertically); or out or in (horizontally).
Vertical scaling is about the size of the individual server, this allows code to be
portable but cannot be done indefinitely. Horizontal scaling is about the number of
servers which has an incremental cost but comes with the issues of a distributed system.
\\[\baselineskip]
This can be performed in tandem with multiple web instances, load handlers, caches and, 
content distribution networks to increase performance for concurrent users.

\subsection{Services}

Infrastructure, platforms and, software can be provided as a service:
\begin{center}
    \begin{tabular}{ c | c | c | c | }
        \cline{2-4}
                         & Infrastructure            & Platform                  & Software \\
        \cline{2-4}
        \hline
        \multicolumn{1}{|c|}{Data}             & \multirow{3}{*}{provided} & \multirow{2}{*}{provided} & provided \\
        \cline{1-1} \cline{4-4}
        \multicolumn{1}{|c|}{Applications}     &                           &                           & \multirow{3}{*}{rented} \\
        \cline{1-1} \cline{3-3}
        \multicolumn{1}{|c|}{OS and Libraries} &                           & \multirow{2}{*}{rented}   & \\
        \cline{1-1} \cline{2-2}
        \multicolumn{1}{|c|}{Hardware}         & rented                    &                           & \\
        \hline
    \end{tabular}
\end{center} The advantages of these services are reliability, lower overhead and,
scalability.

\subsection{Imaging and Configuration Management}

One issue when scaling out is needing to configure new servers with the same
applications as your current server(s). This can be performed with: \begin{itemize}
    \item System images which are snapshots of a system that can be applied
    to another,
    \item Scripts that perform all the required tasks - this can be somewhat
    automated by using a configuration manager,
\end{itemize} these scripts often use package managers like \texttt{apt} and 
\texttt{maven}.

\subsection{Virtualisation}

Library hell is when two applications that you need to run concurrently require
different versions of the same library. To solve this problem, we can separate
problematic applications by putting them in their own virtual machine using
a hypervisor.

