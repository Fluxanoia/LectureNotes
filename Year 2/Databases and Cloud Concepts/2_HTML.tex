\section{HTML - HyperText Markup Language}

\subsection{Tags, Attributes and, Values}

Tags form the structure of HTML, with \texttt{html},
\texttt{head} and, \texttt{body} usually forming the top
levels: \begin{lstlisting}
    <html>
        <head>
            <title>Title<\title>
        <\head>
        <body>
            <p>Paragraph<\p>
        <\body>
    <\html>
\end{lstlisting} Attributes form parts of tags and, as expected,
assign attributes to tags. This can describe the width of elements
(\texttt{width}), the hyperlink attached to text (\texttt{href}) and,
more: \begin{lstlisting}
    <a href="www.bristol.ac.uk">Bristol<\a>
\end{lstlisting}

\subsubsection{Common Tags}

Below is a table containing common HTML tags: \begin{center}
    \begin{tabular}{| c | c |}
        \hline
        Tag & Description \\
        \hline \hline
        \texttt{h1}, $\ldots$ \texttt{h6} & Headings \\
        \hline
        \texttt{p} & Paragraph \\
        \hline
        \texttt{br} & New line \\
        \hline
        \texttt{ul} & Unordered list \\
        \hline
        \texttt{ol} & Ordered list \\
        \hline
        \texttt{li} & List item \\
        \hline
        \texttt{em} & Emphasis \\
        \hline
        \texttt{strong} & Importance \\
        \hline
        \texttt{q} & Quote \\
        \hline
        \texttt{cite} & Citation \\
        \hline
        \texttt{var} & Variable \\
        \hline
        \texttt{code} & Source code \\
        \hline
    \end{tabular}
\end{center}

\subsection{Block and Inline Elements}

Block level elements take up the full width of the container
and start on new lines, so stack vertically. \newline
Inline elements don't start on new lines and only take
up as much width as is necessary, so stack horizontally.

\subsubsection{Block Tags}

Below is a table containing some of the block HTML tags: 
\begin{center}
    \begin{tabular}{| c | p{5cm} |}
        \hline
        Tag & Description \\
        \hline \hline
        \texttt{header} & This is the very top of the page \\
        \hline
        \texttt{main} & This fills the space inbetween the 
        header and footer \\
        \hline
        \texttt{section} & This forms subsections of blocks \\
        \hline
        \texttt{div} & No meaning, for layout purposes \\
        \hline
        \texttt{p} & This forms paragraphs of text \\
        \hline
        \texttt{figure} & This forms images \\
        \hline
        \texttt{nav} & This fills the space left of the main block \\
        \hline
        \texttt{aside} & This fills the space right of the main block \\
        \hline
        \texttt{footer} & This is the very bottom of the page \\
        \hline
    \end{tabular}
\end{center}

\subsection{Common Attributes}

Below is a table containing some of the common HTML attributes: 
\begin{center}
    \begin{tabular}{| c | p{5cm} |}
        \hline
        Attribute & Description \\
        \hline \hline
        \texttt{id} & Uniquely identifies the tag with the value \\
        \hline
        \texttt{class} & Marks tags you want to operate as a group \\
        \hline
    \end{tabular}
\end{center}

\newpage

\subsection{Forms}

The form tag, shown in the example: \begin{lstlisting}
    <form method="post" action="/comment/comment.php">
        <p>
            <label for="name">Name:</label>
            <input type="text" id="name" />
        </p>
        <p>
            <button type="submit">OK</button>
        </p>
    </form>
\end{lstlisting} The \texttt{method} attribute takes two values
\texttt{GET} and \texttt{POST}. The former places the
information in the URL parameters and the latter utilises
a HTTP request. \newline
The \texttt{action} attribute defines an action to be
performed when the form is submitted. In this case,
it's sent to a PHP script. \newline
The \texttt{label for} attribute should link to a \texttt{input 
id}. Additionally, the \texttt{input name} attribute is the key
which accompanies the input value in the request. \newline
The \texttt{button} tag has three types, a \texttt{submit} button
that makes the request, a \texttt{reset} button that resets all fields and,
a \texttt{button} type that does nothing by default but can
be configured using Javascript.
\\[\baselineskip]
Types can be used to make field use a specific format or be
required. Additionally, they can be given placeholder text
and autocompletion properties.