\section{Representing Data}

\subsection{Trees}

Representing data as trees requires three separators for the
start and end of an item, and for fields. Additional
separators are needed for quoting and escaping.

\subsection{XML - Extensible Markup Language}

The goal of XML is to create a straight-forward way of 
representing data that is machine and human readable
that also can give context to data.
\\[\baselineskip]
It allows portable, non-proprietary, hierarchical data storage.
Common parsers are XPath and XQuery.

\subsubsection{The Structure}

XML documents are formed of five components: \begin{itemize}
    \item the XML declaration,
    \item the root element,
    \item attributes,
    \item child elements,
    \item text data,
\end{itemize} illustrated below: \begin{lstlisting}[language=XML]
    <?xml version="1.0" encoding="UTF-8"?>
    <labReport patientId="1234567890" specimenID="750853">
        <testResult date="2005-01-25-T12:15:37-09:00">
            <test>
                <testCode scheme="myCodes">42Hxx</testCode>
                <testName>Potassium</testName>
            </test>
        </testResult>
    </labReport>
\end{lstlisting}

\subsubsection{Validation}

There are two validation methods, DTD (Document Type Definition)
and schema. We consider the example: \begin{lstlisting}[language=XML]
    <candidate>
        <name>Catherine Slade</name>
        <party>
            <name>Green</name>
        </party>
        <ward>
            <name>Bedminster</name>
            <electorate>9951</electorate>
        </ward>
    </candidate>
\end{lstlisting} we have the DTD validation format: 
\begin{lstlisting}
    <?xml version="1.0"?>
    <!DOCTYPE candidate [
        <!ELEMENT candidate (name, party, ward)>
        <!ELEMENT name (#PCDATA)>
        <!ELEMENT party (name)>
        <!ELEMENT ward (name, electorate)>
        <!ELEMENT electorate (#PCDATA)>
    ]>
    <candidate> ... </candidate>
\end{lstlisting} where \texttt{PCDATA} is parsed character data.
Also, we have the XML Schema Definition:
\begin{lstlisting}[language=XML]
    <?xml version="1.0"?>
    <xs:schema xmlns:xs="http://www.w3.org/2001/XMLSchema">
        <xs:element name="candidate">
            <xs:complexType><xs:sequence>
        <xs:element name="name" type="xs:string" />
        <xs:element name="party"><xs:complexType><xs:sequence>
            <xs:element name="name" type="xs:string" />
        </xs:sequence></xs:complexType></xs:element>
        <xs:element name="ward"><xs:complexType><xs:sequence>
            <xs:element name="name" type="xs:string" />
            <xs:element name="electorate"
                        type="xs:nonNegativeInteger" />
            </xs:sequence></xs:complexType></xs:element>
        </xs:sequence></xs:complexType></xs:element>
    </xs:schema>
\end{lstlisting}

\subsection{JSON - Javascript Object Notation}

JSON is a machine and human friendly data format.
As it is formed by text, it can be parsed and generated by most
programming languages and can be transmitted easily.

\subsubsection{Comparisions to XML}

Here are some key differences: \begin{itemize}
    \item JSON allows arrays,
    \item JSON tends to be shorter,
    \item JSON is quicker to read and write,
    \item JSON can be parsed by standard functions.
\end{itemize}