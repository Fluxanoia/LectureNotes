\section{The Internet}

The internet is a world-wide computer network, connecting computing devices
also known as hosts or end systems. These connections can take many forms,
such as cables and radio waves. Intermediate switching devices inbetween
hosts are known as routers.

\subsection{Clients and Servers}

A program or machine that responds to requests from others is called a server.
A program or machine that sends requests to a server is a client.

\subsection{Internet Layers}

There are four internet layers: \begin{center}
    \begin{tabular}{ | c | c | p{6cm} | }
        \hline
        Layer & Common Protocol & Description \\     
        \hline \hline
        Application & HTTP & 
        Web browsers making requests and parsing responses \\
        \hline
        Transport   & TCP  & 
        Breaks requests down into numbered packets and
        can reassemble messages \\
        \hline
        Network     & IP   & 
        Attaches addresses to packets and groups
        packets based on their incoming addresses \\
        \hline
        Physical    &      & 
        Sends bits to the local router and assembles bits 
        into packets \\
        \hline
    \end{tabular}
\end{center}

\subsection{Protocols}

Protocols are an agreement on how to communicate.

\subsubsection{HTTP - HyperText Transfer Protocol}

There are four main operations that can be carried out on HTTP resources:
\begin{center}
    \begin{tabular}{ | c | c |}
        \hline
        Operation & Performed by... \\
        \hline \hline
        Creation & \texttt{HTTP POST} \\
        \hline
        Reading  & \texttt{HTTP GET} \\
        \hline
        Updating & \texttt{HTTP PUT} \\
        \hline
        Deletion & \texttt{HTTP DELETE} \\
        \hline
    \end{tabular}
\end{center} Requests are formed by an operation as well as
a \texttt{host} and \texttt{content-type} parameter to describe 
the format of information.

\subsubsection{URL - Uniform Resource Locator}

Each URL is formed by a scheme (like \texttt{http} or \texttt{https}),
a host (like \texttt{www.bristol.ac}), a path (like \texttt{.uk/home/maths}).
Paths can have queries attached, preceded by \texttt{?} as parameters.


