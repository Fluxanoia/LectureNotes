\section{Digraphs}

A digraph (or directed graph) is a graph where each of the edges
has a direction. This direction means the edge can only be traversed
in a single direction.

\subsection{Neighbourhoods in Digraphs}

For a digraph $G = (V, E)$, we consider the edges entering and 
leaving a vertex (or set of vertices). Take $v$ in $V$:
\\[\baselineskip]
The in-neighbourhood of $v$ is the set of edges that enter $v$
denoted by $N^-(v)$.
\\[\baselineskip]
The out-neighbourhood of $v$ is the set of edges that leave $v$
denoted by $N^+(v)$.

\subsubsection{Degree in Digraphs}

We can consider in-degree and out-degree: \begin{align*}
    \text{in-degree of } v  &= \Deg^-(v) = |N^-(v)| \\ 
    \text{out-degree of } v &= \Deg^+(v) = |N^+(v)|. 
\end{align*}

\subsection{The Directed Handshake Lemma}

For a digraph $G = (V, E)$, we have that: \begin{gather*}
  \sum_{v \in V} \Deg^-(v) = \sum_{v \in V} \Deg^+(v) = |E|.
\end{gather*}

\subsection{Connectivity in Digraphs}

\subsubsection{Strong Connectivity}

A digraph $G = (V, E)$ is strongly connected if for each $u, v$
in $V$, there exists a path from $u$ to $v$ and from $v$ to $u$.

\subsubsection{Weak Connectivity}

A digraph $G = (V, E)$ is weakly connected if for each $u, v$
in $V$, there exists a path from $u$ to $v$ or from $v$ to $u$.

\subsubsection{Connected Components in Digraphs}

A strong component of a digraph is the maximal strongly
connected induced subgraph.
\\[\baselineskip]
A weak component of a digraph is the maximal weakly
connected induced subgraph.

\subsection{Euler Circuits and Trails on Digraphs}

For a digraph $G = (V, E)$, $G$ has an Euler circuit if and only 
if $G$ is strongly connected and every vertex has equal in and out
degree. We can see from this that an Euler trail exists if and only
if $G$ is strongly connected and for some $x, y$ in $V$: \begin{itemize}
    \item Every vertex in $V \setminus \{x, y\}$ has equal in and out degree,
    \item $\Deg^-(x) = \Deg^+(x) + 1$,
    \item $\Deg^+(y) = \Deg^-(y) + 1$.
\end{itemize}
