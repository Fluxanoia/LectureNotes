\section{Output Sensitivity}

We have that some algorithms are output sensitive. This means that their
runtime depends on what the answer to the problem is.

\subsection{Line Intersections}

Suppose we are given a set of line segments (as two coordinates), we would
like to find all the coordinates of the Intersections between these line
segments.

\subsubsection{A Simple Algorithm}

We iterate through all the pairs and output the intersections:
\begin{lstlisting}
points intersections_simple(lines) {
  points ps;
  for (int i = 0; i < lines.size(); i++) {
    for (int j = i + 1; j < lines.size(); j++) {
      if (lines[i] intersects lines[j]) {
        ps.push(intersection);
      }
    }    
  }
  return ps;
}
\end{lstlisting} This algorithm takes $O(n^2)$ time.

\subsubsection{Output Sensitivity in Line Intersections}

It can be seen that certain inputs could potentially have $O(n^2)$ output
but this would make it seem like the simple algorithm is optimal but we
will see that if consider $k$ to be the number of outputs, we can find
an algorithm with $O(n\log_2(n) + k\log_2(n))$ time complexity. However,
if we consider bounds for $k$:
\begin{center}
  \begin{tabular}{ c c c }
    $k \leq 2n$  & $\Rightarrow$ & $O(n\log_2(n))$ \\
    $k \geq n^2$ & $\Rightarrow$ & $O(n^2\log_2(n))$,
  \end{tabular}
\end{center} so for certain inputs this algorithm will be \textbf{worse}
than the simple algorithm.

\newpage

\subsection{An Outline for Finding Intersections}

It can be seen that for two line segments, they can only have an intersection
if the spans of the segments in the $y$ direction intersect also. Thus,
we could consider sweeping a horizontal line through all our line segments
picking up intersections as we go.

\subsubsection{Adjacency} 
We say two line segments are adjacent if there is a
contiguous horizontal line from one segment to the other (not interrupted
by another line segment). It can be
seen that two segments that are never adjacent can't intersect.

\subsubsection{Event Points} 
We can't possibly iterate through all possible $y$
points, thus we only consider 'event points' which are the end points of 
segments and line intersections but this requires that we calculate 
intersections as we go. If we have $k$ intersections,
this gives $O(n + k)$ event points.
\\[\baselineskip]
We consider the set of event points as a priority queue with keys 
equal to their $y$ value, allowing us to \texttt{extractMin} to get
our next event point. However, our process could give rise to duplicate
event points, but these can be dealt with by checking the queue beforehand.

\subsubsection{Status} 
We consider the status of the sweep line to be the
ordered set of line segments currently being interesected by the sweep line
with respect to their $x$ coordinates. The status can clearly only change
at event points, so at each event point we query line segments that have
newly become adjacent.
\\[\baselineskip]
We consider status as a $2-3-4$ tree where: \begin{itemize}
  \item At the top of a line segment, we insert it,
  \item At the bottom of a line segment, we delete it,
  \item At an intersection, we swap the intersecting lines,
\end{itemize} checking for new intersections as these changes occur.

\subsubsection{The Full Process}

We add all the start and end points of our line segments to our priority queue, 
and we iterate through them. We update the status and query for intersections as
we progress, adding intersections to our output and the queue as necessary.
This takes $O((n + k)\log_2(n))$ time.