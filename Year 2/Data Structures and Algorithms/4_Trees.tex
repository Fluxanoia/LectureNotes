\section{Trees and Forests}

A graph is a forest if it is acyclic.
A tree is a connected forest.

\subsection{Leaves}

For a tree $T = (V, E)$, for a vertex $v$ in $V$, $v$ is a leaf
if $\Deg(v) = 1$. 

\subsubsection{Existence of Leaves}

If $|V| \geq 2$, we have that $T$ has at least two leaves.

\subsection{Characterisation of Trees}

We have that for a graph $G = (V, E)$, the following are equivalent:
\begin{itemize}
  \item $G$ is a tree,
  \item $G$ is maximally acyclic ($G$ is acyclic and the addition
  of any edge forms a cycle),
  \item $G$ is minimally connected ($G$ is connected and the removal
  of any edge disconnects it),
  \item $G$ is connected and $|E| = |V| - 1$,
  \item $G$ is acyclic and $|E| = |V| - 1$,
  \item Any two vertices in $G$ are connected by a unique path.
\end{itemize}

\subsection{Rooted Trees}

For a tree $T = (V, E)$, we can root $T$ with some $r$ in $V$.

In the rooting process, we take each $v$ in $V\setminus\{r\}$ and 
define $P_v$ to be the path from $r$ to $v$. We then direct the edges 
from $r$ to $v$ for each $P_v$.
\\[\baselineskip]
For $u, v$ in $V\setminus\{r\}$, we say that: \begin{itemize}
  \item $u$ is an ancestor of $v$ if $u$ lies on $P_v$,
  \item $u$ is the parent of $v$ if $u$ is in the
  in-neighbourhood of $v$,
  \item $L_0 = \{r\}$ and $L_n = \{v \in V : |P_v| = n\}$ are the 
  levels of $T$,
  \item The depth of a tree is the greatest $n$ where
  $L_n$ is non-empty.
\end{itemize} 