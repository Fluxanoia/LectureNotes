\section{Graphs Algorithms}

\subsection{Data Representations of Graphs}

\subsubsection{Adjacency Matrix}

We have for a graph $G = (V, E)$, the adjacency matrix is a
$|V|$ by $|V|$ matrix $A = (a_{ij})$ where: \begin{gather*}
  a_{ij} = \begin{cases}
    1 & \text{if there's an edge from vertex $i$ to $j$} \\
    0 & \text{otherwise} \\
  \end{cases}
\end{gather*}

\subsubsection{Adjacency List}

We can represent a graph also by an array of linked lists or 
hash tables where
each element in the array represents a vertex $v$ and the corresponding
list represents the vertices in the neighbourhood of $v$.

\subsubsection{Comparision of Representations}

We can compare some basic properties of the representations:
\begin{center}
  \renewcommand{\arraystretch}{1.4}
  \begin{tabular}{ | r || c | c | c |}
    \hline
    & Matrix & Linked Lists & Hash Tables \\ 
    \hline\hline
    Space & $\Theta(|V|^2)$ & $\Theta(|V| + |E|)$ & $\Theta(|V| + |E|)$ \\
    Finding an edge from $u$ & $O(1)$ & $O(\text{deg}(u))$ & $O(1)$ \\
    Finding the neighbourhood of $u$ & $O(|V|)$ & $O(\text{deg}(u))$ & $O(\text{deg}(u))$ \\
    \hline
  \end{tabular}
\end{center} This raises the question, why don't we always use
hash tables? Due to the probability of collisions in hash tables,
we opt for the linked list as it's more reliable for large graphs
(additionally, we are almost always are looking for a neighbourhood
not a specific edge).

\vfill

\subsection{Search}

Generally with a graph searching algorithm, we have a data structure 
which is (in the general case) a structure that can store a set of 
vertices and return them one by one in some undefined manner. 
Starting with a graph $G$ with some \texttt{u} in $V(G)$, naming 
our data structure \texttt{data}, we perform the following:
\begin{lstlisting}
data Search(u) {
  add u to data;
  while (data is non-empty) {
    take x from data;
    if (x is not marked) {
      mark x;
      for (each edge {x, y}) {
        put y in data;
      }
    }
  }
}
\end{lstlisting} We have that this process always terminates, visits
every vertex in connected graphs, and has time complexity $O(|E(G)|)$
(assuming the data operations are $O(1)$).

\subsubsection{Breadth-first Search}

If our data structure is a queue, we get breadth-first searching.
This causes vertices to be marked in distance order from the starting
point.
\\[\baselineskip]
By tracking distances, we can find shortest
paths from a given vertex using this searching style. In a graph $G = (V, E)$, 
this takes $O(|V| + |E|)$ time.

\subsubsection{Depth-first Search}

If our data structure is a stack, we get depth-first searching.
This causes vertices to be marked the further they are from
the starting vertex.

\newpage

\subsection{Djikstra's Algorithm}

For a non-negatively weighted, directed graph $G$, we have that Djikstra's 
algorithm returns the shortest path to all vertices from some starting 
vertex. Taking \texttt{distances} to be a relation between each $v$ in $V(G)$ 
and the length of the shortest path to $v$, the algorithm is structured as 
follows: \begin{lstlisting}
distances Djikstra(s) {
  let pq be our priority queue;
  let dist be our array of distances;
  for (each v) {
    dist[v] = infinity;
  }
  dist[s] = 0;
  for (each v) {
    pq->insert(v, dist(v));
  }
  while (pq is non-empty) {
    u = pq->extractMin();
    for (each edge (u, v)) {
      if (dist[v] > dist[u] + weight(u, v)) {
        dist[v] = dist[u] + weight(u, v);
        pq->decreaseKey(v, dist(v));
      }
    }
  }
  return dist;
}
\end{lstlisting} We have that the time complexity of the algorithm
varies across different implementations of priority queues: \begin{center}
  \begin{tabular} {| r || c |}
    \hline
    & Runtime \\
    \hline \hline
    Linked List & $O(|V|^2 + |V||E|)$ \\
    \hline
    Binary Heap & $O((|V| + |E|)\log{(|V|)})$ \\
    \hline
    Fibonacci Heap & $O(|E| + |V|\log{(|V|)})$ \\
    \hline
  \end{tabular}
\end{center}
and has $O(|V| + |E|)$ space complexity across all queues.

\newpage

\subsection{Bellman-Ford's Algorithm}

For a weighted, directed graph $G$, we have that Bellman-Ford's algorithm 
returns the fastest path to all vertices from some starting vertex. 
Taking \texttt{distances} to be a relation between each $v$ in $V(G)$ 
and the length of the shortest path to $v$, the algorithm is structured as 
follows: \begin{lstlisting}
distances BellmanFord(s) {
  let dist be our array of distances;
  for (each v) {
    dist[v] = infinity;
  }
  dist[s] = 0;
  do (|V| times) {
    for (each edge (u, v)) {
      // Relaxing (u, v)
      if (dist[v] > dist[u] + weight(u, v)) {
        dist[v] = dist[u] + weight(u, v);
      }
    }
  }
}
\end{lstlisting} This runs in $O(|V||E|)$ time.

\subsubsection{Negative Weight Cycles}

Suppose our graph has a cycle which has a negative weight. 
This must mean that we can choose an arbitrarily negative 
path in the graph by traversing the cycle multiple times.
This is why we require that there are no negative weight cycles.
\\[\baselineskip]
We can run the algorithm on graphs with negatives cycles and
simply run a final check at the end to see if we have a negative weight cycle.
If we relax each edge again and decrease a path, there must be a negative cycle
as we should already have all the shortest paths.

\newpage

\subsection{Johnson's Algorithm}

For a weighted, directed graph we have that Johnson's algorithm 
returns the fastest path between all vertex pairs. It does this 
by re-weighting the graph and performing Djikstra's
repeatedly.

\subsubsection{Re-weighting based on Vertex Potential}

For a graph $G = (V, E)$ with a weighting function $w : E \to \mathbb{Z}$, 
we define a potential function $h : V \to \mathbb{Z}$ to associate vertices with 
potentials. We define a re-weighting function $w' : E \to \mathbb{Z}$: \begin{gather*}
  w'((u, v)) = w((u, v)) + h(u) - h(v).
\end{gather*} We find the vertex potentials by adding a vertex $s$ to the graph with
an edge $(s, v)$ for each $v$ in $V$ of weight zero forming a new graph $G'$. 
We then run Bellman-Ford on $G'$ and define $h$ as follows: \begin{gather*}
  h(v) = \text{ the shortest path length from $s$ to $v$ in } G'.
\end{gather*} Note that: \begin{itemize}
  \item Any path is a shortest path in $G$ if and only if it's a shortest path
  in the re-weighted $G$,
  \item The length of the shortest path in $G$ can be calculated from its
  length in the re-weighted $G$,
  \item A cycle has negative weight in $G$ if and only if it has negative
  weight in the re-weighted $G$,
  \item If there are no negative cycles in $G$, the re-weighted graph
  has no negative weights.
\end{itemize}

\subsubsection{The Algorithm}

Starting with a graph $G = (V, E)$ with a weighting function $w : E \to \mathbb{Z}$,
form $G' = (V \cup \{s\}, E \cup S)$ where: \begin{gather*}
  S = \{(s, v) : v \in V\} \qquad \text{and} \qquad w(e) = 0 \text{ for all } e \text{ in } S.
\end{gather*} Run Bellman-Ford on $G'$ starting at $s$ (detecting any negative weight cycles)
to define our vertex potentials. Using the potentials, re-weight each edge as above
in $G$. Run Djikstra's on every vertex in $G$ to create our set of paired shortest paths.
We can then convert our path weights back into their weights as inputted and retrieve the
values if necessary. This takes $O(|V||E|\log_2(|V|))$ time.

\subsection{Minimum Spanning Trees}

In a graph $G = (V, E)$, a spanning tree $T = (V, E_T)$ is a tree
with $E_T \subseteq E$.
\\[\baselineskip]
A spanning tree on $G$ is minimal if there is no other spanning tree
on $G$ with a lower weight.

\subsubsection{Kruskal's Algorithm}

For a weighted, connected, and undirected graph $G = (V, E)$, we have 
the following steps to the algorithm: \begin{enumerate}
  \item Generate a graph $T = (V, \emptyset)$
  \item Generate a disjoint set data structure $X$ of size $|V|$
  \item For each $v$ in $V$, perform \texttt{makeSet(v)} (where
  each vertex is defined by some unique integer in $\{1, \ldots, |V|\}$)
  \item Sort the edges by weight
  \item For each edge $(u, v)$ (in increasing order): \begin{itemize}
    \item If \texttt{findSet}$(u) \neq$ \texttt{findSet}$(v)$,
    perform \texttt{union}$(u, v)$ and add $(u, v)$ to $T$.
  \end{itemize}
\end{enumerate} Overall, this runs in $O(|E| \log_2(|V|))$ time.