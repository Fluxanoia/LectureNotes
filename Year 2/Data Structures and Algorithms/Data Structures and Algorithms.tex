\documentclass[a4paper, 12pt, twoside]{article}
\usepackage[left = 3cm, right = 3cm]{geometry}
\usepackage[english]{babel}
\usepackage[utf8]{inputenc}
\usepackage{mathtools}
\usepackage{amssymb}
\usepackage{amsmath}
\usepackage{multicol}
\usepackage{pgfplots}
\usepackage{pgfplotstable}
\usepackage{listings}
\usepackage{xcolor}
\usepackage{color}

\pgfplotsset{compat=1.5.1}

\lstset{frame=none,
  language=C++,
  aboveskip=3mm,
  belowskip=3mm,
  showstringspaces=false,
  columns=flexible,
  basicstyle={\small\ttfamily},
  numbers=none,
  numberstyle=\tiny\color{gray},
  keywordstyle=\color{blue},
  commentstyle=\color{gray},
  stringstyle=\color{orange},
  breaklines=true,
  breakatwhitespace=true,
  tabsize=2
}

\begin{document}

\title{Data Structures and Algorithms Notes}
\date{}
\author{\textit{paraphrased by} Tyler Wright}
\maketitle

\vfill

\textit{An important note, these notes are absolutely \textbf{NOT}
  guaranteed to be correct, representative of the course, or rigorous.
  Any result of this is not the author's fault.}

\newpage

\section{Data Structures}

\subsection{Stacks}

A stack is a list of variables. It supports three operations:

\begin{center}
  \begin{tabular}{ || c | p{6.5cm} | c || }
    \hline
    Name & Description & Worst case runtime \\
    \hline
    \texttt{create()} & Creates a new stack & $O(1)$ \\
    \hline
    \texttt{push(x)} & Adds \texttt{x} to the end of the stack & $O(1)$ \\
    \hline
    \texttt{pop()} & Removes and returns the last element of the stack & $O(1)$ \\
    \hline
  \end{tabular}
\end{center}

\subsection{Queues}

A queue is a list of variables. It supports three operations:

\begin{center}
  \begin{tabular}{ || c | p{6.5cm} | c || }
    \hline
    Name & Description & Worst case runtime \\
    \hline
    \texttt{create()} & Creates a new queue & $O(1)$ \\
    \hline
    \texttt{add(x)} & Adds \texttt{x} to the end of the queue & $O(1)$ \\
    \hline
    \texttt{serve()} & Removes and returns the first element of the queue & $O(1)$ \\
    \hline
  \end{tabular}
\end{center}

\subsection{Linked List}

A linked list is a list of variables represented by nodes which
point to the next and previous element in the list (null if one does
not exist). 
It supports four operations:

\begin{center}
  \begin{tabular}{ || c | p{6.5cm} | c || }
    \hline
    Name & Description & Worst case runtime \\
    \hline
    \texttt{create()} & Creates a new linked list & $O(1)$ \\
    \hline
    \texttt{insert(x, i)} & Inserts \texttt{x} after node \texttt{i} & $O(1)$ \\
    \hline
    \texttt{delete(i)} & Removes node \texttt{i} & $O(1)$ \\
    \hline
    \texttt{lookup(i)} & Returns node \texttt{i} & $O(1)$ \\
    \hline
  \end{tabular}
\end{center}

\subsection{Arrays}

An array is a list of variables of fixed length. 
It supports three operations:

\begin{center}
  \begin{tabular}{ || c | p{6.5cm} | c || }
    \hline
    Name & Description & Worst case runtime \\
    \hline
    \texttt{create(n)} & Creates a new array of size \texttt{n} & $O(1)$ \\
    \hline
    \texttt{update(x, i)} & Overwrites the data at position \texttt{i} with \texttt{x} & $O(1)$ \\
    \hline
    \texttt{lookup(i)} & Returns the value at \texttt{i} & $O(1)$ \\
    \hline
  \end{tabular}
\end{center}

\subsection{Hash Tables}

A hash table is an array of linked lists storing key-value pairs. 
We use a \textbf{hash function} to map data to a linked list. As we 
are using linked lists, if multiple keys map to the same index, 
we can just add them to the list - and when looking up data, we 
can find the right list with the hash function and then match our key.
\\[\baselineskip]
It supports four operations:

\begin{center}
  \begin{tabular}{ || c | p{7.5cm} | c || }
    \hline
    Name & Description & Average runtime \\
    \hline
    \texttt{create(n)} & Creates a \texttt{n} sized array
    of linked lists and chooses a hash function \texttt{h} & $O(1)$ \\
    \hline
    \texttt{insert(k, v)} & Inserts the pair (\texttt{k, v}),
    if $\frac{\texttt{n}}{2}$ pairs are stored, we create a hash
    table of double the size and copy the contents into it & $O(1)$ \\
    \hline
    \texttt{delete(k)} & Deletes the pair corresponding to the key \texttt{k} & $O(1)$ \\
    \hline
    \texttt{lookup(k)} & Returns the pair corresponding to the key \texttt{k} & $O(1)$ \\
    \hline
  \end{tabular}
\end{center}

\subsubsection{Markov's Inequality}

For $X \geq 0$ a random variable with mean $\mu$, for all
$t$ in $\mathbb{R}_{\geq 0}$: \begin{gather*}
  \mathbb{P}(X \geq t) \leq \frac{\mu}{t}.
\end{gather*} So, if $X$ is the expected time it takes for an 
algorithm to terminate, we can say how likely it is for an algorithm
to terminate based on our prediction.

\subsection{Binary Heaps}

Binary heaps are rooted binary trees where each level is full except
possibly the last (which is filled from left to right). The
elements of the tree are ordered according to a 
\textbf{heap property}. These have the following properties: \begin{itemize}
  \item For a heap of size $n$, the height of the heap is $\log_2(n)$
  \item For an index $i$: \begin{itemize}
    \item The parent has index $\left\lfloor \dfrac{i}{2} \right\rfloor$
    \item The left child has index $2i$
    \item The right child has index $2i + 1$
  \end{itemize}
\end{itemize}

\newpage

\subsection{Priority Queues}

A priority queue is a set of distinct elements with associated value
called the key. We can use a binary heap as a priority queue with the
elements as the keys and the heap property that the parents are
less than or equal to the children. This supports the following:
\begin{center}
  \begin{tabular}{ || c | p{7.5cm} | c || }
    \hline
    Name & Description & Runtime \\
    \hline
    \texttt{insert(x, k)} & Inserts \texttt{x} with key \texttt{k}
    & $O(\log_2(n))$ \\
    \hline
    \texttt{decreaseKey(x, d)} & Decreases the key of \texttt{x} to \texttt{d}
    & $O(\log_2(n))$ \\
    \hline
    \texttt{extractMin()} & Removes and returns the \texttt{x} in the queue
    with the smallest key & $O(\log_2(n))$ \\
    \hline
  \end{tabular}
\end{center}

\subsection{Disjoint Set}

Stores a collection of disjoint sets where each set has elements
$1, 2, \ldots n$ for some natural $n$. This supports the following:
\begin{center}
  \begin{tabular}{ || c | p{7.5cm} | c || }
    \hline
    Name & Description & Runtime \\
    \hline
    \texttt{makeSet(x)} & Creates a new set containing only \texttt{x}
    this fails if \texttt{x} is already in a set
    & $O(1)$ \\
    \hline
    \texttt{union(x, y)} & Merges the sets containing 
    \texttt{x} and \texttt{y}
    & $O(\log_2(n))$ \\
    \hline
    \texttt{findSet(x)} & Finds the identifier of the set containing
    \texttt{x} (the identifier is an element of the set)
    & $O(\log_2(n))$ \\
    \hline
  \end{tabular}
\end{center} This is stored as an array size $n$ where each cell
is empty or points to the identifier of the set it was originally
added to.
\\[\baselineskip]
\textit{So, adding $3$ to $\{7\}$ would make $3$ always point to 
$7$. But creating a set with $3$, will make it point to itself.}

\subsection{Dynamic Search Structures}

This structure stores a set of elements, each with a unique key.
This supports the following:
\begin{center}
  \begin{tabular}{ || c | p{7.5cm} | c || }
    \hline
    Name & Description & Runtime \\
    \hline
    \texttt{insert(x, k)} & Inserts \texttt{x}
    with key \texttt{k}
    & $O(\log_2(n))$ \\
    \hline
    \texttt{find(k)} & Returns the element with 
    unique key \texttt{k}
    & $O(\log_2(n))$ \\
    \hline
    \texttt{delete(k)} & Deletes the element with unique key
    \texttt{k}
    & $O(\log_2(n))$ \\
    \hline
    \texttt{predecessor(k)} & Returns the element with unique key
    \texttt{n} such that \texttt{n} $<$ \texttt{k}
    & $O(\log_2(n))$ \\
    \hline
    \texttt{rangeFind(a, b)} & Returns the elements with unique key
    \texttt{k} such that \texttt{a} $\leq$ \texttt{k} $\leq$ \texttt{b}
    & $O(\log_2(n))$ \\
    \hline
  \end{tabular}
\end{center}

\vfill

\section{Graph Theory}

\subsection{Definition of a Graph}

A graph is a pair of sets $G = (V, E)$, where $V$ is a set of 
vertices (or nodes) and $E$ is a set of edges (or arcs).

\subsection{Definition of an Edge}

An edge of a graph $G = (V, E)$ is $e = \{u, v\}$ in $E$ where $u$,
$v$ are vertices in $V$.

\subsection{Definition of a Neighbourhood}

For a graph $G = (V, E)$ with $v$ in $V$, the neighbourhood
of $v$ is the set $V' \subseteq V$ of vertices connected to
$v$ by an edge in $E$.
\\[\baselineskip]
The neighbourhood of $v$ is denoted by $N(v)$.
\\[\baselineskip]
The neighbourhood of a set of vertices is the union of
the neighbourhoods of each vertex.

\subsection{Definition of Degree}

For a graph $G = (V, E)$ with $v$ in $V$, the degree of $v$
is the size of its neighbourhood.
\\[\baselineskip]
The degree of $v$ is denoted by $d(v)$.

\subsection{The Handshake Lemma}

For a graph $G = (V, E)$, we have that: \begin{gather*}
  |E| = \frac{\sum_{v \in V} d(v)}{2}.
\end{gather*} \textit{This is because each edge visits two vertices,
so by counting the degree of each vertex we count each edge exactly
twice.}

\subsection{$k$-regular Graphs}

For a graph $G = (V, E)$, we have that $G$ is $k$-regular for some
$k$ in $\mathbb{Z}_{>0}$ if for all $v$ in $V$, we have: \begin{gather*}
  d(v) = k.
\end{gather*} We cannot have a $k$-regular graph where $k$ is odd
and $|V|$ is odd by the Handshake Lemma.

\subsection{Isomorphic Graphs}

Graphs $G_1 = (V_1, E_1)$ and $G_2 = (V_2, E_2)$ are called 
isomorphic if there exists a bijection $f : V_1 \to V_2$ such
that: \begin{gather*}
  \{u, v\} \in E_1 \Longleftrightarrow \{f(u), f(v)\} \in E_2.
\end{gather*} This relationship is denoted by $G_1 \cong G_2$.

\subsection{Definition of a Subgraph}

A graph $G' = (V', E')$ is a subgraph of $G = (V, E)$ if
$V' \subseteq V$ and $E' \subseteq E$.

\subsection{Definition of an Induced Subgraph}

An induced subgraph generated from $G = (V, E)$ by $V' \subseteq V$
is the graph $G' = (V', E')$ where: \begin{gather*} 
  E' = \{\{u, v\} \in E \text{ such that } u, v \in V'\}.
\end{gather*} \textit{Essentially, you generate an induced
subgraph from a subset of the vertices of a graph by selecting
edges that join vertices in the subset.}

\subsection{Walks}

\subsubsection{Definition of a walk}

A walk in a graph $G = (V, E)$ is a set of vertices in $V$ connected
by edges in $E$. The length of the walk is the number of edges
traversed in the walk.

\subsubsection{Definition of a path}

A path is a walk where no vertices are repeated.

\subsubsection{Definition of an Euler walk}

An Euler walk is a walk such that every edge is traversed exactly
once. Thus, for a graph $G = (V, E)$, the length is $|E|$.

\subsubsection{Conditions for an Euler walk}

For an Euler walk to be possible on a given graph,
all vertices must have an even degree \textbf{or} exactly
two vertices have odd degree. 
\\[\baselineskip]
If all vertices have even degree we 
have that the Euler walk is a cycle, if exactly two vertices have
odd degree then we have that these vertices are the start and end
points of our Euler walk. 

\subsection{Definition of a Connected Graph}

A connected graph is a graph where for each pair of vertices,
there is a path connecting them.

\subsection{Definition of a Component}

A component of a graph $G$ is a maximal connected 
induced subgraph of $G$. This means an induced subgraph of $G$
that is connected but is not longer connected if a vertex is
removed.

\subsection{Digraphs}

\subsubsection{Definition of a digraph}

A digraph (or directed graph) is a graph where each of the edges
has a direction. This direction means the edge can only be traversed
in a single direction.

\subsubsection{The Directed Handshake Lemma}

For a digraph $G = (V, E)$, we have that: \begin{gather*}
  \sum_{v \in V} d^-(v) = \sum_{v \in V} d^+(v) = |E|.
\end{gather*} \textit{This is because if we consider the 'tail' of an
edge (the vertex it leaves), each edge has exactly one tail.}

\subsubsection{Definition of a strongly connected digraph}

A digraph $G = (V, E)$ is strongly connected if for each $u, v$
in $E$, there exists a path from $u$ to $v$ \textbf{and} 
from $v$ to $u$.

\subsubsection{Definition of a weakly connected digraph}

A digraph $G = (V, E)$ is weakly connected if for each $u, v$
in $E$, there exists a path from $u$ to $v$ \textbf{or} 
from $v$ to $u$.

\subsubsection{Definition of components of digraphs}

A strong component of a digraph is the maximal \textit{strongly}
connected induced subgraph.
\\[\baselineskip]
A weak component of a digraph is the maximal \textit{weakly}
connected induced subgraph.
\\[\baselineskip]
\textit{So, these are induced subgraphs that are strongly/weakly
connected but are no longer strongly/weakly connected once a
vertex is removed.}

\subsubsection{Definition of neighbourhoods in digraphs}

The neighbourhood of a vertex in a digraph can be considered by
looking at the edges \textit{from} the vertex and the edges
\textit{to} the vertex.
\\[\baselineskip]
The in-neighbourhood of a vertex $v$ are the edges that enter $v$.
The out-neighbourhood of a vertex $v$ are the edges that exit $v$.
These are denoted by $N^-(v)$ and $N^+(v)$ respectively.

\subsubsection{Definition of degrees in digraphs}

For a vertex $v$, the in-degree of the vertex $d^-(v)$ is the size of
the in-neighbourhood and the out-degree of the vertex $d^+(v)$
is the size of the out-neighbourhood.
\\[\baselineskip]
It can be seen that the degree of a given vertex is the sum of
its in and out degree (in a digraph).

\subsubsection{Conditions for an Euler walk in a digraph}

For an Euler walk to be possible on a given digraph, we have
two cases, either:\begin{itemize}
  \item the digraph is strongly connected and every vertex
  has equal in and out degrees, or
  \item one vertex has an in-degree one greater than its out-degree,
  another has an out-degree one greater than its in-degree, and all
  remaining vertices have equal in and out degrees.
\end{itemize}
In the first case we have that the Euler walk is a cycle, 
in the second we have that the special vertices are the start and end
points of our Euler walk. 

\subsubsection{Cycles}

\subsubsection{Definition of a cycle}

A cycle is a walk where the first and last vertices are the same
and each vertex appears at most once (barring the first and last
vertex).

\subsubsection{Definition of a Hamiltonian cycle}

A Hamiltonian cycle is a cycle where each vertex is visited.

\subsubsection{Conditions for a Hamiltonian cycle}

Whilst the conditions necessary for a Hamiltonian cycle in general
are unknown, by Dirac's theorem, we know that for a graph with
$n$ vertices, if every vertex has degree $\frac{n}{2}$ or greater
then a Hamiltonian cycle exists.

\subsection{Trees}

\subsubsection{Definition of a forest}

A forest is a graph with no cycles.

\subsubsection{Definition of a tree}

A tree is a connected forest (or a connected graph with no cycles).

\subsubsection{Path uniqueness of trees}

For a tree $T = (V, E)$, we have that for any $u, v$ in $V$, there
exists a unique path from $u$ to $v$.
\\[\baselineskip]
\textit{To prove this, suppose there are two unique paths between
$u$ and $v$. These paths must diverge and if we connect them, they
form a cycle which contradicts the definition of a tree.}

\subsubsection{The magnitude of edges in trees}

For a tree $T = (V, E)$, we have that $|E| = |V| - 1$.

\subsubsection{Rooted trees}

For a tree $T = (V, E)$, we can root $T$ with some $r$ in $V$.
For $v$ in $V\setminus{r}$, we define $P_v$ to be the path from 
$r$ to $v$, we then direct the edges from $r$ to $v$ for each $P_v$.
\\[\baselineskip]
For $u, v$ in $V\setminus\{r\}$, we say that: \begin{itemize}
  \item $u$ is an \textbf{ancestor} of $v$ if $u$ lies on $P_v$
  \item $u$ is the \textbf{parent} of $v$ if $u$ is in the
  in-neighbourhood of $v$
  \item $v$ is a \textbf{leaf} if it has degree $1$
  \item $L_0 = \{r\}$ and $L_n = \{v : |P_v| = n\}$ are the 
  \textbf{levels} of $T$
  \item The \textbf{depth} of a tree is the greatest $n$ where
  $L_n$ is non-empty.
\end{itemize} 

\subsubsection{Lower bound on the amount of leaves in a tree}

For a tree with $T = (V, E)$, if $V > 1$, there must be at least
$2$ leaves.

\subsubsection{Equivalent statements to the tree definition}

For a graph $T = (V, E)$, we have that the following are
equivalent: \begin{itemize}
  \item $T$ is a tree
  \item $T$ is connected and has no cycles
  \item $|E| = n - 1$ and T is connected
  \item $|E| = n - 1$ and T has no cycles
  \item T has a unique path between any two vertices
\end{itemize}

\subsection{Bipartitions}

\subsubsection{Definition of a bipartite graph}

For $G = (V, E)$, we have that $G$ is bipartite if there exists
$A \subset V$, $B \subset V$ such that $A$ and $B$ are disjoint
and the induced subgraphs of $A$ and $B$ have no edges. $A$
and $B$ are bipartitions of $G$.
\\[\baselineskip]
Saying $G$ is bipartite is equivalent to saying $G$ has no
cycles of odd length.
 
\subsubsection{Definition of a matching}

A matching in a graph is a set of disjoint edges.
\\[\baselineskip]
A matching is \textbf{perfect} if each vertex is contained in
some matching edge.

\subsubsection{Definition of a semi-matching}

For $k$ in $\mathbb{Z}_{>0}$, a $k$ to 1 semi-matching in a
bipartite graph $G$ with a bipartition $\{A, B\}$ is a subgraph
of $G$ where each vertex in $A$ has degree at most $k$ and
each vertex in $B$ has degree at most $1$.

\subsubsection{Definition of an augmenting path}

Given a matching $M$ in a bipartite graph $G = (V, E)$, 
an augmenting path is a set of vertices in $V$ connected
by edges $e_i$ in $E$ such that: \begin{align*}
  e_i \text{ is } \begin{cases}
    \text{in } M & \text{for } i \text{ odd} \\
    \text{not in } M & \text{for } i \text{ even}. \\
  \end{cases}
\end{align*} With the condition that the first and last vertices
in the path are not in the matching.

\subsubsection{Hall's Theorem}

For a bipartite graph $G = (V, E)$ with the bipartition $(A, B)$ has
a perfect matching if and only if $|A| = |B|$ and for all 
$X \subseteq A$, $|N(X)| \geq |X|$. 

\vfill

\section{Working on Graphs}

\subsection{Data Representations of Graphs}

\subsubsection{Adjacency matrix}

We have for a graph $G = (V, E)$, the adjacency matrix is a
$|V|$ by $|V|$ matrix $A = (a_{ij})$ where: \begin{gather*}
  a_{ij} = \begin{cases}
    1 & \text{if there's an edge from vertex $i$ to $j$} \\
    0 & \text{otherwise} \\
  \end{cases}
\end{gather*}

\subsubsection{Adjacency list}

We can represent a graph also by an array of linked lists or 
hash tables where
each element in the array represents a vertex and the corresponding
list represents the vertices in the neighbourhood of the vertex.

\newpage

\subsubsection{Comparision of representations}

We can compare some basic properties of the representations:
\begin{center}
  \renewcommand{\arraystretch}{1.4}
  \begin{tabular}{ | r || c | c | c |}
    \hline
    & Matrix & Linked Lists & Hash Tables \\ 
    \hline\hline
    Space & $\Theta(|V|^2)$ & $\Theta(|V| + |E|)$ & $\Theta(|V| + |E|)$ \\
    Finding an edge from $u$ & $O(1)$ & $O(\text{deg}(u))$ & $O(1)$ \\
    Finding the neighbourhood of $u$ & $O(|V|)$ & $O(\text{deg}(u))$ & $O(\text{deg}(u))$ \\
    \hline
  \end{tabular}
\end{center} This raises the question, why don't we always use
hash tables? Due to the probability of collisions in hash tables,
we opt for the linked list as it's more reliable for large graphs
(additionally, we are almost always are looking for a neighbourhood
not a specific edge).

\subsection{Search}

Generally with a graph searching algorithm, we have a data structure 
which is left undefined here (besides the fact we can add vertices to 
it and take vertices out). Starting with a vertex \texttt{u}, naming 
our data structure \texttt{data}, we perform the following:
\begin{lstlisting}
data Search(u) {
  add u to data;
  while (data is non-empty) {
    take x from data;
    if (x is not marked) {
      mark x;
      for (each edge (x, y)) {
        put y in data;
      }
    }
  }
}
\end{lstlisting} We have that this process always terminates, visits
every vertex in connected graphs, and has time complexity $O(|E|)$
(assuming the data operations are $O(1)$) where $E$ is the edge set.

\subsubsection{Breadth-first search}

If our data structure is a queue, we get breadth-first searching.
This causes vertices to be marked in distance order from the starting
point.
\paragraph{Shortest paths} By tracking distances, we can find shortest
paths using this searching style.
This is $O(|V| + |E|)$ in a graph $G = (V, E)$.

\subsubsection{Depth-first search}

If our data structure is a stack, we get depth-first searching.
This causes vertices to be marked the further they are from
the starting vertex.

\subsection{Djikstra's Algorithm}

For a weighted (non-negatively), directed graph we have that Djikstra's algorithm 
returns the fastest path to all vertices from some starting vertex. 
It is structured as follows: \begin{lstlisting}
distances Djikstra(s) {
  let pq be our priority queue;
  let dist be our array of distances;
  for (each v) {
    dist[v] = infinity;
  }
  dist[s] = 0;
  for (each v) {
    pq->insert(v, dist(v));
  }
  while (pq is non-empty) {
    u = pq->extractMin();
    for (each edge (u, v)) {
      if (dist[v] > dist[u] + weight(u, v)) {
        dist[v] = dist[u] + weight(u, v);
        pq->decreaseKey(v, dist(v));
      }
    }
  }
  return dist;
}
\end{lstlisting} We have that the time complexity of the algorithm
varies across queues: \begin{center}
  \begin{tabular} {| r || c |}
    \hline
    & Runtime \\
    \hline \hline
    Linked List & $O(|V|^2 + |V||E|)$ \\
    \hline
    Binary Heap & $O((|V| + |E|)\log{(|V|)})$ \\
    \hline
    Fibonacci Heap & $O(|E| + |V|\log{(|V|)})$ \\
    \hline
  \end{tabular}
\end{center}
and has $O(|V| + |E|)$ space complexity across all queues.

\subsection{Bellman-Ford's Algorithm}
For a weighted, directed graph we have that Bellman-Ford's algorithm 
returns the fastest path to all vertices from some starting vertex. 
It is structured as follows: \begin{lstlisting}
distances BellmanFord(s) {
  let dist be our array of distances;
  for (each v) {
    dist[v] = infinity;
  }
  dist[s] = 0;
  do (|V| times) {
    for (each edge (u, v)) {
      // Relaxing (u, v)
      if (dist[v] > dist[u] + weight(u, v)) {
        dist[v] = dist[u] + weight(u, v);
      }
    }
  }
}
\end{lstlisting} This runs in $O(|V||E|)$ time.

\subsubsection{Negative weight cycles}

Suppose our graph has a cycle which has a negative weight. 
This must mean that we can choose an arbitrarily small/negative 
path in the graph by traversing the cycle multiple times.
This is why we require that there are no negative weight cycles.
\\[\baselineskip]
We can run the algorithm on graphs with negatives cycles and
simply run a final check at the end to see if we have a negative weight cycle.
If we relax each edge again and decrease a path, there must be a negative cycle
as we should already have all the shortest paths.

\newpage

\subsection{Johnson's Algorithm}

For a weighted, directed graph we have that Johnson's algorithm 
returns the fastest path between all vertex pairs. It does this 
by \textbf{re-weighting} the graph and performing Djikstra's
repeatedly.

\subsubsection{Reweighting based on vertex potential}

For a graph $G = (V, E)$ with a weighting function $w : E \to \mathbb{Z}$, 
we define a potential function $h : V \to \mathbb{Z}$ to associate vertices with 
potentials. We define a re-weighting function $w' : E \to \mathbb{Z}$: \begin{gather*}
  w'((u, v)) = w((u, v)) + h(u) - h(v).
\end{gather*} We find the vertex potentials by adding a vertex $s$ to the graph with
an edge $(s, v)$ for each $v$ in $V$ of weight zero forming a new graph $G'$. 
We then run Bellman-Ford on $G'$ and define $h$ as follows: \begin{gather*}
  h(v) = \text{ the shortest path length from $s$ to $v$ in } G'.
\end{gather*} Note that $G'$ has the same number of negative weight cycles as $G$
as all our edges are directed away from $s$.

\subsubsection{The algorithm}

Starting with a graph $G = (V, E)$ with a weighting function $w : E \to \mathbb{Z}$,
form $G' = (V \cup \{s\}, E \cup S)$ where: \begin{gather*}
  S = \{(s, v) : v \in V\} \qquad \text{and} \qquad w(e) = 0 \text{ for all } e \text{ in } S.
\end{gather*} Run Bellman-Ford on $G'$ starting at $s$ (detecting any negative weight cycles)
to define our vertex potentials. Using the potentials, re-weight each edge as above
in $G$. Run Djikstra's on every vertex in $G$ to create our set of paired shortest paths.
We can then convert our path weights back into their weights as inputted and retrieve the
values if necessary. This takes $O(|V||E|\log_2(|V|))$ time.

\subsection{Minimum Spanning Trees}

\subsubsection{Definition of a spanning tree}

In a connected, undirected graph $G = (V, E)$, we have that a
spanning tree $T = (V', E')$ of $G$ is a subgraph of $G$ where
$T$ is a tree and $V = V'$.
\\[\baselineskip]
A spanning tree on $G$ is minimal if there is no other spanning tree
on $G$ with a lower weight.

\subsubsection{Kruskal's Algorithm}

For a weighted, connected, and undirected graph $G = (V, E)$, we have 
the following steps to the algorithm: \begin{enumerate}
  \item Generate a graph $T = (V, \emptyset)$
  \item Generate a disjoint set data structure $X$ of size $|V|$
  \item For each $v$ in $V$, perform \texttt{makeSet(v)} (where
  each vertex is defined by some unique integer in $\{1, \ldots, |V|\}$)
  \item Sort the edges by weight
  \item For each edge $(u, v)$ (in increasing order): \begin{itemize}
    \item If \texttt{findSet}$(u) \neq$ \texttt{findSet}$(v)$,
    perform \texttt{union}$(u, v)$ and add $(u, v)$ to $T$
  \end{itemize}
\end{enumerate} Overall, this runs in $O(|E| \log_2(|V|))$ time.

\vfill

\section{Fast Fourier Transforms}

\subsection{Polynomials}

\subsubsection{Definition of a Polynomial}

A polynomial of degree $n$ in $\mathbb{Z}_{\geq 0}$ is a function $A$: 
\begin{gather*}
  A(x) = \sum_{i = 0}^n a_ix^i,
\end{gather*} where $a_i$ are the coefficients of $A$. We say for 
$k > n$, $k$ is a degree-bound of $A$. We can represent this by listing
the coefficients, called the \textbf{coefficient representation}.

\subsubsection{Fast Polynomial Evaluation}

We can evaluate polynomials quickly using \textit{Horner's Rule},
for a polynomial $A$ degree $n$: \begin{gather*}
  A(x) = a_0 + x(a_1 + x(a_2 + \cdots + x(a_n)))).
\end{gather*} This can be simplified in the following code:
\begin{lstlisting}
int polynomial(coeffs, x) {
  output = 0;
  for (i = n; i >= 0; i--) {
    output = (output * x) + coeffs[i];
  }
  return output;
}
\end{lstlisting} We have that this is $O(n)$.

\subsubsection{Point Intersection with Polynomials}

For a given set of points of size $n$, we have that there exists a 
unique polynomial with degree-bound $n$ such that the polynomial
intersects all the given points.

\subsubsection{Point-Value Representation}

We can represent a polynomial by a set of points it intersects like so:
\begin{gather*}
  \{(x_0, y_0), \ldots, (x_n, y_n)\},
\end{gather*} for a polynomial degree $n + 1$.

\subsubsection{Polynomial Addition}

For two polynomials $A, B$ with coefficients $a_i, b_i$ and degrees $n, m$ 
respectively, we have that: \begin{gather*}
  (A + B)(x) =  \sum_{i = 0}^{\text{max}(n, m)}(a_i + b_i)x^i.
\end{gather*} If $m > n$ or vice versa, we pad out the shorter
polynomial with zeroes. We can do this with the point-value representation
by adding the '$y$-values'. We have that addition as it's defined here
is $O(n)$.

\subsubsection{Polynomial Multiplication}

For two polynomials $A, B$ with coefficients $a_i, b_i$ and degrees $n, m$ 
respectively, we have that: \begin{gather*}
  C(x) = (A \cdot B)(x) = \sum_{i = 0}^{k} c_ix_i,
\end{gather*} where $k = 2 \cdot \text{max}(n, m)$ and: \begin{gather*}
  c_i = \sum_{j = 0}^ia_jb_{j - 1}.
\end{gather*} We can do this with the point-value representation,
for: \begin{gather*}
  A = \{(x_0, y_0), \ldots, (x_n, y_n)\}, \\
  B = \{(x_{n + 1}, z_0), \ldots, (x_{n + m}, z_m)\},
\end{gather*} We have that: \begin{gather*}
  C = A \cdot B = \{(x_0, y_0 \cdot z_0), \ldots, (x_k, y_k \cdot z_k)\}
\end{gather*} This is much easier, yielding
an $O(n)$ algorithm rather than an $O(n^2)$ algorithm.

\subsection{Fast Fourier Transform}

\subsubsection{Roots of Unity}

The idea is that we evaluate a polynomial to perform pointwise
multiplication and then interpolate back into a polynomial.
We need to evaluate a polynomial of degree $n$ at $n + 1$ points to
convert it to point-value form. We use the $n + 1$ roots of unity:
\begin{gather*}
  \omega_{n+1}^k = e^{\frac{2\pi i}{n + 1}k},
\end{gather*} for $k$ in $\{0, 1, \ldots n\}$. Therefore considering:
\begin{gather*}
  y_k = A(\omega_{n + 1}^k),
\end{gather*} for $A$ a polynomial, $k$ as above, and the vector of 
all ordered $y_k$ being the \textbf{Discrete Fourier Transform (DFT)} 
of the coefficient vector of $A$.

\paragraph{Cancellation Lemma:} we have that 
$\omega_{dn}^{dk} = \omega_{n}^{k}$.

\paragraph{Halving Lemma:} we have that if $n$ is even, the set of all
the squared roots of unity is just the set of roots of unity for $\frac{n}{2}$.
\\[\baselineskip]
\textit{This is true due to the Cancellation Lemma, we have:} \begin{gather*}
  (\omega_{2k}^j)^2 = \omega_{2k}^{2j} = \omega_{k}^j.
\end{gather*}

\subsubsection{Method of the Fast Fourier Transform}

For a polynomial $A$ degree $n$, 
we define $A^{[0]}$ and $A^{[1]}$ as: \begin{align*}
  A^{[0]} &= a_0 + a_2x + \cdots + a_{n - 2}x^{(n / 2) - 1} \\
  A^{[1]} &= a_1 + a_3x + \cdots + a_{n - 1}x^{(n / 2) - 1},
\end{align*} so we have that: \begin{gather*}
  A(x) = A^{[0]}(x^2) + xA^{[1]}(x^2).
\end{gather*} So, we can split a DFT computation into two
equally sized parts, compute them, and then combine them in linear time.

\subsection{Polynomial Multiplication}

So, the steps are laid out, for polynomials $A, B$ with degree bound 
$n$, as follows: \begin{itemize}
  \item Set the degree of $A$ and $B$ to $2n$, padding with zeroes
  \item Perform the fast Fourier transform
  \item Form our point-value representation and multiply pointwise
  \item Interpolate with the inverse fast Fourier transform.
\end{itemize} This process is $O(n \log(n))$.

\vfill

\section{Dynamic Programming}

Dynamic programming is the process of solving programming problems
by breaking them down into \textbf{overlapping} subproblems,
computing the base cases and storing the solutions to be later
composed into a solution.

\newpage

\subsection{Largest Empty Square}

This problem is about finding the largest square in a $n \times n$
black and white image such that the square does not contain a black
pixel.

\subsubsection{A recursive algorithm}

To find the largest square at the position $(x, y)$ 
(bottom-right corner at $(x, y)$), we use the
following algorithm: \begin{lstlisting}
size LargestSquare(x, y) {
  if ((x, y) is black) return 0;
  if ((x == 1) or (y == 1)) return 1;
  return min(
    LargestSquare(x - 1, y - 1),
    LargestSquare(x - 1, y),
    LargestSquare(x, y - 1));
}
\end{lstlisting} The time complexity of this algorithm, however,
is exponential.
\\[\baselineskip]
We get this as each cell barring the first and last columns and rows 
have cells where \texttt{LargestSquare} is computed three times 
(as they  are checked from below, to the right, and below and to 
the right).

\subsubsection{Storing the solutions to subproblems}

We now consider storing our solutions to cells so we do not repeat
ourselves, take \texttt{A} to be an $n \times n$ array
where each cell is undefined as first: \begin{lstlisting}
  size LargestSquare_Stored(x, y) {
    if ((x, y) is black) A[x, y] = 0;
    if ((x == 1) or (y == 1)) A[x, y] = 1;
    if (A[x, y] is undefined) A[x, y] = min(
      LargestSquare_Stored(x - 1, y - 1),
      LargestSquare_Stored(x - 1, y),
      LargestSquare_Stored(x, y - 1));
    return A[x, y];
  }
\end{lstlisting} This can be adapted to work iteratively from
$(1, 1)$ down to $(x, y)$ with a complexity of $O(n^2)$.

\newpage

\subsection{Weighted Interval Scheduling}

We have a set of $n$ intervals, a triple containing a
start time $s_i$, finishing time $f_i$, and a weight $w_i$.
A schedule is a set of intervals such that they do not overlap
(with respect to their starting and finishing times).
\\[\baselineskip]
The intervals are provided as an array $A$, sorted ascending by
finishing times.

\subsubsection{The rightmost compatible interval $p$}

We define $p$ as a function from $\textbf{Interval} \to \textbf{Interval}$,
which takes an interval $i$ and returns the \textbf{latest} interval that 
finishes \textbf{before} $i$.
\\[\baselineskip]
This can be precomputed beforehand in $O(n\log_2(n))$ time by using 
binary search ($O(\log_2(n))$).

\subsubsection{A recursive algorithm}

For $n$ intervals indexed by $\{1, \ldots, n\}$ in $A$: \begin{lstlisting}
weight WIS(i) {
  if (i == 0) return 0;
  return max(WIS(i - 1), WIS(p(i)) + w_i);
}
\end{lstlisting} However, this leads to WIS(i) being calculated more than once
for some $i$.

\subsubsection{Storing the solutions to subproblems}

Now, we consider a global array of schedules $S$ where
$S[i]$ contains \texttt{WIS}($i$) or is undefined: \begin{lstlisting}
weight WIS_Stored(n) {
  if (n == 0) return 0;
  for (int i = 1; i <= n; i++) {
    S[i] = max(S[i - 1], S[p(i)] + w_i);
  }
  return S[n];
}
\end{lstlisting} This takes $O(n)$ time.

\newpage

\subsubsection{Returning the schedule}

We can find the schedule using our
stored $S$ from the previous section: \begin{lstlisting}
schedule FindSchedule(i) {
  if (i == 0) return [];
  if (S(i - 1) <= S(p(i)) + w_i) {
    return FindSchedule(p(i)) ++ [i]; 
  }
  return FindSchedule(i - 1);
}
\end{lstlisting} This takes $O(n)$ time.

\subsection{Self-balancing Trees}

\paragraph{Perfect balance} a tree where each path from the root to a leaf has the
same length is perfectly balanced.
\\[\baselineskip]
We want to use self-balancing trees as an optimisation over linked lists in a
dynamic search structure.
Consider a tree where each node can have between $2$ and $4$ (inclusive) children 
(where a child can be empty) called a $2-3-4$ tree. Take note of the following: 

  \paragraph{2-node} a node with value $v$, $2$ children, and $1$ key
  where the left child is less than or equal to $v$ and 
  the right child is greater than or equal to $v$.
  \paragraph{3-node} a node with values $v_1, v_2$, $3$ children, and $2$ keys
  where the left child is less than or equal to $v_1$, 
  the middle child is between $v_1$ and $v_2$ (inclusive),
  and the right child is greater than or equal to $v_2$.
  \paragraph{4-node} a node with values $v_1, v_2, v_3$, $4$ children, and $3$ keys
  where the left child is less than or equal to $v_1$, 
  the left-middle child is between $v_1$ and $v_2$ (inclusive),
  the right-middle child is between $v_2$ and $v_3$ (inclusive),
  and the right child is greater than or equal to $v_3$.

\newpage

\subsubsection{The height of $2-3-4$ trees}

If we suppose all the nodes in the tree are $2$/$4$ nodes we get the
worst/best case heights for a $2-3-4$ tree with $n$ elements: \begin{center}
  \begin{tabular}{| c | c |}
    \hline
    Node Type & Height \\
    \hline 
    \hline 
    2 & $O(\log_2(n))$ \\
    \hline 
    4 & $O(\log_4(n))$ \\
    \hline
  \end{tabular}
\end{center}

\subsubsection{Insertion on $2-3-4$ trees}

\paragraph{Splitting} this operation works on a $4$-node. The middle value of the node
is added to the parent and two 2-nodes are formed from the remains.
\\[\baselineskip]
When inserting an element $k$ we search for where the element belongs whilst
splitting any $4$-nodes into $2$-nodes as we recurse. We convert the bottom node from
type $t$ to $t + 1$ ($t \neq 3$ by our algorithm structure) and insert our value.

\subsubsection{Deletion on $2-3-4$ trees}

\paragraph{Fusion} this operation works on two $2$-nodes with a shared parent. A relevant key
is taken from the parent and used to form a $4$-node. Fusing the root decreases the height of 
tree and is the only operation with this property.

\paragraph{Transferring} this operation works on a $2$-node and a $3$-node with a shared parent.
A key from the parent is added to the 2-node whilst a key from the 3-node is added to the parent
\\[\baselineskip]
We will consider the cases when deleting a value $k$. For leaves, we search for the value,
transferring and fusing to convert 2-nodes on the path, we delete the value, converting the
node from a node type $t$ to a type $t - 1$ ($t \neq 2$ by our algorithm structure). For
non-leaves, we delete the predecessor of $k$, $k'$ (always a leaf) and replace $k$ with $k'$.

\subsubsection{Binary Search Trees}

For an element $k$ in a binary search tree (acting as our dynamic search structure), 
the left child of an element is less than or equal to $k$ and the right child is 
greater than or equal to $k$.
\\[\baselineskip]
However, this results in $O(\log_2(n))$ time for \texttt{insert}, \texttt{find}, 
and \texttt{delete} for balanced trees like the $2-3-4$ tree above 
but becomes $O(n)$ for unbalanced trees.

\subsection{Skip Lists}

We want to use skip lists as an optimisation over linked lists in a
dynamic search structure. Building on a linked list, we require it is sorted and then 
we can add 'shortcut' levels. Each level is a subset of the linked list in 
the level below with the bottom level being the full list and each level containing
the minimum and maximum.

\subsubsection{Insertion in skip lists}

When inserting an entry, we choose randomly whether it appears in the level above.
We insert it into the lowest level and flip a coin to see if we should insert it 
into the level above. We repeat these coin flips until it fails to be inserted again
(note that each level must contain the minimum and maximum of the list and the top
level should be exactly the minimum and maximum).
\\[\baselineskip]
If there is a level which isn't the bottom layer that contains all entries, we can
delete all levels below it.

\subsubsection{Deletion in skip lists}

When deleting an entry, we simply delete all occurances of the entry. If this is the minimum
or maximum, we ensure that each level contains the minimum or maximum unless the
whole list is empty.
\\[\baselineskip]
If there is a level which isn't the top layer that contains only the minimum and maximum 
entries, we can delete it.

\subsubsection{Finding in skip lists}

We start at the minimum of the top layer, iterating across it until we find it or we find
a greater value. If the next value is greater, we move to the layer below and repeat the
process: \begin{lstlisting}
value find(int key) {
  while (entry.key != max_key) {
    if (entry.key == key) return entry.value;
    else if (entry.key >  key) move down;
    else move right;
  }
  return undefined;
}
\end{lstlisting}

\subsubsection{Runtime of skip lists}

All processes take $O(\log_2(n))$ time on average with randomised levels. Also,
for large $n$, the amount of levels is also $O(\log_2(n))$ on average.

\section{Line Intersections}

Suppose we are given a set of line segments (as two coordinates), we would
like to find all the coordinates of the Intersections between these line
segements.

\subsection{A Simple Algorithm}

We iterate through all the pairs and output the intersections:
\begin{lstlisting}
points intersections_simple(lines) {
  points ps;
  for (int i = 0; i < lines.size(); i++) {
    for (int j = i + 1; j < lines.size(); j++) {
      if (lines[i] intersects lines[j]) {
        ps.push(intersection);
      }
    }    
  }
  return ps;
}
\end{lstlisting} this algorithm takes $O(n^2)$.

\subsection{Output Sensitivity}

It can be seen that certain inputs could potentially have $O(n^2)$ output
but this would make it seem like the simple algorithm is optimal but we
will see that if consider $k$ to be the number of outputs, we can find
an algorithm with $O(n\log_2(n) + k\log_2(n))$ time complexity. However,
if we consider bounds for $k$:
\begin{center}
  \begin{tabular}{ c c c }
    $k \leq 2n$     & $\qquad$ & $k \geq n^2$ \\
    $\Rightarrow$   & $\qquad$ & $\Rightarrow$ \\
    $O(n\log_2(n))$ & $\qquad$ & $O(n^2\log_2(n))$,
  \end{tabular}
\end{center} so for certain inputs this algorithm will be \textbf{worse}
than the simple algorithm.

\subsection{An Outline for Finding Intersections}

It can be seen that for two line segments, they can only have an intersection
if the spans of the segments in the $y$ direction intersect also. Thus,
we could consider sweeping a horizontal line through all our line segments
picking up intersections as we go.

\subsubsection{Adjacency} 
We say two line segments are adjacent if there is a
contiguous horizontal line from one segment to the other (not interrupted
by another line segment). It can be
seen that two segments that are never adjacent can't intersect.

\subsubsection{Event points} 
We can't possibly iterate through all possible $y$
points, thus we only consider 'event points' which are the end points of 
segments and line intersections but this requires that we calculate 
intersections as we go. If we have $k$ intersections,
this gives $O(n + k)$ event points.
\\[\baselineskip]
We consider the set of event points as a priority queue with keys 
equal to their $y$ value, allowing us to \texttt{extractMin} to get
our next event point. However, our process could give rise to duplicate
event points, but these can be dealt with by checking the queue beforehand.

\subsubsection{Status} 
We consider the status of the sweep line to be the
ordered set of line segments currently being interesected by the sweep line
with respect to their $x$ coordinates. The status can clearly only change
at event points, so at each event point we query line segments that have
newly become adjacent.
\\[\baselineskip]
We consider status as a $2-3-4$ tree where: \begin{itemize}
  \item At the top of a line segment, we insert it
  \item At the bottom of a line segment, we delete it
  \item At an intersection, we swap the intersecting lines,
\end{itemize} checking for new intersections as these changes occur.

\subsubsection{The full process}

We add all line segment start and end points to our priority queue, and
iterate through them, updating the status and querying for intersections as
we progress, adding intersections to our output and the queue as necessary.
This takes \newline $O((n + k)\log_2(n))$ time.

\section{Linear Programming}

\subsection{Vector Comparison}

We have that for $v, w$ in $\mathbb{R}^n$ for some $n$ in $\mathbb{Z}_{>0}$ such that: 
\begin{gather*}
  v = \begin{pmatrix}
    v_1 \\ v_2 \\ \vdots \\ v_n
  \end{pmatrix} \qquad
  w = \begin{pmatrix}
    w_1 \\ w_2 \\ \vdots \\ w_n
  \end{pmatrix},
\end{gather*} we have that $v \leq w$ if and only if for all $i$ in $[n]$, $v_i \leq w_i$.
\\[\baselineskip]
A result of this definition is that some vectors are incomparable.

\subsection{Standard Form}

The standard form of a linear programming problem is that we have an objective function
$f : \mathbb{R}^n \to \mathbb{R}$, an $m \times n$ matrix $A$, and an $m$-dimensional
vector $b$ in $\mathbb{R}^m$. The desired output is a vector $x$ in $\mathbb{R}^n$ that 
maximises $f(x)$ subject to $Ax \leq b$ and $x \geq 0$.
\\[\baselineskip]
We can accomodate for minimisation and lower bounds by multiplying by negative one, and
we can write negative variables as the subtraction of positive variables.

\subsection{Integer Linear Programming}

In a linear programming problem where we add the constraint that solutions must be
integers, we can relax our constraints to form approximate solutions.

\section{Flow Algorithms}

\subsection{Flow Networks}

A flow network $F = (G, c, s, t)$ consists of a directed graph $G = (V, E)$, a capacity function
$c : E \to \mathbb{R}$, a source vertex $s$ in $V$ (with empty in-neighbourhood), and a sink
vertex $t$ in $V$ (with empty out-neighbourhood).

\subsubsection{Flows}

A flow in $F$ is a function $f : E \to \mathbb{R}$ with the following properties: \begin{itemize}
  \item No edge has more 'flow' than capacity, for $e$ in $E$: \begin{gather*}
    0 \leq f(e) \leq c(e).
  \end{gather*}
  \item Flow is conserved, for $v$ in $V \setminus \{s, t\}$: \begin{gather*}
    f^-(v) = \sum_{u \in N^-(v)} f((u, v)) = \sum_{w \in N^+(v)} f((v, w)) = f^+(v).
  \end{gather*}
\end{itemize} We denote the value of a flow as $v(f) = f^+(s)$. We can also define $f^+$ and $f^-$
for sets by considering the flow entering and exiting the sets. Similarly, in and out flow
of sets are identical.

\subsubsection{Cuts}

A cut is a partition $A, B$ of $V$ such that the source is in $A$ and the sink is in $B$.
We have that the flow in $A$ minus the flow out is equal to the flow out to $B$ minus the flow in:
\begin{gather*}
  (f^+ - f^-)(A) = (f^- - f^+)(B) = v(f).
\end{gather*}

\subsection{Maximising Flow Value}

We choose a greedy algorithm that works on a modified graph $G_F$, where we add backwards edges
(edges corresponding to edges with non-zero flow) 
and forwards edges 
(edges corresponding to edges with non-capacity flow) 
that allows flow to be pushed back down the edge. So, edges are no longer considered when 
at capacity (except after we choose to push their flow backwards and re-route it).

\subsubsection{Augmenting paths}

An augmenting path is a directed path from $s$ to $t$ in $G_F$.

\subsubsection{Residual capacity}

The residual capacity of an edge is the amount of flow you can add to the forward edge or
the amount of flow you can remove from the backward edge (whichever is greater).

\subsubsection{Pushing}

Considering an augmenting path, pushing the path involves adding as much flow as possible to
all the edges.

\subsubsection{The Ford-Fulkerson algorithm}

We consider all augmenting paths (depth-first) and push each one, updating $G_F$ as we go. 
The resulting flow is maximal. This takes $O(v(f^*)|E|)$ time where $f^*$ is our maximal flow.

\subsubsection{Flow networks and bipartite graphs}

Given a bipartite graph $G = (V, E)$ with bipartition $A, B$, we can form a new graph
$G'$ identical to $G$ except: \begin{itemize}
  \item All edges are directed from $A$ to $B$
  \item All edges have capacity $1$
  \item We add a source vertex connected to all vertices in $A$
  \item We add a sink vertex connected to all vertices in $B$,
\end{itemize} in $G'$, maximal flows corresponding to matchings.

\subsubsection{The Edmonds-Karp algorithm}

If we pick augmenting paths with minimal edges (breadth-first) then we are guaranteed to 
finish Ford-Fulkerson in $O(|V||E|^2)$ time.

\subsubsection{Vertex capacities}

We can add capacities to vertices too, forming a vertex flow network 
$F = (G, c_E, c_V, s, t)$ identical to the flow network except we restrict flow through
vertices. We can form a regular flow network from this by changing each vertex $v$
into two vertices $v^+, v^-$ where the capacity of $(v^+, v^e)$ is $c_V(v)$ and
all the edges going \textit{into} $v$ go into $v^-$ and all the edges going
\textit{out of} $v$ go out of $v^+$.

\subsection{Circulation networks}

A circulation network $C = (G, c, d)$ is a directed graph $G = (V, E)$ and a
capacity function $c : E \to \mathbb{N}$ and demand function $d : V \to \mathbb{N}$.
Vertices with positive demand are sinks, and vertices with negative demand are sources.

\subsubsection{Circulations}

A circulation is a function $f : E \to \mathbb{R}$ with $0 \leq f(e) \leq c(e)$ for each
$e$ in $E$ and $f^-(v) - f^+(v) = D(v)$ for all $v$ in $V$ (flow is conserved except at
sources and sinks).
\\[\baselineskip]
We find circulations by attaching a source vertex to all sources in $C$ with edges equal
to the (negative) demand of the sources and similarly adding a sink vertex to all
sinks in $C$ with edges equal to the demand. This forms a flow network we can run our 
algorithms on.

\vfill

\section{Complexity Theory}

\subsection{Cook Reductions}

Suppose we have algorithms $A_X, A_Y$ which solve the decision problems 
$X$ and $Y$ respectively. If whilst performing $A_X$ we call $A_Y$ as a 
subroutine a $O(x^n)$ number of times (for some finite $n$), we say
we have a Cook reduction from $X$ to $Y$ denoted by $X \leq_c Y$.

\subsubsection{Properties of Cook reductions}

We have that for problems $X, Y, Z$: \begin{itemize}
  \item $X \leq_c Y$, $Y \leq_c Z$ implies that $X \leq_c Z$ (transitivity)
  \item $X \leq_c Y$ implies that if we have a polynomial time algorithm for $Y$, 
  we have one for $X$
  \item $X \leq_c Y$ and there is no polynomial-time algorithm for $X$ implies 
  that there is no polynomial-time algorithm for $Y$.
\end{itemize}

\subsection{Karp Reductions}

For the decision problems $X$ and $Y$, a Karp reduction from $X$ to $Y$
is a map $f$ from the instances of $X$ to the instances of $Y$ such that:
\begin{itemize}
  \item $f(x)$ can be computed in polynomial time (in $|x|$)
  \item $f(x)$ is a \texttt{Yes} instance of $X$ if and only if it's a \texttt{Yes} instance
  of $Y$.
\end{itemize} This is denoted by $X \leq_k Y$.

\subsection{Decision Problems}

A decision problem is a problem such that the answer is in the set $\{\texttt{Yes},
\texttt{ No}\}$.

\subsection{Oracles}

An oracle is a construct that given its corresponding problem, solves it in
$O(1)$ time.

\subsection{The Class, \textbf{NP}}

We have that \textbf{NP} is the class of all decision problems $X$ such that there is some
polynomial-time verification algorithm $A_X$ such that for some input $x$, 
if $x$ is a \texttt{Yes} instance of our problem, there is a witness bit string $w$
such that $A_X(x, w) = \texttt{Yes}$.

\subsubsection{\textbf{NP}-completeness}

We say a problem is \textbf{NP}-complete if it is in \textbf{NP}.

\subsection{The Class, \textbf{P}}

We have that \textbf{P} is the class of all decision problems which have a polynomial-time
algorithm. We have that $\textbf{P} \subseteq \textbf{NP}$ as we can just process the solution
(ignoring the witness).

\subsection{The SAT Problem}

The SAT problem is the problem that asks if when given some formula in conjunctive-normal form
(consisting of AND and OR clauses) we can assign the variables such that the formula is 
satisfied (true).

\subsubsection{Cook-Levin theorem}

We have that all problems in \textbf{NP} is Cook-reducible to the SAT. So, if there's
a polynomial-time algorithm for SAT, $\textbf{P} = \textbf{NP}$.

\subsubsection{\textbf{NP}-hardness}

We say a problem is \textbf{NP}-hard (under Cook reductions) if it is Cook-reducible to SAT.
Similarly, we say a problem is \textbf{NP}-hard (under Karp reductions) if it is Karp-reducible
to all members of \textbf{NP}.

\subsection{The 3-SAT Problem}

We have that the width of a conjunctive-normal form is the number of literals
of all the OR clauses. The 3-SAT problems asks if a width-3 conjunctive-normal 
form is satisfiable. We have that this is \textbf{NP}-complete.

\end{document} 