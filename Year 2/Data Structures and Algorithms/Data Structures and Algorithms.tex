\documentclass[a4paper, 12pt, twoside]{article}
\usepackage[left = 3cm, right = 3cm]{geometry}
\usepackage[english]{babel}
\usepackage[utf8]{inputenc}
\usepackage{mathtools}
\usepackage{amssymb}
\usepackage{amsmath}
\usepackage{multicol}

\begin{document}

\title{Data Structures and Algorithms Notes}
\date{}
\author{\textit{paraphrased by} Tyler Wright}
\maketitle

\vfill

\textit{An important note, these notes are absolutely \textbf{NOT}
  guaranteed to be correct, representative of the course, or rigorous.
  Any result of this is not the author's fault.}

\newpage

\section{Graph Theory}

\subsection{Definition of a Graph}

A graph is a pair of sets $G = (V, E)$, where $V$ is a set of 
vertices (or nodes) and $E$ is a set of edges (or arcs).

\subsection{Definition of an Edge}

An edge of a graph $G = (V, E)$ is $e = \{u, v\}$ in $E$ where $u$,
$v$ are vertices in $V$.

\subsection{Definition of a Neighbourhood}

For a graph $G = (V, E)$ with $v$ in $V$, the neighbourhood
of $v$ is the set $V' \subseteq V$ of vertices connected to
$v$ by an edge in $E$.
\\[\baselineskip]
The neighbourhood of $v$ is denoted by $N(v)$.
\\[\baselineskip]
The neighbourhood of a set of vertices is the union of
the neighbourhoods of each vertex.

\subsection{Definition of Degree}

For a graph $G = (V, E)$ with $v$ in $V$, the degree of $v$
is the size of its neighbourhood.
\\[\baselineskip]
The degree of $v$ is denoted by $d(v)$.

\subsection{Isomorphic Graphs}

Graphs $G_1 = (V_1, E_1)$ and $G_2 = (V_2, E_2)$ are called 
isomorphic if there exists a bijection $f : V_1 \to V_2$ such
that: \begin{gather*}
  \{u, v\} \in E_1 \Longleftrightarrow \{f(u), f(v)\} \in E_2.
\end{gather*} This relationship is denoted by $G_1 \cong G_2$.

\subsection{Definition of a Subgraph}

A graph $G' = (V', E')$ is a subgraph of $G = (V, E)$ if
$V' \subseteq V$ and $E' \subseteq E$.

\newpage

\subsection{Definition of an Induced Subgraph}

An induced subgraph generated from $G = (V, E)$ by $V' \subseteq V$
is the graph $G' = (V', E')$ where: \begin{gather*} 
  E' = \{\{u, v\} \in E \text{ such that } u, v \in V'\}.
\end{gather*} \textit{Essentially, you generate an induced
subgraph from a subset of the vertices of a graph by selecting
edges that join vertices in the subset.}

\subsection{Walks}

\subsubsection{Definition of a walk}

A walk in a graph $G = (V, E)$ is a set of vertices in $V$ connected
by edges in $E$. The length of the walk is the number of edges
traversed in the walk.

\subsubsection{Definition of a path}

A path is a walk where no vertices are repeated.

\subsubsection{Definition of an Euler walk}

An Euler walk is a walk such that every edge is traversed exactly
once. Thus, for a graph $G = (V, E)$, the length is $|E|$.

\subsubsection{Conditions for an Euler walk}

For an Euler walk to be possible on a given graph,
all vertices must have an even degree \textbf{or} exactly
two vertices have odd degree. 
\\[\baselineskip]
If all vertices have even degree we 
have that the Euler walk is a cycle, if exactly two vertices have
odd degree then we have that these vertices are the start and end
points of our Euler walk. 

\subsection{Definition of a Connected Graph}

A connected graph is a graph where for each pair of vertices,
there is a path connecting them.

\subsection{Definition of a Component}

A component of a graph $G = (V, E)$ is a maximal connected 
induced subgraph of $G$. This means an induced subgraph of $G$
that is connected but is not longer connected if a vertex is
removed.
\\[\baselineskip]
Connected graphs have a single component, the entire graph.

\subsection{Digraphs}

\subsubsection{Definition of a digraph}

A digraph (or directed graph) is a graph where each of the edges
has a direction. This direction means the edge can only be traversed
in a single direction.

\subsubsection{Definition of a strongly connected digraph}

A digraph $G = (V, E)$ is strongly connected if for each $u, v$
in $E$, there exists a path from $u$ to $v$ \textbf{and} 
from $v$ to $u$.

\subsubsection{Definition of a weakly connected digraph}

A digraph $G = (V, E)$ is weakly connected if for each $u, v$
in $E$, there exists a path from $u$ to $v$ \textbf{or} 
from $v$ to $u$.

\subsubsection{Definition of components of digraphs}

A strong component of a digraph is the maximal \textit{strongly}
connected induced subgraph.
\\[\baselineskip]
A weak component of a digraph is the maximal \textit{weakly}
connected induced subgraph.
\\[\baselineskip]
\textit{So, these are induced subgraphs that are strongly/weakly
connected but are no longer strongly/weakly connected once a
vertex is removed.}

\newpage

\subsubsection{Definition of neighbourhoods in digraphs}

The neighbourhood of a vertex in a digraph can be considered by
looking at the edges \textit{from} the vertex and the edges
\textit{to} the vertex.
\\[\baselineskip]
The in-neighbourhood of a vertex $v$ are the edges that enter $v$.
The out-neighbourhood of a vertex $v$ are the edges that exit $v$.
These are denoted by $N^-(v)$ and $N^+(v)$ respectively.

\subsubsection{Definition of degrees in digraphs}

For a vertex $v$, the in-degree of the vertex $d^-(v)$ is the size of
the in-neighbourhood and the out-degree of the vertex $d^+(v)$
is the size of the out-neighbourhood.
\\[\baselineskip]
It can be seen that the degree of a given vertex is the sum of
its in and out degree (in a digraph).

\subsubsection{Conditions for an Euler walk in a digraph}

For an Euler walk to be possible on a given digraph, we have
two cases, either:\begin{itemize}
  \item the digraph is strongly connected and every vertex
  has equal in and out degrees, or
  \item one vertex has an in-degree one greater than its out-degree,
  another has an out-degree one greater than its in-degree, and all
  remaining vertices have equal in and out degrees.
\end{itemize}
In the first case we have that the Euler walk is a cycle, 
in the second we have that the special vertices are the start and end
points of our Euler walk. 

\subsubsection{Cycles}

\subsubsection{Definition of a cycle}

A cycle is a walk where the first and last vertices are the same
and each vertex appears at most once (barring the first and last
vertex).

\subsubsection{Definition of a Hamiltonian cycle}

A Hamiltonian cycle is a cycle where each vertex is visited.

\subsubsection{Conditions for a Hamiltonian cycle}

Whilst the conditions necessary for a Hamiltonian cycle in general
are unknown, by Dirac's theorem, we know that for a graph with
$n$ vertices, if every vertex has degree $\frac{n}{2}$ or greater
then a Hamiltonian cycle exists.

\section{Types of Algorithms}

\subsection{Greedy Algorithms}

These types of algorithms start with a trivial solution and
iteratively optimise their solution based on the information
available at the time. They do not retroactively change the solution
based on new data, only add to it.

\end{document}