\section{Linear Programming}

\subsection{Vector Comparison}

We have that for $v, w$ in $\mathbb{R}^n$ for some $n$ in $\mathbb{Z}_{>0}$ such that: 
\begin{gather*}
  v = \begin{pmatrix}
    v_1 \\ v_2 \\ \vdots \\ v_n
  \end{pmatrix} \qquad
  w = \begin{pmatrix}
    w_1 \\ w_2 \\ \vdots \\ w_n
  \end{pmatrix},
\end{gather*} we have that $v \leq w$ if and only if for all $i$ in $[n]$, $v_i \leq w_i$.
\\[\baselineskip]
A result of this definition is that some vectors are incomparable.

\subsection{Standard Form}

The standard form of a linear programming problem is that we have an objective function
$f : \mathbb{R}^n \to \mathbb{R}$, an $m \times n$ matrix $A$, and an $m$-dimensional
vector $b$ in $\mathbb{R}^m$. The desired output is a vector $x$ in $\mathbb{R}^n$ that 
maximises $f(x)$ subject to $Ax \leq b$ and $x \geq 0$.
\\[\baselineskip]
We can accomodate for minimisation and lower bounds by multiplying by negative one, and
we can write negative variables as the subtraction of positive variables.

\subsection{Integer Linear Programming}

In a linear programming problem where we add the constraint that solutions must be
integers, we can relax our constraints to form approximate solutions.