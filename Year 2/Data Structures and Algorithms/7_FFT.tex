\section{Fast Fourier Transforms}

\subsection{Polynomials}

A polynomial of degree $n$ in $\mathbb{Z}_{\geq 0}$ is a function $A$: 
\begin{gather*}
  A(x) = \sum_{i = 0}^n a_ix^i,
\end{gather*} where $a_1, \ldots, a_n$ are the coefficients of $A$. 
We can represent $A$ by listing the coefficients $a_1, \ldots, a_n$, 
called the coefficient representation.
Additionally, we say for $k > n$, $k$ is a degree-bound of $A$.

\subsubsection{Horner's Rule}

We can evaluate polynomials quickly as follows.
For a polynomial $A$ degree $n$: \begin{gather*}
  A(x) = a_0 + x(a_1 + x(a_2 + \cdots + x(a_n)))).
\end{gather*} This can be simplified in the following code:
\begin{lstlisting}
int polynomial(coeffs, x) {
  output = 0;
  for (i = n; i >= 0; i--) {
    output = (output * x) + coeffs[i];
  }
  return output;
}
\end{lstlisting} Taking $O(n)$ time.

\subsubsection{Point-Value Representation}

We can represent a polynomial of degree $n$ in $\mathbb{Z}_{\geq 0}$ 
by a set of points it intersects like so:
\begin{gather*}
  ((x_1, y_1), \ldots, (x_{n + 1}, y_{n + 1})).
\end{gather*} where for $i, j$ in $[n + 1]$ with $i \neq j$, $x_i \neq x_j$. 

\subsubsection{Polynomial Addition}

For two polynomials $A, B$ with degrees $n, m$ in $\mathbb{Z}_{\geq 0}$
and $k = \Max\{n, m\}$:
\\[\baselineskip]
Under coefficient representation, taking: \begin{align*}
    A &= (a_1, \ldots, a_n) \\
    B &= (b_1, \ldots, b_m),
\end{align*} we have that $(A + B) = (a_1 + b_1, \ldots, a_k + b_k)$
where we pad the polynomial of lesser degree with zeroes.
\\[\baselineskip]
Under point-value representation, taking: \begin{align*}
    A = ((x_1, a_1), \ldots, (x_{k + 1}, a_{k + 1})) \\
    B = ((x_1, b_1), \ldots, (x_{k + 1}, b_{k + 1})),
\end{align*} we have that $(A + B) = ((x_1, a_1 + b_1), 
\ldots, (x_{k + 1}, a_{k + 1} + b_{k + 1}))$ where we pad the polynomial 
of lesser degree with zeroes.

\subsubsection{Polynomial Multiplication}

For two polynomials $A, B$ with degrees $n, m$ in $\mathbb{Z}_{\geq 0}$
and $k = 2 \cdot \Max\{n, m\}$:
\\[\baselineskip]
Under coefficient representation, taking: \begin{align*}
    A &= (a_1, \ldots, a_n) \\
    B &= (b_1, \ldots, b_m),
\end{align*} we have that: \begin{gather*}
  (A \cdot B)(x) = (c_1, \ldots, c_k) \qquad \text{where }
  c_i = \sum_{j = 0}^ia_jb_{j - 1}.
\end{gather*} Taking $O(n^2)$ time.
\\[\baselineskip]
We can do this with the point-value representation, taking: \begin{align*}
    A = ((x_1, a_1), \ldots, (x_{k + 1}, a_{k + 1})) \\
    B = ((x_1, b_1), \ldots, (x_{k + 1}, b_{k + 1})),
\end{align*} We have that: \begin{gather*}
  A \cdot B = \{(x_1, a_1 \cdot b_1), \ldots, (x_{k + 1}, a_{k + 1} \cdot b_{k + 1})\}
\end{gather*} Taking $O(n)$ time.

\subsection{The Fast Fourier Transform}

\subsubsection{Roots of Unity}

The idea is that we evaluate a polynomial to perform pointwise
multiplication and then interpolate back into the coefficient representation.
We need to evaluate a polynomial of degree $n$ at $n + 1$ points to
convert it to point-value form. We use the $n + 1$ roots of unity:
\begin{center}
    \scalebox{1.25}{$\omega_{n+1}^k = e^{\frac{2\pi i}{n + 1}k}$},
\end{center} for $k$ in $\{0, 1, \ldots n\}$. So, we consider:
\begin{gather*}
  y_k = A(\omega_{n + 1}^k),
\end{gather*}
The ordered vector $(y_0, \ldots, y_n)$ is the 
Discrete Fourier Transform (DFT) of the coefficient vector of $A$.

\paragraph{Cancellation Lemma:} we have that 
$\omega_{dn}^{dk} = \omega_{n}^{k}$.

\paragraph{Halving Lemma:} we have that if $n$ is even, the set of all
the squared roots of unity is just the set of roots of unity for $\frac{n}{2}$.

\subsubsection{The Discrete Fourier Transform}

For a polynomial $A$ of degree $n$, we call the vector
of evaulations of $A$ at the $n + 1$ roots of unity the
Discrete Fourier transform of the coefficient representation
of $A$.

\subsubsection{The Fast Fourier Transform}

For a polynomial $A$ of even degree $n$,
we define $A^{[0]}$ and $A^{[1]}$ as: \begin{align*}
  A^{[0]} &= a_0 + a_2x + \cdots + a_{n}x^{(n / 2)} \\
  A^{[1]} &= a_1 + a_3x + \cdots + a_{n - 1}x^{(n / 2) - 1},
\end{align*} so we have that: \begin{gather*}
  A(x) = A^{[0]}(x^2) + xA^{[1]}(x^2).
\end{gather*} So, we can split the computation into two
equally sized parts, compute them, and then combine them in linear time.
Combined with the Discrete Fourier transform and the Halving Lemma, we
only have to evaulate $A$ at $\frac{n}{2}$ points.
\begin{lstlisting}
  polynomial FFT(polynomial P, degree n) {
    if (n < 1) {
      // If the degree is less than 1, we just return P
      return P;
    }
    // The (n + 1)th root of unity
    complex root = root_of_unity(n + 1);
    // The cumulative root of unity
    complex root_sum = 1;
    // The polynomial split into two
    polynomial P_0 = (a_0, a_2, a_4, ..., a_n);
    polynomial P_1 = (a_1, a_3, a_5, ..., a_{n - 1});
    // The recursive step
    polynomial y0 = FFT(P_0, n / 2);
    polynomial y1 = FFT(P_1, n / 2);
    // Combine step
    polynomial y;
    for (int k = 0; k <= n / 2; j++) {
      y[k] = y0[k] + root_sum * y1[k];
      y[k + n / 2] = y0[k] - root_sum * y1[k];
      root_sum *= root;
    }
    return y;
  }
\end{lstlisting}

\subsection{Polynomial Multiplication}

For polynomials $A, B$ with degrees $a, b$, let $n = \Max\{a, b\}$: 
\begin{itemize}
  \item Set the degree of $A$ and $B$ to $2n$, padding with zeroes,
  \item Perform the Fast Fourier transform,
  \item Multiply pointwise,
  \item Interpolate with the inverse Fast Fourier transform.
\end{itemize} This process takes $O(n \log(n))$ time.