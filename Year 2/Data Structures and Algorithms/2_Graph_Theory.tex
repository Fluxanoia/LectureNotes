\section{Graphs}

A graph $G$ is a set system $(V, E)$ where the elements of $E$ have
size $2$. Some definitions and facts follow from the definition:
\begin{itemize}
  \item The elements of $V$ are \textbf{vertices},
  \item The elements of $E$ are called \textbf{edges},
  \item The size of $V$ is often called the \textbf{order} of $G$,
  \item $G$ is a $2$-uniform set with ground set $V$,
  \item $u, v$ in $V$ are adjacent if $\{u, v\}$ is in $E$.
\end{itemize}  

\subsubsection{Graph Isomorphisms}

For two graphs $G_1 = (V_1, E_1)$, $G_2 = (V_2, E_2)$, we say that
$G_1$ and $G_2$ are isomorphic ($G_1 \cong G_2$) if there exists a
bijection $\phi : V_1 \to V_2$ such that for each pair of vertices
$u, v$ in $V$ we have that: \begin{gather*}
  \{u, v\} \in E_1 \Longleftrightarrow \{\phi(u), \phi(v)\} \in E_2.
\end{gather*}

\subsection{Neighbourhood and Degree}

For a graph $G = (V, E)$ the neighbourhood of $v$ in $V$ 
is the set of all adjacent vertices (denoted by $N_G(v)$). The
neighbourhood of a set $S$ is simply the union of the neighbourhoods
of the elements of $S$ (minus the vertices in $S$).
The degree of $v$ is simply the size of $N_G(v)$ denoted 
by $\Deg(v)$.

\subsubsection{Minimum and Maximum Degree}

For a graph $G = (V, E)$ we have that the following to represent
minimum and maximum degree: \begin{align*}
  \delta(G) &:= \Min\{\text{deg}(v) : v \in V\} \\
  \Delta(G) &:= \Max\{\text{deg}(v) : v \in V\}.
\end{align*}

\subsubsection{$k$-regular Graphs}

For a graph $G = (V, E)$, we have that $G$ is $k$-regular for some
$k$ in $\mathbb{Z}_{>0}$ if for all $v$ in $V$, we have $\Deg(v) = k$.


\subsection{Subgraphs}

A graph $G' = (V', E')$ is a subgraph of $G = (V, E)$ if
$V' \subseteq V$ and $E' \subseteq E$ such that for all $e$
in $E'$ we have that $e \subseteq V'$.

\subsubsection{Induced Subgraphs}

An induced subgraph generated of $G = (V, E)$
is a subgraph $G' = (V', E')$ where: \begin{gather*} 
  E' = \{\{u, v\} \in E \text{ such that } u, v \in V'\}.
\end{gather*} Induced subgraphs are generated from a subset of 
the vertices of a graph by selecting all the edges that
are subsets of our chosen vertex set.

\subsection{Complements of Graphs}

For a graph $G = (V, E)$, we have that $\bar{G} = (V, \bar{E})$ is
the complement of $G$ where $\bar{E} = \{\{u, v\} : u, v \in V\} \setminus E$.

\subsection{Walks}

A walk of length is a set of vertices connected by edges.
Its length is the number of edges it traverses.
\\[\baselineskip]
A walk is closed if its first and last vertex are identical.

\subsubsection{Types of Walks}

\begin{center}
    \begin{tabular} {| c | c | c | c | c |}
        \hline
        Name & Closed? & Repeats vertices? & Repeats edges? \\
        \hline \hline
        Walk        & Not necessarily & Can    & Can    \\ \hline
        Trail       & Not necessarily & Can    & Cannot \\ \hline
        Paths       & Not necessarily & Cannot & Cannot \\ \hline
        Circuit     & Yes             & Can    & Cannot \\ \hline
        Cycles      & Yes             & Cannot & Cannot \\ \hline
    \end{tabular}
\end{center}

\subsection{Connected Graphs}

A graph is connected if there exists a path between any two vertices 
in the graph.

\subsubsection{Connected Components}

A component of a graph $G$ is a maximally connected subgraph of $G$.

\subsection{Euler Circuits and Trails}

An Euler trail in a graph $G = (V, E)$ is a trail in $G$ that traverses every edge
exactly once. An Euler circuit is a closed Euler trail.

\subsubsection{Conditions for Euler Circuits and Trails}

An Euler circuit in a connected graph $G = (V, E)$ exists if and only 
if each vertex in $V$ has even degree. We can see from this that
an Euler trail exists if and only if each vertex in $V$ has even
degree except exactly two vertices.

\subsection{Hamiltonian Cycles and Paths}

A Hamiltonian path is a path that visits each vertex exactly once.
A closed Hamiltonian path is a Hamiltonian cycle.

\subsubsection{Dirac's Theorem}

For a graph $G = (V, E)$ where $n = |V| \geq 3$: \begin{gather*}
  \delta(G) \geq \frac{n}{2} \Rightarrow G \text{ is Hamiltonian.}
\end{gather*}