\section{Flow Algorithms}

\subsection{Flow Networks}

A flow network $F = (G, c, s, t)$ consists of a directed graph $G = (V, E)$, a capacity function
$c : E \to \mathbb{R}$, a source vertex $s$ in $V$ (with empty in-neighbourhood), and a sink
vertex $t$ in $V$ (with empty out-neighbourhood).

\subsubsection{Flows}

A flow in $F$ is a function $f : E \to \mathbb{R}$ with the following properties: \begin{itemize}
  \item No edge has more 'flow' than capacity, for $e$ in $E$: \begin{gather*}
    0 \leq f(e) \leq c(e),
  \end{gather*}
  \item Flow is conserved, for $v$ in $V \setminus \{s, t\}$: \begin{gather*}
    f^-(v) = \sum_{u \in N^-(v)} f((u, v)) = \sum_{w \in N^+(v)} f((v, w)) = f^+(v).
  \end{gather*}
\end{itemize} We denote the value of a flow as $v(f) = f^+(s)$. We can also define $f^+$ and $f^-$
for sets by considering the flow entering and exiting the sets. Similarly, in and out flow
of sets are identical.

\subsubsection{Cuts}

A cut is a partition $A, B$ of $V$ such that the source is in $A$ and the sink is in $B$.
We have that the flow out of $A$ minus the flow in is equal to the flow in 
to $B$ minus the flow out:
\begin{gather*}
  (f^+ - f^-)(A) = (f^- - f^+)(B) = v(f).
\end{gather*} The capacity of a cut is the sum of all the capacities of all
the edges leaving the $A$ ($c^+(A)$).

\subsection{Residual Graphs}

On a flow network $F = (G, c, s, t)$, we consider the residual graph 
$G_F = (V, E_F)$, where for each $e$ in $E(G)$ we add: \begin{itemize}
  \item A forward edge if $f(e) \leq c(e)$,
  \item A backward edge if $f(e) \geq 0$.
\end{itemize} to $E_F$. The backward edges allow flow to be pushed back 
down edges in $G$. The forward edges make it so that edges at capacity in $G$ 
are no longer considered.

\subsubsection{Augmenting Paths in Residual Graphs}

An augmenting path is a directed path from $s$ to $t$ in $G_F$.

\subsection{Residual Capacity}

On a flow network $F = (G, c, s, t)$,
the residual capacity of an edge $\{x, y\}$ in $G$ is the amount of flow 
you can add to $(x, y)$ in $G_F$ or the amount of flow you can remove from 
$(y, x)$ in $G_F$ (whichever is greater). For paths, the residual
capacity of a path $P$ is the minimum of the residual capacities
of all the edges.

\subsubsection{Pushing}

Considering an augmenting path $P$, pushing $P$ involves adding the
residual capacity of $P$ to the edges used by the path.

\subsection{The Ford-Fulkerson Algorithm}

For a weakly connected flow network $F = (G, c, s, t)$, we generate $G_F$ and
repeatedly push the augmenting paths on $G_F$ onto $G$, updating $G_F$
as the flow on $G$ changes. 
The resulting flow is maximal, taking $O(v(f^*)|E|)$ time where 
$f^*$ is our maximal flow.
\begin{proof}
  To show the resulting flow $f$ is maximal, we first consider a random
  cut $(X, Y)$. We have that: \begin{gather} \label{ffa}
    v(f) = f^+(A) - f^-(A) \leq c^+(A).
  \end{gather} So, if $v(f) = c^+(A)$ we have a maximal flow.
  Now, we consider the cut ($A, B$):
  \begin{align*}
    A &= \{v \in V(G) : \text{there is a path from $s$ to $v$ in } G_F\} \\
    B &= V(G) \setminus A.
  \end{align*} We can see that $f^+(A)$, the flow out of $A$, is equal to
  $c^+(A)$, the capacity of the cut:
  \\[\baselineskip]
  For each $v$ in $B$, there is no path from $s$ to $v$. Hence, there are
  no forward edges in $G_F$ from $A$ to $B$. Thus, all the edges from $A$ 
  to $B$ in $G$ are at capacity. Thus, the flow out of $A$ ($f^+(A)$) 
  must be the capacity of the cut, $c^+(A)$. 
  \\[\baselineskip]
  Similarly, there are no backward edges in $G_F$ from $A$ to $B$. Thus, 
  all the edges from $B$ to $A$ in $G$ have zero flow, hence the flow into 
  $A$ ($f^-(A)$) is zero.
  \\[\baselineskip]
  By the properties of cuts, $f^+(A) - f^-(A) = v(f)$. Thus,
  $c^+(A) = v(f)$ as required by (\ref{ffa}).
\end{proof}

\subsubsection{Max-flow Min-cut Theorem}

The value of a maximum flow is always equal to the minimum capacity of a cut.
\begin{proof}
  We let $f$ be the maximum flow and take $(A, B)$ to be our cut of minimum
  capacity.
  We know that there are no augmenting paths for $f$ as that would 
  contradict the maximality of $f$. By the proof of Ford-Fulkerson proof, 
  (as the output flow of Ford-Fulkerson doesn't have any augmenting paths)
  we pick a cut $(X, Y)$ such that $c^+(X) = v(f)$. But as $(A, B)$ is a
  cut of minimum capacity: \begin{gather*}
    c^+(A) \leq c^+(X) = v(f).
  \end{gather*} But by (\ref{ffa}), we know that $v(f) \leq c^+(A)$. Thus,
  $v(f) = c^+(A)$ as required.
\end{proof}

\subsection{The Edmonds-Karp Algorithm}

If we pick augmenting paths with minimal edges (breadth-first) then we are guaranteed to 
finish Ford-Fulkerson in $O(|V||E|^2)$ time.

\subsection{Additional Uses of Flows}

\subsubsection{Flow Networks and Bipartite Graphs}

Given a bipartite graph $G = (V, E)$ with bipartition $A, B$, we can form a new graph
$G'$ identical to $G$ except: \begin{itemize}
  \item We add a source vertex connected to all vertices in $A$,
  \item We add a sink vertex connected to all vertices in $B$,
  \item All edges are directed from $s$ to $A$ to $B$ to $t$,
  \item All edges have capacity $1$,
\end{itemize} in $G'$, maximal flows corresponding to matchings.

\subsubsection{Vertex Capacities}

We can add capacities to vertices too, forming a vertex flow network 
$F = (G, c_E, c_V, s, t)$ identical to the flow network except we restrict flow through
vertices. We can form a regular flow network from this by changing each vertex $v$
into two vertices $v^+, v^-$ where the capacity of $(v^-, v^+)$ is $c_V(v)$ and
all the edges going into $v$ go into $v^-$ and all the edges going
out of $v$ go out of $v^+$.

\subsubsection{Circulation Networks}

A circulation network $C = (G, c, d)$ is a directed graph $G = (V, E)$ and a
capacity function $c : E \to \mathbb{N}$ and demand function $d : V \to \mathbb{N}$.
Vertices with positive demand are sinks, and vertices with negative demand are sources.
\\[\baselineskip]
A circulation is a function $f : E \to \mathbb{R}$ with $0 \leq f(e) \leq c(e)$ for each
$e$ in $E$ and $f^-(v) - f^+(v) = D(v)$ for all $v$ in $V$ (flow is conserved except at
sources and sinks).
\\[\baselineskip]
We find circulations by attaching a source vertex to all sources in $C$ with edges equal
to the (negative) demand of the sources and similarly adding a sink vertex to all
sinks in $C$ with edges equal to the demand. This forms a flow network we can run our 
algorithms on.