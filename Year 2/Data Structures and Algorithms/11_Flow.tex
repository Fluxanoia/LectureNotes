\section{Flow Algorithms}

\subsection{Flow Networks}

A flow network $F = (G, c, s, t)$ consists of a directed graph $G = (V, E)$, a capacity function
$c : E \to \mathbb{R}$, a source vertex $s$ in $V$ (with empty in-neighbourhood), and a sink
vertex $t$ in $V$ (with empty out-neighbourhood).

\subsubsection{Flows}

A flow in $F$ is a function $f : E \to \mathbb{R}$ with the following properties: \begin{itemize}
  \item No edge has more 'flow' than capacity, for $e$ in $E$: \begin{gather*}
    0 \leq f(e) \leq c(e),
  \end{gather*}
  \item Flow is conserved, for $v$ in $V \setminus \{s, t\}$: \begin{gather*}
    f^-(v) = \sum_{u \in N^-(v)} f((u, v)) = \sum_{w \in N^+(v)} f((v, w)) = f^+(v).
  \end{gather*}
\end{itemize} We denote the value of a flow as $v(f) = f^+(s)$. We can also define $f^+$ and $f^-$
for sets by considering the flow entering and exiting the sets. Similarly, in and out flow
of sets are identical.

\subsubsection{Cuts}

A cut is a partition $A, B$ of $V$ such that the source is in $A$ and the sink is in $B$.
We have that the flow in $A$ minus the flow out is equal to the flow out to $B$ minus the flow in:
\begin{gather*}
  (f^- - f^+)(A) = (f^+ - f^-)(B) = v(f).
\end{gather*}

\subsection{Maximising Flow Value}

We choose a greedy algorithm that works on a modified graph $G_F$, where we add backwards edges
(edges corresponding to edges with non-zero flow) 
and forwards edges 
(edges corresponding to edges with non-capacity flow) 
that allows flow to be pushed back down the edge. So, edges are no longer considered when 
at capacity (except after we choose to push their flow backwards and re-route it).

\subsubsection{Augmenting Paths in Flow Networks}

An augmenting path is a directed path from $s$ to $t$ in $G_F$.

\subsubsection{Residual Capacity}

The residual capacity of an edge is the amount of flow you can add to the forward edge or
the amount of flow you can remove from the backward edge (whichever is greater).

\subsubsection{Pushing}

Considering an augmenting path, pushing the path involves adding as much flow as possible to
all the edges.

\subsubsection{The Ford-Fulkerson Algorithm}

We consider all augmenting paths (depth-first) and push each one, updating $G_F$ as we go. 
The resulting flow is maximal. This takes $O(v(f^*)|E|)$ time where $f^*$ is our maximal flow.

\subsubsection{Flow Networks and Bipartite Graphs}

Given a bipartite graph $G = (V, E)$ with bipartition $A, B$, we can form a new graph
$G'$ identical to $G$ except: \begin{itemize}
  \item All edges are directed from $A$ to $B$,
  \item All edges have capacity $1$,
  \item We add a source vertex connected to all vertices in $A$,
  \item We add a sink vertex connected to all vertices in $B$,
\end{itemize} in $G'$, maximal flows corresponding to matchings.

\subsubsection{The Edmonds-Karp Algorithm}

If we pick augmenting paths with minimal edges (breadth-first) then we are guaranteed to 
finish Ford-Fulkerson in $O(|V||E|^2)$ time.

\subsubsection{Vertex Capacities}

We can add capacities to vertices too, forming a vertex flow network 
$F = (G, c_E, c_V, s, t)$ identical to the flow network except we restrict flow through
vertices. We can form a regular flow network from this by changing each vertex $v$
into two vertices $v^+, v^-$ where the capacity of $(v^-, v^+)$ is $c_V(v)$ and
all the edges going into $v$ go into $v^-$ and all the edges going
out of $v$ go out of $v^+$.

\subsection{Circulation Networks}

A circulation network $C = (G, c, d)$ is a directed graph $G = (V, E)$ and a
capacity function $c : E \to \mathbb{N}$ and demand function $d : V \to \mathbb{N}$.
Vertices with positive demand are sinks, and vertices with negative demand are sources.

\subsubsection{Circulations}

A circulation is a function $f : E \to \mathbb{R}$ with $0 \leq f(e) \leq c(e)$ for each
$e$ in $E$ and $f^-(v) - f^+(v) = D(v)$ for all $v$ in $V$ (flow is conserved except at
sources and sinks).
\\[\baselineskip]
We find circulations by attaching a source vertex to all sources in $C$ with edges equal
to the (negative) demand of the sources and similarly adding a sink vertex to all
sinks in $C$ with edges equal to the demand. This forms a flow network we can run our 
algorithms on.