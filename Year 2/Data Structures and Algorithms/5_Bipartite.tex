\section{Bipartite Graphs}

A graph $G = (V, E)$ is bipartite if $V$ can be partitioned into
two vertex sets $V_1, V_2$ such that each edge connects a vertex
from $V_1$ to a vertex in $V_2$.

\subsection{Matchings}

For $G = (V, E)$ a bipartite graph with bipartition $X, Y$,
a matching from $X$ to $Y$ is a set of edges $M \subseteq E$
such that $f : X \to Y$ defined by: \begin{gather*}
  f(x) := y \qquad \text{ where } \{x, y\} \in M,
\end{gather*} is injective on $M$. The matching is called perfect
if $f$ is an isomorphism.

\subsection{Augmenting Paths}

An augmenting path is a path in a bipartite graph $G$ that alternates between
edges in a matching $M$ for $G$ and edges in $E(G) \setminus M$ starting
and finishing with the latter. Additionally, the first and last vertices 
must not be matched in the matching $M$.

\subsubsection{Partitioning Augmenting Paths}

Thus, an augmenting path $P$ associated to a matching $M$ can
have its edge set partitioned into the edges in $M$ and the
edges not in $M$: \begin{align*}
  P^- &= \{e \in E(P) : e \in M\}, \\
  P^+ &= E(P) \setminus P^-.
\end{align*} Considering the fact that the first and
last vertices in $P$ are unmatched so $P$ must start and end with
edges not in the matching (in $P^+$). 
As the edges alternate between being in and out of the matching, 
$|P^+| = |P^-| + 1$.

\subsubsection{Switching Augmenting Paths}

We define the operation \texttt{switch} on an augmenting path $P$ 
associated with a matching $M$ that removes all the matching edges 
in $P$ from $M$ and adds the remaining edges from $P$.
So, if $M'$ is our matching after \texttt{switch} then: 
\begin{gather*}
  M' = (M \setminus P^-) \cup P^+.
\end{gather*} We have that $|M'| = |M| + 1$ as $|P^+| = |P^-| + 1$.

\subsubsection{Finding Augmenting Paths}

For a bipartite graph $G = (V, E)$ bipartite with bipartition $A, B$
and a matching $M$ from $A$ to $B$, we can find an augmenting path in
$G$ by forming the bipartite digraph $G' = (V, E')$. We form $E'$ by
taking each $\{a, b\}$ in $E$ where $a$ and $b$ are in $A$ and $B$ 
respectively and mapping:
\begin{gather*}
  \{a, b\} \mapsto \begin{cases}
    (b, a) \in E' & \text{if } \{a, b\} \in M \\
    (a, b) \in E' & \text{otherwise},
  \end{cases}
\end{gather*} directing matching edges towards $A$ and all other edges
towards $B$. Finding an augmenting path in $G$ reduces to finding a
path in $G'$ which can be done using breadth-first search.

\subsection{Berge's Lemma}

For a bipartite graph $G$ and a matching $M$, if $G$ admits no augmenting
paths under $M$ then $M$ is maximal.

\subsection{$k$-to-1 Semi-matchings}

For $G$ a bipartite graph with bipartition $X, Y$,
a $k$-to-1 semi-matching from $X$ to $Y$ is a subgraph $G'$ of $G$
such that: \begin{itemize}
  \item Each vertex in $X$ in $G'$ has degree at most $k$,
  \item Each vertex in $Y$ in $G'$ has degree at most $1$.
\end{itemize} Finding maximum $k$-to-1 semi-matchings reduces to 
a matching problem if we simply create $k - 1$ copies of each
vertex in $X$ and find a maximum matching. We can then delete the
copies and add their edges to the original vertex.

\newpage

\subsection{Hall's Marriage Theorem}

For $G = (V, E)$ a bipartite graph with bipartition $X, Y$: \begin{gather*}
  G \text{ has a perfect matching from } X \text{ to } Y \\
  \Longleftrightarrow \\
  \text{For all } S \subseteq X, |N(S)| \geq |S|.
\end{gather*}
\begin{proof}
  Suppose $G$ has a perfect matching from $X$ to $Y$, we have that
  for each set of vertices in $X$, they each match to a
  vertex in $Y$ under our matching. Thus, for each $S \subseteq X$,
  $|N(S)| \geq |S|$. Suppose for all $S \subseteq X$, 
  $|N(S)| \geq |S|$. If $|X| = 1$, $S = X = \{x\}$ meaning
  $|N(x)| \geq 1$, so there is some edge from $x$ to an
  element in $y$. This edge is our matching.
  Suppose now that $|X| > 1$ and for all sizes of $X$ in
  $\{1, \ldots, |X| - 1\}$, the statement holds.
  \paragraph{Case 1}
  Suppose for all $S \subseteq X$ we actually have that 
  $|N(S)| > |S|$ (a stronger condition). Choose some
  $x$ in $X$ and some $y$ in $Y$ connected to $x$. We
  form $G'$ by removing $x$, $y$ and, all their incident
  edges from $G$. The vertex class of $X$ in $G'$
  is strictly smaller than that of $X$ in $G$. For
  each vertex in $X$ in $G'$, we have that at most
  one edge was removed (the one connected to $y$).
  Thus, we have that $|N_{G'}(S)| \geq |S|$ so by the
  inductive hypothesis, $G'$ has a matching. By adding
  $\{x, y\}$ to this matching, we get a matching for $G$.
  \paragraph{Case 2}
  Suppose there is some $S \subseteq X$ such that $|N(S)| = |S|$.
  Consider the graph $G_S = (S \cup N_G(S), E_S)$ the induced
  subgraph on $S \cup N_G(S)$. By the inductive hypothesis,
  we have a matching in $G_S$. Consider 
  $G_S' = (V \setminus V(G_S), E_S')$ the induced subgraph on
  $V \setminus V(G_S)$. Observe for each $T \subseteq V(G_S')$,
  $|N_G(T \cup S)| \geq |T| + |S|$, thus: \begin{gather*}
      |N_{G_S'}(T)| = |N_G(T \cup S) \setminus N_G(S)| \geq |T|.
  \end{gather*} Thus, by the inductive hypothesis, we have a matching
  from $S$ to $N_G(S)$ and from $X \setminus S$ to $Y \setminus N_G(S)$.
  Combining these gives a perfect matching in $G$. 
\end{proof}