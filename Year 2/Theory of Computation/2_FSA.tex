\section{Finite State Automaton}

\subsubsection{Deterministic Finite State Automaton}

A deterministic finite state automaton (DFA) is a 5-tuple 
$M = \langle Q, \Sigma, \delta, q_0, F \rangle$
where: \begin{align*}
    Q      &= \text{ any finite set, called the states,} \\
    \Sigma &= \text{ any alphabet,} \\ 
    \delta &\in \{ Q \times \Sigma \to Q \} \text{ is the the transition function,} \\
    q_0    &\in Q \text{ is the initial state,} \\
    F      &\subseteq Q \text{ is the set of accept states.}
\end{align*} We say that $M$ accepts a word $w$ in $\Sigma$
if there is a sequence of states $r_0, \ldots, r_n$ in $Q$
satisfying: \begin{itemize}
    \item $r_0 = q_0$,
    \item $\delta(r_i, w_{i + 1}) = r_{i + 1}$,
    \item $r_n$ is in $F$.
\end{itemize}

\subsubsection{Product Automaton}

For the two DFA: \begin{align*}
    M_1 = \langle Q_1, \Sigma, \delta_1, q_1, F_1 \rangle,
    M_2 = \langle Q_2, \Sigma, \delta_2, q_2, F_2 \rangle,
\end{align*} the product automaton $M$ is: \begin{gather*}
    M = M_1 \times M_2 = \langle Q, \Sigma, \delta, q_0, F \rangle,
\end{gather*} where: \begin{align*}
    Q &= Q_1 \times Q_2 \\
    \delta((p_1, p_2), a) &= (\delta_1(p_1, a), \delta_2(p_2, a)), \\
    q_0 &= (q_1, q_2),
    F = F_1 \times F_2.
\end{align*}

\subsection{Non-deterministic Finite State Automaton}

A non-deterministic finite state automaton (NFA) is identical
to a DFA except our transition function is from 
$Q \times \Sigma_\epsilon \to \mathcal{P}(Q)$ where
$\Sigma_\epsilon$ is an alphabet $\Sigma$ with the empty word
added.
\\[\baselineskip]
Transitioning on the empty word doesn't consume a letter of our
input word and arbitrary choices are made by the automaton when
choices present themselves. However, we have that each NFA is
arbitrarily lucky and each word that could be accepted, is.

\subsubsection{NFA Simulated by DFA}

We can simulate an arbitrary NFA: \begin{gather*}
    M =  \langle Q, \Sigma, \delta, q_0, F \rangle
\end{gather*}
with a DFA: \begin{gather*}
    M' = \langle Q', \Sigma_\epsilon, \delta', q_0', F' \rangle
\end{gather*} where: \begin{align*}
    Q'   &= \mathcal{P}(Q), \\
    q_0' &= \{q_0\}, \\
    F'   &= \{q' \in Q' : \text{ for some } q \in q', q \in F\}.
\end{align*} and we have that for some letter $a$ in $\Sigma$
and $q \in Q$: \begin{gather*}
    \delta(q, a) = \delta'(q, a),
\end{gather*} and the empty word is dealt with by the union of
the states reachable through each $\epsilon$ transition.
    

\subsection{Languages}

For a DFA $M$, the language set of $M$ denoted by $L(M)$
is the maximal set of words in the alphabet of $M$ such that
for each $w$ in $L(M)$, $M$ accepts $w$. We say $M$ recognises
a language $A$ if $L(M) = A$.

\subsubsection{Regular Languages}

A language is regular if it is recognised by some DFA.

\subsubsection{The Intersection of Regular Languages}

For the two DFA $M_1$ and $M_2$ with languages
$A$ and $B$ (resp.), we have that $A \cap B$ is
recognised by $M_1 \times M_2$ the product automaton.
