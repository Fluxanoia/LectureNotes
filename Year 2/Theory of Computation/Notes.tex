\documentclass[a4paper, 12pt, twoside]{article}
\usepackage[left = 3cm, right = 3cm]{geometry}
\usepackage[english]{babel}
\usepackage[utf8]{inputenc}
\usepackage{mathtools}
\usepackage{amssymb}
\usepackage{amsmath}
\usepackage{amsthm}
\usepackage{multicol}
\usepackage{multirow}
\usepackage{pgfplots}
\usepackage{pgfplotstable}
\usepackage{listings}
\usepackage{xcolor}
\usepackage{color}
\usepackage{graphicx}

\pgfplotsset{compat=1.5.1}

\lstset{frame=none,
  language=C++,
  aboveskip=3mm,
  belowskip=3mm,
  showstringspaces=false,
  columns=flexible,
  basicstyle={\small\ttfamily},
  numbers=none,
  numberstyle=\tiny\color{gray},
  keywordstyle=\color{blue},
  commentstyle=\color{gray},
  stringstyle=\color{orange},
  breaklines=true,
  breakatwhitespace=true,
  tabsize=2
}

\newcommand*{\bfrac}[2]{\genfrac{}{}{0pt}{}{#1}{#2}}

\begin{document}

\title{Theory of Computation Notes}
\date{}
\author{by Tyler Wright \\
  \\
  github.com/Fluxanoia $\qquad$ fluxanoia.co.uk
}
\maketitle

\vfill

\textit{These notes are not necessarily correct,
consistent, representative of the course as it stands today or, 
rigorous. Any result of the above is not the author's fault.}
\\[\baselineskip]
\textbf{Furthermore, these notes are incomplete and will remain so for the 
foreseeable future.}

% \addtocounter{section}{-1}
% \section{Notation}

We commonly deal with the following concepts in 
Language Engineering
which I will abbreviate as follows for brevity:
\begin{center}
    \begin{tabular}{ | r | c | }
        \hline
        Term & Notation \\
        \hline \hline
        \hline
    \end{tabular}
\end{center}

\newpage

\tableofcontents

\section{Data Acquisition}

\subsection{Analogue to Digital Conversion}

There are two steps to this conversion, sampling and quantisation.
They can be done in any order.

\subsubsection{Nyquist-Shannon Sampling Theorem}
If a function $f$ contains no frequencies higher than some
$h_{\text{max}}$ hertz, it is completely determined by
sampling at points spaced $\frac{1}{2 \cdot h_{\text{max}}}$
apart.

\section{Data Characteristics}

\subsection{Measures of Distance}

A valid distance measure $D : A \times A \to \mathbb{R}$
for some data set $A$ has the following properties, it is: \begin{itemize}
    \item non-negative,
    \item reflexive ($D(a, b) = 0 \Longleftrightarrow a = b$),
    \item symmetric,
    \item satisfies the triangle inequality
    ($D(a, b) + D(b, c) \geq D(a, c)$).
\end{itemize}

\subsubsection{Euclidean Distance in $\mathbb{R}^n$ ($p$-norm distance)}

For two vectors $x$ and $y$ in $\mathbb{R}^n$, we have the Euclidean
distance $D$ is: \begin{gather*}
    D(x, y) := \sqrt[\uproot{2}p]{\sum_{i = 1}^n |x_i - y_i|^p}.
\end{gather*}

\subsubsection{Chebyshev Distance in $\mathbb{R}^n$ ($\infty$-norm distance)}

For two vectors $x$ and $y$ in $\mathbb{R}^n$, we have the Chebyshev
distance $D$ is: \begin{gather*}
    D(x, y) := \lim_{n \to \infty} \left(
        \sqrt[\uproot{2}p]{\sum_{i = 1}^n |x_i - y_i|^p}
    \, \right) = \max_{i \in [n]}(|x_i - y_i|).
\end{gather*}

\subsubsection{Time Series Distance}

Finding the distance between two time series $x$ and $y$
of length $n$ and $m$ (resp.) can be found using Dynamic Time 
Warping: \begin{gather*}
    D_{tw}(x, y) := D(x_1, y_1) + 
    \min\{
        D_{tw}(x, y'),
        D_{tw}(x', y),
        D_{tw}(x', y')
    \},
\end{gather*} where $D$ is some numerical distance measure and
$x'$ and $y'$ are the time series length $n - 1$ and $m - 1$
(resp.) corresponding to $x$ and $y$ with the first element
removed.

\subsubsection{Hamming Distance}

When given two strings of the same length, the Hamming distance
between them is how many characters differ in the strings at each
index.

\subsubsection{Edit Distance}

When given two strings of any length, the edit distance between
them is the smallest number of insertions, substitutions and,
deletions that can transform one string into the other (or vice versa).

\subsubsection{Wu and Palmer Distance}

This measure is based on a hierarchy of word semantics, a
graph of relationships between words based on meaning.
Using the shortest distance between the words $d_1$ and the 
shortest distance from a most specific ancestor to the path
$d_2$ we have the distance measure: \begin{gather*}
    D(w_1, w_2) := \frac{2 \cdot d_2}{d_1 + 2 \cdot d_2} - 1.
\end{gather*}

\subsection{Summary Statistics}

\subsubsection{Mean}

We take $X = \{x_1, \ldots, x_n\}$ to be a data set.
The mean $\bar{X}$ is defined as follows: \begin{gather*}
    \bar{X} := \frac{1}{n}\sum_{i = 1}^n x_i.
\end{gather*}

\subsubsection{Standard Deviation and Variance}

We take $X = \{x_1, \ldots, x_n\}$ to be a data set.
We have that for $\sigma_X$, the standard deviation of $X$,
the variance of $X$ is $\sigma_X^2$. We define the variance
(and thus the standard deviation) as follows: \begin{gather*}
    \sigma_X^2 = \frac{1}{n - 1}\sum_{i = 1}^n(x_i - \bar{X})^2.
\end{gather*} 

\subsubsection{Covariance}

We take $X = \{x_1, \ldots, x_n\}$ to be a data set consisting
of $m$-dimensional row vectors. We define the covariance matrix: 
\begin{gather*}
    \Sigma = \frac{1}{n - 1}\sum_{i = 1}^n (x_i - \mu)^T(x_i - \mu),
\end{gather*} where $\mu$ is the $m$-dimensional row vector
where the $j^{\text{th}}$ entry corresponds to the mean of the
$j^{\text{th}}$ entry of each $x_i$. This yields a $m \times m$
matrix that is square and symmetric with the variance of the
$j^{\text{th}}$ entry of each $x_i$ on the $j^{\text{th}}$ value
on the diagonal.

\paragraph{Eigenvalues and Eigenvectors} As the matrix is symmetric
we can diagonalise it and find the eigenvalues and eigenvectors.
The major axis is the eigenvector corresponding to the largest
eigenvalue and the minor axis is the eigenvector corresponding
to the smallest value.

\subsubsection{Linear Regression}

For a two dimensional set of data: \begin{gather*}
    D = \{(x_1, y_1), (x_2, y_2), \ldots, (x_n, y_n)\}.
\end{gather*} We calculate a gradient $r$ and an intercept $c$ to 
uniquely define the regression line $(y = rx + c)$ on $D$: \begin{align*}
    r &= \frac{
        \sum_{i = 1}^n (x_i \cdot y_i) - n\bar{x}\bar{y}
    }{
        \sum_{i = 1}^n (x_i^2) - n\bar{x}^2
    }, \\
    c &= \bar{y} - r \cdot \bar{x}.
\end{align*} Outliers have disproportionate effects due to the
squares used by the measure. 

\newpage

\noindent
For a two dimensional set of data: 
\begin{gather*}
    D = \{(x_1, y_1), (x_2, y_2), \ldots, (x_n, y_n)\}.
\end{gather*} where each $x_i$ is $k$-dimensional,
we can use matrices to calculate our coefficients: \begin{gather*}
    R = (X^{\text{T}}X)^{-1}X^{\text{T}}Y,
\end{gather*} where: \begin{gather*}
    R = \begin{pmatrix}
        r_0 \\ r_1 \\ \vdots \\ r_k
    \end{pmatrix} 
    \qquad
    Y = \begin{pmatrix}
        y_1 \\ y_2 \\ \vdots \\ y_n
    \end{pmatrix}
    \qquad
    X = \begin{pmatrix}
        1      & \leftarrow & x_1    & \rightarrow \\
        1      & \leftarrow & x_2    & \rightarrow \\
        \vdots &            & \vdots &             \\
        1      & \leftarrow & x_n    & \rightarrow
    \end{pmatrix}.
\end{gather*} We have a least squares linear regression line
$y = r_0 + r_1x_1 + \cdots + r_nx_n$.

\subsubsection{Data Normalisation}

Data may need to be normalised before we use our distance measures
on it. We consider a data set $X = \{x_1, \ldots, x_n\}$
with mean $\mu$ and standard deviation $\sigma$.

\paragraph{Scaling} We map each $x_i$ for $i \in [n]$ as follows:
\begin{gather*}
    x_i \mapsto \frac{x_i - \min(X)}{\max(X) - \min(X)}.
\end{gather*}

\paragraph{Standardisation} We map each $x_i$ for $i \in [n]$ as follows:
\begin{gather*}
    x_i \mapsto \frac{x_i - \mu}{\sigma}.
\end{gather*}

\paragraph{Scaling to Unit Length} We map each $x_i$ for $i \in [n]$ as follows:
\begin{gather*}
    x_i \mapsto \frac{x_i}{|x_i|}.
\end{gather*} where $|x_i|$ denotes the magnitude of $x_i$.

\subsubsection{Outliers}

A small amount of values significantly different to the remainder
of the data set.
\section{Deterministic Finite State Automaton}

A deterministic finite state automaton (DFA) is a 5-tuple 
$M = \langle Q, \Sigma, \delta, q_0, F \rangle$
where: \begin{align*}
    Q      &= \text{ any finite set, called the states,} \\
    \Sigma &= \text{ any alphabet,} \\ 
    \delta &\in \{ Q \times \Sigma \to Q \} \text{ is the the transition function,} \\
    q_0    &\in Q \text{ is the initial state,} \\
    F      &\subseteq Q \text{ is the set of accept states.}
\end{align*} We say that $M$ accepts a word $w$ in $\Sigma$
if there is a sequence of states $r_0, \ldots, r_n$ in $Q$
satisfying: \begin{itemize}
    \item $r_0 = q_0$,
    \item $\delta(r_i, w_{i + 1}) = r_{i + 1}$,
    \item $r_n$ is in $F$.
\end{itemize}

\subsection{Product Automaton}

For the two DFA: \begin{align*}
    M_1 = \langle Q_1, \Sigma, \delta_1, q_1, F_1 \rangle,
    M_2 = \langle Q_2, \Sigma, \delta_2, q_2, F_2 \rangle,
\end{align*} the product automaton $M$ is: \begin{gather*}
    M = M_1 \times M_2 = \langle Q, \Sigma, \delta, q_0, F \rangle,
\end{gather*} where: \begin{align*}
    Q &= Q_1 \times Q_2 \\
    \delta((p_1, p_2), a) &= (\delta_1(p_1, a), \delta_2(p_2, a)), \\
    q_0 &= (q_1, q_2), \\
    F &= F_1 \times F_2.
\end{align*}

\section{Regular Languages}

For a DFA $M$, the language set of $M$ denoted by $L(M)$
is the maximal set of words in the alphabet of $M$ such that
for each $w$ in $L(M)$, $M$ accepts $w$. We say $M$ recognises
a language $A$ if $L(M) = A$.
\\[\baselineskip]
A language is regular if it is recognised by some DFA.

\subsection{Operations}

We can calculate the union and intersection of regular languages
as expected and for two DFA $M_1$ and $M_2$ with languages
$A$ and $B$ (resp.), we have that $A \cap B$ is
recognised by $M_1 \times M_2$ the product automaton.
\\[\baselineskip]
Additionally, we can concatenate two regular languages $A$ and
$B$: \begin{gather*}
    A \circ B = \{xy : x \in A \text{ and } y \in B\},
\end{gather*} and form the Kleene Star: \begin{gather*}
    A^* = \{x_0 \cdots x_k : k \in \mathbb{Z}_{\geq 0}
        \text{ and for each } i \in \{0, 1, \ldots, k\}, 
        x_i \in A\}.
\end{gather*} We have that each of these operations are closed
in the set of regular languages.

\subsection{Regular Expressions}

We have that $R$ is a regular expression over an alphabet
$\Sigma$ if it has one of the following shapes:
\begin{center}
    \begin{tabular}{ c l }
        $\emptyset$ & \\
        $\epsilon$ & \\
        $a$ & for some $a$ in $\Sigma$ \\
        $R_1 \cup R_2$ & for some regular expressions $R_1$ and $R_2$ \\
        $R_1 \circ R_2$ & for some regular expressions $R_1$ and $R_2$ \\
        $R^*$ & for some regular expression $R$ \\
    \end{tabular}
\end{center} The language of regular expressions $R_1$ and $R_2$
can be formed as follows: \begin{align*}
    L(\emptyset) &= \emptyset \\
    L(\epsilon) &= \{\epsilon\} \\
    L(a) &= \{a\} \\
    L(R_1 \cup R_2) &= L(R_1) \cup L(R_2) \\
    L(R_1 \circ R_2) &= L(R_1) \circ L(R_2) \\
    L(R_1^*) &= L(R_1)^*
\end{align*} We have that a language $L$ is regular
if and only if $L = L(R)$ for some regular expression $R$.

\subsection{Limitations of Regular Languages}

We can see, for example, that $\{0^k1^k : k \in \mathbb{Z}_{>0}\}$
over the alphabet $\{0, 1\}$ is not a regular language. By
using the following Pumping Lemma, we can generate a contradiction.

\subsubsection{The Pumping Lemma}

Supposing $A$ is regular, then there is some $p$ in $\mathbb{Z}_{>0}$
such that for any word $w$ longer than $p$, we can write $w = xyz$ 
such that: \begin{itemize}
    \item For each $k$ in $\mathbb{Z}_{\geq 0}$, $xy^kz$ is in $A$,
    \item $y$ is non-empty,
    \item $xy$ is shorter than $p$.   
\end{itemize}

\section{Non-deterministic Finite State Automaton}

A non-deterministic finite state automaton (NFA) is identical
to a DFA except our transition function is from 
$Q \times \Sigma_\epsilon \to \mathcal{P}(Q)$ where
$\Sigma_\epsilon$ is an alphabet $\Sigma$ with the empty word
added.
\\[\baselineskip]
Transitioning on the empty word doesn't consume a letter of our
input word and arbitrary choices are made by the automaton when
choices present themselves. We have that a word is accepted
in an NFA if and only if there is at least one computation where 
the word is accepted.

\subsection{Epsilon Closure}

For the NFA $M =  \langle Q, \Sigma, \delta, q_0, F \rangle$, 
and $R \subseteq Q$, we define the epsilon closure
of $R$ to be: \begin{gather*}
    E(R) := \left\{
        q \in Q : \bfrac{
        \text{ where there is a series of 
        transitions solely over }
    }{
        \epsilon \text{ from some } r \text{ in } 
        R \text{ to } q
    }\right\}
\end{gather*}

\subsection{Simulation via a DFA}

We can simulate an arbitrary NFA: \begin{gather*}
    M =  \langle Q, \Sigma, \delta, q_0, F \rangle
\end{gather*}
with a DFA: \begin{gather*}
    M' = \langle Q', \Sigma_\epsilon, \delta', q_0', F' \rangle
\end{gather*} where: \begin{align*}
    Q'   &= \mathcal{P}(Q), \\
    \delta'(q, a) &= \{q : \text{for some }
        r \in R, q \in E(\delta(r, a))
    \} \\
    q_0' &= E(\{q_0\}), \\
    F'   &= \{q' \in Q' : \text{for some } q \in q', q \in F\}.
\end{align*} Now that we have this, we know that languages are
regular if and only if they are accepted by some NFA as
all DFA are NFA and each NFA can be expressed by a DFA.
\section{Generalised NFA}

A generalised non-deterministic finite state automaton is a 5-tuple
$M = \langle Q, \Sigma, \delta, p, p' \rangle$ where:
\begin{align*}
    Q      &= \text{ the set of states,} \\
    \Sigma &= \text{ any alphabet,} \\ 
    \delta &\in \big\{ 
        (Q \setminus \{p'\})
        \times 
        (Q \setminus \{p\})
        \to \mathcal{R} 
        \big\} \text{ is the the transition function,} \\
    p      &\in Q \text{ is the initial state,} \\
    p'     &\in Q \text{ is the accept state,}
\end{align*} where $\mathcal{R}$ is the set of all regular
expressions. We say that that for a word $w = w_1 \cdots w_n$,
$M$ accepts $w$ if for a corresponding series of states
$q_0, \ldots, q_n$: \begin{itemize}
    \item $q_0 = p$,
    \item $q_n = p'$,
    \item for each $i$ in $[n]$, $w_i$ is in 
        $L(\delta(q_{i - 1}, q_i))$.
\end{itemize}

\subsection{Conversion to a Regular Expression}

We can distill a GNFA into a regular expression by iteratively
removing states from it until there is only the start and accept
state, joined by the sole transition which describes the GNFA
as a regular expression.
\\[\baselineskip]
Taking $M = \langle Q, \Sigma, \delta, p, p' \rangle$ to be a
GNFA, we can choose $q$ in $Q \setminus \{p, p'\}$ and form
$M' = \langle Q \setminus \{q\}, \Sigma, \delta', p, p' \rangle$
where: \begin{align*}
    \delta'& : \big\{ 
        (Q \setminus \{q, p'\})
        \times 
        (Q \setminus \{q, p\})
        \to \mathcal{R} 
    \big\} \\
    \delta'&(q_1, q_2) = R_1R_2^*R_3 \cup R_4,
\end{align*} and $R_1, R_2, R_3, R_4$ are defined as: \begin{align*}
    R_1 &= \delta(q_1, q), \\
    R_2 &= \delta(q, q), \\
    R_3 &= \delta(q, q_2), \\
    R_4 &= \delta(q_1, q_2).
\end{align*}
\section{Context-free Grammars}

We use context-free grammars to generate context-free langauages. 
A context-free grammar
is a 4-tuple $G = \langle V, \Sigma, R, S \rangle$ where: \begin{align*}
    V      &= \text{ the set of variables (non-terminals),}\\
    \Sigma &= \text{ the set of terminals, disjoint from } V, \\
    R      &= \text{ the set of rules},\\
    S      &\in V \text{ is the start variable}.
\end{align*} We have that each rule is a pair of a variable $A$ and a string $w$
which it maps to.
\\[\baselineskip]
By using the rules of the context-free grammar on the start variable, we can
generate a language from it: \begin{gather*}
    L(G) := \{w \in \Sigma^* : S \Rightarrow^* w \},
\end{gather*} where $\Rightarrow^*$ denotes some amount of applications
of the rules of $G$ onto $S$.

\subsection{Ambiguity}

A derivation of a string $w$ in a grammar $G$ is a
left-most derivation if, at every step in the derivation,
the left-most remaining variable is evaluated.
\\[\baselineskip]
A string $w$ is generated ambiguously if it has more than
one unique left-most derivation under $G$. $G$ is ambiguous
if it generates some string ambiguously.

\subsection{Push-down Automata}

A push-down automaton is a 6-tuple
$M = \langle Q, \Sigma, \Gamma, \delta, q_0, F \rangle$ where:
\begin{align*}
    Q      &= \text{ the set of states,} \\
    \Sigma &= \text{ any alphabet,} \\ 
    \Gamma &= \text{ any alphabet, called the stack alphabet,} \\ 
    \delta &\in \big\{ 
        (Q \times (\Sigma \cup \epsilon) \times (\Gamma \cup \epsilon))
        \to \mathcal{P}(Q \times (\Gamma \cup \epsilon))
        \big\} \text{ is the the transition function,} \\
    q_0    &\in Q \text{ is the initial state,} \\
    F      &\subset Q \text{ is the set of accept states,}
\end{align*} 

\newpage
\noindent
We say that that for a word $w = w_1 \cdots w_n$,
$M$ accepts $w$ if for a corresponding series of states
$p_0, \ldots, p_n$ and strings $s_0, \ldots, s_n$ in $\Gamma^*$: 
\begin{itemize}
    \item $p_0 = q_0$,
    \item $s_0 = \epsilon$,
    \item $p_n \in F$,
    \item for each $i$ in $[n - 1]$ and for some $x, y$ in $(\Gamma \cup \epsilon)$: 
    \begin{itemize}
        \item $(p_{i + 1}, x) \in \delta(p_{i}, w_{i + 1}, y)$,
        \item $s_i$ is of the form $ys$ and 
        $s_{i + 1}$ is of the form $xs$ for some $s$ in $\Gamma^*$.
    \end{itemize} 
\end{itemize} The language recognised by $M$ is the set of words accepted by it.

\subsubsection{Equivalence of PDA to CFG}

Each PDA has a corresponding CFG which generates its language.
Similarly, each CFG has a PDA which recognises its generated language.

\end{document}