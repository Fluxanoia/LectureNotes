\section{Non-deterministic Finite State Automaton}

A non-deterministic finite state automaton (NFA) is identical
to a DFA except our transition function is from 
$Q \times \Sigma_\epsilon \to \mathcal{P}(Q)$ where
$\Sigma_\epsilon$ is an alphabet $\Sigma$ with the empty word
added.
\\[\baselineskip]
Transitioning on the empty word doesn't consume a letter of our
input word and arbitrary choices are made by the automaton when
choices present themselves. We have that a word is accepted
in an NFA if and only if there is at least one computation where 
the word is accepted.

\subsection{Epsilon Closure}

For the NFA $M =  \langle Q, \Sigma, \delta, q_0, F \rangle$, 
and $R \subseteq Q$, we define the epsilon closure
of $R$ to be: \begin{gather*}
    E(R) := \left\{
        q \in Q : \bfrac{
        \text{ where there is a series of 
        transitions solely over }
    }{
        \epsilon \text{ from some } r \text{ in } 
        R \text{ to } q
    }\right\}
\end{gather*}

\subsection{Simulation via a DFA}

We can simulate an arbitrary NFA: \begin{gather*}
    M =  \langle Q, \Sigma, \delta, q_0, F \rangle
\end{gather*}
with a DFA: \begin{gather*}
    M' = \langle Q', \Sigma_\epsilon, \delta', q_0', F' \rangle
\end{gather*} where: \begin{align*}
    Q'   &= \mathcal{P}(Q), \\
    \delta'(q, a) &= \{q : \text{for some }
        r \in R, q \in E(\delta(r, a))
    \} \\
    q_0' &= E(\{q_0\}), \\
    F'   &= \{q' \in Q' : \text{for some } q \in q', q \in F\}.
\end{align*} Now that we have this, we know that languages are
regular if and only if they are accepted by some NFA as
all DFA are NFA and each NFA can be expressed by a DFA.