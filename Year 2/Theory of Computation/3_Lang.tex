\section{Languages}

For a DFA $M$, the language set of $M$ denoted by $L(M)$
is the maximal set of words in the alphabet of $M$ such that
for each $w$ in $L(M)$, $M$ accepts $w$. We say $M$ recognises
a language $A$ if $L(M) = A$.

\subsection{Regular Languages}

A language is regular if it is recognised by some DFA.

\subsubsection{Operations}

We can calculate the union and intersection of regular languages
as expected and for two DFA $M_1$ and $M_2$ with languages
$A$ and $B$ (resp.), we have that $A \cap B$ is
recognised by $M_1 \times M_2$ the product automaton.
\\[\baselineskip]
Additionally, we can concatenate two regular languages $A$ and
$B$: \begin{gather*}
    A \circ B = \{xy : x \in A \text{ and } y \in B\},
\end{gather*} and form the Kleene Star: \begin{gather*}
    A^* = \{x_0 \cdots x_k : k \in \mathbb{Z}_{\geq 0}
        \text{ and for each } i \in \{0, 1, \ldots, k\}, 
        x_i \in A\}.
\end{gather*} We have that each of these operations are closed
in the set of regular languages.

\subsubsection{Regular Expressions}

We have that $R$ is a regular expression over an alphabet
$\Sigma$ if it has one of the following shapes:
\begin{center}
    \begin{tabular}{ c l }
        $\emptyset$ & \\
        $\epsilon$ & \\
        $a$ & for some $a$ in $\Sigma$ \\
        $R_1 \cup R_2$ & for some regular expressions $R_1$ and $R_2$ \\
        $R_1 \circ R_2$ & for some regular expressions $R_1$ and $R_2$ \\
        $R^*$ & for some regular expression $R$ \\
    \end{tabular}
\end{center} The language of regular expressions $R_1$ and $R_2$
can be formed as follows: \begin{align*}
    L(\emptyset) &= \emptyset \\
    L(\epsilon) &= \{\epsilon\} \\
    L(a) &= \{a\} \\
    L(R_1 \cup R_2) &= L(R_1) \cup L(R_2) \\
    L(R_1 \circ R_2) &= L(R_1) \circ L(R_2) \\
    L(R_1^*) &= L(R_1)^*
\end{align*} We have that a language $L$ is regular
if and only if $L = L(R)$ for some regular expression $R$.