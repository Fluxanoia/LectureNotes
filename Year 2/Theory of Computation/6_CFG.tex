\section{Context-free Grammars}

We use context-free grammars to generate context-free langauages. 
A context-free grammar
is a 4-tuple $G = \langle V, \Sigma, R, S \rangle$ where: \begin{align*}
    V      &= \text{ the set of variables (non-terminals),}\\
    \Sigma &= \text{ the set of terminals, disjoint from } V, \\
    R      &= \text{ the set of rules},\\
    S      &\in V \text{ is the start variable}.
\end{align*} We have that each rule is a pair of a variable $A$ and a string $w$
which it maps to.
\\[\baselineskip]
By using the rules of the context-free grammar on the start variable, we can
generate a language from it: \begin{gather*}
    L(G) := \{w \in \Sigma^* : S \Rightarrow^* w \},
\end{gather*} where $\Rightarrow^*$ denotes some amount of applications
of the rules of $G$ onto $S$.

\subsection{Ambiguity}

A derivation of a string $w$ in a grammar $G$ is a
left-most derivation if, at every step in the derivation,
the left-most remaining variable is evaluated.
\\[\baselineskip]
A string $w$ is generated ambiguously if it has more than
one unique left-most derivation under $G$. $G$ is ambiguous
if it generates some string ambiguously.

\subsection{Push-down Automata}

A push-down automaton is a 6-tuple
$M = \langle Q, \Sigma, \Gamma, \delta, q_0, F \rangle$ where:
\begin{align*}
    Q      &= \text{ the set of states,} \\
    \Sigma &= \text{ any alphabet,} \\ 
    \Gamma &= \text{ any alphabet, called the stack alphabet,} \\ 
    \delta &\in \big\{ 
        (Q \times (\Sigma \cup \epsilon) \times (\Gamma \cup \epsilon))
        \to \mathcal{P}(Q \times (\Gamma \cup \epsilon))
        \big\} \text{ is the the transition function,} \\
    q_0    &\in Q \text{ is the initial state,} \\
    F      &\subset Q \text{ is the set of accept states,}
\end{align*} 

\newpage
\noindent
We say that that for a word $w = w_1 \cdots w_n$,
$M$ accepts $w$ if for a corresponding series of states
$p_0, \ldots, p_n$ and strings $s_0, \ldots, s_n$ in $\Gamma^*$: 
\begin{itemize}
    \item $p_0 = q_0$,
    \item $s_0 = \epsilon$,
    \item $p_n \in F$,
    \item for each $i$ in $[n - 1]$ and for some $x, y$ in $(\Gamma \cup \epsilon)$: 
    \begin{itemize}
        \item $(p_{i + 1}, x) \in \delta(p_{i}, w_{i + 1}, y)$,
        \item $s_i$ is of the form $ys$ and 
        $s_{i + 1}$ is of the form $xs$ for some $s$ in $\Gamma^*$.
    \end{itemize} 
\end{itemize} The language recognised by $M$ is the set of words accepted by it.

\subsubsection{Equivalence of PDA to CFG}

Each PDA has a corresponding CFG which generates its language.
Similarly, each CFG has a PDA which recognises its generated language.