\section{Context-free Grammars}

We use context-free grammars to generate langauages. A context-free grammar
is a 4-tuple $G = \langle V, \Sigma, R, S \rangle$ where: \begin{align*}
    V      &= \text{ the set of variables (non-terminals),}\\
    \Sigma &= \text{ the set of terminals, disjoint from } V, \\
    R      &= \text{ the set of rules},\\
    S      &\in V \text{ is the start variable}.
\end{align*} We have that each rule is a pair of a variable $A$ and a string $w$
which it maps to.
\\[\baselineskip]
By using the rules of the context-free grammar on the start variable, we can
generate a language from it: \begin{gather*}
    L(G) := \{w \in \Sigma^* : S \Rightarrow^* w \},
\end{gather*} where $\Rightarrow^*$ denotes some amount of applications
of the rules of $G$ onto $S$.

\subsection{Ambiguity}

A derivation of a string $w$ in a grammar $G$ is a
left-most derivation if, at every step in the derivation,
the left-most remaining variable is evaluated.
\\[\baselineskip]
A string $w$ is generated ambiguously if it has more than
one unique left-most derivation under $G$. $G$ is ambiguous
if it generates some string ambiguously.