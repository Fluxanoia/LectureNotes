\section{Rank and Determinants}

\subsection{Elementary Row Operations}

For a field $K$, take $A$ in $M_{m,n}(K)$. For some
$c$ in $K$, the elementary
row operations are: \begin{itemize}
    \item Swapping,
    \item Multiplying a row by $c \neq 0$,
    \item Adding $c$ multiples of one row to another.
\end{itemize}

\subsubsection{Elementary Matrices}

The $n \times n$ elementary matrices are: \begin{itemize}
  \item $E_1(i, j)$ : obtained by swapping the $i^{th}$ and $j^{th}$ rows
  of the identity
  \item $E_2(c, i)$ : obtained by scaling the $i^{th}$ row of the identity
  by $c$ non-zero
  \item $E_3(c, i, j)$ : obtained by adding $c$ times row $i$ to row $j$
  where $i \neq j$.
\end{itemize} We have that any elementary row operation can be realised as
left-multiplication by a corresponding elementary matrix. As a consequence
of the definition, we have that elementary matrices are invertible and
have elementary inverses.

\subsubsection{Echelon Form}

A matrix $A$ is in echelon form if each row has the form: \begin{gather*}
  (0, \ldots, 0, 1, *, \ldots, *),
\end{gather*} where each row has more leading zeroes than the one above and
the first row has any amount of leading zeroes. Every matrix can be put in
this form via Gaussian elimination.

\subsubsection{Decomposition via Elementary Matrices}

For an $n \times n$ matrix $A$, there exists elementary matrices
$E_1, \ldots, E_k$ such that $E_1\cdots E_kA = B$ where:
\begin{gather*}
  B = \begin{cases}
    \text{the identity} & \text{if $A$ is invertible} \\
    \text{a matrix with a final row consisting of all zeroes} & \text{otherwise.}
  \end{cases}
\end{gather*}

\subsection{Rank}

For $A = (a_{ij})$ a matrix in $M_{m,n}(K)$, we denote its rows by
$A_{(1)}, \ldots, A_{(m)}$ and columns by $A^{(1)}, \ldots, A^{(n)}$. We
say: \begin{itemize}
  \item The row rank of $A$ is the dimension of the subspace of spanned by 
  $A_{(1)}^t, \ldots, A_{(m)}^t$ in $K^m$
  \item The column rank of $A$ is the dimension of the subspace of spanned by 
  \newline $A^{(1)}, \ldots, A^{(n)}$ in $K^n$.
\end{itemize} We have these are equal, so can generally refer to the rank 
of a matrix.
\\[\baselineskip]
If $E_1, \ldots, E_k$ are elementary matrices, we have that the rank of $A$
is equal to the rank of $E_1\cdots E_kA$. Similarly, matrices that are similar 
have the same rank.

\subsubsection{Rank of Matrices from Linear Maps}

For $A$ an $m \times n$ matrix on $K$, we can define a map 
$f : K^n \to K^m$ by $v \mapsto Av$. We have that
the rank of $A$ is the dimension of the image of $f$.
Thus, invertible $n \times n$ matrices have rank $n$.

\subsubsection{Partially Diagonalising Matrices}

For an $m \times n$ matrix $A$, there exists some: \begin{itemize}
    \item $p \times m$ matrix $P$,
    \item $n \times q$ matrix $Q$,
\end{itemize} such that $PAQ = D = (d_{ij})$ where: \begin{gather*}
    d_{ij} = \begin{cases}
        \delta_{ij} & \text{for } i \leq \Rank(A) \\
        0           & \text{otherwise,}
    \end{cases}
\end{gather*} the matrix with $\Rank(A)$ units on the diagonal.

\subsection{Permutations}

For some $n$ in $\mathbb{Z}_{> 0}$, a permutation of $[n]$ is a bijection
$\sigma : [n] \to [n]$. We define the set of all permutations on $[n]$ as
$S_n$.

\subsection{Properties of $S_n$}

For some $n$ in $\mathbb{Z}_{> 0}$, $S_n$ is: \begin{itemize}
    \item A group under function composition,
    \item Of order $n!$,
    \item Non-abelian for $n > 2$.
\end{itemize}

\subsection{Decomposition of permutations}

All permutations can be written as a product of disjoint cycles. Thus,
all permutations can be written as a product of transpositions.

\subsection{Parity of Permutations}

Even permutations are permutations that can be expressed as the
product of an even number of transpositions. Otherwise, a
permutation is odd. 

\subsubsection{The Signature}

We define the sgn function for a given
permutation $\sigma$: \begin{gather*}
  \text{sgn}(\sigma) = \begin{cases}
    1 & \text{if } \sigma \text{ is even} \\
    -1 & \text{otherwise.}
  \end{cases}
\end{gather*} We have that for another permutation $\tau$: \begin{gather*}
  \text{sgn}(\sigma\tau) = \text{sgn}(\sigma)\text{sgn}(\tau).
\end{gather*} In other words, sgn is a homomorphism from $S_n$ to $\{1, -1\}$.

\subsubsection{The Alternating Group}

We have that $A_n$ the set of even permutations in $S_n$ is a
subgroup as it is the kernel of sgn.

\subsection{Determinants}

For $A = (a_{ij})$ a $n \times n$ matrix over $K$, we have
the determinant is a scalar defined by: \begin{gather*}
  \text{det}(A) := \sum_{\sigma \in S_n} \text{sgn}(\sigma)a_{\sigma(1)1} \cdots a_{\sigma(n)n}.
\end{gather*} A more pratical but equivalent definition would be: \begin{gather*}
  \text{det}(A) = \sum_{j = 1}^n (-1)^{i + j}a_{ij}\text{det}(\tilde{A}^{ij}),
\end{gather*} where $i$ is in $[n]$ and $\tilde{A}^{ij}$ is $A$
with the $i^{th}$ row and $j^{th}$ column removed. 
We may represent the $k^{th}$ column vector of $A$ by 
$A^{(k)}$ and then write: \begin{gather*}
  \text{det}(A) = \text{det}(A^{(1)}, \ldots, A^{(n)}),
\end{gather*} as a function on the columns of $A$.

\subsubsection{Multi-linearity of the Determinant}

For $A = (a_{ij})$ a $n \times n$ matrix over $K$ with
$A^{(k)} = c_1v_1 + c_2v_2$ with $c_1, c_2$ in $K$ and
$v_1, v_2$ in $K^n$ for some $k$ in $[n]$, we have that: \begin{align*}
  \text{det}(A) &= \text{det}(A^{(1)}, \ldots, A^{(n)}) \\
  &= c_1 \cdot \text{det}(A^{(1)}, \ldots, v_1, \ldots, A^{(n)}) \\
  & \qquad + c_2 \cdot \text{det}(A^{(1)}, \ldots, v_2, \ldots, A^{(n)}).
\end{align*} From this we can show that any square matrix with a column
of all zeroes has zero determinant.

\subsubsection{Alternativity of the Determinant}

For $A = (a_{ij})$ a $n \times n$ matrix over $K$ with
$i \neq j$, we have that: \begin{gather*}
  \text{det}(A^{(1)}, \ldots, A^{(i)}, \ldots, A^{(j)} \ldots, A^{(n)}) \\
  =\\
  -\text{det}(A^{(1)}, \ldots, A^{(j)}, \ldots, A^{(i)} \ldots, A^{(n)}).
\end{gather*} From this we can show that a matrix with a pair of
identical columns must have zero determinant.

\subsubsection{Normality of the Determinant}

For a square upper (or lower) triangular matrix $A$, we have that
the determinant is the product of the diagonal entries.
\\[\baselineskip]
From this we can show that the determinant of the identity is $1$.

\subsubsection{The Determinants of Elementary Matrices}

We have the following: \begin{align*}
  \text{det}\left[E_1(i, j)\right] &= -1 \\
  \text{det}\left[E_2(c, i)\right] &= c \\
  \text{det}\left[E_3(c, i, j)\right] &= 1.
\end{align*}

\subsubsection{The Determinant of the Transpose}

For a matrix $A$, we have that: $\text{det}(A) = \text{det}(A^t).$

\subsubsection{The Determinant under Matrix Multiplication}

For $A, B$ two $n \times n$ matrices, we have that:
$\text{det}(AB) = \text{det}(A)\cdot\text{det}(B).$

\subsubsection{The Determinant and Invertibility}

A square matrix is invertible if and only if it has non-zero determinant.