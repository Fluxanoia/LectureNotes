\section{Jordan}

\subsection{Jordan Blocks}

For a field $K$, $h$ in $\mathbb{Z}_{> 0}$, a Jordan block of size 
$h \times h$
on $\lambda$ in $K$ is the matrix of the form: \begin{gather*}
  J_h(\lambda) = \begin{pmatrix}
    \lambda & 1       & 0      & \cdots  & 0       \\
    0       & \lambda & 1      & \ddots  & \vdots  \\
    \vdots  & \ddots  & \ddots & \ddots  & 0       \\
    \vdots  & \ddots  & \ddots & \lambda & 1       \\
    0       & \cdots  & \cdots & 0       & \lambda
  \end{pmatrix},
\end{gather*} or alternatively: \begin{gather*}
  J_h(\lambda) = (a_{ij}), \, a_{ij} = \begin{cases}
    \lambda & i = j \\
    1       & j = i + 1 \\
    0       & \text{otherwise.}
  \end{cases}
\end{gather*}

\subsubsection{Jordan Matrices}

For a field $K$, a Jordan matrix consisting of Jordan blocks of sizes 
$\{h_1, \ldots, h_n\}$ in $\mathbb{Z}_{>0}$ and values 
$\{\lambda_1, \ldots, \lambda_n\}$ in $K$ has the form: \begin{gather*}
  J = \begin{pmatrix}
    J_{h_1}(\lambda_1) & 0                  & \cdots & \cdots & 0      \\
    0                  & J_{h_2}(\lambda_2) & \ddots &        & \vdots \\
    \vdots             & \ddots             & \ddots & \ddots & \vdots \\
    \vdots             &                    & \ddots & \ddots & 0      \\
    0                  & \cdots             & \cdots & 0      & J_{h_n}(\lambda_n)
  \end{pmatrix}
\end{gather*}

\subsubsection{Jordan Normal Form}

A Jordan normal form of a matrix $A$ is a Jordan matrix that is similar to $A$.

\subsection{Jordan Bases}

For an algebraically closed field $K$, a finite dimensional vector space $V$ over
$K$, with $f$ in $\mathcal{L}(V, V)$, we have that a basis $B$ of $V$ is a 
Jordan basis for $f$ if $M_{BB}(f)$ is a Jordan matrix.

\subsubsection{Existence of Jordan Bases}

Let $K$ be an algebraically closed field, $V$ be a finite dimensional 
vector space over $K$, $f$ be in $L = \mathcal{L}(V, V)$. There exists a Jordan
basis for $f$ and the Jordan normal form of $f$ is unique up to permutations
of the Jordan blocks.
\begin{proof}
    We first suppose $f$ is nilpotent, so for some $t$ in 
    $\mathbb{Z}_{\geq 0}$, $f^t = 0_L$. We know that $V = V(0)$
    as $f$ has a single eigenvalue, zero. Take $V_0 = E(0)$
    and $V_i$ for $i$ in $[t]$ to be defined as follows:
    \begin{gather*}
        V_i = \Ima(f^i) \cap V_0,
    \end{gather*} so $V_i$ is the eigenvectors of $f^i$ (specifically
    the ones in $E(0)$). Take $v$ in $V_i$, 
    there exists some $v'$ such that $f^i(v') = v$ 
    as $v$ is in $\Ima(f^i)$. So: \begin{align*}
        v &= f^i(v') \\
        &= f^{i - 1}(f(v')),
    \end{align*} meaning that $v$ is in $\Ima(f^{i - 1})$. But, we also
    have that $v$ must be in $V_0$ by the definition of $V_i$ so
    $v$ is in $V_{i - 1}$. Hence, $V_i \subseteq V_{i - 1}$ and we 
    can say that: \begin{gather*}
        \{0_V\} \subseteq V_t \subseteq \cdots \subseteq V_1 \subseteq V_0.
    \end{gather*} Let $h : E(0) \to \mathbb{Z}_{\geq 0}$ be the height
    function.
    Consider $B_{t - 1} = \{e_1, \ldots, e_{b_1}\}$ a basis
    for $V_{t - 1}$. For each $b$ in $[b_1]$, let $h(e_b) = t - 1$ and
    choose $v_b^{h(e_b)} = v_b^{t - 1}$ in $V$ such that: \begin{gather*}
        f^{h(e_b)}(v_b^{h(e_b)}) = e_b.
    \end{gather*} We can choose such a value in $V$ as $V_{t - 1} = V_{h(e_b)}$
    is in the image of $f^{h(e_b)}$. Now, for $k$ in $[h(e_b)]$, we let: \begin{gather*}
        v_b^{k - 1} = f(v_b^k).
    \end{gather*} so that: \begin{align*}
        v_b^{t - 2} &= f(v_b^{t - 1}) \\
        v_b^{t - 3} &= f^2(v_b^{t - 1}) \\
        &\cdots \\
        v_b^1 &= f^{t - 2}(v_b^{t - 1}) \\
        e_b = v_b^0 &= f^{t - 1}(v_b^{t - 1}).
    \end{align*}
    We extend $B_{t - 1}$ to $B_{t - 2}$
    (a basis for $V_{t - 2}$) with new basis elements: \begin{gather*}
        B_{t - 2} = \{e_1, \ldots, e_{b_1}, e_{b_1 + 1}, \ldots, e_{b_2}\}.
    \end{gather*} 
    Similarly to the above, take $b$ in $\{b_1 + 1, \ldots, b_2\}$ and
    let $h(e_b) = t - 2$ and choose $v_b^{t - 2}$ such that
    $f^{t - 2}(v_b^{t - 2}) = e_b$. We then define for $k$ in $[h(e_b)]$: 
    \begin{gather*}
        v_b^{k - 1} = f(v_b^k).
    \end{gather*} so that: \begin{align*}
        v_b^{t - 3} &= f(v_b^{t - 2}) \\
        v_b^{t - 4} &= f^2(v_b^{t - 2}) \\
        &\cdots \\
        v_b^1 &= f^{t - 3}(v_b^{t - 2}) \\
        e_b = v_b^0 &= f^{t - 2}(v_b^{t - 2}).
    \end{align*} We continue this for all $V_{t - 3}, \ldots, {V_0}$.
    We take $B_0 = V_0 = \{e_1, \ldots, e_n\}$ to be our basis 
    extended to $V_0$ noting that $n = b_1 + \cdots + b_t = \Dim(V_0)$. 
    We have: \begin{gather*}
        B = \{v_i^j : 
        \underbrace{1 \leq i \leq n}_\text{stack number}, 
        \overbrace{0 \leq j \leq h(e_i)}^\text{stack height}
        \},
    \end{gather*} our Jordan basis. 
    \\[\baselineskip]
    We want to show linear independence
    of $B$, so we take $\{a_i^j \in K : 1 \leq i \leq n, 0 \leq j \leq h(e_i)\}$
    such that: \begin{gather} \label{jordanlinindep}
        \sum_{i = 1}^{n} \sum_{j = 0}^{h(e_i)} a_i^j v_i^j = 0_V.
    \end{gather} We first notice that for each $v_i^j$, $f^{j}(v_i^j) = e_i$
    and so $f^{j + 1}(v_i^j) = 0_V$. Thus, applying $f^{t - 1}$ to both
    sides of (\ref{jordanlinindep}) we get a linear combination
    of basis vectors (specifically in $B_{t - 1}$) meaning each
    element of $\{a_i^{t - 1} : 1 \leq i \leq n\}$ must be zero.
    Thus, taking $0 \leq j_0 < j \leq t - 1$, where all elements
    of $\{a_i^j : 1 \leq i \leq n\}$ are zero. By applying
    $f^{j_0}$ to both sides of (\ref{jordanlinindep}) we see 
    we have a linear combination of vectors in $V_{j_0}$ with 
    coefficients $\{a_i^{j_0} : 1 \leq i \leq n\}$ necessarily
    zero. So, by induction, all the coefficients are zero 
    and $B$ is linearly independent.
    \\[\baselineskip]
    We want to show that $B$ spans $V$, remembering that $V = V(0)$,
    we take $v$ in $V$ (necessarily a root vector), and has some
    height $h_v \leq t$. Suppose $h_v = 1$, $B$ contains a basis for
    $V_0$ so $v$ is in the span of $B$. Suppose $h_v = k + 1$ and
    for each vector $v'$ of height less than or equal to $k$ we have 
    that $v'$ is in the span of $B$. We have that $f^k(v)$ is in
    $V_k$ as it's in $\Ima(f^k)$ and $f(f^k(v)) = 0_V$ so is in $V_0$.
    Thus, considering $V_k \cap B = \{e_1, \ldots, e_j\}$ a basis
    for $V_k$, we have $a_1, \ldots, a_j$ in $K$ such that: \begin{gather*}
        f^k(v) = \sum_{i = 1}^j a_i e_i.
    \end{gather*} Considering that each $e_i = f^k(v_i^k)$: \begin{align*}
        f^k(v) &= \sum_{i = 1}^j a_i e_i \\
        &= \sum_{i = 1}^j a_i f^k(v_i^k) \\
        &= f^k \left( \sum_{i = 1}^j a_i v_i^k \right).
    \end{align*} So, we can deduce that $v$ is in the span of $B$. Hence,
    $B$ is a Jordan basis.
    \\[\baselineskip]
    To show uniqueness, the number of $m \times m$ Jordan blocks
    is: \begin{gather*}
        \Dim(\Ima(f^{m - 1}) \cap \Ker(f))) - \Dim(\Ima(f^m) \cap \Ker(f)),
    \end{gather*} which is uniquely defined.
    \\[\baselineskip]
    Finally, we take $f$ to be any linear map. We consider
    $\lambda_1, \ldots, \lambda_k$ to be the distinct eigenvalues of $f$.
    By the Primary Decomposition Theorem, we have that: \begin{gather*}
        V = \bigoplus_{i = 1}^k V( \lambda_i).
    \end{gather*} For each $i$ in $[k]$, by the properties of the
    root space, $V(\lambda_i)$ is $f$-invariant. So, we consider
    $f_i : V(\lambda_i) \to V(\lambda_i); v \mapsto f(v)$.
    We have that $(f_i - \lambda_i(\ID))$ has one eigenvalue,
    zero. By using the Cayley-Hamilton theorem we can see that: \begin{gather*}
        (f_i - \lambda_i(\ID))^{\Dim(V(\lambda_i))} = 0_{\mathcal{L}(V, V)},
    \end{gather*} so $(f_i - \lambda_i(\ID))$ is nilpotent. By the above,
    $(f_i - \lambda_i(\ID))$ has a Jordan basis $B_i$. Putting this together,
    $B = \bigcup_{i = 1}^k B_i$ is a Jordan basis for $f$.
\end{proof}

\subsubsection{Relation to Eigenvalue Multiplicity}

For a given eigenvalue: \begin{itemize}
    \item The geometric multiplicity is the number of 
    Jordan blocks with the eigenvalue,
    \item The algebraic multiplicity is the sum of the sizes of all the Jordan blocks
    corresponding to the eigenvalue,
    \item The algebraic multiplicity in $m_f$ is the maximum size of a
    Jordan block corresponding to the eigenvalue.
\end{itemize}

\subsubsection{Computing Jordan Bases}

The process is as follows: \begin{itemize}
    \item Compute the characteristic polynomial and factorise it,
    \item For each eigenvalue, find the eigenspaces,
    \item If the geometric multiplicity of an eigenvalue is 
    less than the algebraic multiplicity then the difference
    is the number of root vectors that need to be found.
\end{itemize}
