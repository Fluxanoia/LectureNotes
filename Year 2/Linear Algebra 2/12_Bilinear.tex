\section{Bilinear and Quadratic Forms}

\subsection{Definition of a Bilinear Form}

For $V$ a vector space over a field $K$, a bilinear form on $V$ is a map $\langle\,,\rangle : V \times V \to K$
such that: \begin{align*}
  \langle au + bv, w \rangle &= a \cdot \langle u, w \rangle + b \cdot \langle v, w \rangle \\
  \langle u, av + bw \rangle &= a \cdot \langle u, v \rangle + b \cdot \langle u, w \rangle,
\end{align*} for all $a, b$ in $K$, $u, v, w$ in $V$. Additionally, $\langle\,,\rangle$ is symmetric if 
$\langle u, v \rangle = \langle v, u \rangle$.

\subsection{Definition of a Quadratic Form}

For $V$ a vector space over a field $K$ with $\langle \, , \rangle$ a symmetric bilinear form on
$V$. The quadratic form $Q : V \to K$ associated to $\langle \, , \rangle$ is 
$Q(v) := \langle v, v \rangle$.

\subsubsection{Determining bilinear forms from quadratic forms}

We have that if char$(K) \neq 2$, then $\langle \, , \rangle$ is uniquely defined by $Q$ as:
\begin{gather*}
  \langle v, w \rangle = 2^{-1}\left[Q(v + w) - Q(v) - Q(w) \right]. 
\end{gather*} 

\subsection{Definition of Orthogonality}

Let $\langle \, , \rangle$ be a bilinear form on $V$ with $v$ in $V$. We say that $u$ in $V$ is 
orthogonal to $v$ if $\langle v, u \rangle = 0$. Note, be very careful as the bilinear is not
necessarily symmetric.

\subsubsection{Definition of orthogonal spaces}

For $W \subseteq V$, we have that $W^\perp$ is defined as: \begin{gather*}
  W^\perp = \{v \in V : w \in W, \langle w, v \rangle = 0\},
\end{gather*} the set of vectors such that for all $v$ in $W^\perp$, $v$ is orthogonal
to all of $W$. This is a subspace of $V$.

\subsubsection{Definition of the kernel for bilinear maps}

The kernel of $\langle \, , \rangle$ is $V^\perp$. If the kernel is $\{0_V\}$,
then the form is called non-degenerate and is called degenerate otherwise.

\subsubsection{Dimension and orthogonal spaces}

We have that if $V$ is finite dimensional and $\langle \, , \rangle$ is non-degenerate
then: \begin{gather*}
  \text{dim}(W^\perp) + \text{dim}(W) = \text{dim}(V).
\end{gather*}

\subsection{Linear Maps from Bilinear Forms}

We can form a linear map $f : V \to V^*$ from a bilinear form $\langle \, , \rangle$ 
as follows: \begin{gather*}
  f(v)(u) = \langle u, v \rangle.
\end{gather*} We have that a bilinear form is non-degenerate if and only if its corresponding
linear map is an isomorphism.

\subsubsection{Isomorphismic Bilinear Maps}

If $V$ is finite dimensional, we have that $f$ is an isomorphism if and only if 
$\langle \, , \rangle$ is non-degenerate. That is, $\text{Ker}(f) 
= \text{Ker}(\langle \, , \rangle) = V^\perp$.

\subsection{Matrices from Bilinear Forms}

For $V$ a finite $n$-dimensional vector space over $K$ with $S = \{v_1, \ldots, v_n\}$
an ordered basis for $V$. Let $B = \langle \, , \rangle$ be a bilinear form. The matrix corresponding
to $B$ with respect to $S$ is $M_{SS}(B) = (b_{ij})$ where: \begin{gather*}
  b_{ij} = \langle v_i, v_j \rangle.
\end{gather*} Similarly, taking $S^* = \{v^*_1, \ldots, v^*_n\}$ to be a dual basis for
$S^*$, we have a matrix $M_{SS^*}(f) = M_{SS}(B)$ for the linear map corresponding to $B$.

\subsubsection{Determining bilinear forms from matrices}

Take $u, v$ in $V$ decomposed into vectors in $S$ with coefficients $x_1, \ldots, x_n$
and $y_1, \ldots, y_n$ respectively. Thus: \begin{gather*}
  \langle u, v \rangle = (x_1, \ldots, x_n) \cdot M_{SS}(B) \cdot \begin{pmatrix}
    y_1 \\ \vdots \\ y_n
  \end{pmatrix}.
\end{gather*}

\subsubsection{Properties of matrices of bilinear forms}

We have that: \begin{itemize}
  \item If $M_{SS}(B)$ is symmetric, so is $B$
  \item $B$ is non-degenerate if and only if $M_{SS}(B)$ is invertible.
\end{itemize}

\subsection{Similarity of Matrices of Bilinear Forms}

For $V$ a finite $n$-dimensional vector space over the field $K$ with char$(K) \neq 2$, 
let $S = \{v_1, \ldots, v_n\}$, $S' = \{v_1', \ldots, v_n'\}$ be
ordered bases for $V$. Let $B = \langle \, , \rangle$ be a symmetric bilinear form.
Let $C = C_{SS'}$ be the transition matrix. We have that: \begin{gather*}
  M_{S'S'}(B) = C^tM_{SS}(B)C.
\end{gather*}

\subsection{Diagonal Matrices of Bilinear Forms}

For $V$ a finite $n$-dimensional vector space over the field $K$ with char$(K) \neq 2$, 
let $B = \langle \, , \rangle$ be a symmetric bilinear form.
There exists a basis $S = \{v_1, \ldots, v_n\}$ for $V$ consisting of pairwise orthogonal
vectors and thus, the matrix $M_{SS}(B)$ is diagonal.

\subsection{Definition of an Inner Product}

For a vector space $V$ over $K$ with symmetric bilinear form $B : V \times V \to K$, we
have that $B$ is an inner product if for all $v$ in $V$: \begin{gather*}
  B(v, v) \geq 0,
\end{gather*} and $B(v, v) = 0$ if and only if $v = 0_V$.