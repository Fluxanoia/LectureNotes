\section{Vector Spaces}

A vector space over a field $K$ is a set $V$ with an addition
operation $+ : V \times V \to V$ and a scalar multiplication operation
$\circ : K \times V \to V$ such that for all $a, b$ in $K$
and $v, w$ in $V$: \begin{itemize}
  \item $(V, +)$ is an abelian group,
  \item $1_K \circ v = v$,
  \item $(ab) \circ v = a \circ (b \circ v)$
  \item $(a + b) \circ v = a \circ v + b \circ v$
  \item $a \circ (v + w) = a \circ v + a \circ w$.
\end{itemize}

\subsection{Subspaces}

For $V$ a vector space over the field $K$ and $W$ a set, $W$ is a 
subspace of $V$ if and only if it is a subset of $V$ and is a vector 
space with respect to the addition and scalar multiplication defined by $V$.
It is sufficient to verify that $W$ is closed under addition and multiplication.

\subsection{Linear Combinations of Vectors}

For a set $V$ with addition operation $+$, a field $K$ and $n$ in 
$\mathbb{N}$, a linear combination of $v_1, \ldots, v_n$ in $V$ is: 
\begin{gather*}
  \sum_{i = 1}^n a_i \cdot v_i = a_1 \cdot v_1 + \cdots + a_n \cdot v_n,
\end{gather*} for some $a_1, \ldots, a_n$ in $K$. Such a combination is
trivial if each of $a_1, \ldots, a_n$ are zero and non-trivial otherwise.

\subsection{Linear Independence}

For a vector space $V$ and $W \subseteq V$, we say $W$ is linearly independent 
if there does not exists a non-trivial linear combination of all the vectors in 
$W$ equal to zero (and linearly dependent otherwise).

\subsubsection{Properties of Linear Independence}

For a vector space $V$ with $W \subseteq V$: \begin{itemize}
  \item $W$ is linearly dependent if it contains $0_V$,
  \item If $W$ linearly independent, any subset of it is also
  linearly independent,
  \item If there's a linearly dependent subset of $W$, then $W$
  is linearly dependent.
\end{itemize}

\subsection{The Span of a Set}

For a set $V$ with addition operation $+$ and a field $K$, the span 
of $W \subseteq V$ is the set of all the linear combinations of the values
in $W$ denoted by $\Span(W)$.

\subsection{Bases}

For a vector space $V$ with $W \subseteq V$, if $W$ is linearly independent
and $\Span(W) = V$, we say that $W$ is a basis of $V$. It is a minimal
spanning set.

\subsubsection{Properties of Bases}

We have that if a basis is finite, all other bases have the same size.
Additionally, saying $W$ is a basis is equivalent to saying that each vector 
in $V$ can be \textbf{uniquely} written as a linear combination of vectors 
in $W$.

\subsection{Dimension}

For a vector space $V$ with a finite basis, we say that the size of the 
basis is the dimension of $V$ denoted by $\Dim(V)$.
By convention, $\Dim(\{0_V\}) = 0$. Vector spaces with identical
dimension are isomorphic.

\subsubsection{Dimension and Subsets}

For $V \neq \{0\}$ a vector space, if there is a
finite spanning set $S$ of $V$ then: \begin{itemize}
  \item $V$ is finite dimensional, particularly, there is a basis $B$ of $V$
  where $B \subseteq S$,
  \item For $X \subseteq V$ such that $X$ is linearly independent, $X$ can
  be extended to a basis of $V$,
  \item All subspaces of $V$ are finite-dimensional.
\end{itemize}