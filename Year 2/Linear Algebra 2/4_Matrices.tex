\section{Matrices}

Let $m, n$ be in $\mathbb{Z}_{>0}$ and let $K$ be a field. An
$m \times n$ matrix with entries in $K$ is a map
$M : [m] \times [n] \to K$, more commonly written as 
$M = (a_{ij})$ representing the rectangular array of values held
by $M$.

\subsection{Types of Matrices}

For $m, n$ in $\mathbb{Z}_{>0}$ and $K$ a field, let $M$
be in $M_{m \times n}(K)$. We have the following types of
matrix: \begin{itemize}
  \item \textbf{Square}: where $m = n$
  \item \textbf{Upper Triangular}: if $a_{ij} = 0$ for $i > j$
  \item \textbf{Lower Triangular}: if $a_{ij} = 0$ for $i < j$
  \item \textbf{Diagonal}: if $a_{ij} = 0$ for $i \neq j$
  \item \textbf{Symmetric}: if $a_{ij} = a_{ji}$
  \item \textbf{Anti-symmetric}: if $a_{ij} = -a_{ji}$.
\end{itemize}

\subsection{The Space of Matrices}

For $m, n$ in $\mathbb{Z}_{>0}$ and $K$ a field, we define
the set of all $m \times n$ matrices over $K$ by $M_{m \times n}(K)$.
We have that $M_{m \times n}(K)$ is a vector space over $K$
where matrices are added and multiplied by scalars component-wise.
So, for $M_1 = (a_{ij}), M_2 = (b_{ij})$ in $M_{m\times n}(K)$
and $c$ in $K$ we have:
\begin{align*}
  cM_1 &= (ca_{ij}) \\
  M_1 + M_2 &= (a_{ij} + b_{ij}).
\end{align*} Additionally, the zero vector is $M_0 = (0_K)$, the 
multiplicative identity is the diagonal matrix of all $1_K$'s and, the vector
space has a basis consisting of $M_{ij}$ where all entries are zero
except the $(i, j)^{\text{th}}$ entry. This leads to the conclusion
that the dimension is $mn$ and thus that $M_{m \times n} \cong K^{mn}$.

\newpage

\subsection{Matrix Multiplication}

For $a, b, c$ in $\mathbb{Z}_{>0}$ and a field $K$, 
we can define the multiplication of the two matrices $X = (x_{ij})$ in
$M_{a \times b}(K)$ and $Y = (y_{ij})$ in $M_{b \times c}(k)$ as follows: 
\begin{gather*}
  XY := (\sum_{k = 1}^b x_{ik}y_{kj}).
\end{gather*} This operation is not commutative in general but is
associative.
\\[\baselineskip]
For $A, B$ in $M_n(K)$, we have that $AB$ is also in $M_n(K)$. This,
along with matrix addition, makes $M_n$ a ring with unity with
multiplicative identity $I_n := (\delta_{ij})$. However, there exists
non-zero $A, B$ in $M_n$ such that $AB=0$ so, $M_n$ is not a field.

\subsection{Matrices of Linear Maps}

For $V, W$ vector spaces over a field $K$, 
we have $A = \{v_1, \ldots, v_n\}$, $B = \{w_1, \ldots, w_n\}$ 
ordered bases for $V$ and $W$ respectively.
Given $f$ in $\mathcal{L}(V, W)$, the matrix associated to $f$
(with respect to the bases $A$ and $B$) is the $m \times n$ matrix:
\begin{gather*}
  M_{BA}(f) = (a_{ij}),
\end{gather*} where we define $a_{ij}$ by: \begin{gather*}
  f(v_j) = \sum_{i = 1}^m a_{ij}w_i,
\end{gather*} for each $j$ in $[n]$.

\subsubsection{Matrices of Composed Linear Maps}

For $U, V, W$ vector spaces over a field $K$, for some
$l, m, n$ in $\mathbb{Z}_{>0}$ we have $A = \{u_1, \ldots, u_l\}$,
$B = \{v_1, \ldots, v_m\}$, $C = \{w_1, \ldots, w_n\}$ bases for 
$U, V, W$ respectively. Given $f$ in $\mathcal{L}(U, V)$, $g$ 
in $\mathcal{L}(V, W)$, we have:
\begin{gather*}
  M_{CA}(g \circ f) = M_{CB}(g)M_{BA}(f).
\end{gather*}

\subsection{Transition Matrices}

For a finite $n$-dimensional vector space $V$ with bases $A, A'$ and 
the $n$-dimensional identity $I$, we call $M_{A'A}(I) = C_{A'A}$ the transition
matrix from $A$ to $A'$. 
We have that $C_{A'A}$ is invertible and $C_{A'A}^{-1} = C_{AA'}$.

\subsection{Matrix Transitions}

For a finite-dimensional vector space $V$ with bases 
$A, B$ and $f : V \to V$ a linear operator: \begin{align*}
  M_{BB}(f) &= C_{AB}^{-1}M_{AA}(f)C_{AB} \\
  &= C_{BA}M_{AA}(f)C_{AB}.
\end{align*}

\subsection{Similar Matrices}

For matrices $A', A$, we say that $A'$ and $A$ are similar if there
exists an invertible matrix $C$ such that: \begin{gather*}
  A' = C^{-1}AC.
\end{gather*} This is denoted by $A' \sim A$. Similarity forms
an equivalence relation on the space of square matrices.
If we have $A \sim A'$ and $A$ represents some linear operator $f$
for some basis $B$, then we have that for some basis $B'$, $f$ has
matrix $A'$.
