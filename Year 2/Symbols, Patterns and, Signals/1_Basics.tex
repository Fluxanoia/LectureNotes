\section{Data Acquisition}

\subsection{Analogue to Digital Conversion}

There are two steps to this conversion, sampling and quantisation.
They can be done in any order.

\subsubsection{Nyquist-Shannon Sampling Theorem}
If a function $f$ contains no frequencies higher than some
$h_{\text{max}}$ hertz, it is completely determined by
sampling at points spaced $\frac{1}{2 \cdot h_{\text{max}}}$
apart.

\section{Data Characteristics}

\subsection{Measures of Distance}

A valid distance measure $D : A \times A \to \mathbb{R}$
for some data set $A$ has the following properties, it is: \begin{itemize}
    \item non-negative,
    \item reflexive ($D(a, b) = 0 \Longleftrightarrow a = b$),
    \item symmetric,
    \item satisfies the triangle inequality
    ($D(a, b) + D(b, c) \geq D(a, c)$).
\end{itemize}

\subsubsection{Euclidean Distance in $\mathbb{R}^n$ ($p$-norm distance)}

For two vectors $x$ and $y$ in $\mathbb{R}^n$, we have the Euclidean
distance $D$ is: \begin{gather*}
    D(x, y) := \sqrt[\uproot{2}p]{\sum_{i = 1}^n |x_i - y_i|^p}.
\end{gather*}

\subsubsection{Chebyshev Distance in $\mathbb{R}^n$ ($\infty$-norm distance)}

For two vectors $x$ and $y$ in $\mathbb{R}^n$, we have the Chebyshev
distance $D$ is: \begin{gather*}
    D(x, y) := \lim_{n \to \infty} \left(
        \sqrt[\uproot{2}p]{\sum_{i = 1}^n |x_i - y_i|^p}
    \, \right) = \max_{i \in [n]}(|x_i - y_i|).
\end{gather*}

\subsubsection{Time Series Distance}

Finding the distance between two time series $x$ and $y$
of length $n$ and $m$ (resp.) can be found using Dynamic Time 
Warping: \begin{gather*}
    D_{tw}(x, y) := D(x_1, y_1) + 
    \min\{
        D_{tw}(x, y'),
        D_{tw}(x', y),
        D_{tw}(x', y')
    \},
\end{gather*} where $D$ is some numerical distance measure and
$x'$ and $y'$ are the time series length $n - 1$ and $m - 1$
(resp.) corresponding to $x$ and $y$ with the first element
removed.

\subsubsection{Hamming Distance}

When given two strings of the same length, the Hamming distance
between them is how many characters differ in the strings at each
index.

\subsubsection{Edit Distance}

When given two strings of any length, the edit distance between
them is the smallest number of insertions, substitutions and,
deletions that can transform one string into the other (or vice versa).

\subsubsection{Wu and Palmer Distance}

This measure is based on a hierarchy of word semantics, a
graph of relationships between words based on meaning.
Using the shortest distance between the words $d_1$ and the 
shortest distance from a most specific ancestor to the path
$d_2$ we have the distance measure: \begin{gather*}
    D(w_1, w_2) := \frac{2 \cdot d_2}{d_1 + 2 \cdot d_2} - 1.
\end{gather*}

\subsection{Summary Statistics}

\subsubsection{Mean}

We take $X = \{x_1, \ldots, x_n\}$ to be a data set.
The mean $\bar{X}$ is defined as follows: \begin{gather*}
    \bar{X} := \frac{1}{n}\sum_{i = 1}^n x_i.
\end{gather*}

\subsubsection{Standard Deviation and Variance}

We take $X = \{x_1, \ldots, x_n\}$ to be a data set.
We have that for $\sigma_X$, the standard deviation of $X$,
the variance of $X$ is $\sigma_X^2$. We define the variance
(and thus the standard deviation) as follows: \begin{gather*}
    \sigma_X^2 = \frac{1}{n - 1}\sum_{i = 1}^n(x_i - \bar{X})^2.
\end{gather*} 

\subsubsection{Covariance}

We take $X = \{x_1, \ldots, x_n\}$ to be a data set consisting
of $m$-dimensional row vectors. We define the covariance matrix: 
\begin{gather*}
    \Sigma = \frac{1}{n - 1}\sum_{i = 1}^n (x_i - \mu)^T(x_i - \mu),
\end{gather*} where $\mu$ is the $m$-dimensional row vector
where the $j^{\text{th}}$ entry corresponds to the mean of the
$j^{\text{th}}$ entry of each $x_i$. This yields a $m \times m$
matrix that is square and symmetric with the variance of the
$j^{\text{th}}$ entry of each $x_i$ on the $j^{\text{th}}$ value
on the diagonal.

\paragraph{Eigenvalues and Eigenvectors} As the matrix is symmetric
we can diagonalise it and find the eigenvalues and eigenvectors.
The major axis is the eigenvector corresponding to the largest
eigenvalue and the minor axis is the eigenvector corresponding
to the smallest value.

\subsubsection{Data Normalisation}

Data may need to be normalised before we use our distance measures
on it. We consider a data set $X = \{x_1, \ldots, x_n\}$
with mean $\mu$ and standard deviation $\sigma$.

\paragraph{Scaling} We map each $x_i$ for $i \in [n]$ as follows:
\begin{gather*}
    x_i \mapsto \frac{x_i - \min(X)}{\max(X) - \min(X)}.
\end{gather*}

\paragraph{Standardisation} We map each $x_i$ for $i \in [n]$ as follows:
\begin{gather*}
    x_i \mapsto \frac{x_i - \mu}{\sigma}.
\end{gather*}

\paragraph{Scaling to Unit Length} We map each $x_i$ for $i \in [n]$ as follows:
\begin{gather*}
    x_i \mapsto \frac{x_i}{|x_i|}.
\end{gather*} where $|x_i|$ denotes the magnitude of $x_i$.

\subsubsection{Outliers}

A small amount of values significantly different to the remainder
of the data set.