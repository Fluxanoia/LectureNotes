\section{Data Representation}

Data representation is about extracting
features from data.

\subsection{Preprocessing}

We perfom preprocessing to remove
noise, rectify missing values and, remove
redundancies and inconsistencies.
We do this by: \begin{itemize}
    \item Data cleaning, removing noise and
    inconsistent data,
    \item Data integration, combining data
    from multiple sources,
    \item Data selection, filtering and
    extracting the desired data,
\end{itemize} allowing us to transform
the data so it is ready for analysis.

\subsection{Digital Signal Processing}

Digital signal processing is about processing
and manipulating signals using digital
techniques.

\subsubsection{Shannon's Sampling Theorem}

An analogue signal of frequency at most 
$f_{\text{max}}$ should be sampled at 
$2 \cdot f_{\text{max}}$ to stop aliasing
of the reconstructed signal.

\subsubsection{Quantisation}

Quantisation levels are the values you
are assigning your input signal to in
order to represent it digitally.

\subsection{Fourier Series}

For some periodic function $f : \mathbb{R} \to \mathbb{R}$ with period $T$,
we can write $f$ as an infinite sum: \begin{gather*}
    f(x) = \sum_{n = 0}^\infty 
        a_n \cos{\left( \frac{2\pi n}{T} x \right)} +
        b_n \sin{\left( \frac{2\pi n}{T} x \right)},
\end{gather*} where $(a_n)_{n \in \mathbb{Z}_{\geq 0}}$ and 
$(b_n)_{n \in \mathbb{Z}_{\geq 0}}$ are sequences in $\mathbb{R}$.
Also, with some consideration of the properties of sin and cos, we can write
this as: \begin{gather*}
    f(x) = a_0 + \sum_{n = 1}^\infty 
        a_n \cos{\left( \frac{2\pi n}{T} x \right)} +
        b_n \sin{\left( \frac{2\pi n}{T} x \right)}.
\end{gather*} The solution for these coefficients is as follows: \begin{align*}
    a_k &= \frac{2}{T} \int_{-\frac{T}{2}}^{\frac{T}{2}} 
        f(x) \cdot \cos{\left( \frac{2\pi n}{T} x \right)} \, dx \\
    b_k &= \frac{2}{T} \int_{-\frac{T}{2}}^{\frac{T}{2}} 
        f(x) \cdot \sin{\left( \frac{2\pi n}{T} x \right)} \, dx,
\end{align*} or taking $\lambda$ to be our frequency ($\lambda = \frac{1}{T}$): 
\begin{align*}
    a_k &= 2\lambda \int_{-\frac{T}{2}}^{\frac{T}{2}} 
        f(x) \cdot \cos{(2\pi n\lambda x)} \, dx \\
    b_k &= 2\lambda \int_{-\frac{T}{2}}^{\frac{T}{2}} 
        f(x) \cdot \sin{(2\pi n\lambda x)} \, dx.
\end{align*} 

\subsubsection{Fourier Transform}

We can apply the Fourier series process to non-periodic functions too, in that
case yielding a Fourier transform. In the case of some $f : \mathbb{R} \to
\mathbb{R}$ (not necessarily periodic): \begin{gather*}
    F(\lambda) = \int_{-\infty}^\infty f(x) e^{-2\pi\lambda ix} \, dx,
\end{gather*} is the Fourier transform of $f$, where $i = \sqrt{-1}$.
We can also inverse this process: \begin{gather*}
    f(x) = \int_{-\infty}^\infty F(\lambda) e^{2\pi\lambda ix} \, du.
\end{gather*} For $f : [n] \to [n]$ (discrete), we have the corresponding
pair: \begin{gather*}
    F(\lambda) = \frac{1}{n + 1} \sum_{x = 0}^{n} f(x) e^{-\frac{2\pi\lambda i}{n + 1}x}
    , \qquad
    f(x) = \sum_{\lambda = 0}^{n} F(\lambda) e^{\frac{2\pi\lambda i}{n + 1}x}.
\end{gather*} By using Euler's identity, we can expand our expressions using
$e$ into the sum of sin and cos. Furthermore, we can consider taking the real
$R$ and imaginary $I$ parts of $F$ to derive quantities: \begin{center}
    \begin{tabular}{ r c l }
        $F(\lambda)$                     & = & $R(\lambda) + i \cdot I(\lambda)$ \\
        Magnitude, $|F(\lambda)|$        & = & $\sqrt{R^2(\lambda) + I^2(\lambda)}$ \\
        Phase angle, $\phi(\lambda)$  & = & $\tan^{-1}\left( \frac{I(\lambda)}{R(\lambda)} \right)$\\
        Polar coordinate of $F(\lambda)$ & = & $|F(\lambda)|e^{i\phi(\lambda)}$
    \end{tabular}
\end{center}