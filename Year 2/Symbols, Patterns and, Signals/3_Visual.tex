\section{Data Visualisation}

\subsection{Scatter Plots}

Scatter plots are useful for visualising two dimensional data and their relation.
When we get up to higher dimensions, we can represent the data as a matrix of
scatter plots where we choose two of the possible parameters at a time.

\subsection{Histograms}

For discrete data, this is a bar chart. For continuous data, we sort data into
bins of some non-zero width.

\subsection{Box Plots}

Box plots give a good idea of some of the key values of distributions
like the interquartile range, median and, range.

\subsection{Surfaces}

We can use 3D surfaces to represent 3D data but also, we can encode a
fourth dimension into the colour of the plot at each point.

\subsection{The Lie Factor}

We can quantify how much we are over/under-stating effects in our data
with the Lie Factor (LF):
\begin{gather*}
    \text{LF} = \frac{\text{
        size of effect in visualisation
    }}{\text{
        size of effect in data
    }}.
\end{gather*} So, an LF greater than one means we are over-stating
and an LF less than one means we are under-stating.