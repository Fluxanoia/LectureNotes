\section{Interpretation}

Interpretation is one-stage execution, the code is evaluated
at runtime.
\\[\baselineskip]
Most code interpretation is done as expected (refering to the
earlier chapters) but there are some key concepts to consider.

\subsection{Let Expressions}

Let expressions have the form: \begin{lstlisting}
    let x xexpr expr
\end{lstlisting} where \texttt{x} is the variable we are
setting the value of, \texttt{xexpr} is the expression of
which we are setting \texttt{x} to, and \texttt{expr} is the
expression in which \texttt{x} takes on the value described
by \texttt{xexpr}.
\\[\baselineskip]
Variables defined in let expressions take precedence over
variables in the environment. However, there is potential 
(when substituting free variables) for variables to be
captured by our let expressions when they shouldn't.
So, when we substitute variables, we should use some
method (like a counter) to make sure substituted variables
are unique and thus, are not captured.