\section{The Abstract Machine}

We have that the configurations of the abstract machine are of the 
form: \begin{gather*}
  \langle c, e, s \rangle,
\end{gather*} where $c$ is the code being executed, $e$ is the
evaluation stack, and $s$ is the storage. We use $\varepsilon$ to 
denote an empty stack or code.

\subsection{Code}

The code being executed is simply an instruction defined as follows:
\begin{align*}
  c :&= \text{empty} \, | \, \text{inst};c \\
  \text{inst} :&= \text{PUSH}-n \, | \, \text{ADD} \, | \, \text{MULT} \, | \, \text{SUB} \\
  &| \, \text{TRUE} \, | \, \text{FALSE} \, | \, \text{EQ} \, | \, \text{LE} \, | \, \text{AND} \, | \, \text{NEG} \\
  &| \, \text{FETCH}-x \, | \, \text{STORE}-x \, | \, \text{NOOP} \\
  &| \, \text{BRANCH}(c, c) \, | \, \text{LOOP}(c, c). \\
\end{align*}

\subsection{Evaluation Stack}

The evaluation stack is an stack of elements in $\mathbb{Z}$ or $T$.

\subsection{Storage}

For the sake of simplicity, the storage is essentially a state.

\subsection{Terminal States}

A configuration is terminal if the code is empty.

\newpage

\subsection{Operational Semantics for the Abstract Machine}

The table below lays out the operational semantics for the abstract machine.
\begin{center}
  \renewcommand{\arraystretch}{1.6}
  \begin{tabular}{ l c l l }
    \hline \multicolumn{4}{c}{Memory Management} \\ \hline
    $\langle$ PUSH-$n:c, e, s$             $\rangle$ & $\triangleright$ & $\langle c, \mathcal{N}(n):e, s \rangle$ & \\
    $\langle$ FETCH-$x:c, e, s$            $\rangle$ & $\triangleright$ & $\langle c, s(x):e, s \rangle$ & \\
    $\langle$ STORE-$x:c, z:e, s$          $\rangle$ & $\triangleright$ & $\langle c, e, s[x \mapsto z] \rangle$ & if $z$ is in $\mathbb{Z}$ \\
    \hline \multicolumn{4}{c}{Arithmetic Operators} \\ \hline
    $\langle$ ADD$:c, z_1:z_2:e, s$        $\rangle$ & $\triangleright$ & $\langle c, (z_1 + z_2):e, s \rangle$ & if $z_1, z_2$ are in $\mathbb{Z}$ \\
    $\langle$ MULT$:c, z_1:z_2:e, s$       $\rangle$ & $\triangleright$ & $\langle c, (z_1 \star z_2):e, s \rangle$ & if $z_1, z_2$ are in $\mathbb{Z}$ \\
    $\langle$ SUB$:c, z_1:z_2:e, s$        $\rangle$ & $\triangleright$ & $\langle c, (z_1 - z_2):e, s \rangle$ & if $z_1, z_2$ are in $\mathbb{Z}$ \\
    \hline \multicolumn{4}{c}{Boolean Operators} \\ \hline
    $\langle$ TRUE$:c, e, s$               $\rangle$ & $\triangleright$ & $\langle c, \text{tt}:e, s \rangle$ & \\
    $\langle$ FALSE$:c, e, s$              $\rangle$ & $\triangleright$ & $\langle c, \text{ff}:e, s \rangle$ & \\
    $\langle$ EQ$:c, z_1:z_2:e, s$         $\rangle$ & $\triangleright$ & $\langle c, (z_1 == z_2):e, s \rangle$ & if $z_1, z_2$ are in $\mathbb{Z}$ \\
    $\langle$ LEQ$:c, z_1:z_2:e, s$        $\rangle$ & $\triangleright$ & $\langle c, (z_1 \leq z_2):e, s \rangle$ & if $z_1, z_2$ are in $\mathbb{Z}$ \\
   
    \hdashline

    \multirow{2}{*}{
    $\langle$ AND$:c, t_1:t_2:e, s$ $\rangle$
    } & $\triangleright$ & $\langle c, \text{tt}:e, s \rangle$ & if $t_1 == $ tt, $t_2 == $ tt \\
    & $\triangleright$ & $\langle c, \text{ff}:e, s \rangle$ & otherwise \\
    
    \hdashline

    \multirow{2}{*}{
    $\langle$ NEG$:c, t:e, s$ $\rangle$ 
    } & $\triangleright$ & $\langle c, \text{tt}:e, s \rangle$ & if $t$ == ff \\
    & $\triangleright$ & $\langle c, \text{ff}:e, s \rangle$ & if $t$ == tt \\

    \hline \multicolumn{4}{c}{Control Flow} \\ \hline
    $\langle$ NOOP$:c, e, s$ $\rangle$ & $\triangleright$ & $\langle c, e, s \rangle$ & \\
    
    \hdashline

    \multirow{2}{*}{
    $\langle$ BRANCH$(c_1, c_2):c, t:e, s$ $\rangle$ 
    } & $\triangleright$ & $\langle c_1, e, s \rangle$ & if $t ==$ tt \\
    & $\triangleright$ & $\langle c_2, e, s \rangle$ & otherwise \\
    
    \hdashline

    $\langle$ LOOP$(c_1, c_2):c, e, s$ $\rangle$ & $\triangleright$ & 
    \multicolumn{2}{l}{
    $\langle c_1:$BRANCH($c_2:$LOOP($c_1, c_2$), NOOP)$:c, e, s \rangle$
    } \\
  \end{tabular}
\end{center}

\subsubsection{Computation Sequences}

Corresponding to the derivation sequences of structural operational semantics,
we can also define a computation sequence of the abstract machine. For a 
sequence of instructions $c$ and some storage $s$, the length of such a sequence 
is either: \begin{itemize}
  \item \textbf{Finite}: $\gamma_0, \gamma_1, \gamma_2, \ldots, \gamma_k$
  for some $k$ in $\mathbb{Z}_{\geq 0}$ where each $i$ in $\{0, 1, \ldots, k - 1\}$ satisfies
  $\gamma_0 = \langle c, \varepsilon, s \rangle$, $\gamma_i \triangleright \gamma_{i + 1}$ where
  there is no $\gamma$ such that $\gamma_k \triangleright \gamma$,
  \item \textbf{Infinite}: $(\gamma_k)_{k \geq 0}$ where 
  for each $i$ in $\mathbb{Z}_{\geq 0}$ satisfies 
  $\gamma_0 = \langle c, \varepsilon, s \rangle$, $\gamma_i \triangleright \gamma_{i + 1}$.
\end{itemize} We have that the evaluation stack $e$ is always empty
initially. We say the computation sequence is terminating if it's finite
and looping if it's infinite. Terminating sequences lead to terminal or stuck
configurations.
\\[\baselineskip]
The semantics of the abstract machine are deterministic as defined
in the table above, meaning for all
computation sequences $\gamma, \gamma_1, \gamma_2$: \begin{gather*}
  \begin{rcases*}
    \gamma_1 \triangleright \gamma \, \\
    \gamma_2 \triangleright \gamma \,
  \end{rcases*} \Rightarrow \gamma_1 = \gamma_2.
\end{gather*}

\subsection{The Execution Function}

Similarly to the semantic function for our types of semantics,
we define a function for the execution of abstract machine code, 
$\mathcal{M}$: \begin{align*}
  \mathcal{M} &: \textbf{Code} \to (\textbf{State} \hookrightarrow \textbf{State}) \\
  \mathcal{M}(c)(s) &= \begin{cases}
    s' & \text{if } \langle c, \varepsilon, s \rangle \triangleright^* \langle \varepsilon, e, s' \rangle\\
    \text{undefined} & \text{otherwise.}
  \end{cases}
\end{align*}

\vfill

\subsection{Code Generation}

We define functions for generating code for the abstract machine from
expressions and statements.

\subsubsection{Arithmetic Expressions}

This code generation function has the form: \begin{gather*}
  \mathcal{CA} : \textbf{Aexp} \to \textbf{Code}.
\end{gather*} The function is defined as follows: \begin{align*}
  \mathcal{CA}(n) &= \text{PUSH-}n, \\
  \mathcal{CA}(x) &= \text{FETCH-}x, \\
  \mathcal{CA}(a_1 + a_2) &= \mathcal{CA}(a_2):\mathcal{CA}(a_1):\text{ADD}, \\
  \mathcal{CA}(a_1 - a_2) &= \mathcal{CA}(a_2):\mathcal{CA}(a_1):\text{SUB}, \\
  \mathcal{CA}(a_1 \star a_2) &= \mathcal{CA}(a_2):\mathcal{CA}(a_1):\text{MULT}. \\
\end{align*}

\subsubsection{Boolean Expressions}

This code generation function has the form: \begin{gather*}
  \mathcal{CB} : \textbf{Bexp} \to \textbf{Code}.
\end{gather*} The function is defined as follows: \begin{align*}
  \mathcal{CB}(\texttt{true}) &= \text{TRUE}, \\
  \mathcal{CB}(\texttt{false}) &= \text{FALSE}, \\
  \mathcal{CB}(a_1 == a_2) &= \mathcal{CA}(a_2):\mathcal{CA}(a_1):\text{EQ}, \\
  \mathcal{CB}(a_1 \leq a_2) &= \mathcal{CA}(a_2):\mathcal{CA}(a_1):\text{LEQ}, \\
  \mathcal{CB}(\neg b) &= \mathcal{CB}(b):\text{NEG}, \\
  \mathcal{CB}(b_1 \land b_2) &= \mathcal{CB}(b_2):\mathcal{CB}(b_1):\text{AND}.\\
\end{align*}

\subsubsection{Statements}

This code generation function has the form: \begin{gather*}
  \mathcal{CS} : \textbf{Stm} \to \textbf{Code}.
\end{gather*} The function is defined as follows: \begin{align*}
  \mathcal{CS}(x = a) &= \mathcal{CA}(a):\text{STORE-}x, \\
  \mathcal{CS}(\texttt{skip}) &= \text{NOOP}, \\
  \mathcal{CS}(S_1:S_2) &= \mathcal{CS}(S_1):\mathcal{CS}(S_2), \\
  \mathcal{CS}(
    \texttt{if } b \texttt{ then } S_1 \texttt{ else } S_2
  ) &= \mathcal{CB}(b):\text{BRANCH}(\mathcal{CS}(S_1), \mathcal{CS}(S_2)), \\
  \mathcal{CS}(
    \texttt{while } b \texttt{ do } S
  ) &= \text{LOOP}(\mathcal{CB}(b), \mathcal{CS}(s)).
\end{align*}

\subsection{The Semantic Function}

We define the semantic function as follows: \begin{gather*}
  \mathcal{S}: \textbf{Stm} \to (\textbf{State} \hookrightarrow \textbf{State}).
\end{gather*} Thus for a given statement $S$, we have that: \begin{gather*}
  \mathcal{S}(S) = (\mathcal{M} \circ \mathcal{CS})(S).
\end{gather*} We that this function is partial because some statements
may never terminate. Thus giving an indeterminate answer.
\\[\baselineskip]
As there is more than one semantic function (for our different types
of semantics) we denote this function as $\mathcal{S}_{am}$.

\subsection{Correctness}

\subsubsection{Arithmetic Expressions}

We have that for all arithmetic expressions $a$: \begin{gather*}
  \langle \mathcal{CA}(a), \varepsilon, s \rangle 
  \triangleright^*
  \langle \varepsilon, \mathcal{A}(a)(s), s \rangle,
\end{gather*} and all intermediary states have non-empty stacks.

\subsubsection{Boolean Expressions}

We have that for all boolean expressions $b$: \begin{gather*}
  \langle \mathcal{CB}(b), \varepsilon, s \rangle 
  \triangleright^*
  \langle \varepsilon, \mathcal{B}(b)(s), s \rangle,
\end{gather*} and all intermediary states have non-empty stacks.

\subsubsection{Statements}

We have that for all statements $S$: \begin{gather*}
  S_{ns}(S) = S_{sos}(S) = S_{am}(S).
\end{gather*} Also, for all states $s$, we have that: \begin{gather*}
  \begin{rcases*}
    \langle S, s \rangle \to s' \\
    \qquad \text{or} \\
    \langle S, s \rangle \Rightarrow^* s'
  \end{rcases*} \Rightarrow
  \langle \mathcal{CS}(S), \varepsilon, s \rangle \triangleright^* 
  \langle \varepsilon, \varepsilon, s' \rangle 
\end{gather*}