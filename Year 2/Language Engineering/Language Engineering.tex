\documentclass[a4paper, 12pt, twoside]{article}
\usepackage[left = 3cm, right = 3cm]{geometry}
\usepackage[english]{babel}
\usepackage[utf8]{inputenc}
\usepackage{mathtools}
\usepackage{amssymb}
\usepackage{amsmath}
\usepackage{multicol}

\begin{document}

\title{Language Engineering Notes}
\date{}
\author{\textit{paraphrased by} Tyler Wright}
\maketitle

\vfill

\textit{An important note, these notes are absolutely \textbf{NOT}
  guaranteed to be correct, representative of the course, or rigorous.
  Any result of this is not the author's fault.}

\newpage

\section{Syntax and Semantics}

\subsection{What is Syntax?}

Syntax is the grammatical structure of a program. For example,
for the program $x:=y;y:=z;z:=x;$, syntactic analysis of this 
program would conclude that we have three statements concluded
with ';'. Each of said statements are variables followed by
the composite symbol ':=' and another variable.

\subsection{What are Semantics?}

The semantics of a program are what the program evaluates
to or rather, the meaning of a syntactically correct program.
For example, $x:=y;$ evaluates to setting the value of $x$ to the 
value of $y$.

\section{Operational Semantics}

\subsection{Overview of Operational Semantics}

An operational explanation of the meaning of a construct will explain
how to execute said construct. For example, in C, the semicolons provide
chronology and the $=$ symbol demonstrates assignment. These statements
are abstractions as they do not concern themselves with the specific
memory addresses or registers. Thus, these semantics are independent
of machine architecture.

\subsection{Derivation Trees and Natural Semantics}

A program's execution can be modelled by a 'derivation tree'
where the higher parts of the tree are breakdowns of the statements below.
For example, the program $x:=y;y:=z;z:=x;$ can be written as:
\begin{gather*}
  \frac{}
  {\langle x:=y;y:=z;z:=x;, s_0\rangle \to s_3}
\end{gather*}

\end{document}