\section{Compiling}

Compiling is two-stage execution, the code is evaluated
into machine code - then it can be evaluated.

\subsection{Variables}

Variable's symbolic names are lost on compilation and 
replaced by indices refering to the distance to the variable
binding. In this case, our environment becomes a list of values
which we use our variable indices to index.

\subsection{Stack Instructions}

We form a set of stack instructions:
\begin{center}
    \begin{tabular}{| c | c | c | c |}
        \hline
        Instruction & Stack before & Stack after & Description \\
        \hline \hline
        \texttt{SCSTI} $n$ & \ldots               & $n$, \ldots             & Pushes \\
        \hline
        \texttt{SVAR} $n$  & \ldots               & $s[n]$, \ldots          & Indexes the stack \\
        \hline
        \texttt{SADD}      & $n_2$, $n_1$, \ldots & $n_1 + n_2$, \ldots     & Adds \\
        \hline
        \texttt{SSUB}      & $n_2$, $n_1$, \ldots & $n_1 - n_2$, \ldots     & Subtracts \\
        \hline
        \texttt{SMUL}      & $n_2$, $n_1$, \ldots & $n_1 \cdot n_2$, \ldots & Multiplies \\
        \hline
        \texttt{SDUP}      & $v$, \ldots          & $v$, $v$, \ldots        & Duplicates \\
        \hline
        \texttt{SPOP}      & $v$, \ldots          & \ldots                  & Pops \\
        \hline
        \texttt{SSWAP}     & $v_2$, $v_1$, \ldots & $v_1$, $v_2$, \ldots    & Swaps \\
        \hline
    \end{tabular}
\end{center} The element remaining at the top of the stack at
the end of computation is the result of the computation.

\subsection{Lexers}

The purpose of a lexer is to break down a string of characters (a written
program) into tokens. Tokens can be keywords, operators, special symbols, etc.
with identifiers and number literals formed compositionally.
\\[\baselineskip]
Lexers are generated using a specification which describes the format of
tokens.

\subsection{Parsers}

Parsers check that grammar (syntax) is respected, arranging tokens into a syntax tree
with the leaves as tokens and other nodes as operators.
\\[\baselineskip]
Like with lexers, parsers are generated using a specification which describes well-formed 
streams of tokens.

\subsubsection{Ambiguity}

We say that a grammar is ambigious if it has multiple derivation trees. To circumvent this,
we assign precedence and associativity to operators. Higher precedence operators appear
closer to the leaves.

\subsubsection{LR and LL Parsing}

LR parsing parses from left to right, making derivations from the right-most non-terminal
symbol. This corresponds to bottom-up parsing, which is difficult to hand-write but requires
no grammar transformations.
\\[\baselineskip]
LL parsing parses from left to right, making derivations from the left-most non-terminal
symbol. This corresponds to top-down parsing, which is easy to hand-write but requires
grammar transformations.

\subsubsection{Parser State}

The parser state is a stack of values consisting of parser state numbers and grammar symbols,
terminal and non-terminal. The parser can perform the following actions on the state:
\begin{itemize}
    \item Shift, read a symbol in and put it on the stack,
    \item Reduce, take grammar rule symbols off the stack and 
        replace with the corresponding non-terminal, then evaluate
        the semantic action.
\end{itemize} There can be ambiguity in the decision to shift or reduce, this can be solved by
altering precedence and associativity.
