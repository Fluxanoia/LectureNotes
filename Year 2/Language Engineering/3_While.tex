\section{The While Language}

\subsection{Syntactic Categories}

For this language, we have five syntactic categories: \begin{itemize}
  \item \textbf{Numerals} (Num), denoted by $n$,
  \item \textbf{Variables} (Var), denoted by $x$,
  \item \textbf{Arithmetic Expressions} (Aexp), denoted by $a$,
  \item \textbf{Boolean Expressions} (Bexp), denoted by $b$,
  \item \textbf{Statements} (Stm), denoted by $S$.
\end{itemize} We assume that the numerals and variables are defined
elsewhere (for example, the numerals could be strings of digits
and the variables could be strings of letters). The other
structures are detailed below: \begin{align*}
  a &:= n \, | \, x \, | \, a_1 + a_2 \, | \, a_1 \star a_2 \, | \, a_1 - a_2, \\
  b &:= \texttt{true} \, | \, \texttt{false} \, | \, a_1 == a_2 \, 
  | \, a_1 \leq a_2 \, | \, \neg b \, | \, b_1 \land b_2, \\
  S &:= x = a \, | \, \texttt{skip} \, | \, S_1 ; S_2 \, | \, \texttt{if }
  b \texttt{ then } S_1 \texttt{ else } S_2 \, | \, \texttt{while } 
  b \texttt{ do } S.
\end{align*} We define representations of each structure in terms of itself
as composite elements and the rest are basis elements.

\subsubsection{Extensions of the Categories}

We define a language \textbf{Exc} identical to \textbf{While} except we
extend the \textbf{Statements} to: \begin{align*}
  S &:= x = a \, | \, \texttt{skip} \, | \, S_1 ; S_2 \, | \, \texttt{if }
  b \texttt{ then } S_1 \texttt{ else } S_2 \, | \, \texttt{while } 
  b \texttt{ do } S \\
  & \qquad \texttt{begin } S_1 \texttt{ handle } e : S_2 \texttt{ end} \, |
  \, \texttt{raise } e,
\end{align*} and we define the \textbf{Exception} of which $e$ is a member of.
\\[\baselineskip]
The \texttt{raise} instruction, when called, breaks the execution of
the current statement and gives control to the handling statement.

\subsection{State}

When considering a statement, it is almost always important to consider the
state of the program as the role of statements is to change the state
of the program. For example, when evaluating $x + 1$, we must consider
what $x$ is. This requires us to develop the notion of state.
\\[\baselineskip]
We define state as a set of mappings from variables to values: \begin{gather*}
  \textbf{State} = \{f : \textbf{Var} \to \mathbb{Z}\}.
\end{gather*} This is commonly listed out. For example, if $x$ maps to $4$ and 
$y$ maps to $5$, we have our state equal to $\{x \mapsto 4, y \mapsto 5\}$. 

\subsection{Semantic Functions}

The use of semantic functions is to convert syntactic elements into its meaning.
Each of the semantic styles will be used to define said semantic functions
for variable and statements. However, for numerals, arithmetic and boolean
expressions, they are defined as follows. 

\subsubsection{Numerals}

If we assume our numerals are in base-2 (binary), we can define a numeral
as follows: \begin{gather*}
  n := 0 \, | \, 1 \, | \, n \doubleplus 0 \, | \, n \doubleplus 1 \, |.
\end{gather*} In order to determine the value represented by a numeral, we
define a \textit{semantic function} $\mathcal{N} : \textbf{Num} \to \mathbb{Z}$:
\begin{align*}
  \mathcal{N}(0) &= 0, \\
  \mathcal{N}(1) &= 1, \\
  \mathcal{N}(n \doubleplus 0) &= 2 \cdot \mathcal{N}(n), \\
  \mathcal{N}(n \doubleplus 1) &= 2 \cdot \mathcal{N}(n) + 1,
\end{align*} This definition is called \textit{compositional} as it simply
defines an output for each way of constructing a numeral.
Much like in Linear Algebra, it is important to differentiate
the numeral '1' and the integer $1$ and similarly for zero.

\newpage

\subsubsection{Arithmetic Expressions}

For a given arithmetic expression, it may be necessary to analyse the state
to determine the result. Thus, the semantic expression for arithmetic
expressions is a function $\mathcal{A} : \textbf{Aexp} \to (\textbf{State} 
\to \mathbb{Z})$. 
\\[\baselineskip]
This means providing $\mathcal{A}$ with an arithmetic
expression gives us a function from a state to a value. So, for a state $s$,
we can define $\mathcal{A}$ as follows: \begin{align*}
  \mathcal{A}(n)(s) &= \mathcal{N}(n), \\
  \mathcal{A}(x)(s) &= s(x), \\
  \mathcal{A}(a_1 + a_2)(s) &= \mathcal{A}(a_1)(s) + \mathcal{A}(a_2)(s), \\
  \mathcal{A}(a_1 \star a_2)(s) &= \mathcal{A}(a_1)(s) \star \mathcal{A}(a_2)(s), \\
  \mathcal{A}(a_1 - a_2)(s) &= \mathcal{A}(a_1)(s) - \mathcal{A}(a_2)(s).
\end{align*} Similarly to the numerical semantic function, it's important
to differentiate between the syntactic $+, \star, -$ in the arithmetic
expressions like $a_1 + a_2$ and the usual arithmetic operations $+, \star, -$
between integers.

\subsubsection{Boolean Expressions}

We define the set $T$ to be $T = \{\text{tt}, \text{ff}\}$ where tt represents
truth and ff represents falsity.
\\[\baselineskip]
Similarly to arithmetic expressions we can define $\mathcal{B} : \textbf{Bexp}
\to (\textbf{State} \to \mathbb{Z})$ for a state $s$: \begin{align*}
  \mathcal{B}(\texttt{true})(s) &= \text{tt}, \\
  \mathcal{B}(\texttt{false})(s) &= \text{ff}, \\
  \mathcal{B}(a_1 == a_2)(s) &= \begin{cases}
      \text{tt} & \text{if } \mathcal{A}(a_1)(s) = \mathcal{A}(a_2)(s) \\
      \text{ff} & \text{otherwise,}
  \end{cases} \\
  \mathcal{B}(a_1 \leq a_2)(s) &= \begin{cases}
    \text{tt} & \text{if } \mathcal{A}(a_1)(s) \leq \mathcal{A}(a_2)(s) \\
    \text{ff} & \text{if } \mathcal{A}(a_1)(s) > \mathcal{A}(a_2)(s),
  \end{cases} \\
  \mathcal{B}(\neg b)(s) &= \begin{cases}
    \text{tt} & \text{if } \mathcal{B}(b)(s) = \text{ff} \\
    \text{ff} & \text{if } \mathcal{B}(b)(s) = \text{tt},
  \end{cases} \\
  \mathcal{B}(b_1 \land b_2)(s) &= \begin{cases}
    \text{tt} & \text{if } \mathcal{B}(b_1)(s) = \text{tt} \text{ and } \mathcal{B}(b_2)(s) = \text{tt} \\
    \text{ff} & \text{if } \mathcal{B}(b_1)(s) = \text{ff} \text{ or } \mathcal{B}(b_2)(s) = \text{ff},
  \end{cases}
\end{align*}

\subsection{Free Variables}

The free variables of an expression is the set of variables 
occuring within it.

\subsubsection{Arithmetic Expressions}

We define $F_V : \textbf{Aexp} \to \{\textbf{Var}\}$ by: \begin{align*}
  F_V(n) &:= \emptyset, \\
  F_V(x) &:= \{x\}, \\
  F_V(a_1 + a_2) &:= F_V(a_1) \cup F_V(a_2), \\
  F_V(a_1 \star a_2) &:= F_V(a_1) \cup F_V(a_2), \\
  F_V(a_1 - a_2) &:= F_V(a_1) \cup F_V(a_2).
\end{align*} Thus, we have developed a way to decompose expressions
and generate all the variables said expression depends on. This
is formalised for states $s_1, s_2$ and an arithmetic expression 
$a$: \begin{gather*}
  \big[\forall x \in F_V(a)\big]\big[s_1(x) = s_2(x)\big] 
  \Longrightarrow 
  \big[\mathcal{A}(a)(s_1) = \mathcal{A}(a)(s_2)\big].
\end{gather*}

\subsubsection{Boolean Expressions}

We define $F_V : \textbf{Bexp} \to \{\textbf{Var}\}$ by: \begin{align*}
  F_V(\texttt{true}) &:= \emptyset, \\
  F_V(\texttt{false}) &:= \emptyset, \\
  F_V(a_1 == a_2) &:= F_V(a_1) \cup F_V(a_2), \\
  F_V(a_1 \leq a_2) &:= F_V(a_1) \cup F_V(a_2), \\
  F_V(\neg b) &:= F_V(b), \\
  F_V(b_1 \land b_2) &:= F_V(b_1) \cup F_V(b_2).
\end{align*} Similarly, for states $s_1, s_2$ and an boolean expression 
$b$: \begin{gather*}
    \big[\forall x \in F_V(b)\big]\big[s_1(x) = s_2(x)\big] 
    \Longrightarrow 
    \big[\mathcal{B}(b)(s_1) = \mathcal{B}(b)(s_2)\big].
\end{gather*}

\newpage

\subsection{Substitutions}

Within expressions, we will be interested in swapping the occurances
of a variable with another expression. For an expression $e$ with
$x$ in $F_V(e)$ and another expression $e_0$, we write
$e[x\mapsto e_0]$ for the expression $e$ where $x$ is substituted
for $e_0$.

\subsubsection{Arithmetic Expressions}

For an arithmetic expression $a_0$, we can define substitution on
arithmetic expressions: \begin{align*}
  n[y \mapsto a_0] &:= n, \\
  x[y \mapsto a_0] &:= \begin{cases}
    a_0 & \text{if } x = y \\
    x & \text{otherwise,}
  \end{cases} \\
  (a_1 + a_2)[y \mapsto a_0] &:= (a_1[y \mapsto a_0]) + (a_2[y \mapsto a_0]), \\
  (a_1 \star a_2)[y \mapsto a_0] &:= (a_1[y \mapsto a_0]) \star (a_2[y \mapsto a_0]), \\
  (a_1 - a_2)[y \mapsto a_0] &:= (a_1[y \mapsto a_0]) - (a_2[y \mapsto a_0]).
\end{align*}

\subsubsection{Boolean Expressions}

For an arithmetic expression $a_0$, we can define substitution on 
boolean expressions: \begin{align*}
  (\texttt{true})[x \mapsto a_0] &:= \texttt{true}, \\
  (\texttt{false})[x \mapsto a_0] &:=  \texttt{false}, \\
  (a_1 == a_2)[x \mapsto a_0] &:= (a_1[x \mapsto a_0]) == (a_2[x \mapsto a_0]), \\
  (a_1 \leq a_2)[x \mapsto a_0] &:= (a_1[x \mapsto a_0]) \leq (a_2[x \mapsto a_0]), \\
  (\neg b)[x \mapsto a_0] &:= \neg (b)[x \mapsto a_0], \\
  (b_1 \land b_2)[x \mapsto a_0] &:= (b_1)[x \mapsto a_0] \land (b_2)[x \mapsto a_0].
\end{align*}

\subsubsection{States}

A similar process applies to states, for a state $s$ and the variables
$x, v$: \begin{gather*}
  (s[y\mapsto v])(x) = \begin{cases}
    v & \text{if } x = y \\
    s(x) & \text{otherwise}.
  \end{cases}
\end{gather*}