\section{Operational Semantics}

An operational explanation of the meaning of a construct will explain
how to execute said construct. For example, in C, the semicolons provide
chronology and the $=$ symbol demonstrates assignment. The semantics
are not merely interested in the results of a program.
\\[\baselineskip]
More specifically, this means we are interested in how the state of the
program changes. To this end, there are two approaches we shall consider:
\begin{itemize}
  \item \textbf{Natural Semantics}: to describe how the overall
  results of executions are obtained,
  \item \textbf{Structural Operational Semantics}: to describe how
  the individual steps of computation take place.
\end{itemize}

\subsection{Rules and Derivation Trees}

\subsubsection{Transitions}

A transition is a construct representing the transition of
a state $s$ to a state $s'$ via some statement $S$, denoted
by $\langle S, s \rangle \to s'$.

\subsubsection{Rules}

We can define 'rules' which are of the form: \begin{gather*}
  \frac{
    \langle S_1, s_1 \rangle \to s_2, \ldots, \langle S_n, s_n \rangle \to s'
  }{
    \langle S, s \rangle \to s'
  } \qquad \text{if } \cdots.
\end{gather*} Where the premises consisting of immediate constituents 
of a statement $S$ or compositions of said immediate constituents, 
are written above the conclusion. Conditions may be written to the right
of the rule which are necessary for when the rule is applied.

\subsubsection{Axioms}

For a rule where the set of premises is empty, we call 
such a rule an \textit{axiom}. Axioms with meta-variables
are called \textit{axiom schema} where we can obtain an
\textit{instance} of an axiom by defining particular
variables.

\subsubsection{Derivation Trees}

A program's execution can be modelled by a 'derivation tree'
formed from rules.
For example, the program $x=y;y=z;z=x;$ can be written as:
\begin{gather*}
  \frac{\dfrac{
    \langle z=x, s_{570} \rangle \to s_{575} 
    \qquad \langle x=y, s_{575} \rangle \to s_{775}
  }{\langle z=x;x=y, s_{570} \rangle \to s_{775}}
  \qquad \langle y=z, s_{775} \rangle \to s_{755}}
  {\langle x=y;y=z;z=x, s_{570} \rangle \to s_{755}}.
\end{gather*} Where $s_{abc}$ represents the state where 
$\{x \mapsto a, y \mapsto b, z \mapsto c\}$.
\\[\baselineskip]
This tree repesents the decomposition of the composite statement
at the bottom of the tree $x=y;y=z;z=x;$. As $z=x$ transforms
the state $s_{570}$ to $s_{575}$ and $x=y$ transforms the state
$s_{575}$ to $s_{775}$, we know that the composite statement
$z=x;x=y$ transforms $s_{570}$ to $s_{775}$.

\subsection{Natural Semantics}

During this section, we will use $\langle S, s\rangle$
representing the statement $S$ being executed from the state $s$ and
simply $s$ to represent a terminal state.
\\[\baselineskip]
We can define some natural semantics for the While language as follows:
\begin{align*}
  [\text{assignment}]& \qquad \langle x=a, s \rangle \to s[x\mapsto \mathcal{A}(a)(s)] \\ \\
  [\text{skip}]& \qquad \langle \texttt{skip}, s \rangle \to s \\ \\
  [\text{composition}]& \qquad \frac{
    \langle S_1, s \rangle \to s' \qquad \langle S_2, s' \rangle \to s''
  }{\langle S_1;S_2, s \rangle \to s''} \\ \\
  [\text{if}]& \qquad \begin{cases}
    \dfrac{
      \langle S_1, s \rangle \to s'
    }{
      \langle \texttt{if } b \texttt{ then } S_1 \texttt{ else } S_2, s \rangle \to s'
    } & \text{if } \mathcal{B}(b)(s) = \text{tt} \\ \\
    \dfrac{
      \langle S_2, s \rangle \to s'
    }{
      \langle \texttt{if } b \texttt{ then } S_1 \texttt{ else } S_2, s \rangle \to s'
    } & \text{if } \mathcal{B}(b)(s) = \text{ff}
  \end{cases} \\ \\
  [\text{while}]& \qquad \begin{cases}
    \dfrac{
      \langle S, s \rangle \to s' \qquad \langle 
      \texttt{while } b \texttt{ do } S, s' \rangle \to s''
    }{
      \langle \texttt{while } b \texttt{ do } S, s \rangle \to s''
    } & \text{if } \mathcal{B}(b)(s) = \text{tt} \\ \\
    \langle \texttt{while } b \texttt{ do } S, s \rangle \to s
    & \text{if } \mathcal{B}(b)(s) = \text{ff}
  \end{cases}
\end{align*} We have that the semantics detailed above are deterministic, that is,
for identical inputs, we get identical outputs.
\\[\baselineskip]
As there is more than one definition of these semantics
we will subscript these defintions with $ns$ where appropriate.

\subsubsection{The Semantic Function}

We can now condense the meaning of statements into a partial
function from \textbf{State} to \textbf{State}. We define the semantic
function as follows: \begin{gather*}
  \mathcal{S}: \textbf{Stm} \to (\textbf{State} \hookrightarrow \textbf{State}).
\end{gather*} Thus for a given statement $S$, we have that: \begin{align*}
  &\mathcal{S}(S) : \textbf{State} \hookrightarrow \textbf{State} \\
  s &\mapsto \begin{cases}
    s' & \text{if } \langle S, s \rangle \to s' \\
    \text{undefined} & \text{otherwise}.
  \end{cases}
\end{align*} We that this function is partial because some statements
may never terminate. Thus giving an indeterminate answer.
\\[\baselineskip]
As there will be more than one semantic function (for our different types
of semantics) we denote this function as $\mathcal{S}_{ns}$.

\subsection{Structural Operational Semantics}

In structural operation semantics, there is a particular emphasis
on the individual steps of execution. The transition relation in
these semantics is of the form: \begin{gather*}
  \langle S, s \rangle \Rightarrow \gamma,
\end{gather*} where $\gamma$ is of the form:
\begin{center}
\renewcommand{\arraystretch}{1.5}
\begin{tabular}{ | c || p{8.5cm} | }
  \hline
  $\langle S', s' \rangle$ & denoting that $S$ and $s$
  have transformed in some way and that the execution is not
  yet complete, \\
  $s'$ & denoting that the execution has completed with some 
  terminal state $s'$. \\
  \hline
\end{tabular}
\end{center}

\noindent
We shall say that $\langle S, s \rangle$ is \textbf{stuck} if there
does not exist a $\gamma$ such that $\langle S, s \rangle
\Rightarrow \gamma$ and \textbf{unstuck} otherwise. 

\newpage

\noindent
We can define some structural operational semantics for the While 
language as follows:
\begin{align*}
  [\text{assignment}]& \qquad \langle x=a, s \rangle \Rightarrow s[x\mapsto \mathcal{A}(a)(s)], \\ \\
  [\text{skip}]& \qquad \langle \texttt{skip}, s \rangle \Rightarrow s, \\ \\
  [\text{composition}]& \qquad \frac{
    \langle S_1, s \rangle \Rightarrow \langle S'_1, s' \rangle
  }{\langle S_1;S_2, s \rangle \Rightarrow \langle S'_1;S_2, s' \rangle} 
  \qquad \text{or} \qquad \frac{
    \langle S_1, s \rangle \Rightarrow s'
  }{\langle S_1;S_2, s \rangle \Rightarrow \langle S_2, s' \rangle}, \\ \\
  [\text{if}]& \qquad \begin{cases}
    \langle \texttt{if } b \texttt{ then } S_1 \texttt{ else } S_2, s \rangle \Rightarrow 
    \langle S_1, s \rangle
    & \text{if } \mathcal{B}(b)(s) = \text{tt} \\ \\
    \langle \texttt{if } b \texttt{ then } S_1 \texttt{ else } S_2, s \rangle \Rightarrow
    \langle S_2, s \rangle
    & \text{if } \mathcal{B}(b)(s) = \text{ff},
  \end{cases} \\ \\
  [\text{while}]& \qquad \langle \texttt{while } b \texttt{ do } S, s \rangle \Rightarrow
    \langle \texttt{if } b \texttt{ then } (S; \texttt{ while } b \texttt{ do } S)
    \texttt{ else skip}, s \rangle.
\end{align*} We can see by the definition of composition that with these 
semantics we have to fully decompose and execution $S_1$ before
we can begin execution of $S_2$.
\\[\baselineskip]
As there is more than one definition of these semantics
we will subscript these defintions with $sos$ where appropriate.

\subsubsection{Derivation Sequences}

For a statement $S$ and state $s$, we can consider the sequence of
configurations of $S$ in $s$. The length of such a sequence is either:
\begin{itemize}
  \item \textbf{Finite}: $\gamma_0, \gamma_1, \gamma_2, \ldots, \gamma_k$
  for some $k$ in $\mathbb{Z}_{\geq 0}$ where each $i$ in $\{0, 1, \ldots, k - 1\}$ satisfies
  $\gamma_0 = \langle S, s \rangle$, $\gamma_i \Rightarrow \gamma_{i + 1}$ where
  $\gamma_k$ is terminal or stuck,
  \item \textbf{Infinite}: $(\gamma_k)_{k \geq 0}$ where 
  for each $i$ in $\mathbb{Z}_{\geq 0}$ satisfies 
  $\gamma_0 = \langle S, s \rangle$, $\gamma_i \Rightarrow \gamma_{i + 1}$.
\end{itemize} For two configurations $\gamma_i, \gamma_k$, we write
$\gamma_i \Rightarrow^k \gamma_j$ to denote that there are $k$ steps 
in execution between the configurations or $\gamma_i \Rightarrow^* \gamma_j$
to denote that there's some finite number of steps between the two
configurations.

\newpage

\noindent
We say for a statement $S$ and state $s$, the execution of $S$ on $s$:
\begin{itemize}
  \item \textbf{Terminates}: if and only if there's a finite derivation
  sequence starting with $\langle S, s, \rangle$
  \item \textbf{Loops}: if and only if there's a infinite derivation
  sequence starting with $\langle S, s \rangle$.
\end{itemize} Thus, statements can loop on some states and terminate on
others. Similarly, some states always terminate and always loop.

\subsubsection{Partial Execution}

If for some statement $S$, $k$ in $\mathbb{Z}_{\geq 0}$, and states 
$s, s''$, we have that $\langle S, s\rangle \Rightarrow^k s''$,
then there exists $k_1, k_2$ in $\mathbb{Z}_{\geq 0}$ such that 
$k_1 + k_2 = k$, $\langle S, s \rangle \Rightarrow^{k_1} s'$,
and $\langle S, s' \rangle \Rightarrow^{k_2} s''$ for some
intermediate state $s'$.

\subsubsection{The Semantic Function}

We can now condense the meaning of statements into a partial
function from \textbf{State} to \textbf{State}. We define the semantic
function as follows: \begin{gather*}
  \mathcal{S}: \textbf{Stm} \to (\textbf{State} \hookrightarrow \textbf{State}).
\end{gather*} Thus for a given statement $S$, we have that: \begin{align*}
  &\mathcal{S}(S) : \textbf{State} \hookrightarrow \textbf{State} \\
  s &\mapsto \begin{cases}
    s' & \text{if } \langle S, s \rangle \Rightarrow^* s' \\
    \text{undefined} & \text{otherwise}.
  \end{cases}
\end{align*} We that this function is partial because some statements
may never terminate. Thus giving an indeterminate answer.
\\[\baselineskip]
As there is more than one semantic function (for our different types
of semantics) we denote this function as $\mathcal{S}_{sos}$.

\subsection{Semantic Equivalence}

For every statement $S$ of \textbf{While}, we have that: \begin{gather*}
  \mathcal{S}_{ns}(S) = \mathcal{S}_{sos}(S).
\end{gather*} Similarly, we have that if a statement terminates to a state
for one type of semantics, it will terminate in the same state for the other.
