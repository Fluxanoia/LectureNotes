\section{Denotational Semantics}

\subsection{Partial Order}

\subsubsection{Weak Partial order}

A weak partial order on $X$ with $a, b, c$ in $X$ 
has the following properties: \begin{itemize}
  \item reflexive ($a \leq a$),
  \item transitive ($a \leq b$, $b \leq c \Rightarrow a \leq c$),
  \item antisymmetric ($a \leq b, b \leq a \Rightarrow a = b$).
\end{itemize}

\subsubsection{Strong Partial order}

A strong partial order on $X$ with $a, b, c$ in $X$ 
has the following properties: \begin{itemize}
  \item irreflexive ($a \leq a$ for no $a$),
  \item transitive ($a \leq b$, $b \leq c \Rightarrow a \leq c$),
  \item antisymmetric ($a \leq b, b \leq a \Rightarrow a = b$).
\end{itemize}

\subsubsection{Total Partial order}

A total partial order on $X$ with $a, b, c$ in $X$ 
has the following properties: \begin{itemize}
  \item connex ($a \leq b$ or $b \leq a$),
  \item reflexive ($a \leq a$),
  \item transitive ($a \leq b$, $b \leq c \Rightarrow a \leq c$),
  \item antisymmetric ($a \leq b, b \leq a \Rightarrow a = b$).
\end{itemize}

\subsection{Partial Order Sets}

For a partial order $\sqsubseteq$ on a set $D$, we have that
$(\sqsubseteq, D)$ is a partial order set (PO-set).

\subsubsection{Chains}

For a PO-set $(\sqsubseteq, D)$, for $X \subseteq D$, we say that
$X$ is a chain if and only if for all $x_1, x_2$ in $X$: \begin{gather*}
  x_1 \sqsubseteq x_2 \text{ or } x_2 \sqsubseteq x_1,
\end{gather*} meaning $X$ is totally ordered. 

\subsubsection{Least Upper Bounds}

For a PO-set $(\sqsubseteq, D)$, the least upper bound 
$x$ of $X \subseteq D$ is denoted by $x = \sqcup \, X$. 
\\[\baselineskip]
If such a bound exists for all subsets $X$ that are chains, 
$(\sqsubseteq, D)$ is a \textit{chain complete} PO-set.
\\[\baselineskip]
If such a bound exists for all subsets $X$, $(\sqsubseteq, D)$ 
is a \textit{complete lattice}.

\subsubsection{Lifted Partial-Order Sets}

A PO-set is 'lifted' if an artificial bottom element $\bot$
is added to the set.

\subsubsection{A Partial Order on State Transformers}

We have the following PO-set $(\sqsubseteq, T)$ where: \begin{gather*}
  T = \{f \text{ where } f : \textbf{State} \hookrightarrow \textbf{State}\},
\end{gather*} and for $f, g$ in $T$, $f \sqsubseteq g$
is equivalent to saying: \begin{gather*}
  f s = s' \Rightarrow g s = s' \text{ for all states } s. 
\end{gather*} This is a \textbf{weak} partial order on $T$, with
the least element being $0$ in $T$ undefined for all inputs.

\subsubsection{Fixed Point Theorem}

For $f : X \to X$ a continuous function on a chain-complete partial order
$(\sqsubseteq, X)$ with a least element $\bot$, we have that: \begin{gather*}
  \texttt{FIX } f = \sqcup \, \{f^n(\bot) : n \in \mathbb{Z}_{\geq 0}\},
\end{gather*} exists and is the least fixed point of $f$. For clarity: 
\begin{gather*}
  f^0 = \texttt{id} \qquad f^n = f \circ f^{n - 1},
\end{gather*} where $n$ is in $\mathbb{Z}_{\geq 0}$.

\subsubsection{The Conditional Function}

We define the type of \texttt{cond} as follows: \begin{gather*}
  \texttt{cond} : 
  (\textbf{State} \to \textbf{T}) \times
  (\textbf{State} \hookrightarrow \textbf{State}) \times
  (\textbf{State} \hookrightarrow \textbf{State}) \to
  (\textbf{State} \hookrightarrow \textbf{State}),
\end{gather*} with definition: \begin{gather*}
  \texttt{cond}(p, f_1, f_2)s = \begin{cases}
    f_1(s) & \text{if } p(s) == \textbf{tt} \\
    f_2(s) & \text{otherwise.} \\
  \end{cases}
\end{gather*}

\subsubsection{The Fixed Point Function}

We define the type of \texttt{FIX} as follows: \begin{gather*}
  \texttt{FIX} : 
  (\textbf{State} \hookrightarrow \textbf{State}) \to
  (\textbf{State} \hookrightarrow \textbf{State}) \to
  (\textbf{State} \hookrightarrow \textbf{State}),
\end{gather*} where \texttt{FIX}$(F)$ returns the least fixed 'point'
(or rather function) of $F$ ($f$ in Im($F$) such that $F(f) = f$). 
Smallest in this case is the 'most undefined' function.

\subsubsection{A Fibonacci Definition using \texttt{FIX}}

Consider the following definition:

\begin{lstlisting}
fib :: Int -> Int
fib = fix F where
  F :: (Int -> Int) -> (Int -> Int)
  F f n
    | (n == 0) || (n == 1) = 1
    | otherwise            = (f (n - 1)) + (f (n - 2))
\end{lstlisting}

So, if we want to calculate \texttt{fib 4} we get: \begin{center}
  \begin{tabular}{ l l }
    \begin{lstlisting}
      fib 4    = fix F 4
               = F (fix F) 4
               = (fix F 3) + (fix F 2)
      
      fix F 3 = F (fix F) 3
               = (fix F 2) + (fix F 1)
      
      fix F 2 = F (fix F) 2
               = (fix F 1) + (fix F 0)
      
      fix F 1 = F (fix F) 1
               = 1
      
      fix F 0 = F (fix F) 0
               = 1
      \end{lstlisting}
      &
      \begin{lstlisting}
        fix F 2 = F (fix F) 2
                 = (fix F 1) + (fix F 0)
                 = 1 + 1
                 = 2
        
        fix F 3 = F (fix F) 3
                 = (fix F 2) + (fix F 1)
                 = 2 + 1
                 = 3
        
        fib 4    = fix F 4
                 = F (fix F) 4
                 = (fix F 3) + (fix F 2)
                 = 3 + 2
                 = 5
      \end{lstlisting}
  \end{tabular}
\end{center}

\newpage

\subsection{Direct Style}

\subsubsection{The Semantic Function} 

We define the semantic function as follows: \begin{gather*}
  \mathcal{S}: \textbf{Stm} \to (\textbf{State} \hookrightarrow \textbf{State}).
\end{gather*} We that this function is partial because some statements
may never terminate. We have the following: \begin{align*}
  \mathcal{S}(x = a)(s) &= s[x \mapsto \mathcal{A}(a)(s)], \\
  \mathcal{S}(\texttt{skip}) &= \texttt{id}, \\
  \mathcal{S}(S_1 ; S_2) &= \mathcal{S}(S_2) \circ \mathcal{S}(S_1), \\
  \mathcal{S}(\texttt{if } b \texttt{ then } S_1 \texttt{ else } S_2)
  &= \texttt{cond}(\mathcal{B}(b), \mathcal{S}(S_1), \mathcal{S}(S_2)), \\
  \mathcal{S}(\texttt{while } b \texttt{ do } S)
  &= \texttt{FIX } F \text{ (where } F(g) 
  := \texttt{cond}(\mathcal{B}(b), g \circ \mathcal{S}(S), \texttt{id})).
\end{align*} 
As there is more than one semantic function (for our different types
of semantics) we denote this function as $\mathcal{S}_{ds}$.

\subsection{Continuation Style}

A continuation describes the effect of executing the remainder of the
program denoted as \textbf{Cont} and typed as follows: \begin{gather*}
  \textbf{Cont} = \textbf{State} \hookrightarrow \textbf{State}.
\end{gather*} For a sequence of statements
$\{S_1, S_2, \ldots, S_k, \ldots, S_{n - 1}, S_n\}$
we have that the continuation $c$ from $n$ to $k$ can be extended
to $c'$ the continuation from $n$ to $k - 1$ illustrated here: \begin{gather*}
  S_1, S_2, \ldots, 
  \underbrace{S_{k - 1}, \overbrace{S_k, \ldots, S_{n - 1}, S_n}^c}_{c'}.
\end{gather*}

\newpage

\subsubsection{The Semantic Function on \textbf{While}} 

We define the semantic function as follows: \begin{gather*}
  \mathcal{S}: \textbf{Stm} \to (\textbf{Cont} \hookrightarrow \textbf{Cont}).
\end{gather*} We that this function is partial because some statements
may never terminate. We have the following: \begin{align*}
  \mathcal{S}(x = a)(c)(s) &= c(s[x \mapsto \mathcal{A}(a)(s)]), \\
  \mathcal{S}(\texttt{skip})(c)(s) &= c(s), \\
  \mathcal{S}(S_1 ; S_2) &= \mathcal{S}(S_1) \circ \mathcal{S}(S_2), \\
  \mathcal{S}(\texttt{if } b \texttt{ then } S_1 \texttt{ else } S_2)(c)
  &= \texttt{cond}(\mathcal{B}(b), \mathcal{S}(S_1)(c), \mathcal{S}(S_2)(c)), \\
  \mathcal{S}(\texttt{while } b \texttt{ do } S)
  &= \texttt{FIX } F \text{ (where } F(g)(c) 
  := \texttt{cond}(\mathcal{B}(b), \mathcal{S}(S)(g(c)), c)).
\end{align*} 
As there is more than one semantic function (for our different types
of semantics) we denote this function as $\mathcal{S}_{cs}^{\textbf{While}}$.

\subsubsection{The Semantic Function on \textbf{Exc}} 

\paragraph{Exception environments} we define exeception environments
$env_E$ of type \newline 
$\textbf{Env}_E : \textbf{Exception} \to \textbf{Cont}$.
\\[\baselineskip]
We define the semantic function as follows: \begin{gather*}
  \mathcal{S}: \textbf{Stm} \to \textbf{Env}_E 
  \to (\textbf{Cont} \hookrightarrow \textbf{Cont}).
\end{gather*} We that this function is partial because some statements
may never terminate. We have the following: \begin{align*}
  \mathcal{S}(x = a)(env_E)(c)(s) &= c(s[x \mapsto \mathcal{A}(a)(s)]), \\
  \mathcal{S}(\texttt{skip})(env_E)(c)(s) &= c(s), \\
  \mathcal{S}(S_1 ; S_2)(env_E) &= \mathcal{S}(S_1)(env_E) \circ \mathcal{S}(S_2)(env_E), \\
  \mathcal{S}(\texttt{if } b \texttt{ then } S_1 \texttt{ else } S_2)(env_E)(c)
  &= \texttt{cond}(\mathcal{B}(b), \mathcal{S}(S_1)(env_E)(c), \mathcal{S}(S_2)(env_E)(c)), \\
  \mathcal{S}(\texttt{while } b \texttt{ do } S)(env_E)
  &= \texttt{FIX } F \\
  \text{ (where } F(g)(c) &
  := \texttt{cond}(\mathcal{B}(b), \mathcal{S}(S)(env_E)(g(c)), c)), \\
  \mathcal{S}(\texttt{begin } S_1 \texttt{ handle } e: S_2 \texttt{ end}) \\
  (env_E)(c) &= \mathcal{S}(S_1)(env_E[e \mapsto \mathcal{S}(S_2)(env_E)(c)]) c, \\
  \mathcal{S}(\texttt{raise } e)(env_E)(c) &= env_E e.
\end{align*} 
As there is more than one semantic function (for our different types
of semantics) we denote this function as $\mathcal{S}_{cs}^{\textbf{Exc}}$.