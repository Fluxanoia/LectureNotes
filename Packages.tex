\usepackage[left = 3cm, right = 3cm]{geometry}
\usepackage[english]{babel}
\usepackage[utf8]{inputenc}
\usepackage{ragged2e}
\usepackage{mathtools}
\usepackage{amssymb}
\usepackage{amsmath}
\usepackage{amsthm}
\usepackage{multicol}
\usepackage{multirow}
\usepackage{array}
\usepackage{listings}
\usepackage{xcolor}
\usepackage{color}
\usepackage{graphicx}
\usepackage{arydshln}
\usepackage{pifont}
\usepackage{fancyhdr}
\usepackage{hhline}
\usepackage{ifthen}
\usepackage{bussproofs}
\usepackage{tcolorbox}
\usepackage{textcomp}

% Title Page
%%% title, notation?, inprogress?, unsupported?, unsupported_date, incomplete?
\newcommand{\fluxtitle}[6]{

    \title{#1 Notes}
    \date{}
    \author{by Tyler Wright \\
      \\
      github.com/Fluxanoia $\qquad$ fluxanoia.co.uk
    }
    \maketitle

    \vfill

    \begin{center}

    \noindent
    \textit{These notes are not necessarily correct,
    consistent, representative of the course as it stands today, or 
    rigorous. Any result of the above is not the author's fault.}

    \ifthenelse{
        \boolean{#3}
    }{
        \vspace{\baselineskip}
        \noindent
        \textbf{These notes are in progress.}
    }{}
    
    \ifthenelse{
        \boolean{#4}
    }{
        \vspace{\baselineskip}
        \noindent
        \textbf{These notes are marked as unsupported, they were supported
        up until #5.}
    }{}

    \ifthenelse{
        \boolean{#6}
    }{
        \vspace{\baselineskip}
        \noindent
        \textbf{These notes are incomplete and will remain so for the 
        foreseeable future.}
    }{}

    \end{center}

    \ifthenelse{
        \boolean{#2}
    }{
        \addtocounter{section}{-1}
        \section{Notation}

We commonly deal with the following concepts in 
Language Engineering
which I will abbreviate as follows for brevity:
\begin{center}
    \begin{tabular}{ | r | c | }
        \hline
        Term & Notation \\
        \hline \hline
        \hline
    \end{tabular}
\end{center}
    }{}

    \newpage

    \tableofcontents
}

% lstlistings config
% language
\newcommand{\lstconfig}[1]{
    \lstset{frame=none,
      language=#1,
      aboveskip=3mm,
      belowskip=3mm,
      showstringspaces=false,
      columns=flexible,
      basicstyle={\small\ttfamily},
      numbers=none,
      numberstyle=\tiny\color{gray},
      keywordstyle=\color{blue},
      commentstyle=\color{gray},
      stringstyle=\color{orange},
      breaklines=true,
      breakatwhitespace=true,
      tabsize=2
    }
}

% Math operators

\DeclareMathOperator{\ID}{id}

\DeclareMathOperator{\Max}{max}
\DeclareMathOperator{\Min}{min}

\DeclareMathOperator{\Aut}{Aut}
\DeclareMathOperator{\sgn}{sgn}
\DeclareMathOperator{\Sym}{Sym}
\DeclareMathOperator{\Ord}{ord}
\DeclareMathOperator{\Mod}{mod}
\DeclareMathOperator{\Gcd}{gcd}
\DeclareMathOperator{\Lcm}{lcm}
\DeclareMathOperator{\Orb}{Orb}
\DeclareMathOperator{\Stab}{Stab}
\DeclareMathOperator{\Fix}{Fix}
\DeclareMathOperator{\Syl}{Syl}

\DeclareMathOperator{\Deg}{deg}

\DeclareMathOperator{\Char}{char}
\DeclareMathOperator{\Span}{span}
\DeclareMathOperator{\Dim}{dim}
\DeclareMathOperator{\Ker}{Ker}
\DeclareMathOperator{\Ima}{Im}
\DeclareMathOperator{\Rank}{rank}
\DeclareMathOperator{\Null}{nullity}
\DeclareMathOperator{\End}{End}

\DeclareMathOperator{\Dom}{dom}
\DeclareMathOperator{\Ran}{ran}
\DeclareMathOperator{\Field}{Field}

\DeclareMathOperator{\ot}{ot}
\DeclareMathOperator{\suc}{succ}

% Quick Sylow

\newcommand{\Syls}{Sylow $p$-subgroup }
\newcommand{\Sylg}{Sylow $p$-group }

% Quick Math Fonts

\newcommand{\mb}[1]{\mathbb{#1}}
\newcommand{\mbf}[1]{\mathbf{#1}}
\newcommand{\mc}[1]{\mathcal{#1}}

% Quick function range (Set Theory)

\newcommand{\fran}{{\! \; \text{\textquotesingle \textquotesingle}}}

% Quick pre-superscript

\newcommand{\ps}[1]{{{}^{#1}}}

% Quick langle rangle

\newcommand{\ang}[1]{{\langle #1 \rangle}}

% Quick norm/char subgroups

\newcommand{\nsub}{\trianglelefteq}
\newcommand{\csub}{\underset{\text{char}}{\trianglelefteq}}

% Quick beta reduction

\newcommand{\bred}{\twoheadrightarrow_\beta}

% Quick church numeral

\newcommand{\church}[1]{
  \ulcorner
  #1
  \urcorner
}

% Quick type constraint

\newcommand\teq{\stackrel{
  \mathclap{\text{\scriptsize\mbox{?}}}
}{=}}

% Nicer symbols

%%% Empty Set

\let\oldemptyset\emptyset
\let\emptyset\varnothing

%%% Lambda

\makeatletter
\newcommand\Pimathsymbol[3][\mathord]{%
  #1{\@Pimathsymbol{#2}{#3}}}
\def\@Pimathsymbol#1#2{\mathchoice
  {\@Pim@thsymbol{#1}{#2}\tf@size}
  {\@Pim@thsymbol{#1}{#2}\tf@size}
  {\@Pim@thsymbol{#1}{#2}\sf@size}
  {\@Pim@thsymbol{#1}{#2}\ssf@size}}
\def\@Pim@thsymbol#1#2#3{%
  \mbox{\fontsize{#3}{#3}\Pisymbol{#1}{#2}}}
\makeatother
\input{utxmia.fd}
\DeclareFontShape{U}{txmia}{m}{n}{<->ssub * txmia/m/it}{}
\DeclareFontShape{U}{txmia}{bx}{n}{<->ssub * txmia/bx/it}{}
\newcommand{\pilambdaup}{\Pimathsymbol[\mathord]{txmia}{21}}
\let\oldlambda\lambda
\let\lambda\pilambdaup

%%% Chi

\newcommand{\varchi}{\protect\raisebox{2pt}{$\chi$}}

%%% Double Plus

\newcommand\doubleplus{+\kern-1.3ex+\kern0.8ex}
\newcommand\mdoubleplus{\ensuremath{\mathbin{+\mkern-10mu+}}}

%%% Fraction without line

\newcommand*{\bfrac}[2]{\genfrac{}{}{0pt}{}{#1}{#2}}

%%% Under/over arrow

\makeatletter
\newcommand*{\underarrow}{\def\@underarrow{\relax}\@ifstar{\@@underarrow}{\def\@underarrow{\hidewidth}\@@underarrow}}
\newcommand*{\@@underarrow}[2][]{\underset{\@underarrow\substack{\uparrow\if\relax\detokenize{#1}\relax\else\\#1\fi}\@underarrow}{#2}}

\newcommand*{\overarrow}{\def\@overarrow{\relax}\@ifstar{\@@overarrow}{\def\@overarrow{\hidewidth}\@@overarrow}}
\newcommand*{\@@overarrow}[2][]{\overset{\@overarrow\substack{\if\relax\detokenize{#1}\relax\else#1\\\fi\downarrow}\@overarrow}{#2}}
\makeatother
