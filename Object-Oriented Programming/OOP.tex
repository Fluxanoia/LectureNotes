\documentclass[a4paper, 12pt, twoside]{article}
\usepackage[left = 3cm, right = 3cm]{geometry}
\usepackage[english]{babel}
\usepackage[utf8]{inputenc}
\usepackage{amssymb}
\usepackage{amsmath}
\usepackage{multicol}

\begin{document}

\title{Object-Oriented Programming Notes}
\date{}
\author{written by Tyler Wright}
\maketitle

\vfill

\textit{An important note, these notes are absolutely \textbf{NOT}
      guaranteed to be correct, representative of the course, or rigorous.
      Any result of this is not the author's fault.}

\newpage

\section{Classes, Objects, and the Basics of Java}

\subsection{Definition of a Class}

A class is a type for variables and objects, encapsulating many key
ideas, it is:
\begin{itemize}
    \item A structure with fields
    \item A closed system
    \item A module
    \item A namespace (to avoid clashes)
    \item A unit (for testing).
\end{itemize}

\subsection{Definition of an Object}

An object is an entity that combines both data (attributes) and
methods uniquely, objects...
\begin{itemize}
    \item ...can have a state and/or act
    \item ...can have a unique identity
    \item ...are responsible for their own data
    \item ...can communicate with other objects via methods
    \item ...can expose its own information.
\end{itemize}
Object-oriented programming is (quite obviously) based around this
concept.

\subsection{Instances}

Every object is an \textbf{instance} of a class, or rather, an object
is an instantiation of a class. In this way, classes are like a
blueprint for objects.

\vspace{\baselineskip}

Objects are \textbf{constructed} by using the constructor for a class
to instantiate the class. In Java, the constructor always has the
same name as the class.

\newpage

\subsection{The Difference between Classes and Objects}

\begin{multicols}{2}
    Classes are:
    \begin{itemize}
        \item Catergorised
        \item Compile time
        \item Blueprints
        \item Static
        \item Immutable
    \end{itemize}
    \columnbreak
    Objects are:
    \begin{itemize}
        \item Instances
        \item Runtime
        \item Entities
        \item Dynamic
        \item Mutable
    \end{itemize}
\end{multicols}

\subsection{The Java Toolchain}

In Java, the written \texttt{.java} files are compiled into 
\texttt{.class} files which are run by the \textbf{JVM} (Java
Virtual Machine).

\subsection{The Java Entry Point}

Java programs start in a \texttt{main} function which takes in
an array of strings (command line arguments).

\subsection{Exception Handling}

In Java, a \texttt{try-catch} block is used to catch exceptions
at runtime.

\subsection{Exit Codes}

In Java, we can use \texttt{System.exit(i)} to close the program at
anytime. The meaning of the \texttt{int i} passed to the function
is defined by the programmer, but in general 0 is good, anything
non-zero is an error.

\subsection{Immutable Objects}

Immutable objects cannot have their value change after declaration
and their value must be assigned on declaration. The \texttt{final}
keyword is used for this in Java. Final methods cannot be overriden
and final classes cannot be extended.

\section{Memory}

\subsection{The Stack}

The stack is the memory given to a thread of execution.
When a function is called, a block is reserved on the stack for
local variables and bookkeeping data. This memory is freed once
the function ends.

\subsection{The Heap}

The heap is the memory given to dynamic allocation. There's no
pattern to the management of blocks on the heap, it can be
allocated and freed at any time. This memory is used for large
data to avoid overflowing the stack.

\subsection{The Lifecycle of Objects}

Once objects no longer have any references to them, they are
eligible for garbage collection by the Java garbage collector.
This is a system that deallocated unused memory in Java. So,
at the end of function calls, all variables local to that function
become unused and are collected.

\subsection{Pass by value}

Primitive types such as \texttt{int}, \texttt{double} and, 
\texttt{boolean} are passed by value.

\vspace{\baselineskip}

Objects in Java are also passed by value but a \textit{reference}
to the object is passed. This means we can call the methods of the
object and have its values change, but we cannot reassign the
variable inside a function.

\subsection{Equality}

It is important to note that \texttt{==} is reference equality
in Java so two strings with the same underlying data may not
be equal. This is why we have \texttt{String.equals} to test
string equality.

\newpage

\section{Scope}

\subsection{The \texttt{this} Keyword}

\texttt{this} references to the current object whose method
is begin executed. So, the use of this keyword will throw
errors in places where there is no object such as static methods.

\subsection{The Private Modifier}

Private elements are class-scope. Only the class can directly
change or use the element, this underpins encapsulation.

\subsection{The Default (or no) Modifier}

Default elements are package-scope.

\subsection{The Static Modifier}

Static elements are associated to a class rather than an object.
This means they are processed at compile time and can be accessed
\textbf{without} instantiation.

\vspace{\baselineskip}

Static functions cannot use other non-static members unless given
to the function or instantiated inside the function.

\vspace{\baselineskip}

One key use of static is the \texttt{main} function, the entry point
of the program.

\section{Inheritance}

If we need multiple classes with similar attributes, it makes sense
for them all to inherit core functionality from a super (parent) class.
This can be achieved with the \texttt{extends} keyword.

\vspace{\baselineskip}

Additionally, inheriting classes can override the super classes methods
to implement their own functionality (overriding the super classes
base functionality). This can be similarly done by adding the
\texttt{abstract} modifer to your attributes or methods but this will
mean there is no default implementation and the inheriting subclass
must implement the method.

\vspace{\baselineskip}

The abstract modifier can also be applied to classes where it means
it can only be instantiated as a super class.

\subsection{Polymorphism}

When we have classes (say \texttt{ClassTest}) inheriting other Classes
(say \texttt{Test}) we can refer to the subclass as an instance
of the super class (all \texttt{ClassTest} objects are also
\texttt{Test} objects). This doesn't change the object, it just changes
how the object is viewed, so to speak.

\subsection{Interfaces}

We can only extend a single class in Java, but it would be nice to
be able to implement multiple classes in one place, this is where
interfaces come in. They are similar to abstract classes with
only final abstract methods and no attributes - they are essentially
a set of methods your subclass must implement.

\subsection{Encapsulation}

Encapsulation is all about keeping a classes functionality identical
or adding more features without changing the implementation of the
class. 

\vspace{\baselineskip}

This is accomplished by only having class data accessed by
specified 'getter' and 'setter' methods, thus we can change the
attributes and methods of the class at will as long as the
getters and setters function as they did before.

\subsection{Single Dispatch}

Single dispatch is when we call a method of an object but the call is
overridden by a method in the subclass of the referenced object.
Such a decision is made at runtime.

\subsection{Double Dispatch}

Double dispatch is used to resolve interactions between two objects
of known super class but unknown subclass. By making each subclass
implement a set of overloaded methods (\texttt{visit} for example)
with a definition for each subclass and one generic implementation.

\vspace{\baselineskip}

The generic implementation will call the specific implementation of
the other object as it knows its own type (of course). Thus,
the other object then knows what is calling it and its own type
so can produce the corresponding reaction.

\subsection{Downcasting}

Downcasting is a last resort programming technique where classes
are casted to their subclasses. This can lead to complications if
done incorrectly as there's no guarantee that it is in fact
an instance of the subclass. We can use generics to avoid this
technique most of the time.

\subsection{Generics}

We can define classes of a generic type to make them more versatile.
For example, we can have a \texttt{Collection} of any instantiable
object in Java.

\section{Patterns}

\subsection{The Visitor Pattern}

The visitor pattern allows a specific method within a visitor class
to be called by a specific visiting class. So, in a Java board game, 
if we had a game class that processed different types of moves
with a generic move super class we could declare the game class
as a \textbf{visitor}.

\vspace{\baselineskip}

Then, the moves would be the \textbf{visiting} classes. In our
visitor, we would define an accepting move for each move subclass
and then use a method call on the visiting classes to have them
'visit' the visitor. As the visiting class knows its type, it will
call the specific method.

\subsection{The Iterator Pattern}

The iterator pattern is used to sequentially access the
elements of an object without exposing it's underlying structure.
This uses encapsulation as it doesn't rely on how the class
stores or handles its values, it just requires that it be
iterable.

\vspace{\baselineskip}

In this pattern, we have an iterable class and an iterator class.
The iterator will use the iterable interface to extract values
from the class and handle the indexing.

\newpage

\subsection{The Strategy Pattern}

The strategy pattern is used to define a set of algorithms that
can be specifically chosen to carry out specific behaviour.

\vspace{\baselineskip}

We have a context where we require that we can use multiple different
algorithms (strategies) at will. So, we define specific strategy
subclasses and these can be passed to the context who can
use the implementation of the strategy super class to carry
out the strategy on itself.

\subsection{The Observer Pattern}

In this pattern, we have a main subject class and
many observer classes. The subject class registers and
unregisters observers and can announce to update observer classes.
There are two main types of observer, push and pull:
\begin{itemize}
    \item Push observer subjects send their state to the observer when
    it updates
    \item Pull observer observers must request the state from the 
    observer    
\end{itemize}

\subsection{The Composite Pattern}

The composite pattern is used to store multiple components in a
tree-like structure. Each component can store a simple component
with no children or composite classes that can contain a set of
more components.

\subsection{The Model-View-Controller}

This is a pattern of patterns and has no specific correct definition.
It consists of a model and a view and the essence of the idea is 
regulating their communication thorough specific classes.

\vspace{\baselineskip}

If we allow interaction between these intermediary classes, we get
a Model-View-Presenter pattern. 

\end{document}