\documentclass[a4paper, 12pt, twoside]{article}
\usepackage[left = 3cm, right = 3cm]{geometry}
\usepackage[english]{babel}
\usepackage[utf8]{inputenc}
\usepackage{amssymb}
\usepackage{amsmath}
\usepackage{multicol}

\DeclareMathOperator{\Ord}{ord}
\DeclareMathOperator{\Mod}{mod}
\DeclareMathOperator{\Gcd}{gcd}
\DeclareMathOperator{\Lcm}{lcm}

\DeclareMathOperator{\Ker}{Ker}
\DeclareMathOperator{\Ima}{Im}

\begin{document}

\title{Introduction to Group Theory Notes}
\date{}
\author{\textit{paraphrased by} Tyler Wright}
\maketitle

\vfill

\textit{An important note, these notes are absolutely \textbf{NOT}
      guaranteed to be correct, representative of the course, or rigorous.
      Any result of this is not the author's fault.}

\newpage

\section{The Basics of Groups}

\subsection{Binary operations}

A binary operation on a set $G$ is a function:
\begin{align*}
      * : G \times G \to G.
\end{align*}

\textit{It's just a function that takes two values and gives a single
      output. Examples are addition, multiplication, and composition.}

\vspace{\baselineskip}

Such an operation is called \textbf{commutative} if:
\begin{align*}
      x * y = y * x. \tag{$\forall x, y \in G$}
\end{align*}

\subsection{Definition of a Group}

A group is a set $G$ paired with a binary operation $*$ such that they satisfy
the following:

\begin{itemize}
      \item \textbf{Associativity}: For $x, y, z \in G$,
            $(x * y) * z = x * (y * z)$
      \item \textbf{Identity}: $\exists \, e \in G$ such that
            $\forall g \in G$, $e * g = g * e = g$
      \item \textbf{Inverses}: $\forall g \in G$, $\exists \, g^{-1} \in G$
            such that $g * g^{-1} = g^{-1} * g = e$.
\end{itemize}

A group is called commutative or Abelian if all its elements commute with
the given operation.

\subsection{Consequences of the Definition}

\subsubsection{Left and right cancellation}

We can left and right cancel with inverses:
\begin{align*}
      (ax = bx) & \Rightarrow (a = b) \tag{$\forall a, b, x \in G$}  \\
      (xa = xb) & \Rightarrow (a = b). \tag{$\forall a, b, x \in G$}
\end{align*}

\textit{However, $ax = xb$ does not imply $a = b$ unless the group is Abelian.}

\subsubsection{Uniqueness of the identity and inverses}

We have uniqueness of certain elements:

\begin{itemize}
      \item The identity of a group is unique
      \item The inverse of an element is unique.
\end{itemize}

\subsubsection{Inverse properties}

For a group $G$ with elements $x, y$:

\begin{itemize}
      \item $(x^{-1})^{-1} = x$
      \item $(xy)^{-1} = y^{-1}x^{-1}$.
\end{itemize}

\subsubsection{Exponent properties}

For a group $G$ with an element $x$ and $m, n$ in $\mathbb{Z}$:

\begin{itemize}
      \item $x^{-n} = (x^{-1})^n$
      \item $(x^n)(x^m) = x^{n + m}$.
\end{itemize}

\textit{However, $(xy)^n$ may not equal $x^ny^n$ unless $G$ is Abelian.}

\section{Dihedral Groups}

\subsection{Definition of a Dihedral Group}

The dihedral group $D_{2n}$ is the group of symmetries of an $n$-sided polygon.
This group has order $2n$ as is defined as:
\begin{align*}
      D_{2n} & = \langle a \rangle \cup b\langle a \rangle                        \\
             & = \{e, a, a^2, \ldots, a^{n - 1}, b, ba, ba^2, \ldots, ba^{n - 1}\}.
\end{align*}
Where $a$ is a rotation of $\frac{2\pi}{n}$ radians around the centre of the
polygon and $b$ is a reflection in the line through vertex $1$ and the centre
of the polygon.

\subsection{Properties of a Dihedral Group}

For the dihedral group $D_{2n}$:

\begin{itemize}
      \item $a^n = e$
      \item $b^2 = e$
      \item $a^nb = ba^{-n}$
\end{itemize}

\section{Subgroups}

\subsection{Definition of a Subgroup}

A subgroup is a subset $H$ of a group $G$ such that $H$ is also a
group under the binary operation defined by $G$ ($H \leq G$). If we have a subset
$H$ of a group $G$, we can show it is a subgroup by showing the
following properties hold for $H$:

\begin{itemize}
      \item \textbf{Closure}: For $x, y \in H$, $xy \in H$
      \item \textbf{Identity}: $\exists \, e \in H$ such that
            for $x \in H$, $e * x = x * e = x$
      \item \textbf{Inverses}: For $x \in H$, $\exists \, x^{-1} \in
                  H$ such that $x * x^{-1} = x^{-1} * x = e$.
\end{itemize}

A consequence of this definition is that the intersection of subgroups
is a subgroup.

\section{The Order of Elements}

\subsection{The Definition of Order for Elements}

For $x$ an element in some group $G$, we have that the order of $x$
is defined by:
\begin{align*}
      \Ord{(x)} = \begin{cases}
            n \text{ such that } x^n = e & \text{if such $n$ exists} \\
            \infty                       & \text{otherwise.}
      \end{cases}
\end{align*}

\textit{The order is the \textbf{least} possible integer such that
      $x^n = e$. To show the order of $x$ is $n$, you need to show
      $x^n = e$ and $x^k \neq e$ for all $k \in \{1, 2, \ldots, n - 1\}$.}

\subsection{Properties of the Order of Elements}

Let $G$ be a group with element $x$:

\begin{itemize}
      \item $\Ord(x) = \infty \Rightarrow$
            all $x^i$ are distinct $(i \in \mathbb{Z})$
      \item $|G| < \infty \Rightarrow \Ord(x) < \infty$
      \item If $\Ord(x) = n \in \mathbb{N}$, for $i \in \mathbb{N}$,
            $\Ord(x^i) = \frac{n}{\gcd{(n, i)}}$.
\end{itemize}

\newpage

\section{Cyclic Groups}

\subsection{Definition of a Cyclic Group}

For a group $G$, the cyclic group generated by $x$ in $G$ is defined by:
\begin{align*}
      \langle x \rangle = \{x^i : i \in \mathbb{N}\}.
\end{align*}

\subsection{Properties of Cyclic Groups}

For a group $G$ with element $x$:

\begin{itemize}
      \item $\langle x \rangle$ is a subgroup of $G$
      \item $|\langle x \rangle| = \Ord(x)$
      \item Cyclic groups are Abelian
      \item Subgroups of cyclic groups are cyclic
      \item $G$ is cyclic $\Leftrightarrow \exists \, x \in G$ such
            that $\Ord(x) = |G|$.
\end{itemize}

\section{Groups from Modular Arithmetic}

\subsection{Congruence Classes}

A congruence class $[a]$ of the set $\mathbb{Z}/n\mathbb{Z}$ is a set
of integers congruent to $a$ $(\Mod n)$. We define the following
operations:

\begin{itemize}
      \item \textbf{Addition}: $[a] + [b] = [a + b]$
      \item \textbf{Multiplication}: $[a][b] = [ab]$.
\end{itemize}

\textit{For example:}
\begin{align*}
      \mathbb{Z}/7\mathbb{Z} & = \bigcup_{i = 0}^{6} \, [i],
\end{align*}
\textit{with distinct elements $0, 1, 2, 3, 4, 5, 6$.}

\newpage

\subsection{The Set of Congruence Classes under Addition}

We have that the set $\mathbb{Z}/n\mathbb{Z}$ with the operation
of addition $(\mathbb{Z}/n\mathbb{Z}, +)$ is a cyclic group
generated by $1$.

\vspace{\baselineskip}

\textit{This means it's also an Abelian group.}

\subsection{The Set of Congruence Classes under Multiplication}

The trouble with multiplication is that certain congruence classes
never have inverses and as a result, the set under multiplication
can never be a group. We have that an element $[a]$ of
$(\mathbb{Z}/n\mathbb{Z}, \times)$ has an inverse if:
\begin{align*}
      \gcd{(a, n)} = 1.
\end{align*}
We define the set $U_n$ as follows:
\begin{align*}
      U_n = \{a : a \in \mathbb{Z} \text{ with } \gcd{(a, n) = 1}\}.
\end{align*}
Thus, we have $(U_n, \times)$ is an Abelian group.

\subsection{The Set of Congruence Classes under the Direct Product}

For $m, n$ positive integers with $\Gcd(m, n) = 1$, we have:
\begin{align*}
      U_m \times U_n \cong U_{mn}.
\end{align*}

\section{Isomorphisms}

\subsection{Definition of an Isomorphisms}

For $(G, *)$, $(H, \circ)$ groups, an isomorphism $\phi : G \to H$
is a bijective function such that:
\begin{align*}
      \phi(x * y) = \phi(x) \circ \phi(y). \tag{$\forall x, y \in G$}
\end{align*}

\newpage

\subsection{Properties of Isomorphisms}

For the groups $G, H, K$ and an isomorphism $\phi : G \to H$:

\begin{multicols}{2}
      \begin{itemize}
            \item $\phi^{-1}$ is an isomorphism
            \item $G$ and $H$ are isomorphic ($G \cong H$)
            \item If there exists an isomorphism $\psi:H \to K$
                  then $G \cong K$ (transitive)
            \item $\phi(e_G) = e_H$
            \item $\phi(x^{-1}) = \phi(x)^{-1}$
      \end{itemize}
      \columnbreak
      \begin{itemize}
            \item $\phi(x^i) = \phi(x)^i$ ($i \in \mathbb{Z}$)
            \item $\Ord_G(x) = \Ord_H(\phi(x))$
            \item $|G| = |H|$
            \item $G$ is Abelian $\Leftrightarrow$ $H$ is Abelian
            \item $G$ is cyclic $\Leftrightarrow$ $H$ is cyclic
      \end{itemize}
\end{multicols}

\section{Direct Products}

\subsection{Definition of the Direct Product}

For $G, H$ groups, $G \times H$ is the Cartesian product of $G$ and
$H$ with the binary operation:
\begin{align*}
      (x, y)(a, b) = (x * a, y * b). \tag{$\forall x, a \in G, y, b \in H$}
\end{align*}
This is itself a group.

\subsection{Properties of the Direct Product}

For $H, K$ groups, $G = H \times K$:

\begin{itemize}
      \item $G$ is finite $\Leftrightarrow$ $H$ and $K$ are finite
            (in this case $|G| = |H||K|$)
      \item $G$ is Abelian $\Leftrightarrow$ $H$ and $K$ are Abelian
      \item $G$ is cyclic $\Rightarrow$ $H$ and $K$ are cyclic.
\end{itemize}

\subsection{The Direct Product and Cyclic Groups}

\subsubsection{Order of elements}

For $H, K$ groups, $G = H \times K$, $(x, y)$ in $G$:
\begin{align*}
      \Ord(x, y) = \Lcm(\Ord_H(x), \Ord_K(y)).
\end{align*}

\subsubsection{Condition for a cyclic direct product}

For $H, K$ finite cyclic groups, $G = H \times K$, $G$ is cyclic if and only if
$\Gcd(|H|, |K|) = 1$.

\subsubsection{The direct product of cyclic groups}

We denote the cyclic group of order $n$ as $C_n$. We have that
for $C_n$, $C_m$ cyclic groups:
\begin{align*}
      C_n \times C_m \cong C_{mn} \Leftrightarrow \Gcd(m, n) = 1.
\end{align*}

\section{Lagrange's Theorem}

\subsection{Definition of Lagrange's Theorem}

For a finite group $G$ with $H \leq G$ a subgroup. We have that
$|H|$ divides $|G|$.

\subsection{Cyclic Subgroups}

For $G$ a finite group with order $n$, for $x$ in $G$, $\Ord(x)$ divides
$n$ (this is because $\langle x \rangle \leq G$).

\subsection{Cosets}

\subsubsection{Definition of a coset}

For a group $G$ with $H \leq G$ and $x$ in $G$, the left coset $xH$ is
and right coset $Hx$ are the sets:
\begin{align*}
      xH = \{xh : h \in H\}, \, Hx = \{hx : h \in H\}.
\end{align*}
\textit{While this is a sub\textbf{set} of $G$, it is not necessarily
      a sub\textbf{group}.}

\subsubsection{A bijection from a subgroup to its left coset}

For a group $G$ with $H \leq G$, $x$ in $G$, and left coset $xH$,
there existsu a bijection from $H$ to $xH$. This implies that their
order is the same.

\subsubsection{The intersection of cosets}

For a group $G$ with $H \leq G$, $x, y$ in $G$:
\begin{align*}
      xH \cap yH \neq \emptyset \Leftrightarrow xH = yH.
\end{align*}
\textit{Cosets are distinct unless they are equal.}

\subsubsection{Index of a subgroup}

For a group $G$ with $H \leq G$, the index of $H$ in $G$
$|G : H|$ is the number of left cosets of $H$ in $G$. So, since
all cosets of $H$ are distinct, we have:
\begin{align*}
      |G| = |H||G : H|.
\end{align*}

\subsection{Consequences of Lagrange's Theorem}

\subsubsection{Intersection of subgroups}

For a group $G$ with $H, K \leq G$, $\Gcd(|H|, |K|) = 1$ implies
$H \cap K = \{e\}$.

\subsubsection{Prime order groups}

For $G$ a group with $|G| = p \in \mathbb{P}$ (prime):

\begin{itemize}
      \item $G$ is cyclic
      \item Every element of $G$ except the identity has order $p$
            (and generates $G$)
      \item The only subgroups of $G$ are $G$ and $\{e\}$.
\end{itemize}

\section{Fermat-Euler Theorem}

\subsection{Euler's $\phi$ Function}

We define the Euler $\phi$ function over the naturals by:
\begin{align*}
      \phi(n) = \left|\{a : a \in \mathbb{N}, \, \Gcd(a, n) = 1\}\right|.
\end{align*}
We have that $\phi(n)$ is the order of $U_n$ (the group of congruence
classes under multiplication). Also, for $p$ in $\mathbb{P}$ (prime),
$\phi(p) = p - 1$.

\vspace{\baselineskip}

\textit{This is the number of values less than or equal to an integer
      that don't divide it.}

\subsection{Fermat-Euler Theorem}

For $a, n$ in $\mathbb{N}$ with $\Gcd(a, n) = 1$, we have that:
\begin{align*}
      a^{\phi(n)} \equiv 1 \, (\Mod n).
\end{align*}
So, for $p$ in $\mathbb{P}$ (prime):
\begin{align*}
      a^{p - 1} \equiv 1 \, (\Mod p).
\end{align*}

\section{Symmetric Groups}

\subsection{Definition of a Symmetric Group}

For a set $X$, $S(X)$ is the group of all symmetries of $X$.
For $n$ in $\mathbb{N}$, $S_n$ is the group of all symmetries of
$\{1, \ldots, n\}$. We have that $|S_n| = n!$.

\subsection{$k$-cycles in $S_n$}

\subsubsection{Definition of a $k$-cycle}

For $k, n$ in $\mathbb{N}$ with $k \leq n$. A $k$-cycle $f$ in $S_n$
is a permutation of the $k$ distinct elements $\{i_1, i_2, \ldots,
      i_k\}$ in $\{1, \ldots, n\}$ of the form:
\begin{gather*}
      f(i_1) = i_2, f(i_2) = i_3, \ldots, f(i_k) = f(i_1) \\
      f = (i_1, i_2, i_3, \ldots, i_k).
\end{gather*}

\subsubsection{Properties of $k$-cycles}

For $f$ in $S_n$ a $k$-cycle:

\begin{itemize}
      \item $f$ has order $k$
      \item $\Ord(f) = 2 \Rightarrow f$ is a \textbf{transposition}.
\end{itemize}

\subsection{Disjoint Cycles}

\subsubsection{Definition of a disjoint cycle}

We call a set of cycles disjoint if no element of $\{1, \ldots, n\}$ is
moved by more than one of the cycles.

\subsubsection{Elements of $S_n$ as a product of disjoint cycles}

We have that for all $f$ in $S_n$, $f$ can be written as a product
of disjoint cycles.

\subsubsection{Order of elements of $S_n$}

For $f$ in $S_n$ with $f = (f_1)(f_2)\cdots(f_k)$ a product of
disjoint cycles:
\begin{align*}
      \Ord(f) = \Lcm(\Ord(f_1), \Ord(f_2), \ldots, \Ord(f_k)).
\end{align*}

\section{Transpositions}

\subsection{Elements of $S_n$ as a Product of Transpositions}

We have that for all $f$ in $S_n$, $f$ can be written as a product
of transpositions.

\subsection{Odd and Even Permutations}

\subsubsection{Definition of odd and even permutations}

For each $f$ in $S_n$, write $f$ as the product of transpositions,
let $k$ be the number of transpositions needed:
\begin{itemize}
      \item $f$ is odd if $k$ is odd
      \item $f$ is even if $k$ is even.
\end{itemize}

\subsubsection{Composition of Permutations}

For $f, g$ in $S_n$, we have that:

\begin{itemize}
      \item $f, g$ both odd or both even $\Rightarrow$ $fg$ even
      \item $f, g$ odd and even (or vice-versa)
            $\Rightarrow$ $fg$ odd.
\end{itemize}

\subsubsection{$k$-cycles}

For $f$ in $S_n$ a $k$-cycle:

\begin{itemize}
      \item $k$ odd $\Rightarrow$ $f$ even
      \item $k$ even $\Rightarrow$ $f$ odd.
\end{itemize}

\subsubsection{The alternating group}

Let $A_n$ be the set of all even permutations of $S_n$, we have:

\begin{itemize}
      \item $|A_n| = \frac{|S_n|}{2} \, (n \geq 1)$
      \item $A_n \leq S_n$.
\end{itemize}

\newpage

\section{Homomorphisms}

\subsection{Definition of a Homomorphism}

For $(G, *)$, $(H, \circ)$ groups, a homomorphism $\phi : G \to H$
is a function such that:
\begin{align*}
      \phi(x * y) = \phi(x) \circ \phi(y). \tag{$\forall x, y \in G$}
\end{align*}

\textit{This is an isomorphism without the requirement of bijectivity.}

\subsection{Properties of Homomorphisms}

For the groups $G, H$ and a homomorphism $\phi : G \to H$:

\begin{itemize}
      \item $\phi(e_G) = e_H$
      \item $\phi(x^{-1}) = \phi(x)^{-1}$
      \item $\phi(x^i) = \phi(x)^i$ ($i \in \mathbb{Z}$)
\end{itemize}

\subsection{Trivial Homomorphisms}

For the groups $G, H$, the following are all homomorphisms:

\begin{itemize}
      \item $\phi: G \to H; \, \phi(x) = e_H$
      \item $\phi: G \to G \times H; \, \phi(g) = (g, e_H)$
      \item $\phi: H \to G \times H; \, \phi(h) = (e_G, h)$
      \item $\phi: G \times H \to G; \, \phi(g, h) = g$
      \item $\phi: G \times H \to H; \, \phi(g, h) = h$.
\end{itemize}

\subsection{The Kernel and Image}

For the groups $G, H$ and a homomorphism $\phi : G \to H$:

\begin{itemize}
      \item $\Ker(\phi) = \{x : x \in G, \, \phi(x) = e_H\} \leq G$
      \item $\Ima(\phi) = \{\phi(x) : x \in G\} \leq H$.
\end{itemize}

\subsection{Injectivity}

For the groups $G, H$ and a homomorphism $\phi : G \to H$,
$\phi$ is injective if and only if $\Ker(\phi) = \{e_G\}$.

\newpage

\section{Normal Subgroups}

\subsection{Definition of Normal Subgroups}

A normal subgroup of group $G$ is a subgroup $N \leq G$
such that $gNg^{-1} = N$ for all $g \in G$. This is denoted by
$N \unlhd \, G$.

\vspace{\baselineskip}

\textit{We have, $gNg^{-1} = N \Leftrightarrow gN = Ng$. So, we can show a
group is a normal subgroup by showing the left and right
cosets are the same for a given $g$.}

\subsection{Abelian Groups}

All subgroups of Abelian groups are normal.

\subsection{The Kernel of Homomorphisms}

For the groups $G, H$ and a homomorphism $\phi : G \to H$,
$\Ker(\phi)$ is a normal subgroup of $G$.

\section{Quotient Groups}

\subsection{Definition of Quotient Groups}

For $G$ a group with $N \unlhd \, G$ a normal subgroup, the quotient
group $G/N$ is the set of cosets of $N$ in $G$ with the binary
operation defined for $x, y$ in $G$ by:
\begin{align*}
      (xN)(yN) = (xy)N.
\end{align*}

\subsection{The Quotient Homomorphism}

For $G$ a group with $N \unlhd \, G$ a normal subgroup, we can 
define a homomorphism $\phi$ from $G$ to the quotient group $G/N$:
\begin{gather*}
      \phi: G \to G/N \\
      \phi(g) = gN.
\end{gather*}

\textit{It's easy to see that this is surjective also.}

\newpage

\section{The Homomorphism Theorem}

We have that for the groups $G, H$ with a homomorphism 
$\phi : G \to H$, $\Ker(\phi) \unlhd \, G$. So, it makes sense to
construct the quotient group $G/\Ker(\phi)$. We have that this
group is isomorphic to $\Ima(\phi)$.

\section{Group Actions}

\subsection{Definition of a Group Action}

A group action of a group $G$ on a set $X$ is a function 
$(\cdot) : G \times X \to X$ where for all $x$ in $X$, $g, h$ in $G$:
\begin{itemize}
      \item $e \cdot x = x$ 
      \item $g \cdot (h \cdot x) = (gh) \cdot x$.
\end{itemize}

\subsection{The Trivial Group Action}

For $G$ a group, we have $(\cdot) : G \times G \to G$ the
trivial group action defined for $g, h$ in $G$ by:
\begin{align*}
      g \cdot h = gh.
\end{align*}

\subsection{Bijective Functions from Group Actions}

For a group $G$ acting on a set $X$, we have that for each $g$ in 
$G$, $f$ is bijective defined by:
\begin{gather*}
      f: X \to X \\
      f(x) = g \cdot x.
\end{gather*}

\newpage

\subsection{The Orbit and Stabiliser}

\subsubsection{Definition of the orbit and the stabiliser}

For $G$ acting on $X$ with $x$ in $X$:
\begin{itemize}
      \item The orbit of $x$ ($G \cdot x$) is defined by: $$
      G \cdot x = \{g \cdot x : g \in G\}.
      $$
      \item The stabiliser of $x$ ($G_x$) is defined by: $$
      G_x = \{g : g \in G, g \cdot x = x\}.
      $$
\end{itemize}

\textit{So, the orbit of an element is everything that it can
be mapped to under the group action. The stabiliser of an 
element $x$ is the set of elements that have no effect on $x$
under the group action. To loosely put it, the 'identities' of $x$.}

\subsubsection{Disjoint property of orbits}

For $G$ acting on $X$ with $x, y$ in $X$, $G \cdot x$ and $G \cdot y$
are either disjoint or equal. So, we have that $X$ is partitioned
into orbits so that each element of $x$ exists in exactly one orbit.

\subsubsection{Subgroup property of stabilisers}

For $G$ acting on $X$ with $x$ in $X$, $G_x$ is a subgroup of $G$.

\subsubsection{The orbit-stabiliser theorem}

For $G$ acting on $X$ with $x$ in $X$:
\begin{gather*}
      |G:G_x| = |G \cdot X|,
\end{gather*}
and if $G$ is finite:
\begin{gather*}
      |G| = |G \cdot X||G_x|
\end{gather*}

\textit{So, we have that the number of cosets of the stabiliser
in $G$ is equal to the amount of elements in the orbit. The 
second result follows from:}
\begin{gather*}
      |G:G_x| = \frac{|G|}{|G_x|},
\end{gather*}
\textit{if $G$ is finite as $G_x$ is a subgroup.}

\end{document}