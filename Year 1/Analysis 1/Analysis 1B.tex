\documentclass[a4paper, 12pt, twoside]{article}
\usepackage[left = 3cm, right = 3cm]{geometry}
\usepackage[english]{babel}
\usepackage[utf8]{inputenc}
\usepackage{amssymb}
\usepackage{amsmath}
\usepackage{multicol}

\begin{document}

\title{Analysis 1 (TB2) Notes}
\date{}
\author{\textit{paraphrased by} Tyler Wright}
\maketitle

\vfill

\textit{An important note, these notes are absolutely \textbf{NOT}
      guaranteed to be correct, representative of the course, or rigorous.
      Any result of this is not the author's fault.}

\newpage

\section{Continuity}

\subsection{Continuous Functions}

From Analysis 1A, we have that a function $f: A \to \mathbb{R}$
is continuous on $A$ if:
\begin{align*}
      \forall x \in A, \forall \epsilon > 0, \exists\,\delta > 0
      \text{ such that } \forall y \in A,
      (|y - x| < \delta) \Rightarrow (|f(y) - f(x)| < \epsilon).
\end{align*}
It's important to note that $x$ is chosen given before we choose a
$\delta$. Thus, our choice for $\delta$ can depend on $x$ as well as
$\epsilon$.

\vspace{\baselineskip}

\textbf{Uniform} continuity requires that $\delta$ is independent of
$x$.

\vspace{\baselineskip}

\textit{A note, a function being continuous at a value (or set
      of values for that matter), it equivalent to saying that
      there exists a limit for the function at that value and that limit
      is the value of the function applied to that value.}

\subsection{Uniformly Continuous Functions}

Uniform continuity is similar to continuity as we knew it in Analysis
1A. For a function $f : A \to \mathbb{R}$, $f$ is uniformly continuous
on $A$ if:
\begin{align*}
      \forall \epsilon > 0, \exists\,\delta > 0
      \text{ such that } \forall x, y \in A,
      (|y - x| < \delta) \Rightarrow (|f(y) - f(x)| < \epsilon).
\end{align*}
We can see that uniform continuity \textbf{implies} continuity but
\textbf{not} vice versa.

\vspace{\baselineskip}

\textit{A note, for uniform continuity, we are saying that given a value
      $\epsilon$, we can always pick a distance ($\delta$) such that
      if two values are within that distance of each other, the distance
      between the values after the function is applied to them
      will be less than $\epsilon$. This is essentially testing for
      divergence to infinity at a value ($\frac{1}{x}$ is continuous
      but not uniformly continuous on $\mathbb{R}_{>0}$).}

\newpage

\section{Convergence}

We have the notion of convergence for sequences of real numbers
from Analysis 1A, convergence in this section is similar but
specifically for functions.

\subsection{Pointwise Convergence}

A sequence of functions $(f_n)_{n \in \mathbb{N}}$ from
$A \to \mathbb{R}$ converges \textbf{pointwise} to the function
$f$ on $A$ if:
\begin{align*}
      \lim_{n \to \infty}(f_n(x)) = f(x). \tag{$\forall x \in A$}
\end{align*}
$f$ is called the \textbf{pointwise limit} of $(f_n)_{n \in \mathbb{N}}$.

\vspace{\baselineskip}

\textit{A note, for $f_n:[0, 1]\to[0, 1]; x \to x^n$,
$f:[0, 1] \to [0, 1]; x \to \delta_1(x)$, $f_n$ converges pointwise
to $f$.}

\subsection{Uniform Convergence}

A sequence of functions $(f_n)_{n \in \mathbb{N}}$ from
$A \to \mathbb{R}$ converges \textbf{uniformly} to the
function $f$ on $A$ if:
\begin{align*}
      \forall \epsilon > 0, \exists\,N \in \mathbb{N} \text{ such that }
      \forall x \in A, \forall n \in \mathbb{N},
      (n \geq N) \Rightarrow (|f(x) - f_n(x)| < \epsilon).
\end{align*}

\textit{For the same functions outlined in the note under pointwise
convergence, we have that $f_n$ does not converge uniformly to $f$.
Let $\epsilon \in (0, 1)$, $x \in [0, 1)$ and suppose $f_n$ is
uniformly convergent to $f$,}
\begin{align*}
       & |f_n(x)-f(x)| = |x^n| < \epsilon                   \\
       & \Rightarrow 0 \leq x^n < \epsilon < 1              \\
       & \Rightarrow 0 \leq x < \epsilon^{\frac{1}{n}} < 1  \\
       & \Rightarrow \epsilon = 1 \text{ as } x \in [0, 1).
\end{align*}

\textit{This is a contradiction by the definition of $\epsilon$.
      Thus, we have the result.}

\subsection{Weierstrass' Theorem}

For $a, b \in \mathbb{R}$ with $a < b$, if a sequence of continuous
functions on $[a, b]$, $(f_n)_{n \in \mathbb{N}}$ converges
uniformly to $f$ on $[a, b]$, $f$ is continuous on $[a, b]$.

\vspace{\baselineskip}

\textit{Basically, uniform convergence preserves continuity (it also
      preserves regulation).}

\subsection{Supremum Norm}

\subsubsection{Definition of the Supremum Norm}

For $a, b \in \mathbb{R}$ with $a < b$, let $f:[a, b]\to\mathbb{R}$
be a bounded function. The supremum norm of $f$ on $[a,b]$ is
denoted by $\|f\|_{[a,b]}$ and is defined by:
\begin{align*}
      \|f\|_{[a,b]} := \sup{\{|f(x)| : x \in [a,b]\}}.
\end{align*}

\textit{The supremum norm is simply just the furthest distance from
      zero reached by a function over a closed interval. By definition,
      it is a real number and $\exists\,x \in [a, b]$ such that $f(x)$ is
      the supremum norm.}

\subsubsection{Properties of the Supremum Norm}

There are a few key properties of the supremum norm, let $a$ and $b$
be as above and let $\lambda \in \mathbb{R}$, $f, g:[a, b]\to\mathbb{R}$
be bounded functions:

\begin{itemize}
      \item $\|f\|_{[a,b]} > 0$
      \item $\|f\|_{[a,b]} = 0 \Leftrightarrow f = 0 \text{ on } [a, b]$
      \item $\|\lambda f\|_{[a,b]} = |\lambda|\|f\|_{[a,b]}$
      \item $\|f+g\|_{[a,b]} \leq \|f\|_{[a,b]} + \|g\|_{[a,b]}.$
\end{itemize}

\subsection{Cauchy Sequences of Functions}

For $a, b \in \mathbb{R}$ with $a < b$, denote the set of continuous functions
on $[a,b]$ by $C([a,b])$. Let $(f_n)_{n\in \mathbb{N}}$ be a sequence of
functions in $C([a,b])$. We say $(f_n)_{n\in \mathbb{N}}$ is Cauchy if:
\begin{align*}
      \forall \epsilon > 0, \exists\,N \in \mathbb{N} \text{ such that }
      \forall m, n \in \mathbb{N}, (m, n \geq N) \Rightarrow (\|f_n - f_m\|_{[a,b]} <
      \epsilon).
\end{align*}

\textit{This obviously bears an extreme resemblance to the Cauchy sequences of
      Analysis 1A. Just replacing the sequences of reals with sequences of functions
      and the modulus with the supremum norm.}

\vspace{\baselineskip}

For each continuous function, there exists a Cauchy sequence such that the
sequence converges uniformly to said function.

\newpage

\section{Integration}

\subsection{Step Functions}

For $a, b \in \mathbb{R}$ with $a < b$, a partition of the interval
$[a,b]$ is a set $P$ of the form:
\begin{align*}
       & P = \{x_0, x_1, \ldots, x_n\} \, (\text{for some } n \in \mathbb{N}) \\
       & \text{where } a = x_0 < x_1 < \ldots < x_n = b.
\end{align*}

We say a function $\psi:[a,b]\to\mathbb{R}$ is a step function if
there exists a partition $P = \{x_0, \ldots, x_n\}$ and a set of constants in
$\mathbb{R}$ ($\{c_0, c_1, \ldots, c_n\}$) such that:
\begin{align*}
      \psi(x) = c_i \, (\forall x \in (x_{i-1}, x_i)).
\end{align*}
In this case, $P$ and $\psi$ are adapted to each other.

\vspace{\baselineskip}

\textit{$S([a, b])$ is the set of step functions over $[a, b]$.}

\subsection{Integration of Step Functions}

\subsubsection{Definition of integration on step functions}

The integral of the step function is simple:
\begin{align*}
      \int_a^b \psi(x) \, dx := \sum_{i = 1}^n c_i(x_i - x_{i-1}).
\end{align*}
As long as the partition is adapted to $\psi$, the integral doesn't
change.

\subsubsection{Properties of integration on step functions}

Here are some properties of the integration of step functions, let
$\phi, \psi$ be step functions over $[a, b]$, $y \in \mathbb{R}$
with $a < y < b$, $\alpha, \beta \in \mathbb{R}$:

\begin{itemize}
      \item \textbf{Linearity}: $\int_a^b \alpha \psi(x) + \beta \phi(x)
                  \, dx = \alpha \int_a^b \psi(x) \, dx + \beta \int_a^b
                  \phi(x) \, dx$
      \item \textbf{Monotonicity}: $(\psi(x) \leq \phi(x) (\forall
                  x \in [a, b])) \Rightarrow (\int_a^b \psi(x) \, dx
                  \leq \int_a^b \phi(x) \, dx)$
      \item \textbf{Continuity}: $|\int_a^b \psi(x) \, dx| \leq (b-a)
                  \|\psi(x)\|_{[a, b]}$
      \item \textbf{Additivity}: $\int_a^b \psi(x) \, dx = \int_a^y
                  \psi(x) \, dx + \int_y^b \psi(x) \, dx$
\end{itemize}

\newpage

\subsection{Regulated Functions}

\subsubsection{Definition of left and right limits}

Let $A \subseteq \mathbb{R}$ and $f:A\to\mathbb{R}$. For some $\epsilon > 0$,
$a \in A$, and  $\alpha \in \mathbb{R}$:

\begin{enumerate}
      \item f has a \textbf{right limit} of $\alpha$ at $a$ if: \\
            $\exists\,\delta > 0 \text{ such that } (0 < x - a < \delta)
                  \Rightarrow (|f(x) - \alpha| < \epsilon)$
      \item f has a \textbf{left limit} of $\alpha$ at $a$ if: \\
            $\exists\,\delta > 0 \text{ such that } (0 < a - x < \delta)
                  \Rightarrow (|f(x) - \alpha| < \epsilon).$
\end{enumerate}

We can denote right limits by: $\lim_{x\downarrow a}f(x) = \alpha$.
Similarly for left limits: $\lim_{x\uparrow a}f(x) = \alpha$.

\vspace{\baselineskip}

\textit{There is a sequential definition too, for any sequence
      $(x_n)_{n \in \mathbb{N}}$ that satisfies $x_n > a$ for all $n \in \mathbb{N}$
      and $\lim_{n\to\infty} x_n = a$, if $f$ has a right limit,
      $\lim_{n\to\infty}f(x_n) = \alpha$. There is a similar definition for left
      limits.}

\subsubsection{Definition of a regulated function}

A function $f:[a,b]\to\mathbb{R}$ is regulated if:

\begin{itemize}
      \item $f$ has a left limit on all values in $(a, b]$
      \item $f$ has a right limit on all values in $[a, b)$.
\end{itemize}

\textit{All continuous functions are regulated. All increasing and
      decreasing functions are regulated.}

\subsubsection{Properties of regulated functions}

Let $R([a, b])$ be the set of functions regulated over $[a, b]$. We have that
$R([a, b])$ is closed under:

\begin{itemize}
      \item Scalar multiplication (over $\mathbb{R}$)
      \item Addition
      \item Multiplication
      \item Division (if the divisor is greater than zero over $[a, b]$)
      \item Composition
      \item The modulus.
\end{itemize}

\textit{Uniform convergence preserves regulation. Also, all step functions are
      regulated.}

\newpage

For $f$ a regulated function over $[a,b]$, we have that:
\begin{align*}
      \forall \epsilon > 0,\exists\,\psi\in S([a, b]) \text{ such that }
      \|\psi - f\| < \epsilon.
\end{align*}

\textit{Basically, for any regulated function we can always choose an
      arbitrarily accurate approximation that is a step function.}

\subsection{Integeration of Regulated Functions}

\subsubsection{Definition of integration on regulated functions}

For a function $f \in R([a, b])$, say we have two sequences of step
functions, $(\psi_n)_{n\in\mathbb{N}}$ and $(\phi_n)_{n\in\mathbb{N}}$:

\begin{itemize}
      \item $(\psi_n)_{n\in\mathbb{N}}$ is uniformly convergent to $f
                  \Rightarrow (\int_a^b\psi_n(x)\,dx)_{n\in\mathbb{N}}$ is
            convergent
      \item $(\psi_n)_{n\in\mathbb{N}}$ and $(\phi_n)_{n\in\mathbb{N}}$
            are uniformly convergent to $f \Rightarrow$ \\
            $\lim_{n\to\infty}(\int_a^b\psi_n(x)\,dx)
                  = \lim_{n\to\infty}(\int_a^b\phi_n(x)\,dx)$.
\end{itemize}

\textit{Basically, we have that no matter what step function we choose to
      approximate our function, the value of the integral will tend to the same
      value.}

\vspace{\baselineskip}

We define the integral of a regulated function $f \in R([a, b])$ by choosing
a sequence of step functions $(\psi_n)_{n\in\mathbb{N}}$ such that they
converge uniformly to $f$:
\begin{align*}
      \int_a^b f(x) \, dx := \lim_{n\to\infty} \int_a^b \psi_n(x) \, dx.
\end{align*}

\subsubsection{Properties of integration on regulated functions}

The \textbf{linearity}, \textbf{continuity}, and \textbf{additivity} properties
hold similarly to the properties of step functions. The \textbf{monotonicity}
property holds also but the stated definition varies slightly:

\begin{itemize}
      \item \textbf{Monotonicity}: For $f \in R([a,b])$ with $f(x) \geq 0$
            for $x \in [a, b]$, we have that $\int_a^b f(x)\,dx \geq 0$.
\end{itemize}

Some small notes on regulated functions, let $f \in R([a, b])$:

\begin{itemize}
      \item $|\int_a^b f(x) \, dx| \leq \int_a^b |f(x)| \, dx$
      \item For $(f_n)_{n\in\mathbb{N}}$ uniformly convergent to $f$,
            $\lim_{n\to\infty}\int_a^b f_n(x) \, dx = \int_a^b f(x) \, dx$.
\end{itemize}

\textit{The first point is similar to the triangle inequality applied to
      summations. The second was covered similarly but strictly for step functions,
      not all regulated functions.}

\newpage

\subsection{The Mean-Value Theorem of Integeration}

For $f \in C([a,b])$, let $g \in R([a, b])$ and satisfy the following:

\begin{itemize}
      \item $g(x) \geq 0$ for $x \in [a, b]$
      \item $\int_a^b g(x) \, dx > 0$
\end{itemize}

With these assumptions, we have that $\exists\, x \in (a, b)$ with:
\begin{align*}
      f(x) \int_a^b g(t) \, dt = \int_a^b f(t)g(t) \, dt
\end{align*}

\textit{Note that the function $f$ is continuous. This is a stronger
      statement than just saying it's regulated. Also, consider $g = 1$:}
\begin{align*}
      f(x) \int_a^b g(t) \, dt & = f(x) \int_a^b 1 \, dt \tag{1} \\
                               & = f(x) (b-a)                    \\
      \\
      \int_a^b f(t)g(t) \, dt  & = \int_a^b f(t) \, dt \tag{2}   \\
      \\
      \text{(1) and (2) }      & \Rightarrow                     \\
      \int_a^b f(t) \, dt      & = f(x)(b - a)
\end{align*}

\section{Differentiation}

\subsection{Definition of Differentiation}

For a function $f$ defined on a $\delta$-neighbourhood of some
$a \in \mathbb{R}$, we have that $f$ is differentiable at $a$ if
the following exists in $\mathbb{R}$:
\begin{align*}
      \lim_{h\to 0}\left(\frac{f(a + h) - f(a)}{h}\right).
\end{align*}

\textit{Differentiability at a value $a$ implies continuity at $a$.}

\subsection{Properties of Differentiation}

\subsubsection{Closure of the set of differentiable functions}

Let $f$, $g$ be differentiable functions. The set of differentiable
functions is closed under:

\begin{itemize}
      \item \textbf{Addition}: $(f + g)' = f' + g'$
      \item \textbf{Multiplication}: $(fg)' = f'g + fg'$
      \item \textbf{Division}: $\frac{f}{g} =
                  \frac{f'g - fg'}{g^2} \, (\text{for $g$ non-zero})$
      \item \textbf{Composition}: $(f \circ g)' = g'(f' \circ g)$.
\end{itemize}

\subsubsection{The implications of zero derivatives}

For a differentiable function $f$:

\begin{itemize}
      \item $f(x_0)$ is a maximum or minimum $\Rightarrow
                  f'(x_0) = 0$
      \item $f'(x_0) = 0$, $f''(x_0) > 0 \Rightarrow f(x_0)$ is
            a minimum
      \item $f'(x_0) = 0$, $f''(x_0) < 0 \Rightarrow f(x_0)$ is
            a maximum
      \item $f'(x) = 0 \, (\forall x \in [a, b]) \Rightarrow
                  f$ is constant on $[a, b]$.
\end{itemize}

\subsubsection{Rolle's Theorem and the Mean Value Theorem}

For $f$ continuous on $[a, b]$ and differentiable on $(a, b)$,
\begin{align*}
      (f(a) = f(b)) \Rightarrow (\exists \, x_0 \in (a, b)
      \text{ such that } f'(x_0) = 0) \tag{Rolle's Theorem} \\
      \exists \, x_0 \in (a, b) \text{ such that } f'(x_0)
      = \frac{f(b) - f(a)}{b - a}. \tag{Mean Value Theorem}
\end{align*}

\textit{Rolle's Theorem is a special case of the Mean Value Theorem.
      The Mean Value Theorem says that over an interval, the derivative is
      equal to the average derivative across the interval at some value.}

\subsubsection{Cauchy's Mean Value Theorem}

For $f, g$ continuous on $[a, b]$ and differentiable on $(a, b)$,
\begin{align*}
      \exists \, x_0 \in (a, b) \text{ such that }
      \left[(f(b) - f(a))g'(x_0) = (g(b) - g(a))f'(x_0)\right].
\end{align*}

\subsubsection{Other properties of the derivative}

For $f, g$ differentiable functions,

\begin{itemize}
      \item $(f'(x) = g'(x) \, (\forall x \in [a, b])) \Rightarrow
                  (f(x) = g(x) + c \, (c \in \mathbb{R}))$
      \item $f'(x) > 0 \, (\forall x \in [a, b]) \Rightarrow f$ is
            strictly increasing (similarly for strictly decreasing).
\end{itemize}

\subsection{L'Hôpital's Rule}

For $f, g$ differentiable functions defined on a $\delta$-neighbourhood
of $a \in \mathbb{R}$, if:

\begin{itemize}
      \item $g'(x) \neq 0 \, (\forall x \in
                  (a - \delta, a + \delta)\backslash\{a\})$
      \item $f(a) = g(a) = 0$
      \item $\lim_{x\to a}\frac{f'(x)}{g'(x)}$ exists in $\mathbb{R}$.
\end{itemize}

Then,
\begin{align*}
      \lim_{x\to a}\frac{f(x)}{g(x)} = \lim_{x\to a}\frac{f'(x)}{g'(x)}.
\end{align*}

\textit{So, if we have two functions that equal zero at a value, this
      rule helps us find the derivative of the their quotient as long
      as the denominator isn't zero nearby.}

\section{Calculus}

\subsection{Differentiability on a Closed Interval}

We say that a function $f:[a, b] \to \mathbb{R}$ is differentiable on $[a, b]$
if it's differentiable on $(a, b)$ and the left and right derivatives at $a$
and $b$ exist in $\mathbb{R}$.

\subsection{The Fundamental Theorem of Calculus}

Let $f:[a, b] \to \mathbb{R}$ be a continuous function, we have the following
is differentiable on $[a, b]$ and $F' = f$ on $[a, b]$:
\begin{align*}
      F(x) := \int_a^x f(t) \, dt.
\end{align*}
A result is that, for $G$ differentiable on $[a, b]$ with $G' = f$:
\begin{align*}
      \int_a^b f(x) \, dx = G(b) - G(a).
\end{align*}

\subsection{Integration by Parts}

For $f, g : [a, b] \to \mathbb{R}$ continuously differentiable (differentiable
with their derivatives being continuous):
\begin{align*}
      \int_a^b f'(x)g(x) \, dx = [f(x)g(x)]_a^b - \int_a^b f(x)g'(x) \, dx.
\end{align*}

\subsection{Integration by Substitution}

Let $f:[a, b] \to \mathbb{R}$ be a continuous function, suppose $\phi:
      [c, d] \to [a, b]$ is continuously differentiable (differentiable
with their derivatives being continuous) and bijective. Then we
have:
\begin{align*}
      \int_a^b f(x) \, dx = \int_c^d f(\phi(t))\phi'(t) \, dt
\end{align*}

\subsection{Taylor's Theorem}

\subsubsection{Polynomial coefficients}

If for $p : \mathbb{R} \to \mathbb{R}$ a polynomial function with degree $n$
and coefficients $a_0, a_1, \ldots, a_n$ we have:
\begin{align*}
      p(x) = \sum_{i = 0}^n a_ix^i,
\end{align*}
and we have that:
\begin{align*}
      a_k = \frac{1}{k!}f^{(k)}(0). \tag{$k \in \{0, 1, \ldots, n\}$}
\end{align*}

\subsubsection{Taylor Polynomials}

For $a \in \mathbb{R}$, suppose the function $f$ is $n$ times differentiable
on a $\delta$-neighbourhood of $a$, the Taylor polynomial for $f$ degree $n$
around $a$ is defined by:
\begin{align*}
      T_n(a, x) & := f(a) + f'(a)(x - a) + \cdots
      + \frac{f^{(n)}(a)}{n!}(x - a)^n \tag{$x \in \mathbb{R}$}     \\
                & := \sum_{i = 0}^n \frac{f^{(i)}(a)}{i!}(x - a)^i.
\end{align*}

\subsubsection{Taylor's Theorem}

For $a \in \mathbb{R}$, suppose the function $f$ is $n$ times differentiable
on a $\delta$-neighbourhood of $a$. For some $c$ between $a$ and $x$:
\begin{align*}
       & f(x) = f(a) + f'(a)(x - a) + \cdots + \frac{f^{(n-1)}(a)}{(n - 1)!}(x-a)^{n-1}
      + R_n                                                                             \\
       & R_n = \frac{f^{(n)}(c)}{n!}(x - a)^n. \tag{Lagrange Form}
\end{align*}

\section{Series}

\subsection{The Limit Superior and Limit Inferior}

\subsubsection{Definition of the limit superior and limit inferior}

The limit inferior and superior gives bounds on the limit of a
subsequence. This can be used in conjunction with the
Bolzano-Weierstrass Theorem (bounded sequences have convergent
subsequences). For a sequence $(a_n)_{n \in \mathbb{N}}$, we
have the following definitions:

\begin{itemize}
      \item \textbf{Limit superior}: $\limsup_{n \to \infty}(a_n)
                  := \lim_{n \to \infty}(\sup\{a_k : k \geq n\})$
      \item \textbf{Limit inferior}: $\liminf_{n \to \infty}(a_n)
                  := \lim_{n \to \infty}(\inf\{a_k : k \geq n\})$.
\end{itemize}

\subsubsection{Properties of the $\limsup$ and $\liminf$}

For some sequence $(a_n)_{n \in \mathbb{N}}$, some direct consequences
of the definition are:

\begin{itemize}
      \item $a_n$ bounded above $\Rightarrow \limsup_{n\to\infty}(a_n)
                  \in \mathbb{R}$
      \item $a_n$ bounded below $\Rightarrow \liminf_{n\to\infty}(a_n)
                  \in \mathbb{R}$
      \item $a_n$ not bounded above $\Rightarrow \limsup_{n\to\infty}(a_n)
                  = \infty$
      \item $a_n$ not bounded below $\Rightarrow \liminf_{n\to\infty}(a_n)
                  = -\infty$
      \item If the $\limsup$ or $\liminf$ exists in $\mathbb{R}$, then there
            exists a subsequence such that the limit of the sequence is
            the $\limsup$ or $\liminf$ respectively
      \item A sequence is convergent if and only if it's $\limsup$ and $\liminf$
            exist in $\mathbb{R}$ and are equal.
\end{itemize}

\newpage

\subsubsection{Alternate definition of the $\limsup$ and $\liminf$}

We have that for some sequence $(a_n)_{n \in \mathbb{N}}$, let
$a \in \mathbb{R}$, $\epsilon > 0$:

\begin{itemize}
      \item $\limsup_{n\to\infty}(a_n) = a \Leftrightarrow$ \begin{itemize}
                  \item $\exists \, N \in \mathbb{N}$ such that
                        $a_n < a + \epsilon$ for $n \geq N$
                  \item $a_m > a - \epsilon$ for infinitely many $m \in \mathbb{N}$
            \end{itemize}
      \item $\liminf_{n\to\infty}(a_n) = a \Leftrightarrow$ \begin{itemize}
                  \item $\exists \, N \in \mathbb{N}$ such that
                        $a_n > a - \epsilon$ for $n \geq N$
                  \item $a_m < a + \epsilon$ for infinitely many $m \in \mathbb{N}$.
            \end{itemize}
\end{itemize}

\subsection{Subsequential Limits}

A subsequential limit of a sequence $(a_n)_{n \in \mathbb{N}}$ is a value
$a \in \mathbb{R}$ such that there exists a subsequence
$(a_{n_k})_{k\in\mathbb{N}}$ where:
\begin{align*}
      \lim_{k\to\infty}(a_{n_k}) = a.
\end{align*}
If we consider the set of all subsequential limits $S$, if the $\limsup$ or
$\liminf$ of $a_n$ exist in $\mathbb{R}$ we have:

\begin{itemize}
      \item $\limsup_{n\to\infty}(a_n) = \max(S)$
      \item $\liminf_{n\to\infty}(a_n) = \min(S)$,
\end{itemize}

respectively.

\subsection{Types of Convergence}

\subsubsection{Absolute Convergence}

For a series $\sum_{n = 1}^\infty(a_n)$, we have that it is absolutely
convergent if $\sum_{n = 1}^\infty(|a_n|)$ is convergent.

\vspace{\baselineskip}

We have that all absolutely convergent series are convergent.

\subsubsection{Conditional Convergence}

For a series $\sum_{n = 1}^\infty(a_n)$, we have that it is conditionally
convergent if it's convergent but not absolutely convergent.

\subsection{Limit Theorems for Sequences of Functions}

\subsubsection{Uniform convergence under integration}

For $(f_n)_{n\in\mathbb{N}}$ a sequence of regulated functions on $[a, b]$,
assume $f_n$ converges uniformly on $[a, b]$ to some function $f$. Let:
\begin{align*}
      F_n(x) & := \int_a^x f_n(t) \, dt \\
      F(x)   & := \int_a^x f(t) \, dt.
\end{align*}
We have that $(F_n)_{n\in\mathbb{N}}$ converges uniformly to $F$ on $[a,b]$.

\vspace{\baselineskip}

\textit{So, we have that uniform convergence is preserved under integration.}

\subsubsection{Uniform convergence under differentiation}

Let $(f_n)_{n\in\mathbb{N}}$ be a sequence of continuously differentiable
(differentiable with their derivatives being continuous) functions on $[a, b]$.
Assume:

\begin{itemize}
      \item $(f'_n)_{n\in\mathbb{N}}$ converges uniformly to a function $g$
      \item $(f_n(a))_{n\in\mathbb{N}}$ converges in $\mathbb{R}$.
\end{itemize}

Then we have that $(f_n)_{n\in\mathbb{N}}$ converges uniformly to a function
$f$ such that $f' = g$ on $[a, b]$.

\subsubsection{Uniform convergence of series}

For $(g_n)_{n\in\mathbb{N}}$ a sequence of real valued functions defined on
$[a, b]$, define:
\begin{align*}
      f := \sum_{k = 1}^\infty g_k \\
      f_n := \sum_{k = 1}^n g_k.
\end{align*}
We say $f$ converges uniformly if $f_n$ converges pointwise and uniformly to
$f$ on $[a, b]$.

\subsection{Tests for Series Convergence}

\subsubsection{Root Test}

For a series $\sum_{n = 1}^\infty(a_n)$, where each $a_k$ $(k\in\mathbb{N})$
is non-negative, set:
\begin{align*}
      \lambda := \limsup_{n\to\infty}(a_n^{\frac{1}{n}}).
\end{align*}
We have that:

\begin{itemize}
      \item $\lambda < 1 \Rightarrow$ convergence
      \item $\lambda > 1 \Rightarrow$ divergence
      \item $\lambda = 1 \Rightarrow$ may be convergent or divergent
\end{itemize}

\subsubsection{Alternating Series Test}

For a series $\sum_{n = 1}^\infty(a_n)$, where each $a_k$ $(k\in\mathbb{N})$
is positive and $a_n$ is decreasing with limit $0$, we have that the following
is convergent:
\begin{align*}
      \sum_{n=1}^\infty(-1)^na_n.
\end{align*}

\subsubsection{Weierstrass M-Test}

For $(g_n)_{n\in\mathbb{N}}$ a sequence of continuous functions on $[a, b]$,
assume that for each $k\in\mathbb{N}$ there exists $M_k$ such that
$\|g_k\| \leq M_k$ and $\sum_{k = 1}^\infty M_k < \infty$. Then the following
converges uniformly on $[a, b]$:
\begin{align*}
      \sum_{k = 1}^\infty g_k.
\end{align*}

\textit{This is saying if we can find a sequence that is convergent as a series
      and bounds our sequence of functions then the series of our functions is
      uniformly convergent. Note that $M_k$ can never be negative as the
      supremum norm of any function is non-negative.}

\newpage

\subsubsection{Simpler Forms of the M-Test}

For $(g_n)_{n\in\mathbb{N}}$ a sequence of continuously differentiable (differentiable
with their derivatives being continuous) functions on $[a, b]$. For:
\begin{align*}
      f := \sum_{n = 1}^\infty g_n \Rightarrow f' := \sum_{n = 1}^\infty g_n',
\end{align*}
we have that the sequence of partial sums converges uniformly on $[a, b]$
to $f$ and $f$ is continuously differentiable on $[a, b]$ if:
\begin{align*}
      \sum_{n = 1}^\infty \left[\|g_n\| + \|g'_n\|\right] < \infty.
\end{align*}
Additionally, for $(g_n)_{n\in\mathbb{N}}$ a sequence of regulated functions,
we define:
\begin{align*}
      f := \sum_{n = 1}^\infty g_n.
\end{align*}
So, if:
\begin{align*}
      \sum_{n = 1}^\infty \|g_n\| < \infty,
\end{align*}
then $f$ converges uniformly on $[a, b]$ and is regulated.

\section{Power Series}

\subsection{Definition of Radius of Convergence}

The radius of convergence $R$ for a power series:
\begin{align*}
      \mathcal{P} = \sum_{n = 0}^\infty a_nx^n,
\end{align*}
is defined as:
\begin{align*}
      R := \begin{cases}
            \infty & \text{if $\mathcal{P}$ converges for all } x \in \mathbb{R}                \\
            r      & \text{if $\mathcal{P}$ converges if and only if } |x| < r \in \mathbb{R}   \\
            0      & \text{if $\mathcal{P}$ diverges for all } x \in \mathbb{R}\backslash\{0\}.
      \end{cases}
\end{align*}

\newpage

\subsection{Tests for the Radius of Convergence}

\subsubsection{Cauchy-Hadamard Theorem}

Let $(a_n)_{n\in\mathbb{N}\cup\{0\}}$ be a sequence of real numbers and define:
\begin{align*}
      \alpha & := \limsup_{n\to\infty}|a_n^{\frac{1}{n}}| \\
      R      & := \begin{cases}
            \infty           & \text{for } \alpha = 0      \\
            0                & \text{for } \alpha = \infty \\
            \frac{1}{\alpha} & \text{otherwise}.
      \end{cases}
\end{align*}
We have that the series $\sum_{n = 0}^\infty a_nx^n$ has radius of convergence $R$.

\subsubsection{Ratio Test}

Let $(a_n)_{n\in\mathbb{N}\cup\{0\}}$ be a sequence of real numbers such that for:
\begin{align*}
      \beta := \lim_{n\to\infty}\left|\frac{a_{n + 1}}{a_n}\right|,
\end{align*}
if we have that $\beta$ exists in $\mathbb{R}$, $\sum_{n = 0}^\infty a_nx^n$
has radius of convergence $R$ defined by:

\begin{align*}
      R & := \begin{cases}
            \infty           & \text{for } \beta = 0      \\
            0                & \text{for } \beta = \infty \\
            \frac{1}{\beta} & \text{otherwise}.
      \end{cases}
\end{align*}

\subsection{Consequences of the Radius of Convergence}

\subsubsection{Preservation of the Radius of Convergence}

For a power series $\sum_{n = 0}^\infty a_nx^n$ with radius of convergence $R$,
the following have the same radius of convergence:
\begin{align*}
      \sum_{n = 1}^\infty & \, na_nx^{n-1} \\
      \sum_{n = 0}^\infty & \, \frac{a_n}{n + 1}x^{n + 1}.
\end{align*}

\textit{The radius of convergence is preserved by integration and
differentiation.}

Furthermore, if we have the power series $\sum_{n = 0}^\infty a_nx^n$ with 
radius of convergence $R$, then we can define:
\begin{align*}
      f : (-R, R) \to \mathbb{R}; f(x) = \sum_{n = 0}^\infty a_nx^n.
\end{align*}
We have that $f$ is continuously differentiable on $(-R, R)$ with:
\begin{align*}
      f'(x) &= \sum_{n = 1}^\infty \, na_nx^{n-1} \\
      \int_0^x f(t) \, dt &= \sum_{n = 0}^\infty \, \frac{a_n}{n + 1}x^{n + 1},
\end{align*}
defined on $(-R, R)$.

\section{Elementary Functions}

\subsection{The Exponential and Logarithm}

\subsubsection{Definition of the Exponential}

We define the exponential function $\exp : \mathbb{R} \to \mathbb{R}$ by:
\begin{align*}
      \exp(x) := \sum_{n = 0}^\infty \frac{x^n}{n!}.
\end{align*}

\subsubsection{Properties of the exponential}

Let $f$ be the exponential function:

\begin{multicols}{2}
      \begin{itemize}
            \item $f$ is differentiable
            \item $f = f'$
            \item $f(0) = 1$
            \item $f(x)f(y) = f(x + y)$
            \item $f > 0$
      \end{itemize}
      \columnbreak
      \begin{itemize}
            \item $f$ is strictly increasing
            \item $f \to \infty$ as $x \to \infty$
            \item $f \to 0$ as $x \to -\infty$
            \item The range of $f$ is $(0, \infty)$
            \item $\frac{f(x)}{x^k} \to \infty$ as 
                  $x \to \infty$ $\forall k \in \mathbb{N}$
      \end{itemize}
\end{multicols}

\subsubsection{Euler's number}

We call Euler's number $e := \exp(1)$. We have that for $x \in \mathbb{R}$:
\begin{align*}
      \exp(x) = e^x.
\end{align*}

\subsubsection{Definition of the natural logarithm}

We define the natural logarithm as the inverse of the exponential.

\subsubsection{Properties of the natural logarithm}

Let $f$ be the natural logarithm:

\begin{multicols}{2}
      \begin{itemize}
            \item $f$ is differentiable
            \item $f$ is increasing
            \item $f(xy) = f(x) + f(y)$
            \item $f(x / y) = f(x) - f(y)$
            \item $f(1/x) = -f(x)$
      \end{itemize}
      \columnbreak
      \begin{itemize}
            \item $f(1) = 0$
            \item $f'(x) = 1/x$
            \item $f \to \infty$ as $x \to \infty$
            \item $f \to -\infty$ as $x \downarrow 0$
            \item $\frac{f(x)}{x^k} \to 0$ as 
                  $x \to \infty$ $\forall k \in \mathbb{N}$.
      \end{itemize}
\end{multicols}

\subsubsection{The natural logarithm as a power series}

For $x \in (-1, 1)$:
\begin{align*}
      \ln(1 + x) = \sum_{n = 1}^\infty(-1)^{n + 1} \frac{x^n}{n}.
\end{align*}

\subsubsection{Exponentials}

For $a > 0$, $x \in \mathbb{R}$ we can write:
\begin{align*}
      a^x := e^{xln(a)}.
\end{align*}

\subsection{Trigonometric Functions}

\subsubsection{Definition of $\sin$ amd $\cos$}

We can define the $\sin$ and $\cos$ functions as follows:
\begin{align*}
      \sin(x) &:= x - \frac{x^3}{3!} + \frac{x^5}{5!} - \ldots \\
      \cos(x) &:= 1 - \frac{x^2}{2} + \frac{x^4}{4!} - \ldots. \\      
\end{align*}

\subsubsection{Differentiability of $\sin$ amd $\cos$}

These are both differentiable with:
\begin{align*}
      (\sin)' &= \cos \\
      (\cos)' &= -\sin.
\end{align*}

\subsubsection{The Pythagorean Identity}

We have that $\sin^2(x) + \cos^2(x) = 1$.

\section{Improper Integrals}

\subsection{Improper Integrals of the First Kind}

For $a \in \mathbb{R}$, $f:[a, \infty) \to \mathbb{R}$ regulated
on $[a, b]$ for all $b \geq a$. We define the improper integral 
of the first kind as:
\begin{align*}
      \int_a^\infty f(x) \, dx := \lim_{b\to\infty}\int_a^b f(x) \, dx.
\end{align*}
Similarly, we define:
\begin{align*}
      \int_{-\infty}^b f(x) \, dx &:= \lim_{a\to-\infty}\int_a^b f(x) \, dx \\
      \int_{-\infty}^{\infty} f(x) \, dx &:=
            \lim_{a\to-\infty}\int_a^0 f(x) \, dx
            + \lim_{b\to\infty}\int_0^b f(x) \, dx.
\end{align*}

\subsection{Convergence of Functions}

For $f:[a, \infty) \to \mathbb{R}$ continuous:
\begin{gather*}
      \lim_{x\to\infty}f(x) \in \mathbb{R} \\
      \Leftrightarrow \\
      \forall \epsilon > 0, \exists \, R > a \text{ such that }
      \forall x, y > R, [|f(x) - f(y)| < \epsilon].
\end{gather*}

\newpage

\subsection{Sandwiched Improper Integrals}

For $a \in \mathbb{R}$ with $g: [a, \infty) \to \mathbb{R}$ 
non-negative with $\int_a^\infty g(x) \, dx$ convergent. So, for
$f: [a, \infty) \to \mathbb{R}$ with $|f(x)| < g(x)$ for each $x$ in
$[a, \infty)$:
\begin{align*}
      \int_a^\infty f(x) \, dx \in \mathbb{R}.
\end{align*}

\subsection{Integral Test}

For $f : [1, \infty) \to (0, \infty)$ a positive decreasing function,
for any $N \in \mathbb{N}$:
\begin{align*}
      \sum_{n = 2}^N f(n) \leq \int_1^N f(x) \, dx 
      \leq \sum_{n = 1}^{N - 1} f(n).
\end{align*}

\textit{This link may help  with understanding:
\textbf{desmos.com/calculator/me8mtv1slr}}.

\subsection{Improper Integrals of the Second Kind}

For $f:(a, b] \to \mathbb{R}$ regulated on $[a + \delta, b]$ 
for each $\delta \geq 0$. We define the improper integral 
of the second kind as:
\begin{align*}
      \int_a^b f(x) \, dx := 
      \lim_{\delta\downarrow 0} \int_{a + \delta}^b f(x) \, dx
\end{align*}
Like the integral of the first kind, this has similar definitions
for the upper and both endpoints being unbounded.

\end{document}