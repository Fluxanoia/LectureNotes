\documentclass[a4paper, 12pt, twoside]{article}
\usepackage[left = 3cm, right = 3cm]{geometry}
\usepackage[english]{babel}
\usepackage[utf8]{inputenc}
\usepackage{mathtools}
\usepackage{amssymb}
\usepackage{amsmath}
\usepackage{multicol}

\begin{document}

\title{Algorithms Notes}
\date{}
\author{\textit{paraphrased by} Tyler Wright}
\maketitle

\vfill

\textit{An important note, these notes are absolutely \textbf{NOT}
      guaranteed to be correct, representative of the course, or rigorous.
      Any result of this is not the author's fault.}

\newpage

\section{Bounding}

\subsection{Racetrack Principle}

For $f, g : \mathbb{N} \to \mathbb{N}$ functions, $n, k$ in $\mathbb{N}$
we have that:
\begin{align*}
      \begin{rcases*}
            f(k) \geq g(k) \\
            f'(n) \geq g'(n) \quad (\forall n \geq k)
      \end{rcases*} \Rightarrow
      f(n) \geq g(n) \quad (\forall n \geq k)
\end{align*}
\textit{If a function $f$ is greater than another function $g$ 
at a value $k$ and has a greater gradient for all values after
and including $k$, $f$ is greater than $g$ for all values 
after and including $k$.}

\subsection{Big $O$ Notation}

\subsubsection{Definition of the big $O$ notation}

For $g : \mathbb{N} \to \mathbb{N}$ a function, $0(g)$ is a set of
functions $f : \mathbb{N} \to \mathbb{N}$ such that each for
$f$ in $O(g)$:
\begin{gather*}
      \exists \, c \in \mathbb{R}, n_0 \in \mathbb{N}
      \text{ such that } \forall n \in N, \\
      (n \geq n_0) \Rightarrow (0 \leq f(n) \leq cg(n)).
\end{gather*}

\subsubsection{The big $O$ notation under multiplication}

For $f_1, f_2, g_1, g_2 : \mathbb{N} \to \mathbb{N}$ functions where:
\begin{itemize}
      \item $f_1 \in O(g_1)$
      \item $f_2 \in O(g_2)$,
\end{itemize}
we have that:
\begin{itemize}
      \item $f_1 + f_2$ is in $O(g_1 + g_2)$
      \item $f_1 \cdot f_2$ is in $O(g_1 \cdot g_2)$.
\end{itemize}

\subsubsection{Closure of the big $O$ notation}

For $g : \mathbb{N} \to \mathbb{N}$ a function, $O(g)$ is closed
under addition (this follows from the above).

\subsubsection{Polynomials and the big $O$ notation}

For $p : \mathbb{N} \to \mathbb{N}$ a polynomial of degree $k$,
$p$ is in $O(n^k)$.

\subsection{$\Theta$ Notation}

\subsubsection{Definition of the $\Theta$ notation}

For $g : \mathbb{N} \to \mathbb{N}$ a function, $\Theta(g)$ is a set of
functions $f : \mathbb{N} \to \mathbb{N}$ such that each for
$f$ in $\Theta(g)$:
\begin{gather*}
      \exists \, c_0, c_1 \in \mathbb{R}, n_0 \in \mathbb{N}
      \text{ such that } \forall n \in N, \\
      (n \geq n_0) \Rightarrow (0 \leq c_1g(n) \leq f(n) \leq c_2g(n)).
\end{gather*}
\textit{$f$ is sandwiched by multiples of $g$.}

\subsubsection{Equivalency of the $\Theta$ notation}

For $f, g : \mathbb{N} \to \mathbb{N}$ functions:
\begin{gather*}
      f \in \Theta(g) \Longleftrightarrow g \in \Theta(f).
\end{gather*}

\subsubsection{$\Theta$ and $O$ notation}
For $f, g : \mathbb{N} \to \mathbb{N}$ functions:
\begin{gather*}
      f \in \Theta(g) \Longleftrightarrow f \in O(g).
\end{gather*}
\textit{Which also means $g \in O(f)$ by the above equivalency.}

\subsubsection{Definition of the $\Omega$ notation}

For $g : \mathbb{N} \to \mathbb{N}$ a function, $\Omega(g)$ is a set of
functions $f : \mathbb{N} \to \mathbb{N}$ such that each for
$f$ in $\Omega(g)$:
\begin{gather*}
      \exists \, c \in \mathbb{R}, n_0 \in \mathbb{N}
      \text{ such that } \forall n \in N, \\
      (n \geq n_0) \Rightarrow (0 \leq cg(n) \leq f(n)).
\end{gather*}

\subsubsection{Equivalency of the $\Omega$ notation}

For $f, g : \mathbb{N} \to \mathbb{N}$ functions:
\begin{gather*}
      f \in \Omega(g) \Longleftrightarrow g \in O(f).
\end{gather*}

\end{document}