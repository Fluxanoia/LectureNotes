\section{Ordinal Applications}

We collate the properties of ordinals covered so far
for some ordinals $\alpha$, $\beta$, and $\gamma$:
\begin{itemize}
    \item ordinals are transitive and well-ordered by $\in$ by definition,
    \item for $x$ in $\alpha$, $x$ is an ordinal with $x = \alpha_x$,
    \item $\alpha \cong \beta$ implies that $\alpha = \beta$,
    \item we have exactly one of the following $\alpha = \beta$, 
        $\alpha$ in $\beta$, or $\beta$ in $\alpha$.
\end{itemize}

\subsection{Principle of Transfinite Induction on Ordinals (3.24)} \label{3.24}

For a well-defined property of ordinals $\Phi$, we have
that for all ordinals $\alpha$: \begin{align*}
    \bigl[
        \forall \, \beta < \alpha, \Phi(\beta) 
        \Longrightarrow \Phi(\alpha)
    \bigr] \Longrightarrow \Phi(\alpha). \tag{$\ast$}
\end{align*} Hence, the class of ordinals is well-ordered.

\begin{proof}
    We take $C = \{\alpha \in \text{On} : \neg \Phi(\alpha)\}$ and $\alpha_0$
    in $C$. If $\alpha_0$ is not the least element of $C$, we have that
    $\emptyset \neq (\alpha_0 \cap C) \subseteq \alpha_0$ has an $\in$-least 
    element $\alpha_1$ as $\alpha_0$ is an ordinal, which is well-ordered by $\in$.
    Thus, $\alpha_1$ is the $\in$ least element of $C$.
    As we have a least element $\gamma$ of $C$, we see that for all $\beta$ in $C$ with
    $\beta < \gamma$, we have $\Phi(\beta)$. But, our assumption implies that
    we have $\Phi(\gamma)$, a contradiction. Thus, $C = \emptyset$ as required.
\end{proof}

\subsection{The Class of Ordinals (3.25)} \label{3.25}

The class of ordinals is a proper class.

\begin{proof}
    Suppose the class of ordinals is a set $z$.
    We have that $\ang{z, \in}$ is transitive and
    well-ordered by (\ref{3.24}). Thus, $z$ is an ordinal,
    as such $z$ is in $z$. But, this contradicts the
    strict ordering of $\in$.
\end{proof}

\subsection{Sum of Orderings (3.26)} \label{3.26}

For strict total orderings $\ang{A, R}$ and $\ang{B, S}$ 
with $A \cap B$ empty, we define the sum ordering $\ang{C, T}$ as: \begin{align*}
    C &= A \cup B, \\
    x T y &\Longleftrightarrow \begin{cases}
        x R y & \text{for } x \text{ and } y \in A \\
        x S y & \text{for } x \text{ and } y \in B \\
        x \in A \text{ and } y \in B & \text{otherwise}.
    \end{cases}
\end{align*} We can avoid the disjoint constraint by taking
the sum of $\ang{A \times \{0\}, R}$ and \linebreak
$\ang{B \times \{1\}, S}$. We name this operation $+'$ so 
for ordinals $\alpha$ and $\beta$: \begin{align*}
    \alpha +' \beta &= \ang{\ot(
        (\alpha \times \{0\}) \cup (\beta \times \{1\})
    ), T}, \\
    \ang{\gamma, i} T \ang{\delta, j} 
    &\Longleftrightarrow (i = j \text{ and } \gamma < \delta)
    \text{ or } (i < j).
\end{align*}

\subsection{Product of Orderings (3.28)} \label{3.28}

For strict total orderings $\ang{A, R}$ and $\ang{B, S}$,
we define the product of these orderings $\ang{A, R} \times \ang{B, S}$
to be the ordering $\ang{C, U}$: \begin{align*}
    C &= A \times B \\
    \ang{x, y}U\ang{x', y'} 
    &\Longleftrightarrow 
    (ySy') \text{ or } (y = y' \text{ and } xRx'),
\end{align*} defining an operation for ordinals, denoted by $\cdot'$.

\subsection{Supremum of Ordinals (3.30, 3.32)} \label{3.30} \label{3.32}

For a set of ordinals $A$, $\sup(A)$ is the least ordinal $\gamma$ 
such that for all $\delta$ in $A$, $\delta \leq \gamma$.
We also have the strict supremum $\sup^+(A)$ as the least ordinal $\gamma^+$
such that for all $\delta$ in $A$, $\delta < \gamma^+$.
We have that $\sup(A) = \bigcup A$.
\begin{proof}
    We know the supremum is well-defined as if we suppose there isn't an 
    ordinal which is an upper bound for $A$, there's some $\delta$ in 
    $A$ such that $\delta > \gamma$ for each ordinal $\gamma$. 
    However, this means $\bigcup A$ must be equal to $\text{On}$, 
    which is a contradiction as $\bigcup A$ is a set by the Axiom of Unions.
    \\[\baselineskip]
    We take $S = \sup(A)$ and $u$ in $\bigcup A$, we know that there must be 
    some $a$ in $A$, such that $u < a < S$. Thus, $u$ is in $S$ as $S$ is transitive, 
    hence $\bigcup A \subseteq S$.
    Conversely, for $s$ in $S$, $s < S$ so there
    is some $a$ in $A$ with $s < a \leq S$. Thus, $s$ is in $\bigcup A$, 
    so $S \subseteq \bigcup A$. Thus $S = \bigcup A$.
\end{proof}

\subsection{Types of Ordinals (3.33)} \label{3.33}

We can consider three types of ordinals: \begin{itemize}
    \item the zero ordinal,
    \item successor ordinals, ordinals with immediate predecessors,
    \item limit ordinals, ordinals that are not of the other types.
\end{itemize}

\subsection{Recursion Theorem on Ordinals (3.35)} \label{3.35}

For a function $F$ from $V$ to $V$, there exists a unique function
$H$ from the class of ordinals to $V$ such that for all $\alpha$: \begin{align*}
    H(\alpha) = F(H \upharpoonright \alpha).
\end{align*}

\begin{proof}
    We define a function $u$ to be a $\delta$-approximation if 
    $\Dom(u) = \delta$ and for all $\alpha < \delta$, 
    $u(\alpha) = F(u \upharpoonright \alpha)$. 
    For a $\delta$-approximation $u$ and $\delta > 0$, we see that 
    $u(0) = F(u \upharpoonright 0) = F(\emptyset)$ so a $1$-approximation is 
    equal to $\{\ang{0, F(\emptyset)}\}$ with domain $\{0\} = 1$. Additionally, 
    for some $\gamma < \delta$, $u \upharpoonright \gamma$ is a $\gamma$-approximation.
    Furthermore, $u \cup \{\ang{\delta, F(u)}\}$ is a 
    $(\delta + 1)$-approximation.
    \\[\baselineskip]
    \textbf{Agreement on Domain}
    For a $\delta$-approximation $u$ and any $\gamma$-approximation $v$
    with $\delta < \gamma$, $u = v \upharpoonright \delta$.
    \begin{proof}
        We appeal to the contrary and take $\tau$ be the least ordinal such that
        \linebreak $u(\tau) \neq \gamma(\tau)$. Thus, $(u \upharpoonright \tau)
        = (v \upharpoonright \tau)$ but then: \begin{align*}
            u(\tau) 
            = F(u \upharpoonright \tau)
            = F(v \upharpoonright \tau)
            = v(\tau),
        \end{align*} which is a contradiction.
    \end{proof}
    \noindent
    \textbf{Uniqueness}
    If such $H$ exists, it is unique.
    \begin{proof}
        We appeal to the contrary, taking $H'$ to be some differing
        derivation of $H$. We consider the least $\tau$ such that
        $H(\tau) \neq H'(\tau)$ and apply the same argument as
        the \textbf{Agreement on Domain} case.
    \end{proof}
    \noindent
    \textbf{Limits}
    For some limit ordinal $\lambda$, if for all $\alpha < \lambda$
    we have that $u_\alpha$ is an $\alpha$-approximation, 
    $\bigcup_{\alpha < \lambda} u_\alpha$ is a $\lambda$-approximation.
    \begin{proof}
        This union is of an increasing sequence of sets so:
        \begin{align*}
            \alpha < \beta < \lambda \Longrightarrow u_\alpha \subseteq u_\beta.
        \end{align*}
        As each element is a function, and the functions agree on domain,
        the union is also a function and has domain $\lambda$. Thus, this 
        union is a $\lambda$-approximation.
    \end{proof}
    \noindent
    \textbf{Existence}
    We define $H = \bigcup B$ which is a function with $\Dom(H)$ being the set
    of ordinals.
    \begin{proof}
        We know that $H$ is a function by the \textbf{Agreement on Domain} case.
        We take $C = \{\delta : \text{There's no $\delta$-approximation}\}$
        and suppose $C$ is non-empty. By the Principle of Transfinite Induction
        on Ordinals, $C$ has a least elememt $\psi$. We know that
        $\psi > 1$ as we defined a $1$-approximation and by \textbf{Limits}
        it cannot be a limit ordinal. If $\psi = \mu + 1$ then there's
        a $\mu$-approximation $v$ by the minimality of $\psi$. However,
        we can extend $v$ to a $\psi$-approximation $u$ by setting
        $u(\mu) = F(v)$. This is a contradiction.
    \end{proof}
    \noindent 
    Thus, we have that $H$ exists and is a unique function as required.
\end{proof}

\subsection{Recursion Theorem on Ordinals, Second Form (3.38)} \label{3.38}

For $a$ in $V$, and functions $F_0$ and $F_1$ from $V$ to $V$, there's a unique
function $H$ from the class of ordinals to $V$ such that for an ordinal $\alpha$
and a limit ordinal $\lambda$: \begin{align*}
    H(0) &= a, \\
    H(\alpha + 1) &= F_0(H(\alpha)), \\
    H(\lambda) &= F_1(H \upharpoonright \lambda).
\end{align*}

\begin{proof}
    We define a function $F$ from $V$ to $V$ by: \begin{align*}
        F(u) = \begin{cases}
            a & \text{for } u = \emptyset \\
            F_0(u) & \text{if } u \text{ is a function with a successor domain} \\
            F_1(u) & \text{if } u \text{ is a function with a limit domain} \\
            \emptyset & \text{otherwise},
        \end{cases}
    \end{align*} and apply (\ref{3.35}).
\end{proof}

\subsection{Ordinal Addition (3.39)} \label{3.39a}

We define ordinal addition $A_\alpha$ for some ordinals
$\alpha$ and $\beta$, and a limit ordinal $\lambda$ as: \begin{align*}
    A_\alpha(0) &= \alpha, \\
    A_\alpha(\beta + 1) &= S(A_\alpha(\beta)), \\
    A_\alpha(\lambda) &= \sup(\{A_\alpha(x) : x < \lambda\}).
\end{align*}

\subsection{Ordinal Multiplication (3.39)} \label{3.39m}

We define ordinal multiplication $M_\alpha$ for some ordinals
$\alpha$ and $\beta$, and a limit ordinal $\lambda$ as: \begin{align*}
    M_\alpha(0) &= 0, \\
    M_\alpha(\beta + 1) &= M_\alpha(\beta) + \alpha, \\
    M_\alpha(\lambda) &= \sup(\{M_\alpha(x) : x < \lambda\}).
\end{align*}

\subsection{Ordinal Exponentiation (3.39)} \label{3.39e}

We define ordinal exponentiation $A_\alpha$ for some ordinals
$\alpha$ and $\beta$, and a limit ordinal $\lambda$ as: \begin{align*}
    E_\alpha(0) &= 1, \\
    E_\alpha(\beta + 1) &= E_\alpha(\beta) \cdot \alpha, \\
    E_\alpha(\lambda) &= \sup(\{E_\alpha(x) : x < \lambda\}).
\end{align*}

\subsection{Monotonicity of Ordinal Arithmetic (3.40-41)} \label{3.40} \label{3.41}

For ordinals $\alpha$, $\beta$, and $\gamma$ with $\beta > 0$ and $\gamma > 1$,
the functions $A_\alpha$, $M_\beta$, and $E_\gamma$ are strictly increasing 
and thus injective.

\begin{proof}
    We take $\beta$, $\gamma$ and $\delta$ to be ordinals and we proceed by induction,
    supposing that: \begin{align*}
        [\beta < \gamma] \Longrightarrow [A_\alpha(\beta) < A_\alpha(\gamma)], \tag{$\ast$}
    \end{align*} for all $\gamma \leq \delta$. The base case is trivial. 
    For $\beta < \delta + 1$, if $\beta = \delta$, then: \begin{align*}
        A_\alpha(\delta) < S(A_\alpha(\delta)).
    \end{align*} Otherwise, $\beta < \delta$ so by our hypothesis: \begin{align*}
        A_\alpha(\beta) &< A_\alpha(\delta) < S(A_\alpha(\delta)) = A_\alpha(\delta + 1). 
    \end{align*} Now, we suppose ($\ast$) holds for all $\gamma < \lambda$ for some 
    limit ordinal $\lambda$. For $\beta < \lambda$, clearly $\beta < \beta + 1 < \lambda$ 
    as $\lambda$ has no immediate predecessor. By the hypothesis: \begin{align*}
        A_\alpha(\beta) &< A_\alpha(\beta + 1) 
        \leq \sup(\{A_\alpha(\gamma) : \gamma < \lambda\}) 
        = A_\alpha(\lambda),
    \end{align*} as required. The arguments for $M_\alpha$ and $E_\alpha$ are 
    similar.
\end{proof}

\subsection{Remainders (3.43)} \label{3.43}

For $\alpha$ and $\beta$ ordinals with $0 < \alpha \leq \beta$, there's a unique: \begin{enumerate}
    \item ordinal $\gamma$ such that $\alpha + \gamma = \beta$,
    \item pair of ordinals $\zeta$ and $\kappa$ such that
        $\alpha \cdot \zeta + \kappa = \beta$ and $\kappa < \alpha$.
\end{enumerate} 

\begin{proof}
    (1) As $A_\alpha$ is strictly increasing, we consider 
    $Z = \{x : \alpha + x \geq \beta\}$ which must be non-empty as
    $A_\alpha$ is strictly increasing.
    We take $\gamma = \min(Z)$ and see that $\alpha + \gamma = \beta$
    since if $\alpha + \gamma > \beta$ either: \begin{itemize}
        \item $\gamma = \delta + 1$ so $\alpha + \delta < \beta$ as
            $\delta$ is not in $Z$. But then,
            $ \alpha + \gamma = \alpha + (\delta + 1) \leq \beta$,
            a contradiction,
        \item $\gamma$ is a limit ordinal, $\alpha + \gamma =
            \sup(\{\alpha + \delta : \delta < \gamma\})$. But,
            as $\alpha + \gamma > \beta$ there's some $\delta < \gamma$
            such that $\alpha + \delta \geq \beta$. 
            This contradicts the minimality of $\gamma$.
    \end{itemize} 
    (2) As $M_\alpha$ is strictly increasing, we choose the least
    $\zeta$ such that
     $\alpha \cdot \zeta \leq \beta < \alpha \cdot (\zeta + 1)$. We apply
     (1) to find some $\kappa$ such that $\alpha \cdot \zeta + \kappa = \beta$.
     For some $\zeta'$ and $\kappa'$ also satisfying (2), if $\zeta = \zeta'$
     then by the uniqueness of (1), $\kappa = \kappa'$. We suppose
     $\zeta < \zeta'$ so $\zeta + 1 \leq \zeta'$: \begin{align*}
         \beta = \alpha \cdot \zeta + \kappa &< \alpha \cdot \zeta + \alpha \\
         &= \alpha \cdot (\zeta + 1) \\
         &\leq \alpha \cdot \zeta' \\
         &\leq \alpha \cdot \zeta' + \kappa' \\
         &= \beta,
     \end{align*} which is a contradiction. Hence, $\zeta = \zeta'$.
\end{proof}

\subsection{Ordinal Arithmetic (3.44)} \label{3.44}

We have that ordinal addition is associative, ordinal multiplication is distributive 
over addition and associative, and for ordinals $\alpha$, $\beta$, and $\gamma$: \begin{align*}
    \alpha^{\beta + \gamma} = \alpha^\beta \cdot \alpha^\gamma.
\end{align*}
