\section{Well-orderings and Ordinals}

\subsection{The Principle of Transfinite Induction}

Let $\ang{X, \prec}$ be a well-ordering. We have that: \begin{align*}
    [\forall \, x \in X, 
        (\forall \, y \prec x, \Phi(y)) \Rightarrow \Phi(x)
    ] \Rightarrow \forall \, x \in X, \Phi(x).
\end{align*}

\begin{proof}
    We appeal to the contrary and assume the antecedent
    but suppose that $\emptyset \neq Z = \{x \in X : \neg \Phi(x)\}$.
    As $\ang{Z, \prec}$, there is $\prec$-least element $z_0$.
    But then for all $x \prec z_0$, $\Phi(x)$ holds. But, by the
    antecedent, this means $\Phi(z_0)$ holds, a contradiction.
\end{proof}

\subsection{Initial Segments}

For a well-ordering $\ang{X, \prec}$, the $\prec$-initial
segment of some element $z$ in $X$ is the set of predecessors of $z$,
denoted by $X_z$. Note that $X_z$ does not contain $z$.

\subsection{Order Preserving Maps on Well-orderings}

For a well-ordering $\ang{X, \prec}$ with $f : \ang{X, \prec} 
\to \ang{X, \prec}$ an order preserving map, we have that
for all $x$ in $X$, $x \prec f(x)$.

\begin{proof}
    We appeal to the contrary, that for some $x$ in $X$, 
    we have $f(x) \prec x$.
    As $\ang{X, \prec}$ is a well-ordering, there's a
    $\prec$-least $x_0$ in $X$ with the property that 
    $f(x_0) \prec x_0$. But $f(f(x_0)) \prec f(x_0)$
    as $f$ is order preserving. Thus, a contradiction
    to the minimality of $x_0$.
\end{proof}

\subsubsection{Uniqueness of Order Isomorphisms}

For well-orderings $\ang{X, \prec_x}$, $\ang{Y, \prec_y}$
with $f : \ang{X, \prec_x} \to \ang{Y, \prec_y}$ an
order isomorphism. We have that $f$ is unique.

\begin{proof}
    Suppose we have two such isomorphisms $f$ and $g$.
    We have that $(f^{-1} \circ g)$ is also an order
    isomorphism. Taking $x$ arbitrary in $X$: \begin{align*}
        &x \preceq_x (f^{-1} \circ g)(x) \\
        \Longrightarrow \qquad & f(x) \preceq_y f(f^{-1} \circ g)(x) \\
        \Longrightarrow \qquad & f(x) \preceq_y g(x).
    \end{align*} By applying this argument again with the roles
    of $f$ and $g$ swapped, we can also see that $g(x) \preceq_y f(x)$.
    Thus, $f(x) = g(x)$.
    \\[\baselineskip]
    In particular, if $\ang{X, \prec_x} = \ang{Y, \prec_y}$
    then this isomorphism is the identity map.
\end{proof}

\subsection{Non-existence of Order Isomorphisms to Segments}

A well-ordered set is not order isomorphic to any segment of itself.

\begin{proof}
    We appeal to the contrary and suppose there is such
    an order isomorphism on a well-ordering $\ang{X, \prec}$
    to $\ang{X_z, \prec}$ for some $z$ in $X$.
    But, we have that $x \preceq f(x)$ for any $x$ in $X$
    and $f(z) \prec z$ as $f(z)$ is in $X_z$.
    Thus, we have that $z \preceq f(z)$ and $z \succ f(z)$,
    a contradiction.
\end{proof}
