\section{Well-orderings and Ordinals}

\subsection{The Principle of Transfinite Induction}

Let $\ang{X, \prec}$ be a well-ordering. We have that: \begin{align*}
    [\forall \, x \in X, 
        (\forall \, y \prec x, \Phi(y)) \Rightarrow \Phi(x)
    ] \Rightarrow \forall \, x \in X, \Phi(x).
\end{align*}

\begin{proof}
    We appeal to the contrary and assume the antecedent
    but suppose that $\emptyset \neq Z = \{x \in X : \neg \Phi(x)\}$.
    As $\ang{Z, \prec}$, there is $\prec$-least element $z_0$.
    But then for all $x \prec z_0$, $\Phi(x)$ holds. But, by the
    antecedent, this means $\Phi(z_0)$ holds, a contradiction.
\end{proof}

\subsection{Initial Segments}

For a well-ordering $\ang{X, \prec}$, the $\prec$-initial
segment of some element $z$ in $X$ is the set of predecessors of $z$,
denoted by $X_z$. Note that $X_z$ does not contain $z$.

\subsection{Order Preserving Maps on Well-orderings}

For a well-ordering $\ang{X, \prec}$ with $f : \ang{X, \prec} 
\to \ang{X, \prec}$ an order preserving map, we have that
for all $x$ in $X$, $x \prec f(x)$.

\begin{proof}
    We appeal to the contrary, that for some $x$ in $X$, 
    we have $f(x) \prec x$.
    As $\ang{X, \prec}$ is a well-ordering, there's a
    $\prec$-least $x_0$ in $X$ with the property that 
    $f(x_0) \prec x_0$. But $f(f(x_0)) \prec f(x_0)$
    as $f$ is order preserving. Thus, a contradiction
    to the minimality of $x_0$.
\end{proof}

\subsubsection{Uniqueness of Order Isomorphisms}

For well-orderings $\ang{X, \prec_x}$, $\ang{Y, \prec_y}$
with $f : \ang{X, \prec_x} \to \ang{Y, \prec_y}$ an
order isomorphism. We have that $f$ is unique.

\begin{proof}
    Suppose we have two such isomorphisms $f$ and $g$.
    We have that $(f^{-1} \circ g)$ is also an order
    isomorphism. Taking $x$ arbitrary in $X$: \begin{align*}
        &x \preceq_x (f^{-1} \circ g)(x) \\
        \Longrightarrow \qquad & f(x) \preceq_y f(f^{-1} \circ g)(x) \\
        \Longrightarrow \qquad & f(x) \preceq_y g(x).
    \end{align*} By applying this argument again with the roles
    of $f$ and $g$ swapped, we can also see that $g(x) \preceq_y f(x)$.
    Thus, $f(x) = g(x)$.
    \\[\baselineskip]
    In particular, if $\ang{X, \prec_x} = \ang{Y, \prec_y}$
    then this isomorphism is the identity map.
\end{proof}

\subsubsection{Non-existence of Order Isomorphisms to Segments}

A well-ordered set is not order isomorphic to any segment of itself.

\begin{proof}
    We appeal to the contrary and suppose there is such
    an order isomorphism on a well-ordering $\ang{X, \prec}$
    to $\ang{X_z, \prec}$ for some $z$ in $X$.
    But, we have that $x \preceq f(x)$ for any $x$ in $X$
    and $f(z) \prec z$ as $f(z)$ is in $X_z$.
    Thus, we have that $z \preceq f(z)$ and $z \succ f(z)$,
    a contradiction.
\end{proof}

\subsubsection{Order Isomorphism to Set of Segments}

A well-ordered set $\ang{X, \prec}$ is order isomorphic to the set
of its initial segments ordered by $\subset$.

\begin{proof}
    We let $Y = \{X_a : a \in X\}$, we have that $\varphi$
    characterised by $a \mapsto X_a$ is an injective map
    as segments do not contain the element which determines it.
    As $a \prec b \Leftrightarrow X_a \subset X_b$, the mapping
    is order preserving.
\end{proof}

\subsection{Ordinal Numbers}

We say that $\ang{X, \in}$ is an ordinal if and only it
$X$ is transitive, and where $\prec = \in$, 
$\ang{X, \prec}$ is a well-ordering. We have that
$\ang{\omega, \in}$ is an ordinal.

\subsubsection{Elements and Segments}

For an ordinal $\ang{X, \in}$, then every element $z$ in $X$
is identical to $X_z$.

\begin{proof}
    Suppose $X$ is transitive and $\in$ well-orders $X$.
    Taking $z$ in $X$: \begin{align*}
        w \in X_z \qquad
        \Longleftrightarrow \qquad&
        w \in X \text{ and } w \in z \\
        \Longleftrightarrow \qquad&
        w \in z \tag{as $z \subseteq X$},
    \end{align*} thus, $X_z = z$ as required.
\end{proof} 
\noindent
So, for any elements $a$, $b$ of an ordinal: \begin{align*}
    a \in b \Longleftrightarrow a \subset b 
    \Longleftrightarrow X_a \subset X_b.
\end{align*}

\subsubsection{Subsets and Segments}

For an ordinal $\ang{X, \in}$ with $Y \subset X$, if $\ang{Y, \in}$
is also an ordinal, then $Y$ is an $\in$-initial segment of $X$.

\begin{proof}
    Taking $a$ in $Y$ as supposed, as $Y$ is an ordinal so $Y_a = a$.
    As $Y \subset X$, $a$ is in $X$ so $X_a = a$. Thus, $X_a = Y_a$.
    Furthermore, as $Y \neq X$, we consider 
    $c = \inf\{z \in X : z \notin Y\}$ which exists as the set is
    non-empty and $\ang{X, \in}$ is a well-ordering. Hence, $Y = X_c$.
\end{proof}

\subsubsection{Segments}

For an ordinal $\ang{X, \in}$ any $\in$-initial segment of 
$\ang{X, \in}$ is an ordinal.

\begin{proof}
    We take some $u$ in $X$ and $w$ in $X_u$. 
    As $\in$ well-orders $X$, it well-orders
    any subset of $X$ so $\ang{X_u, \in}$ is a well-ordering. 
    We have that: \begin{align*}
        t \in w \in u \Longrightarrow t \in u = X_u,
    \end{align*} thus $X_u$ is transitive as required.
\end{proof}

\subsubsection{The Intersection of Ordinals}

For ordinals $X$, $Y$, $X \cap Y$ is also an ordinal.

\begin{proof}
    We take $\in$ to be the ordering on $X$.
    We know that $X \cap Y$ is transitive as $X$ and $Y$
    are transitive. Any subset of $X$ is a well-ordering
    under $\in$, in particular $X \cap Y$
    is well-ordered by $\in$. 
\end{proof}

\subsection{Classification Theorem for Ordinals}

For two ordinals $X$, $Y$, either $X = Y$ or one is an initial
segment of the other (or equivalently a member).

\begin{proof}
    Suppose that $X \neq Y$. We know that $X \cap Y$ is an ordinal
    also. We have two cases.
    \\[\baselineskip]
    If $X = X \cap Y$ or $Y = X \cap Y$, one must
    be an initial segment of the other as it must
    be a proper subset under our assumption $X \neq Y$.
    \\[\baselineskip]
    If $X \cap Y$ is a proper subset of $X$ and $Y$,
    it is an initial segment of $X$ and $Y$ simultaneously
    we set $X \cap Y = X_a = Y_b$ for some $a$ in $X$ and
    $b$ in $Y$. But, we know that as $X$ and $Y$ are ordinals,
    $a = X_a = Y_b = b$. However, this means: \begin{align*}
        a = b \in X \cap Y = X_a,
    \end{align*} but $a \notin X_a$, a contradiction.
\end{proof}

\subsection{Equality under Isomorphisms}

For two ordinals $X$ and $Y$, if $X$ is order isomorphic to
$Y$ then $X = Y$.

\begin{proof}
    Suppose $X \neq Y$, then without loss of generality we take
    $X$ to be an initial segment of $Y$. But, this would
    mean $Y$ is order isomorphic to a segment of itself
    which is a contradiction.
\end{proof}

\subsection{Bound on Isomorphisms}

A well-ordering is order isomorphic to at most one
ordinal.

\begin{proof}
    If a well-ordering is isomorphic to more than one
    ordinals, then these ordinals are isomorphic to each other
    and thus, equal.
\end{proof}

\subsection{Criterion for Ordinals}

If every segment of a well-ordered set $\ang{A, \prec}$ is 
order isomorphic to some ordinal, $\ang{A, \prec}$ itself
is order isomorphic to an ordinal.

\begin{proof}
    Each segment must be order isomorphic to at most one
    ordinal (thus exactly one). We define a function $F$ from
    the segments of $A$ to the ordinal $F(A_b)$ such
    that $\ang{A_b, \prec} \cong \ang{F(b), \in}$.
    We know that this ordinal is unique as non-identical
    segments have differing sizes.
    We let $Z = \Ran(F)$: \begin{align*}
        Z = \{F(b) : \exists \, b \in A, 
        \exists \text{ an isomorphism } g_b 
        \text{ from } \ang{A_b, \prec} 
        \text{ to } \ang{F(b), \in}\},
    \end{align*} noting that the isomorphism between
    $A_b$ and $F(b)$ is unique. If $c$, $b$ are in $A$ with
    $c \prec b$ then $A_c = (A_b)_c$ implying that
    $F(c) \neq F(b)$ so $F$ is injective and thus bijective
    between $A$ and $Z$.
    Continuing with $c \prec b$: \begin{align*}
        g_b \upharpoonright A_c :
        \ang{A_c, \prec} \cong 
        \ang{(F(b))_{g_b(c)}, \in},
    \end{align*} by the uniqueness of order isomorphisms
    $(g_b \upharpoonright A_c) = g_c$ and $F(c) = (F(b))_{g_b(c)}$.
    Thus, $F(c) \in F(b)$.
    \\[\baselineskip]
    We know that $Z$ is well-ordered by $\in$ as 
    $A$ is well-ordered by $\prec$ and $F$ is an
    order isomorphism. For $u \in F(b) \in Z$,
    as $g_b$ is surjective, $u = g_b(c)$ for some $c \prec b$.
    Then, $u = F(b)_u = F(b)_{g_b(c)} = F(c)$. Hence, $u$ is
    in $Z$, so $Z$ is transitive. Thus, $Z$ is an ordinal.
\end{proof}

\newpage

\subsection{Representation Theorem for Well-orderings}

Every well-ordering is order isomorphic to 
exactly one ordinal.

\begin{proof}
    For a well-ordering $\ang{X, \prec}$, we know that if it
    is isomorphic to an ordinal, this is the only such ordinal.
    We take: \begin{align*}
        Z = \{v \in X : X_v \text{ is not isomorphic to an ordinal}\},
    \end{align*} and want to show it's empty as this will suffice
    combined with our criterion. Supposing the
    contrary, we take $v_0$ to be the $\prec$-least element
    of $Z$. We have that $\ang{X_{v_0}, \prec}$ is a well-ordering
    with each element preceding $v_0$, as such $(X_{v_0})_w = X_w$
    for each $w$ in $X_{v_0}$ is order isomorphic to an ordinal.
    Which means $X_{v_0}$ must also be order isomorphic to an ordinal
    via our criterion, a contradiction. Thus, $Z$ is empty, as required.
\end{proof}

\subsection{Order Type of Well-orderings}

For a well-ordering $\ang{X, \prec}$, the order type of $\ang{X, \prec}$
is the unique ordinal isomorphic to $\ang{X, \prec}$, written as
$\ot(\ang{X, \prec})$.

\subsection{Classification Theorem for Well-orderings}

For two well-orderings $\ang{A, \prec_A}$ and $\ang{B, \prec_B}$
we have that exactly one of the following holds: \begin{itemize}
    \item $\ang{A, \prec_A} \cong \ang{B, \prec_B}$,
    \item There exists $b$ in $B$ such that 
        $\ang{A, \prec_A} \cong \ang{B_b, \prec_B}$,
    \item There exists $a$ in $A$ such that 
        $\ang{A_a, \prec_A} \cong \ang{B, \prec_B}$.
\end{itemize}

\begin{proof}
    We take $\ang{X, \in}$ and $\ang{Y, \in}$ to be the unique ordinals
    isomorphic to $\ang{A, \prec_A}$ and $\ang{B, \prec_B}$ (resp.)
    via maps: \begin{align*}
        f &: \ang{X, \in} \to \ang{A, \prec_A}, \\
        g &: \ang{Y, \in} \to \ang{B, \prec_B}. \\
    \end{align*} We know that either these ordinals are order isomorphic
    or order isomorphic to an initial segment of the other. If the
    former is true, then we have that our well-orderings are isomorphic
    via $f$ and $g$ and their inverses. If the latter is true,
    we know that (without loss of generality) 
    $\ang{X, \in} \cong \ang{Y_y, \in}$ for some $y$ in $Y$.
    Thus: \begin{align*}
        f(\ang{X, \in}) \cong g(\ang{Y_y, \in})
        \Longrightarrow
        \ang{A, \prec_A} \cong \ang{B_{g(y)}, \prec_B}),
    \end{align*} as required.
\end{proof}
