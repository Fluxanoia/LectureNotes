\section{The Universe of Sets}

\subsection{Classes}

We have that classes are collection of objects, these could also be sets.
Classes that are not sets are called proper classes.

\subsection{Russell's Theorem (1.4)} \label{1.4}

We have that $R = \{x : x \notin x\}$ is a proper class.

\begin{proof}
    Suppose we have a set $z$ such that $z = R$, we consider the membership of $z$
    in $R$. If we suppose $z$ is in $R$, by the definition of $R$, $z$ is not in
    $z = R$, a contradiction. If we suppose $z$ is not in $R$, by the definition 
    of $R$, $z$ is in $z = R$, a contradiction. Thus, $z$ cannot be a set,
    so $R$ is a proper class.
\end{proof}

\subsection{The Universe of Sets (1.5)} \label{1.5}

We define the universe of sets as $V = \{x : x = x\}$.
We have that $V$ is a proper class.
\begin{proof}
    If we suppose $V$ is a set, we apply the Axiom of Subsets with
    $\Phi(x) = x \notin x$ and reach a contradiction via (\ref{1.4}).
\end{proof}

\subsection{The Well-founded Hierarchy of Sets (6.1)} \label{6.1}

For an ordinal $\alpha$ and a limit ordinal $\lambda$, we define the function 
$V_\alpha$ by transfinite recursion: \begin{align*}
    V_0 &= \emptyset, \\
    V_{\alpha + 1} &= \mc{P}(V_\alpha), \\
    V_{\lambda} &= \bigcup_{\alpha < \lambda} V_\alpha, \\
    V &= \bigcup_{\alpha \in \text{On}} V_\alpha.
\end{align*}

\subsection{Transitivity of $V_\alpha$ (6.2)} \label{6.2}

For any ordinal $\alpha$, we have that $V_\alpha$ is transitive and for all
$\beta < \alpha$, $V_\beta$ is in $V_\alpha$.

\begin{proof}
    We proceed by induction, for $\alpha = 0$, $V_0 = \emptyset$
    which trivially satisfies both statements. For $\alpha = \beta + 1$,
    we use the fact that if $\beta$ is transitive, then $\mc{P}(\beta)$
    is also. By the inductive hypothesis, $V_\alpha = \mc{P}(\beta)$
    is transitive. As $V_\beta$ is in $\mc{P}(V_\beta)$, we have that
    $V_\beta$ is in $V_\alpha$ and if $\beta' < \beta$ then
    by the inductive hypothesis, $V_{\beta'}$ is in $V_{\beta}$ and
    hence $V_{\beta'}$ is in $V_\alpha$ by transitivity. 
    For a limit ordinal $\alpha$,
    $V_\alpha = \bigcup_{\beta < \alpha} V_\beta$ is transitive by
    the inductive hypothesis. For $\beta < \alpha$, it must be that
    $V_\beta$ is in $V_\alpha$ by the definition and transitivity 
    of $V_\alpha$.
\end{proof}

\subsection{The Rank Function (6.3, 6.5)} \label{6.3} \label{6.5}

For any $x$ in $V$, $\rho(x)$ is the least 
$\tau$ such that $x \subseteq V_\tau$ (or rather, $x$ is in $V_{\tau + 1}$). 
We have that: \begin{enumerate}
    \item $V_\alpha = \{x \in V : \rho(x) < \alpha\}$,
    \item For $x$ in $V$, and all $y$ in $x$, $y$ is in $V$ and
        $\rho(y) < \rho(x)$,
    \item For $x$ in $V$, $\rho(x) 
        = \sup(\{\rho(y) + 1 : y \in x\})
        = \sup^+(\{\rho(y) : y \in x\})$.
\end{enumerate} So, the relation on sets: \begin{align*}
    xRy \Longleftrightarrow \rho(x) < \rho(y),
\end{align*} is a strict partial order that is well-founded, meaning
there is a $R$-least element of every non-empty $X \subseteq V$.

\begin{proof}
    (1) For $x$ in $V$, then $\rho(x) < \alpha$ is equivalent to
    saying that there is some $\beta < \alpha$ such that $x \subseteq V_\beta$
    or rather $x$ is in $V_{\beta + 1}$. This is then equivalent to
    saying $x$ is in $V_\alpha$ as $V_{\beta + 1} \subseteq V_\alpha$
    by the transitivity of $V$.
    \\[\baselineskip]
    (2) We take $\rho(x) = \alpha$ so $x \subseteq V_\alpha$ and as such
    $y$ in $x$ must be in $V_\alpha$ so $\rho(y) < \alpha$.
    \\[\baselineskip]
    (3) We take $\alpha = \sup^+(\{\rho(y) : y \in x\})$ and $y$ in $x$.
    By (2), $\rho(y) < \rho(y) + 1 \leq \rho(x)$ so, $\alpha \leq \rho(x)$. 
    By (1), $\rho(y) < \rho(y) + 1 \leq \alpha$ so, $y$ is in $V_\alpha$.
    Thus, $x \subseteq V_\alpha$ so $\rho(x) \leq \alpha$.
\end{proof}

\subsection{Rank and Ordinals (6.6)} \label{6.6}

For an ordinal $\alpha$, $\rho(\alpha) = \alpha$ and 
$(\text{On} \cap V_\alpha) = \alpha$.

\begin{proof}
    The result is trivial for $\alpha = 0$, so we proceed by
    induction with $\alpha > 0$. By (\ref{6.5}): \begin{align*}
        \rho(\alpha) 
        &= {\sup}^+(\{\rho(\beta) : \beta < \alpha\}) \\
        &= {\sup}^+(\{\beta : \beta < \alpha\}) \tag{IH} \\
        &= \alpha.
    \end{align*} From this, we know that 
    $\alpha \subseteq (\text{On} \cap V_\alpha)$. We take
    $\beta$ in $(\text{On} \cap V_\alpha)$ so 
    $\beta = \rho(\beta) < \alpha$. 
    Thus, $(\text{On} \cap V_\alpha) \subseteq \alpha$ and as such,
    $\alpha = (\text{On} \cap V_\alpha)$.
\end{proof}

\subsection{Principle of $\in$-induction (6.7)} \label{6.7}

For a well-defined and definite property of sets $\Phi$: \begin{align*}
    \bigl[ \forall \, z \in y, \Phi(z) \Longrightarrow \Phi(y) \bigr] 
    \Longrightarrow 
    \Phi(y), \tag{$\ast$}
\end{align*} and if $x$ is a transitive set, we have ($\ast$)
for all $y$ in $x$.

\begin{proof}
    For a transitive set $x$, we take $Z = \{y \in x : \neg \Phi(y)\}$,
    supposing $Z \neq \emptyset$. By the Axiom of Foundation, we 
    have $y_0$ in $Z$ such that $y_0$ is $\in$-minimal
    (meaning $(y_0 \cap Z) = \emptyset$). For any $u$ in $y_0$,
    $u$ must be in $x$ as $x$ is transitive. By the minimality
    of $y_0$, it must be that $\Phi(u)$ holds as otherwise $(y_0 \cap Z)$
    would contain $u$. As such, assuming the antecedent, we get
    $\Phi(y_0)$ which is a contradiction of the membership of $y_0$
    in $Z$. For the case on classes, we just take $Z = \{y : \neg \Phi(y)\}$
    and use the same argument.
\end{proof}

\subsection{Theorem of $\in$-recursion (6.8)} \label{6.8}

For a function $G$ from $V$ to $V$, there is exactly one function
$H$ from $V$ to $V$ such that for all $x$: \begin{align*}
    H(x) = G(H \upharpoonright x) = G(\{\ang{y, H(y)} : y \in x\}).
\end{align*}

\begin{proof}
    The proof operates similarly to that on ordinals, but is omitted.
\end{proof}
