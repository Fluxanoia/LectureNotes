\section{Properties of Ordinals}

We collate the properties of ordinals covered so far
for some ordinals $\alpha$, $\beta$, and $\gamma$:
\begin{itemize}
    \item Ordinals are transitive,
    \item Ordinals are well-ordered by $\in$,
    \item $\alpha \in \beta \in \gamma$ implies that $\alpha \in \gamma$,
    \item If $X \in \alpha$ then $X$ is an ordinal with $X = \alpha_X$,
    \item If $\alpha \cong \beta$ then $\alpha = \beta$,
    \item Exactly one of the following holds: \begin{itemize}
        \item $\alpha = \beta$,
        \item $\alpha \in \beta$,
        \item $\beta \in \alpha$.
    \end{itemize}
\end{itemize}

\subsection{Principle of Transfinite Induction}

For $\Phi$ a well-defined and definite property of ordinals, we have
that for all ordinals $\alpha$: \begin{align*}
    \bigl[
        \forall \, \beta < \alpha, \Phi(\beta) 
        \Rightarrow \Phi(\alpha)
    \bigr] \Rightarrow
    \bigl[
        \Phi(\alpha)
    \bigr].
\end{align*} Hence, the class of ordinals is well-ordered.

\begin{proof}
    We consider $C = \{\alpha \text{ an ordinal} \, : \neg \Phi(\alpha)\}$
    and suppose it's non-empty. We take $\alpha_0$ in $C$, if it is not
    the least element we have that $\alpha_0 \cap C$ is non-empty
    as there is some $\beta$ in $C$ with $\beta < \alpha$ which
    is equivalent to saying that $\beta \in \alpha$.
    As $\alpha_0$ is an ordinal, we have that $\alpha_0$ is well-ordered
    by $\in$, hence $C \cap \alpha_0 \subseteq \alpha_0$ has an
    $\in$-least element $\alpha_1$ which is the least element of $C$.
    \\[\baselineskip]
    Thus, we have that $\neg \Phi(\alpha_1)$ holds, but as this is
    the least element of $C$, for all ordinals $\beta$ less than $\alpha_1$
    we have that $\Phi(\beta)$ holds. However, by our antecedent, this
    means $\Phi(\alpha_1)$ holds, a contradiction. Thus, $C$ is empty
    as required.
    \\[\baselineskip]
    Note that our argument showed than any non-empty class on
    the ordinals has a least element. Thus, the class of ordinals
    is well-ordered by $\in$.
\end{proof}

\subsection{The Class of Ordinals}

The class of ordinals is a proper class.

\begin{proof}
    Suppose the class of ordinals is a set $z$.
    We have that $\ang{z, \in}$ is transitive and
    well-ordered by $\in$. Thus, $z$ is an ordinal,
    as such $z$ is in $z$. But, this contradicts the
    strict ordering of $\in$.
\end{proof}

\subsection{Sum of Orderings}

For $\ang{A, R}$ and $\ang{B, S}$ strict total orderings
with $A \cap B$ empty, we
define the sum ordering $\ang{C, T}$ as: \begin{align*}
    C &= A \cup B, \\
    x T y &\Leftrightarrow \begin{cases}
        x R y & \text{for } x, y \in A \\
        x S y & \text{for } x, y \in B \\
        x \in A \text{ and } y \in B & \text{otherwise}.
    \end{cases}
\end{align*} We can avoid the disjoint constraint by taking
the sum of $\ang{A \times \{0\}, R}$ and \linebreak
$\ang{B \times \{1\}, S}$. We can built this functionality into
our operation. We name this operation $+'$ and for $\alpha$, 
$\beta$ ordinals: \begin{align*}
    \alpha +' \beta &= \ot(\ang{
        \alpha \times \{0\} \cup \beta \times \{1\}
    }, T), \\
    \ang{\gamma, i}T\ang{\delta, j} 
    &\Leftrightarrow (i = j \text{ and } \gamma < \delta)
    \text{ or } (i < j).
\end{align*}

\subsection{Product of Orderings}

For $\ang{A, R}$ and $\ang{B, S}$ strict total orderings.
We define the product of these orderings $\ang{A, R} \times \ang{B, S}$
to be the ordering $\ang{C, U}$: \begin{align*}
    C &= A \times B \\
    \ang{x, y}U\ang{x', y'} 
    &\Leftrightarrow 
    (ySy') \text{ or } (y = y' \text{ and } xRx'),
\end{align*} taking the latter and replacing each member by the former.
We can again construct an operation for ordinals $\alpha$
and $\beta$: \begin{align*}
    \alpha \cdot' \beta &= \ot(\ang{A \times B, U}),
\end{align*} with $U$ defined as above.

\subsection{Supremum of Ordinals}

For a set of ordinals $A$, $\sup(A)$ is the least ordinal $\gamma$ 
such that for all $\delta$ in $A$, $\delta \leq \gamma$. Furthermore,
we have the strict supremum $\sup^+(A)$ as the least ordinal $\gamma^+$
such that for all $\delta$ in $A$, $\delta < \gamma^+$.
\\[\baselineskip]
We can also write $\sup(A) = \bigcup A$.
\begin{proof}
    If we suppose there isn't an ordinal which is an upper bound
    for $A$, there's some $\delta$ in $A$ such that $\delta > \gamma$
    for each ordinal $\gamma$. But, $\bigcup A$ must be a set and equal to
    the set of ordinals, a contradiction.
    \\[\baselineskip]
    Take $S = \sup(A)$ and take $u$ in $\bigcup A$. For some $a$ in $A$,
    $u < a < A$, so $u < S$ and so $u$ is in $S$, $\bigcup A \subseteq S$.
    Conversely, we consider $s$ in $S$, then $s < S = \sup(A)$ so there
    is some $a$ in $A$ with $s < a \leq S$. Thus, $s$ is in $A$ so
    $s$ is in $\bigcup A$, $S \subseteq \bigcup A$. Thus $S = \bigcup A$.
\end{proof}

\subsection{Types of Ordinals}

We can consider three types of ordinals: \begin{itemize}
    \item The zero ordinal, $0$,
    \item Successor ordinals, ordinals with immediate predecessors,
    \item Limit ordinals, ordinals that are not of the other types.
\end{itemize}

\subsection{Transfinite Recursion Theorem on Ordinals}

For $F : V \to V$ a function, there exists a unique function
$H$ from the ordinals to $V$ such that for all $\alpha$: \begin{align*}
    H(\alpha) = F(H \upharpoonright \alpha).
\end{align*}

\begin{proof}
    We define a function $u$ to be a $\delta$-approximation if 
    $\Dom(u) = \delta$ and for all $\alpha < \delta$, 
    $u(\alpha) = F(u \upharpoonright \alpha)$. 
    \\[\baselineskip]
    \textbf{Observations} \newline
    We consider $u$ to be a $\delta$-approximation.
    For $\delta > 0$, we see that $u(0) = F(u \upharpoonright 0) = F(\emptyset)$
    so a $1$-approximation is equal to $\{\ang{0, F(\emptyset)}\}$
    with domain $\{0\} = 1$. Additionally, for some $\gamma < \delta$, 
    $u \upharpoonright \gamma$ is a $\gamma$-approximation.
    Furthermore, $u \cup \{\ang{\delta, F(u)}\}$ is a 
    $(\delta + 1)$-approximation. We take $B = \{u : \exists \, \delta
    \text{ such that } u \text{ is a $\delta$-approximation}\}$.
    \\[\baselineskip]
    \textbf{Agreement on Domain} \newline
    For $u$ a $\delta$-approximation and $v$ any $\gamma$-approximation
    with $\delta < \gamma$, $u = v \upharpoonright \delta$.
    \begin{proof}
        We appeal to the contrary and take $\tau$ be the least value such that
        \linebreak $u(\tau) \neq \gamma(\tau)$. Thus, $(u \upharpoonright \tau
        = v \upharpoonright \tau)$ but then: \begin{align*}
            u(\tau) 
            = F(u \upharpoonright \tau)
            = F(v \upharpoonright \tau)
            = v(\tau),
        \end{align*} which is a contradiction.
    \end{proof}
    \vspace{3mm}\noindent
    \textbf{Uniqueness} \newline
    If such $H$ exists, it is unique.
    \begin{proof}
        We appeal to the contrary, taking $H'$ to be some differing
        derivation of $H$. We consider the least $\tau$ such that
        $H(\tau) \neq H'(\tau)$ and apply the same reasoning
        to the above.
    \end{proof}
    \vspace{3mm}\noindent
    \textbf{Limits} \newline
    For some limit ordinal $\lambda$, if for all $\alpha < \lambda$
    we have that $u_\alpha$ is an $\alpha$-approximation, 
    $\bigcup_{\alpha < \lambda} u_\alpha$ is a $\lambda$-approximation.
    \begin{proof}
        This union is a union of an increasing sequence of sets
        (so \linebreak 
        $\alpha < \beta < \lambda \Rightarrow u_\alpha \subseteq u_\beta$).
        As each element is a function, the union is also a function with
        domain $\lambda$. Thus, this union is a $\lambda$-approximation.
    \end{proof}
    \vspace{3mm}\noindent
    \textbf{Existence} \newline
    We define $H = \bigcup B$ which is a function with $\Dom(H)$ being the set
    of ordinals.
    \begin{proof}
        We know that $H$ is a function by \textbf{Agreement on Domain}.
        We take $C = \{\delta : \text{There's no $\delta$-approximation}\}$
        and suppose $C$ is non-empty. By the principle of transfinite induction
        on the ordinals, $C$ has a least elememt $\psi$. We know that
        $\psi > 1$ as we defined a $1$-approximation and by \textbf{Limits}
        it cannot be a limit ordinal. If $\psi = \mu + 1$ then there's
        a $\mu$-approximation $v$ by the minimality of $\psi$. However,
        we can extend $v$ to a $\psi$-approximation $u$ by setting
        $u(\mu) = F(v)$. This is a contradiction.
    \end{proof}
    \noindent
    Thus, we have that $H$ exists and is a unique function as required.
\end{proof}

\subsection{Alternative Transfinite Recursion on Ordinals}

For $a$ in $V$ and $F_0, F_1 : V \to V$ functions, there's a unique
function $H$ from the set of ordinals to $V$ such that: \begin{align*}
    H(0) &= a, \\
    \suc(\alpha) &\Longrightarrow H(\alpha) = F_0(H(\beta)) \text{ where } \alpha = \beta + 1, \\
    \lim(\alpha) &\Longrightarrow H(\alpha) = F_1(H \upharpoonright \alpha).
\end{align*}

\begin{proof}
    We define $F : V \to V$ by: \begin{align*}
        F(u) = \begin{cases}
            a & \text{for } u = \emptyset \\
            F_0(u) & \text{if } u \text{ is a function with a successor domain} \\
            F_1(u) & \text{if } u \text{ is a function with a limit domain} \\
            \emptyset & \text{otherwise},
        \end{cases}
    \end{align*} and apply the previous recursion theorem.
\end{proof}

\subsection{Ordinal Addition}

We define ordinal addition $A_\alpha$ for some successor ordinal
$\beta + 1$ and limit ordinal $\lambda$ as: \begin{align*}
    A_\alpha(0) &= \alpha + 0 = \alpha, \\
    A_\alpha(\beta + 1) &= S(A_\alpha(\beta)) = A_\alpha(\beta) + 1, \\
    A_\alpha(\lambda) &= \sup(\{A_\alpha(x) : x < \lambda\}).
\end{align*}

\subsection{Ordinal Multiplication}

We define ordinal multiplication $A_\alpha$ for some successor ordinal
$\beta + 1$ and limit ordinal $\lambda$ as: \begin{align*}
    M_\alpha(0) &= 0, \\
    M_\alpha(\beta + 1) &= M_\alpha(\beta) + \alpha, \\
    M_\alpha(\lambda) &= \sup(\{M_\alpha(x) : x < \lambda\}).
\end{align*}

\subsection{Ordinal Exponentiation}

We define ordinal exponentiation $A_\alpha$ for some successor ordinal
$\beta + 1$ and limit ordinal $\lambda$ as: \begin{align*}
    E_\alpha(0) &= 1, \\
    E_\alpha(\beta + 1) &= E_\alpha(\beta) \cdot \alpha, \\
    E_\alpha(\lambda) &= \sup(\{E_\alpha(x) : x < \lambda\}).
\end{align*}

\subsection{Monotonicity of $A_\alpha$}

The functions $A_\alpha$ are strictly increasing and thus injective.

\begin{proof}
    We consider $\beta$, $\gamma$, $\delta$ ordinals with: \begin{align*}
        [\beta < \gamma] \Longrightarrow [A_\alpha(\beta) < A_\alpha(\gamma)], \tag{1}
    \end{align*} for all $\gamma \leq \delta$. 
    The base case is trivial, we consider $\delta + 1$. For $\beta < \delta + 1$,
    if $\beta = \delta$, then: \begin{align*}
        A_\alpha(\delta) &= A_\alpha(\beta) \\ 
        &< A_\alpha(\beta + 1) \\
        &= S(A_\alpha(\beta)) \\
        &= S(A_\alpha(\delta)).
    \end{align*} Otherwise, $\beta < \delta$ so by our hypothesis: \begin{align*}
        A_\alpha(\beta) &< A_\alpha(\delta) \\
        &< S(A_\alpha(\delta)) \\
        &= A_\alpha(\delta + 1). 
    \end{align*} Now, we suppose (1) holds for all 
    $\gamma < \lambda$ for some limit ordinal $\lambda$. For $\beta < \lambda$,
    clearly $\beta < \beta + 1 < \lambda$ as $\lambda$ has no immediate predecessor.
    By the hypothesis: \begin{align*}
        A_\alpha(\beta) &< A_\alpha(\beta + 1) \\
        &\leq \sup(\{A_\alpha(\gamma) : \gamma < \lambda\}) \\
        &= A_\alpha(\lambda),
    \end{align*} as required.
\end{proof} \noindent
Similarly, both $M_\alpha$ and $E_\alpha$ are strictly
increasing, for ordinals $\alpha, \beta, \gamma$ with $\beta > \gamma$: 
\begin{itemize}
    \item If $\alpha > 0$ then $M_\alpha(\beta) < M_\alpha(\gamma)$,
    \item If $\alpha > 1$ then $E_\alpha(\beta) < E_\alpha(\gamma)$.
\end{itemize}
