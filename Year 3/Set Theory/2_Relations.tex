\section{Relations}

We will first state the significant properties relations
can have. Taking a relation $R$ on $X$ with $x$, $y$, $z$
arbitrary in $X$: \begin{center}
    \begin{tabular}{ c l }
        \textbf{Name} & \textbf{Property} \\
        \hline
        Reflexive & $xRx$ \\
        Irreflexive & $\neg(xRx)$ \\
        \\
        Symmetric & $xRy \Rightarrow yRx$ \\
        Antisymmetric & $[xRy$ and $yRx] \Rightarrow [x = y]$ \\
        \\
        Connected & $[x = y]$ or $[xRy]$ or $[yRx]$ \\
        \\
        Transitive & $[xRy$ and $yRz] \Rightarrow [xRz]$ \\
    \end{tabular}
\end{center} For example, equivalence relations must satisfy 
reflexivity, symmetry, and transitivity.

\subsection{Partial Orderings}

We say that a relation $\prec$ on a set $X$ is a (strict) partial
ordering if it is irreflexive and transitive.
\\[\baselineskip]
Similarly, we say that a relation $\preceq$ on a set $X$ is a non-strict
partial ordering if it is reflexive, antisymmetric, and transitive.
\\[\baselineskip]
A partial ordering $(X, \prec)$ is wellfounded if for any non-empty subset
$Y$ of $X$, $Y$ has a least element under $\prec$.

\vfill

\subsection{Bounding}

For a partially ordered set $(X, \prec)$: \begin{itemize}
    \item $x_0$ in $X$ is the minimum of $X$ if for all $x$ in $X$,
        $x_0 \preceq x$,
    \item $x'$ in $X$ is minimal in $X$ if for all $x$ in $X$,
        $\neg(x \prec x')$,
    \item $x_1$ in $X$ is the maximum of $X$ if for all $x$ in $X$,
        $x \preceq x_1$,
    \item $x'$ in $X$ is maximal in $X$ if for all $x$ in $X$,
        $\neg(x' \prec x)$.
\end{itemize} Taking a non-empty subset $Y$ of $X$, we consider
the subordering $(Y, \prec)$ and for some $\alpha$ in $X$ we say:
\begin{itemize}
    \item $\alpha$ is a lower bound for $Y$ if for all $y$ in $Y$,
        $\alpha \prec y$,
    \item $\alpha$ is the infimum of $Y$ if it's a lower bound and
        for all lower bounds $\lambda$ of $Y$, $\alpha \preceq \lambda$,
    \item $\alpha$ is an upper bound for $Y$ if for all $y$ in $Y$,
        $y \prec \alpha$,
    \item $\alpha$ is the supremum of $Y$ if it's an upper bound and
        for all upper bounds $\tau$ of $Y$, $\tau \preceq \alpha$.
\end{itemize}

\subsection{Order Preserving Maps}

We say that $f : (X, \prec_1) \to (Y, \prec_2)$ is an order
preserving map if for each $x_1$, $x_2$ in $X$: \begin{align*}
    x_1 \prec_1 x_2 \Longrightarrow f(x_1) \prec_2 f(x_2).
\end{align*} Two orderings are (order) isomorphic if there is
a bijective order preserving map between them.

\subsection{Representation Theorem for Partially Ordered Sets}

For a partially ordered set $(X, \prec)$, there is a set 
$Y \subseteq \mathcal{P}(X)$ which is such that $(X, \preceq)$
is order isomorphic to $(Y, \subseteq)$.

\begin{proof}
   For some $x$ in $X$, we set $X^x = \{x' \in X : x' \preceq x\}$,
   the set of elements preceding or equal to $x$. For $x$, $y$
   in $X$, $x \neq y$ implies that $X^x \neq X^y$ as these sets
   contain $x$ and $y$ (resp.) so $x \mapsto X^x$ is bijective.
   We have that: \begin{align*}
       x \preceq y \Longleftrightarrow X^x \subseteq X^y,
   \end{align*} by our definition. Thus, $x \mapsto X^x$ is
   an order isomorphism.
\end{proof}

\subsection{Total Orderings}

A relation $\prec$ on a set $X$ is a (strict) total ordering if it
is a connected strict partial ordering.
\\[\baselineskip]
Similarly, we say that a relation $\preceq$ on a set $X$ is a non-strict
total ordering if it is a connected non-strict partial ordering.

\subsection{Well-orderings}

A relation $\prec$ on a set $X$ is a well-ordering if it is a
strict total ordering and for any non-empty subset $Y$ of $X$,
$Y$ has a least element under $\prec$. We denote this with 
$(X, \prec) \in WO$.
