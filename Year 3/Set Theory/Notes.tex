\documentclass[a4paper, 12pt, twoside]{article}
\usepackage[left = 3cm, right = 3cm]{geometry}
\usepackage[english]{babel}
\usepackage[utf8]{inputenc}
\usepackage{ragged2e}
\usepackage{mathtools}
\usepackage{amssymb}
\usepackage{amsmath}
\usepackage{amsthm}
\usepackage{multicol}
\usepackage{multirow}
\usepackage{array}
\usepackage{listings}
\usepackage{xcolor}
\usepackage{color}
\usepackage{graphicx}
\usepackage{arydshln}
\usepackage{pifont}
\usepackage{fancyhdr}
\usepackage{hhline}

% Math operators

\DeclareMathOperator{\ID}{id}
\DeclareMathOperator{\Max}{max}
\DeclareMathOperator{\Min}{min}
\DeclareMathOperator{\sgn}{sgn}
\DeclareMathOperator{\Deg}{deg}
\DeclareMathOperator{\Char}{char}
\DeclareMathOperator{\Span}{span}
\DeclareMathOperator{\Dim}{dim}
\DeclareMathOperator{\Ker}{Ker}
\DeclareMathOperator{\Ima}{Im}
\DeclareMathOperator{\Rank}{rank}
\DeclareMathOperator{\Null}{nullity}
\DeclareMathOperator{\End}{End}
\DeclareMathOperator{\Sym}{Sym}
\DeclareMathOperator{\Dom}{dom}
\DeclareMathOperator{\Ran}{ran}
\DeclareMathOperator{\Field}{Field}

% Empty Set Symbol

\let\oldemptyset\emptyset
\let\emptyset\varnothing

% Nicer Lambda

\makeatletter
\newcommand\Pimathsymbol[3][\mathord]{%
  #1{\@Pimathsymbol{#2}{#3}}}
\def\@Pimathsymbol#1#2{\mathchoice
  {\@Pim@thsymbol{#1}{#2}\tf@size}
  {\@Pim@thsymbol{#1}{#2}\tf@size}
  {\@Pim@thsymbol{#1}{#2}\sf@size}
  {\@Pim@thsymbol{#1}{#2}\ssf@size}}
\def\@Pim@thsymbol#1#2#3{%
  \mbox{\fontsize{#3}{#3}\Pisymbol{#1}{#2}}}
\makeatother
\input{utxmia.fd}
\DeclareFontShape{U}{txmia}{m}{n}{<->ssub * txmia/m/it}{}
\DeclareFontShape{U}{txmia}{bx}{n}{<->ssub * txmia/bx/it}{}
\newcommand{\pilambdaup}{\Pimathsymbol[\mathord]{txmia}{21}}
\let\oldlambda\lambda
\let\lambda\pilambdaup

% lstlisting config

\lstset{frame=none,
  language=Haskell,
  aboveskip=3mm,
  belowskip=3mm,
  showstringspaces=false,
  columns=flexible,
  basicstyle={\small\ttfamily},
  numbers=none,
  numberstyle=\tiny\color{gray},
  keywordstyle=\color{blue},
  commentstyle=\color{gray},
  stringstyle=\color{orange},
  breaklines=true,
  breakatwhitespace=true,
  tabsize=2
}

% Double plus symbol

\newcommand\doubleplus{+\kern-1.3ex+\kern0.8ex}
\newcommand\mdoubleplus{\ensuremath{\mathbin{+\mkern-10mu+}}}

% Fraction without line

\newcommand*{\bfrac}[2]{\genfrac{}{}{0pt}{}{#1}{#2}}

\begin{document}

\title{Set Theory Notes}
\date{}
\author{by Tyler Wright \\
  \\
  github.com/Fluxanoia $\qquad$ fluxanoia.co.uk
}
\maketitle

\vfill

\textit{These notes are not necessarily correct,
consistent, representative of the course as it stands today or, 
rigorous. Any result of the above is not the author's fault.}

\addtocounter{section}{-1}

\section{Notation}

We commonly deal with the following concepts in Group Theory
which I will abbreviate as follows for brevity:
\begin{center}
    \begin{tabular}{ | r | c | }
        \hline
        Term & Notation \\
        \hline \hline
        Additive identity of set $X$ & $0_X$ \\
        Multiplicative identity of a set $X$ & $1_X$ \\
        For a field $F$, $(F \backslash \{0_F\}, \times)$ & $F^*$ \\
        $($invertible $n \times n$ matrices on $F$,$\,\times)$ & $GL_n(F)$ \\
        \hline
    \end{tabular}
\end{center}

\newpage

\tableofcontents

\section{The Fundamentals}

\subsection{Binary Operations}

A binary operation on a set $X$ is a map $X \times X \to X$.
\\[\baselineskip]
Take a binary operation $\ast$ on a set $X$, we say that $\ast$ is 
associative if for all $x, y, z$ in $X$: \begin{align*}
    x \ast (y \ast z) = (x \ast y) \ast z.
\end{align*} Furthermore, we say $e$ in $X$ is an identity element of $\ast$ if
for all $x$ in $X$: \begin{align*}
    e \ast x = x \ast e,
\end{align*} and we say that $y$ in $X$ is the inverse to $x$ if
$x \ast y$ and $y \ast x$ are both identities of $\ast$.

\subsection{Groups}

A group $(G, \ast)$ is a non-empty set $G$ combined with a binary operation
$\ast$ such that: \begin{itemize}
    \item $\ast$ is associative,
    \item $G$ contains an identity for $\ast$,
    \item for each element in $G$, there exists some inverse in $G$ 
        with respect to $\ast$.
\end{itemize}

\subsubsection{Distinct Powers of Group Elements}

For an element $x$ in a group $G$, we have that the powers of $x$ are distinct
up to the order of $x$.

\subsubsection{Symmetric Groups}

For a set $X$, the set of bijections $X \to X$ is a group under function
composition denoted by $\Sym(X)$. We typically write 
$\Sym(\{1, 2, \ldots, n\})$ as $S_n$.

\subsubsection{Cyclic Groups}

If we consider a regular $n$-gon $P_n$, we take rotations of
$\frac{2\pi}{n}$ radians about the centre to be $r$ and can define: \begin{align*}
    C_n = \{e, r, r^2, \ldots, r^{n - 1}\},
\end{align*} to be the group of rotational symmetries of $P_n$, the cyclic
group on $P_n$.

\subsubsection{Dihedral Groups}

If we consider again, a regular $n$-gon $P_n$ and take: \begin{align*}
    r &= \text{a rotation of } \frac{2\pi}{n} \text{ radians about the centre}, \\
    s &= \text{reflection in some fixed line of symmetry},
\end{align*} then we have that: \begin{align*}
    \Sym(P_n) = \{e, r, r^2, \ldots, r^{n - 1}, s, rs, r^2s, \ldots, r^{n - 1}s\},
\end{align*} called the dihedral group, denoted by $D_{2n}$.

\subsubsection{The Infinite Cyclic/Dihedral Group}

A map $\varphi$ from $\mathbb{Z} \to \mathbb{Z}$ is a symmetry if for some
$n$ and $m$ in $\mathbb{Z}$: \begin{align*}
    |\varphi(m) - \varphi(n)| = |m - n|.
\end{align*} Taking $r$ to be the symmetry $n \mapsto n + 1$, we can define the
infinite cyclic group: \begin{align*}
    C_\infty = \{\ldots, r^{-2}, r^{-1}, e, r, r^2, \ldots\}.
\end{align*} Taking $s$ to be the symmetry $n \mapsto -n$, we can define the
infinite dihedral group: \begin{align*}
    D_\infty = \{\ldots, r^{-2}, r^{-1}, e, r, r^2, \ldots, r^{-2}s, r^{-1}s, s, rs, r^2s\}.
\end{align*}

\subsubsection{Torsion Groups}

A group is a torsion group if every element has finite order and torsion-free
if every non-identity element has infinite order.

\subsection{$p$-groups}

For $p$ in $\mathbb{P}$, we say that a group $G$ is a $p$-group if the order
of each element of $G$ is a power of $p$.

\newpage

\subsection{Subsets of Groups}

\subsubsection{Set Multiplication}

For $X$, $Y$ subsets of a group $(G, \ast)$, we define: \begin{align*}
    X \ast Y = \{x \ast y : x \in X, y \in Y\},
\end{align*} the product set of $X$ and $Y$ (which is a subset of $G$).
We have that $\ast$ is an associative binary operation on 
$\mathcal{P}(G)$. Additionally, we define: \begin{align*}
    X^{-1} = \{x^{-1} : x \in X\}.
\end{align*} However, these definitions do not define a group on
$\mathcal{P}(G)$ as an inverse does not necessarily exist for each
element, despite the existence of an identity $\{e_G\}$.

\subsubsection{Centre}

For a group $G$, the centre of $G$ is the set of elements that commute with all
elements of $G$, denoted by $Z(G)$: \begin{align*}
    Z(G) = \{z \in G : gz = zg, \forall \, g \in G\}.
\end{align*} We have that $Z(G)$ is a subgroup.

\subsubsection{Properties of Sets}

For a group $(G, \ast)$ with $X \subseteq G$, we have some defined properties:
\begin{itemize}
    \item $X$ is symmetric if for each $x$ in $X$, $x^{-1}$ is also in $X$,
    \item $X$ is closed under $\ast$ if for all $x$, $y$ in $X$, $x \ast y$ is in $X$.
\end{itemize}

\subsection{Order}

For a group $G = (X, \ast)$, $G$ has order $|X|$. The order of an element $x$ of
$X$ is defined as follows: \begin{center}
    \begin{tabular}{ r c l l }
        $|x|$ & $=$ & $\infty$ & if $x^n \neq e_G$ for any $n$ in $\mathbb{N}$, \\
        $|x|$ & $=$ & $\min\{n \in \mathbb{N} \, | \, x^n = e_G\}$ & otherwise. 
    \end{tabular}
\end{center} Taking $x$ in $X$, if $x$ has finite order, then: \begin{enumerate}
    \item $x^n = e_G$ if and only if $|x|$ divides $n$,
    \item $x^n = x^m$ if and only if $|x|$ divides $m - n$,
\end{enumerate} and if $x$ has infinite order: \begin{enumerate}
    \item[3.] $x^n = x^m$ if and only if $n = m$.
\end{enumerate}

\newpage

\begin{proof}
    For (1), we take $n = q|x| + r$ for some $q$ in $\mathbb{Z}$, 
    $r$ in $\{0, 1, \ldots, |x| - 1\}$. Thus: \begin{align*}
        x^n &= x^{q|x|}x^r, \\
        &= e_G^qx^r, \\
        &= x^r,
    \end{align*} and we can see that $x^r = e_G$ if and only if $r = 0$ as $r < |x|$
    and $|x|$ is minimal. Thus, $x^n = e_G$ if and only if $r = 0$ which occurs
    if and only if $|x|$ divides $n$.
    \\[\baselineskip]
    For (2) and (3), we take $x$ to have any order and consider: \begin{align*}
        x^n = x^m, \\
        x^{m - n} = e_G.
    \end{align*} Thus, if $|x| < \infty$ then $|x|$ divides $m - n$ by (1) and
    if $|x| = \infty$ then $m - n = 0$ by the definition of order.
\end{proof}

\subsection{Isomorphisms}

For $(G, \ast)$, $(H, \circ)$ groups, an isomorphism $\varphi : G \to H$ is a
bijection such that $\varphi(x \ast y) = \varphi(x) \circ \varphi(y)$ for all
$x$, $y$ in $G$. If such a map exists, we say $G$ is isomorphic to $H$, denoted
by $G \cong H$.
\\[\baselineskip]
For $G$, $H$, and $K$ groups, $\varphi : G \to H$ and $\psi : H \to K$ isomorphisms,
we have that: \begin{itemize}
    \item $\varphi^{-1}$ is an isomorphism,
    \item $(\psi \circ \varphi)$ is an isomorphism,
\end{itemize} which means $\cong$ is an equivalence relation on any set of groups.

\subsection{Subgroups}

A subset $X$ of a group $(G, \ast)$ is a subgroup if and only if $(X, \ast)$
(with $\ast$ restricted to $X$, for which $X$ must be closed under $\ast$) 
is a group, denoted by $X \leq G$ (or if $X \neq G$, $X < G$).
\\[\baselineskip]
Alternatively, we have that $X$ is a subgroup if and only if: \begin{itemize}
    \item $e_G$ is in $X$,
    \item $X$ is closed under $\ast$,
    \item $X$ is symmetric under $\ast$.
\end{itemize}

\subsubsection{The Product of Subgroups}

For $H$, $K \leq G$, $HK$ is a subgroup of $G$ if and only if $HK = KH$.

\begin{proof}
    By the alternate definition of a subgroup above, we know that for a subgroup
    $X$ of $G$, $X$ contains $e_G$, and $X$ is closed and symmetric under $\ast$.
    \\[\baselineskip]
    Suppose $HK \leq G$, thus: \begin{align*}
        HK &= (HK)^{-1} \\
        &= K^{-1}H^{-1} \\
        &= KH
    \end{align*} Now, suppose $HK = KH$: \begin{itemize}
        \item $e_G$ = $e_Ge_G$ is in $HK$,
        \item $(HK)(HK) = H(KH)K = H(HK)K = (HH)(KK) = HK$,
        \item $(HK)^{-1} = K^{-1}H^{-1} = KH = HK$,
    \end{itemize} so $HK$ is a subgroup.
\end{proof}

\subsubsection{The Subgroup Test}

For $X$ a subset of a group $G$, $X$ is a subgroup if and only if $X \neq \varnothing$
and $x^{-1}y$ is in $X$ for each $x$, $y$ in $X$.

\begin{proof}
    Suppose $X \leq G$, then $e_G$ is in $X$ so $X \neq \varnothing$. For $x$, $y$ in $X$,
    $x^{-1}$ is also in $X$ by the inverse rule of subgroups, so $x^{-1}y$ is also
    in $X$ by the closure of subgroups.
    \\[\baselineskip]
    Suppose $X \neq \varnothing$ and for each $x$, $y$ in $X$, $x^{-1}y$ is also in $X$.
    Taking $x$, $y$ in $X$, we have that $x^{-1}x = e_G$ is also in $X$. 
    Also, $x^{-1}e_G = x^{-1}$ is in $X$. Finally, $xy = (x^{-1})^{-1}y$.
\end{proof}

\subsubsection{The Intersection of Subgroups}

We have that for a group $G$ with $\mathcal{A}$ a set of subgroups of $G$: \begin{align*}
    \bigcap_{a \in \mathcal{A}} a,
\end{align*} is a subgroup of $G$.

\begin{proof}
    We will use the subgroup test. We set $X$ to be the intersection of
    the subgroups in $\mathcal{A}$, $X$ must be non-empty as each subgroup
    must contain $e_G$. Taking $x$, $y$ in $X$, for each $a$ in $\mathcal{A}$,
    we know that $x$ and $y$ are in $a$. As $a$ is a subgroup, $x^{-1}$ and
    thus $x^{-1}y$ are in $a$. As $a$ is arbitrary, $x^{-1}y$ must be in $X$.
\end{proof}

\subsection{Generated Subgroups}

For a group $G$ with $X \subseteq G$ non-empty, we define the subgroup generated by $X$ as:
\begin{align*}
    \langle X \rangle = \bigcap_{A \leq G : X \subseteq A} A,
\end{align*} the intersection of all the subgroups containing $X$.
This can also be called the smallest subgroup containing $X$.
\\[\baselineskip]
Alternatively, we have that: \begin{align*}
    \langle X \rangle = \Gamma(X) = \{x_1 x_2 \cdots x_n : x_i \in X \cup X^{-1}, m \in \mathbb{N} \}.
\end{align*}

\begin{proof}
    We can see that $\Gamma(X) \subseteq \langle X \rangle$ as $\langle X \rangle$
    contains $X$ and is a subgroup so it contains all the finite products
    of elements of $X \cup X^{-1}$ by closure and existence of inverses.
    \\[\baselineskip]
    If we can show that $\Gamma(X)$ is a subgroup, then that would mean
    $\langle X \rangle \subseteq \Gamma(X)$ as $\Gamma(X)$ contains $X$
    so would have been included in the intersection used to generate 
    $\langle X \rangle$. We know that $\Gamma(X)$ is non-empty as $X$ is
    non-empty and taking $x$, $y$ in $\Gamma(X)$, for some $n$, $m$ in $\mathbb{N}$,
    we have that: \begin{align*}
        x &= x_1 x_2 \cdots x_n, \\
        y &= y_1 y_2 \cdots y_m, 
    \end{align*} by the definition of $\Gamma(X)$. For each $x_i$ with
    $i$ in $[n]$, we know that $x_i^{-1}$ is in $\Gamma(X)$ as 
    $X^{-1} \subseteq \Gamma(X)$ so: \begin{align*}
        x^{-1}y &= (x_1 x_2 \cdots x_n)^{-1}y \\
        &= x_n^{-1}x_{n - 1}^{-1} \cdots x_1^{-1} y_1 y_2 \cdots y_m,
    \end{align*} is in $\Gamma(X)$ by its definition. Thus, $\Gamma(X)$
    is a subgroup as required.
\end{proof}

\subsection{Cyclic Groups}

A group $G$ is cyclic if it is generated by a single element. Elements in
$G$ that generate $G$ are called generators. Supposing $G$ is cyclic: \begin{itemize}
    \item For $x$ a generator of $G$, $G = \{x^n : n \in \mathbb{Z}\}$,
    \item $G$ is abelian,
    \item $G \cong C_{|G|}$,
    \item For $X \leq G$, $X$ is cyclic.
\end{itemize}

\subsection{Cosets}

For a group $G$ with $H \leq G$ and $x$ in $G$, the subset $xH$ is a left
coset of $H$ in $G$ and similarly, $Hx$ is a right coset. We have
some properties of left cosets: \begin{itemize}
    \item For $h$ in $H$, $hH = H = Hh$,
    \item For $g$ in $G \, \backslash \, H$ we cannot say $gH = Hg$ in general,
    \item $G$ is the union of all the left cosets,
    \item For $x$, $y$ in $G$, $xH = yH$ if and only if $x$ is in $yH$,
    \item For $x$, $y$ in $G$, either $xH = yH$ or $xH \cap yH = \varnothing$,
    \item For all $x$ in $G$, $|xH| = |H|$.
\end{itemize}

\subsubsection{A Bijection from Left to Right Cosets}

For a group $G$ with $H \leq G$, the map $xH \mapsto Hx^{-1}$ is a 
bijection from the set of left cosets to the set of right cosets.

\subsubsection{A Equivalence Relation on Cosets}

We can define an equivalence relation $\sim$ on a group $G$ with $H \leq G$ 
by setting: \begin{align*}
    x \sim y \Longleftrightarrow y \in xH,
\end{align*} where $xH$ is the equivalence class containing $x$.

\subsubsection{Index}

For a group $G$ with $H \leq G$, the number of distinct left cosets of $H$
in $G$ is called the index of $H$ in $G$, denoted by $[G : H]$ (the choice
of left cosets here is arbitrary due to the bijection between the coset types).

\subsubsection{Lagrange's Theorem}

For a finite group $G$ with $H \leq G$, $|G| = [G : H]|H|$.
\\[\baselineskip]
This means, for any subgroup $H \leq G$, its index and order divide the 
order of $G$. Thus, for $G$ a finite group: \begin{itemize}
    \item For $x$ in $G$, $|x|$ divides $|G|$,
    \item If $G$ has prime order, $G$ is cyclic and every
        non-identity element is a generator,
    \item For $p$ in $\mathbb{P}$ with $P, Q \leq G$ and $|P| = |Q| = p$,
        $P \cap Q = \varnothing$ or $P = Q$.
\end{itemize}

\subsection{Outer Direct Product}

For $G_1, \ldots, G_n$ groups, we set: \begin{align*}
    G_1 \times \cdots \times G_n = \{(a_1, \ldots, a_n) : a_i \in G_i, i \in [n] \},
\end{align*} and define a binary operation on 
$G = G_1 \times \cdots \times G_n$ by: \begin{align*}
    (a_1, \ldots, a_n)(b_1, \ldots, b_n) = (a_1b_1, \ldots, a_nb_n).
\end{align*} $G$ is a group under this operation.

\subsubsection{Properties of the Outer Direct Product}

For $G_1, \ldots, G_n$ groups, with $G = \prod_{i \in [n]} G_i$: \begin{itemize}
    \item $|G| = \prod_{i \in [n]}|G_i|$,
    \item $Z(G) = \prod_{i \in [n]}Z(G_i)$,
    \item If $G$ is cyclic, $G_i$ is cyclic for each $i$ in $[n]$,
    \item For all $\sigma$ in $S_n$, $G \cong \prod_{i \in [n]} G_{\sigma(i)}$,
    \item For the integers $1 \leq n_1 < n_1 < \cdots < n_r < n$, 
        \begin{gather*}
            G \cong 
            (G_1 \times \cdots \times G_{n_1}) \times 
            (G_{n_1 + 1} \times \cdots \times G_{n_2}) \times \cdots \times
            (G_{n_r + 1} \times \cdots \times G_n),
        \end{gather*}
    \item For $H_1, \ldots, H_n$ groups with $G_i \cong H_i$ for each $i$ in $[n]$
        $G \cong \prod_{i \in [n]} H_i$.
\end{itemize}

\section{Relations}

We will first state the significant properties relations
can have. Taking a relation $R$ on $X$ with $x$, $y$, $z$
arbitrary in $X$: \begin{center}
    \begin{tabular}{ c l }
        \textbf{Name} & \textbf{Property} \\
        \hline
        Reflexive & $xRx$ \\
        Irreflexive & $\neg(xRx)$ \\
        \\
        Symmetric & $xRy \Rightarrow yRx$ \\
        Antisymmetric & $[xRy$ and $yRx] \Rightarrow [x = y]$ \\
        \\
        Connected & $[x = y]$ or $[xRy]$ or $[yRx]$ \\
        \\
        Transitive & $[xRy$ and $yRz] \Rightarrow [xRz]$ \\
    \end{tabular}
\end{center} For example, equivalence relations must satisfy 
reflexivity, symmetry, and transitivity.

\subsection{Partial Orderings}

We say that a relation $\prec$ on a set $X$ is a (strict) partial
ordering if it is irreflexive and transitive.
\\[\baselineskip]
Similarly, we say that a relation $\preceq$ on a set $X$ is a non-strict
partial ordering if it is reflexive, antisymmetric, and transitive.
\\[\baselineskip]
A partial ordering $(X, \prec)$ is wellfounded if for any non-empty subset
$Y$ of $X$, $Y$ has a least element under $\prec$.

\vfill

\subsection{Bounding}

For a partially ordered set $(X, \prec)$: \begin{itemize}
    \item $x_0$ in $X$ is the minimum of $X$ if for all $x$ in $X$,
        $x_0 \preceq x$,
    \item $x'$ in $X$ is minimal in $X$ if for all $x$ in $X$,
        $\neg(x \prec x')$,
    \item $x_1$ in $X$ is the maximum of $X$ if for all $x$ in $X$,
        $x \preceq x_1$,
    \item $x'$ in $X$ is maximal in $X$ if for all $x$ in $X$,
        $\neg(x' \prec x)$.
\end{itemize} Taking a non-empty subset $Y$ of $X$, we consider
the subordering $(Y, \prec)$ and for some $\alpha$ in $X$ we say:
\begin{itemize}
    \item $\alpha$ is a lower bound for $Y$ if for all $y$ in $Y$,
        $\alpha \prec y$,
    \item $\alpha$ is the infimum of $Y$ if it's a lower bound and
        for all lower bounds $\lambda$ of $Y$, $\alpha \preceq \lambda$,
    \item $\alpha$ is an upper bound for $Y$ if for all $y$ in $Y$,
        $y \prec \alpha$,
    \item $\alpha$ is the supremum of $Y$ if it's an upper bound and
        for all upper bounds $\tau$ of $Y$, $\tau \preceq \alpha$.
\end{itemize}

\subsection{Order Preserving Maps}

We say that $f : (X, \prec_1) \to (Y, \prec_2)$ is an order
preserving map if for each $x_1$, $x_2$ in $X$: \begin{align*}
    x_1 \prec_1 x_2 \Longrightarrow f(x_1) \prec_2 f(x_2).
\end{align*} Two orderings are (order) isomorphic if there is
a bijective order preserving map between them.

\subsection{Representation Theorem for Partially Ordered Sets}

For a partially ordered set $(X, \prec)$, there is a set 
$Y \subseteq \mathcal{P}(X)$ which is such that $(X, \preceq)$
is order isomorphic to $(Y, \subseteq)$.

\begin{proof}
   For some $x$ in $X$, we set $X^x = \{x' \in X : x' \preceq x\}$,
   the set of elements preceding or equal to $x$. For $x$, $y$
   in $X$, $x \neq y$ implies that $X^x \neq X^y$ as these sets
   contain $x$ and $y$ (resp.) so $x \mapsto X^x$ is bijective.
   We have that: \begin{align*}
       x \preceq y \Longleftrightarrow X^x \subseteq X^y,
   \end{align*} by our definition. Thus, $x \mapsto X^x$ is
   an order isomorphism.
\end{proof}

\subsection{Total Orderings}

A relation $\prec$ on a set $X$ is a (strict) total ordering if it
is a connected strict partial ordering.
\\[\baselineskip]
Similarly, we say that a relation $\preceq$ on a set $X$ is a non-strict
total ordering if it is a connected non-strict partial ordering.

\subsection{Well-orderings}

A relation $\prec$ on a set $X$ is a well-ordering if it is a
strict total ordering and for any non-empty subset $Y$ of $X$,
$Y$ has a least element under $\prec$. We denote this with 
$(X, \prec) \in WO$.


\end{document}