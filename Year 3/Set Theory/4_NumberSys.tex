\section{Number Systems}

\subsection{Von Neumann Numerals}

We have the von Neumann numerals defined as: \begin{align*}
    0 &= \emptyset, \\
    1 &= \{\emptyset\} = \{0\}, \\
    2 &= \{\emptyset, \{\emptyset\}\} = \{1, 2\}, \\
    & \cdots \\
    n + 1 &= \{0, 1, \ldots, n\}.
\end{align*}

\subsection{Inductive Sets}

A set $X$ is called inductive if $\emptyset$ is in $X$ and
for all $x$ in $X$, $S(x)$ is in $X$.

\subsection{Axiom of Infinity}

There exists an inductive set.

\subsection{Natural Numbers}

We say that $x$ is a natural number if for all $X$: \begin{align*}
    X \text{ is an inductive set} \Rightarrow x \in X.
\end{align*} We define $\omega$ as the class of natural numbers,
$\omega = \cap\{X : X \text{ is an inductive set}\}$. We have that
$\omega$ is the smallest inductive set.

\begin{proof}
    Let $z$ be an inductive set (by the Axiom of Infinity it 
    exists). By the Axiom of Subsets, we define a set $N$: 
    \begin{align*}
        N = \{x \in z : \forall \, Y, Y \text{ is inductive} \Rightarrow x \in Y \},
    \end{align*} the elements of $z$ in every inductive set.
    But $N = \omega$, so $\omega$ is a set.
    \\[\baselineskip]
    We know that $\emptyset$ is in every inductive set by definition,
    so $\emptyset$ is in $\omega$ as it is the intersection of all
    inductive sets. For any $x$ in $\omega$, we know that for any
    inductive set $Y$ that $x$ is in $Y$ (by the definition of $\omega$)
    and thus $S(x)$ is also in $Y$ (by the definiton of an
    inductive set). Thus, $S(x)$ is also in $\omega$ as $Y$ was chosen
    arbitrarily. Hence, $\omega$ is an inductive set and the smallest
    such set by its definition.
\end{proof}

\newpage

\subsection{Principle of Mathematical Induction}

We suppose $\Phi$ is a well-defined property of sets, then we have that:
\begin{align*}
    \Bigl[
        \Phi(0) \text{ and } 
        \forall \, x \in \omega \text{ we have that }
        \Phi(x) \Rightarrow \Phi(S(x)) 
    \Bigr] 
    \Rightarrow 
    \Bigl[
        \forall \, x \in \omega \text{ we have that } \Phi(x)
    \Bigr].
\end{align*}

\begin{proof}
    We assume the antecedent, it suffices to show that the collection
    of $x$ in $\omega$ where $\Phi(x)$ holds is inductive (as we
    assume $\Phi(0)$ holds).
    \\[\baselineskip]
    Let $Y = \{x \in \omega : \Phi(x)\}$. As we assumed $\Phi(0)$, we
    know that $0$ is in $Y$. Then, by the second half of our assumption,
    we can see that $Y$ is closed under the successor function.
    Thus, $Y$ is inductive and as $\omega$ is the smallest inductive set,
    $\omega \subseteq Y$ as required.
\end{proof}

\subsection{Representation of Natural Numbers}

We have that every natural number is either $0$ or $S(x)$ for some
natural number $x$.

\begin{proof}
    Let $Z = \{y \in \omega : y = 0 \text{ or } 
    \exists \, x \in \omega \text{ such that } S(x) = y\}$. It 
    suffices to show that $Z$ is inductive. Clearly, $0$ is in $Z$.
    Suppose we have some $u$ in $Z$, then $u$ is in $\omega$.
    As $\omega$ is inductive, $S(u)$ is also in $\omega$ so
    $S(u)$ is in $Z$. Thus, $Z$ is inductive as required.
\end{proof}

\subsection{Transitivity of $\omega$}

We have that $\omega$ is transitive.

\begin{proof}
    Let $X = \{n \in \omega : n \subseteq \omega\}$. If $X = \omega$
    then by definition $\omega$ is transitive. It suffices to show that
    $X$ is inductive. We know that $\emptyset$ is in $X$ as $0$ is in
    $\omega$. Taking $n$ in $X$, then clearly $\{n\} \subseteq \omega$
    as $n$ is in $\omega$. Furthermore, $n \subseteq \omega$ as $n$
    is in $X$. Thus, $n \cup \{n\} \subseteq \omega$ so $S(n) \in X$
    which means $X$ is inductive as required.
\end{proof}

\subsection{Ordering on the Naturals}

For $m$, $n$ in $\omega$, we define: \begin{align*}
    m < n &\Longleftrightarrow m \in n, \\
    m \leq n &\Longleftrightarrow m = n \text{ or } m \in n. \\
\end{align*} By definition, $n < S(n)$.

\newpage
\noindent
We have that: \begin{enumerate}
    \item This ordering is transitive,
    \item For all $n$ in $\omega$ and for all $m$ we have that
        $m < n$ if and only if $S(m) < S(n)$,
    \item For all $n$ in $\omega$, $n \nless n$.
\end{enumerate}

\begin{proof}
    (1) This follows from the transitivity of set inclusion.
    \\[\baselineskip]
    (2) We take $\Phi(k) = [(m < k) \Rightarrow (S(m) < S(k))]$.
    We see $\Phi(0)$ holds. Supposing $\Phi(k)$ holds for some
    $k$, let $m < S(k)$ then $m$ is in $k \cup \{k\}$. If $m$ 
    is in $k$ then by $\Phi(k)$ we have that $S(m) < S(k) < S(S(k))$.
    If $m = k$ then $S(m) = S(k) < S(S(k))$. Thus, by induction, 
    we have our result.
    \\[\baselineskip]
    Assume $S(m) < S(n)$, $m$ is in $S(m) = m \cup \{m\}$ which
    is in $S(n) = n \cup \{n\}$. If $S(m) = n$, then $m$ is in $n$
    so $m < n$. If $S(m)$ is in $n$ then $m$ is in $n$ as $n$ is
    transitive.
    \\[\baselineskip]
    (3) We know that $0 \nless 0$ as $0 \notin 0$. If $k \notin k$
    then $S(k) \notin S(k)$ by Part (ii). Thus, 
    $X = \{k \in \omega : k \notin k\}$ is inductive which makes
    it equal to $\omega$ as required.
\end{proof}

\subsection{Total Ordering on the Naturals}

We have that $<$ is a (strict) total ordering on the naturals.

\subsection{Well-ordering Theorem for $\omega$}

Let $X \subseteq \omega$, then either $X = \emptyset$ or there is some
$n_0$ in $X$ such that for any $m$ in $X$ either $n_0 = m$ or $n_0 < m$.

\begin{proof}
    Suppose $X \subseteq \omega$ but has no least element.
    Let $Z = \{k \in \omega : \forall \, n < k, n \notin X\}$.
    We want to show $Z$ is inductive, meaning $Z = \omega$ and
    so $X = \emptyset$. 
    \\[\baselineskip]
    Vacuously, $0$ is in $Z$. Suppose we have $k$ in 
    $Z$, we let $n < S(k) = k \cup \{k\}$ and consider: \begin{itemize}
        \item If $n \in k$ then $n \notin X$ as $n < k \in Z$,
        \item If $n = k$ then $n \notin X$ because if $n$ was in $X$
            then it would be the least element of $X$, a contradiction.
    \end{itemize} Thus, $S(k)$ is in $Z$ so $Z$ is inductive.
\end{proof}

\subsection{Recursion Theorem on $\omega$}

Let $A$ be any set with $a$ in $A$ and $f : A \to A$ any function.
There exists a unique function $h : \omega \to A$ such that
for any $n$ in $\omega$:
\begin{align*}
    h(0) &= a, \\
    h(S(n)) &= f(h(n)).
\end{align*}