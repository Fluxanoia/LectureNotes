\section{Cardinality}

\subsection{Equinumerosity}

We say that two sets, $A$ and $B$, are equinumerous if there is a bijection 
between them, written as $A \approx B$.
\\[\baselineskip]
We have that $\approx$ is an equivalence relation with equivalence classes
as collections of all equinumerous sets of a size.

\subsection{Finite Sets}

A set is finite if it is equinumerous with a natural number. Sets that are
not finite are infinite.

\subsection{Pidgeon-hole Principle}

No natural number is equinumerous to a proper subset of itself and thus: \begin{itemize}
    \item No finite set is equinumerous to a proper subset of itself,
    \item Any set equinumerous to a proper subset of itself is infinite,
    \item Any finite set is equinumerous to a unique natural number,
    \item $\omega$ is infinite.
\end{itemize}

\begin{proof}
    We take $Z = \{n \in \omega : \forall \, f, 
    (f : n \to n \text{ and injective}) \Rightarrow (\Ran(f) = n)\}$. Trivially,
    $Z$ contains $0$. For $n$ in $Z$, we consider $f : (n + 1) \to (n + 1)$
    an injective function.

    \paragraph{Case 1} We suppose that $f \upharpoonright n : n \to n$
    is an injective function and by our inductive hypothesis,
    $\Ran(f \upharpoonright n) = n$. Thus, $\Ran(f) = n + 1$.

    \paragraph{Case 2} We suppose that $f(m) = n$ for some $m < n$.
    As $f$ is injective, for some $k < n$, $f(n) = k$. We
    define $g$ identically to $f$ except $g(m) = k$ and $g(n) = n$
    so that $g : (n + 1) \to (n + 1)$ and injective so \textbf{Case 1}
    applies to $g$. Hence, \linebreak $\Ran(g) = n + 1 = \Ran(f)$.
\end{proof}

\subsection{Cantor's Diagonal Argument}

The natural numbers are not equinumerous with the real numbers.

\begin{proof}
    We appeal to the contrary and suppose we have some injective map
    $f : \omega \to \mb{R}$: \begin{align*}
        f(0) &= 2.72938 \ldots \\
        f(1) &= 3.47000 \ldots \\
        f(2) &= 9.32789 \ldots \\
        &\vdots 
    \end{align*} We can generate some $x$ not in $\Ran(f)$ by
    setting the $i^{\text{th}}$ decimal place to the $i^{\text{th}}$
    decimal place of $f(i)$ mapped by: \begin{align*}
        k \mapsto \begin{cases}
            1 & k \text{ even} \\
            2 & k \text{ odd.} \\
        \end{cases}
    \end{align*} Thus, $x$ would differ from every element of $\Ran(f)$.
    A contradiction.
\end{proof}

\subsection{Infinite Infinities}

No set is equinumerous to its powerset.

\begin{proof}
    We appeal to the contrary and suppose $f : X \to \mc{P}(X)$ is a bijection
    for some set $X$. We set $Z = \{u \in X : u \notin f(u)\}$ and see that
    $Z \subseteq X$ so $Z$ is in $\mc{P}(X)$. As such, $Z = f(u)$ for some
    $u$, but:
    \begin{align*}
        u \in Z &\Longrightarrow u \notin f(u), \\
        u \notin Z &\Longrightarrow u \in f(u),
    \end{align*} which is a contradiction.
\end{proof}
