\section{Cardinality}

\subsection{Equinumerosity}

We say that two sets, $A$ and $B$, are equinumerous if there is a bijection 
between them, written as $A \approx B$.
\\[\baselineskip]
We have that $\approx$ is an equivalence relation with equivalence classes
as collections of all equinumerous sets of a size.

\subsection{Finite Sets}

A set is finite if it is equinumerous with a natural number. Sets that are
not finite are infinite.

\subsection{Pidgeon-hole Principle}

No natural number is equinumerous to a proper subset of itself and thus: \begin{itemize}
    \item No finite set is equinumerous to a proper subset of itself,
    \item Any set equinumerous to a proper subset of itself is infinite,
    \item Any finite set is equinumerous to a unique natural number,
    \item $\omega$ is infinite.
\end{itemize}

\begin{proof}
    We take $Z = \{n \in \omega : \forall \, f, 
    (f : n \to n \text{ and injective}) \Rightarrow (\Ran(f) = n)\}$. Trivially,
    $Z$ contains $0$. For $n$ in $Z$, we consider $f : (n + 1) \to (n + 1)$
    an injective function.

    \paragraph{Case 1} We suppose that $f \upharpoonright n : n \to n$
    is an injective function and by our inductive hypothesis,
    $\Ran(f \upharpoonright n) = n$. Thus, $\Ran(f) = n + 1$.

    \paragraph{Case 2} We suppose that $f(m) = n$ for some $m < n$.
    As $f$ is injective, for some $k < n$, $f(n) = k$. We
    define $g$ identically to $f$ except $g(m) = k$ and $g(n) = n$
    so that $g : (n + 1) \to (n + 1)$ and injective so \textbf{Case 1}
    applies to $g$. Hence, \linebreak $\Ran(g) = n + 1 = \Ran(f)$.
\end{proof}

\subsection{Cantor's Diagonal Argument}

The natural numbers are not equinumerous with the real numbers.

\begin{proof}
    We appeal to the contrary and suppose we have some injective map \linebreak
    $f : \omega \to \mb{R}$: \begin{align*}
        f(0) &= 2.72938 \ldots \\
        f(1) &= 3.47000 \ldots \\
        f(2) &= 9.32789 \ldots \\
        &\vdots 
    \end{align*} We can generate some $x$ not in $\Ran(f)$ by
    setting the $i^{\text{th}}$ decimal place to the $i^{\text{th}}$
    decimal place of $f(i)$ mapped by: \begin{align*}
        k \mapsto \begin{cases}
            1 & k \text{ even} \\
            2 & k \text{ odd.} \\
        \end{cases}
    \end{align*} Thus, $x$ would differ from every element of $\Ran(f)$.
    A contradiction.
\end{proof}

\subsection{Cantor's Theorem}

No set is equinumerous to its powerset.

\begin{proof}
    We appeal to the contrary and suppose $f : X \to \mc{P}(X)$ is a bijection
    for some set $X$. We set $Z = \{u \in X : u \notin f(u)\}$ and see that
    $Z \subseteq X$ so $Z$ is in $\mc{P}(X)$. As such, $Z = f(u)$ for some
    $u$, but:
    \begin{align*}
        u \in Z &\Longrightarrow u \notin f(u), \\
        u \notin Z &\Longrightarrow u \in f(u),
    \end{align*} which is a contradiction.
\end{proof}

\subsection{Cantor-Schröder-Bernstein Theorem}

For sets $X$, $Y$, $X \preceq Y$ if there's an injection from $X$ to $Y$
and $X \prec Y$ if $X \preceq Y$ and $Y \npreceq X$.
We have that $X \preceq Y$ and $Y \preceq X$ is equivalent
to $X \approx Y$.
\\[\baselineskip]
We write $A \propto B$ if $A \preceq B$ but $B \npreceq A$.

\newpage

\begin{proof}
    ($\Rightarrow$) We have $f : X \to Y$ and $g : Y \to X$ both
    injective and want to form some $h : X \to Y$ bijective.
    We consider $C_0 = X \backslash \Ran(g)$, the values suppressing
    the surjectivity of $g$. For $n$ in $\mb{N}$, we define: \begin{align*}
        D_{n}     &= f''C_n \\
        C_{n + 1} &= g''D_n = g''(f''C_n), \\
        \\
        h(v) &= \begin{cases}
            f(v) & \text{if $v$ is in $C_n$ for some $n$} \\
            g^{-1}(v) & \text{otherwise.}
        \end{cases}
    \end{align*} To see that $h$ is injective, we consider $u$ and $v$ in
    $X$ and note that as both $f$ and $g$ are injective, it's sufficient to
    show that $h$ is injective under $u$ in some $C_n$ and $v$ not in any
    $C_n$ (without loss of generality). In this case, we take $u$ to be in $C_m$ 
    and see that: \begin{align*}
        h(u) &= f(u) \in D_m, \\
        h(v) &= g^{-1}(v) \notin D_m,
    \end{align*} as otherwise $g(g^{-1}(v)) = v \in C_{m + 1}$ which is
    a contradiction to how we selected $v$. So, $u \neq v$ implies that
    $h(u) \neq h(v)$ and as such, $h$ is injective. To see that $h$ is
    surjective, we first note that $U = \bigcup_{m \in \mb{N}} D_m \subseteq \Ran(h)$.
    We consider $u$ in $Y \backslash U$, $g(u)$ is not in $C_0 = X \backslash \Ran(g)$
    and not in any $C_{n + 1}$ either as $u$ is not in any $D_n$ and as $g$ is
    injective, there's no $v$ in $D_n$ such that $g(v) = g(u)$. As such,
    $h(g(u)) = g^{-1}(g(u)) = u$. So, $h$ is surjective and as such, bijective.
    \\[\baselineskip]
    ($\Leftarrow$) This direction follows from the definitions.
\end{proof}

\subsection{Characteristic Function}

For a set $X$, we define the the characteristic function of any $Y \subseteq X$
to be $\varchi_Y : X \to 2$ defined by: \begin{align*}
    \varchi_Y(a) = \begin{cases}
        1 & \text{if $a$ is in $Y$} \\
        0 & \text{if $a$ is in $X \backslash Y$}.
    \end{cases}
\end{align*}

\subsection{Countability}

A set $X$ is countably infinite if $X \approx \omega$ and countable if
$X \preceq \omega$. 
\\[\baselineskip]
The union of countably infinite sets is also countably infinite.
Subsets of countable sets are countable.

\subsection{Countably Infinite Subsets}

For an infinite set $X$ with $\ang{X, R}$ a well-ordering, $X$ has a
countably infinite subset.

\subsection{Well-ordering Principle}

For a set $X$, there is a well-ordering $\ang{X, R}$.
This implies that the union of countably infinite sets
is countably infinite.

\subsection{Cardinality}

For a set $X$, the cardinality of $X$, $|X|$ is the
least ordinal $\alpha$ such that $X \approx \alpha$.
We have that for $X$, $Y$ sets: \begin{align*}
    X \approx Y &\Longleftrightarrow |X| = |Y|, \\
    X \preceq Y &\Longleftrightarrow |X| \leq |Y|, \\
    X \prec Y &\Longleftrightarrow |X| < |Y|.
\end{align*}

\subsection{Cardinal Numbers}

An ordinal $\alpha$ is a cardinal if $\alpha = |\alpha|$.

\subsection{Cardinality Capture}

For $\alpha$, $\gamma$ ordinals,
$|\alpha| \leq \gamma < \alpha$ then $|\alpha| = |\gamma|$.

\begin{proof}
    By definition, there's a bijection $f$ from $\alpha$
    to $|\alpha|$, so $||a|| = |a|$. We know that $\gamma
    \subseteq \alpha$, so we know that
    $f \upharpoonright \gamma$ is an injection from $\gamma$ to 
    $|\alpha|$ so $\gamma \preceq |\alpha|$. But, 
    $|\alpha| \preceq \gamma$ by our assumption,
    so $|\alpha| \approx \gamma$ which implies that
    $|\gamma| \approx ||\alpha|| = |\alpha|$.
\end{proof}

\subsection{Cardinal Addition and Multiplication}

For cardinals $\kappa$ and $\lambda$, and sets $K$ and $L$ with cardinality
$\kappa$ and $\lambda$ respectively, we define: \begin{align*}
    \kappa \oplus \lambda &= |K \cup L|, \tag{for $K$, $L$ disjoint} \\
    \kappa \otimes \lambda &= |K \times L|.
\end{align*} We note that these operations are commutative and associative.
Furthermore, for $n$ and $m$ in $\omega$: \begin{align*}
    n + m &= n \oplus m < \omega, \\
    n \cdot m &= n \otimes m < \omega.
\end{align*}

\subsection{Hessenberg's Theorem}

For an infinite cardinal $\kappa$, there is a bijection from $\kappa \times \kappa$
to $\kappa$. Thus, $\kappa \otimes \kappa = \kappa$.

\begin{proof}
    We already know that $\omega \times \omega \approx \omega$ so we proceed by
    induction on $\lambda \geq \omega$. We assume $\lambda < \kappa$ so
    $\lambda \times \lambda \approx \lambda$. We consider Gödel's ordering: \begin{align*}
        \bigl[
            \ang{\alpha, \beta} \triangleleft \ang{\lambda, \delta}
        \bigr]
        \Longleftrightarrow&
        \bigl[
            (\max(\{\alpha, \beta\}) < \max(\{\gamma, \delta\})) \\
            & \text{ or } \bigl[
            (\max(\{\alpha, \beta\}) = \max(\{\gamma, \delta\})) \\
            & \text{ and }
            (\alpha < \gamma \text{ or } (\alpha = \gamma \text{ and } \beta < \delta))
        \bigr]\bigr].
    \end{align*} We know that $\ang{\kappa \times \kappa, \triangleleft}$ is a
    well-ordering and each $\ang{\alpha, \beta}$ in $\kappa \times \kappa$ has
    no more than 
    $|(\max(\{\alpha, \beta\}) + 1) \times (\max(\{\alpha + \beta\}) + 1)| < \kappa$
    $\triangleleft$-predecessors as shown here:
    \begin{proof}
        We can see that: \begin{align*}
            (\kappa \times \kappa)_{\ang{\alpha, \beta}} \subset \gamma \times \gamma,
        \end{align*} where $\gamma = \max(\{\alpha, \beta\}) + 1$. But, we know that
        $|\gamma \times \gamma| = |\gamma| \otimes |\gamma|$ and as $\gamma < \kappa$,
        $|\gamma|$ is an infinite cardinal less than $\kappa$, $|\gamma| \otimes |\gamma|
        = \gamma$ by the inductive hypothesis.
    \end{proof} 
    \noindent
    All initial segments $\ang{(\kappa \times \kappa)_{\ang{\alpha, \beta}}, \triangleleft}$
    have order type less than $\kappa$ so $\ot(\kappa \times \kappa, \triangleleft) \leq \kappa$.
    But $\kappa \times \kappa \geq \kappa$ so $\ot(\kappa \times \kappa, \triangleleft) = \kappa$.
\end{proof}

\subsection{Confluence of Addition and Multiplication}

For infinite cardinals $\kappa$, $\lambda$,
$\kappa \oplus \lambda = \kappa \otimes \lambda = \max(\{\kappa, \lambda\})$.

\begin{proof}
    Without loss of generality, we assume $\lambda \leq \kappa$ so
    $\max{\{\kappa, \lambda\}} = \kappa$. For $X$ and $Y$ disjoint with
    cardinality $\kappa$ and $\lambda$ (resp.): \begin{align*}
        X \preceq X \cup Y \preceq (X \times \{0\}) \cup (X \times \{1\})
        = X \times 2 \preceq X \times X.
    \end{align*} So, in terms of cardinals we have: \begin{align*}
         \kappa \leq \kappa \oplus \lambda \leq \kappa \oplus \kappa
         = \kappa \otimes 2 \leq \kappa \otimes \kappa.
    \end{align*} But, by Hessenberg's Theroem, $\kappa = \kappa \otimes \kappa$
    which induces equality on all the above statements. 
    As such, $\kappa = \kappa \oplus \lambda$ and similarly: \begin{align*}
        \kappa \leq \kappa \otimes \lambda \leq \kappa \otimes \kappa = \kappa,
    \end{align*} we have that $\kappa = \kappa \otimes \lambda$, as required.
\end{proof}

\subsection{Cardinality of a Countable Union of Infinite Cardinals}

For a set $A$, $\ps{<\omega}A = \bigcup_{n \in \omega} \ps{n}A$.
For an infinite cardinal $\kappa$, $|\ps{<\omega}\kappa| = \kappa$.

\subsection{Cardinal Exponentiation}

For cardinals $\kappa$ and $\lambda$, $\kappa^{\lambda} = |\ps{L}K|$ where $K$ and $L$
are sets of cardinality $\kappa$ and $\lambda$ respectively. 
For cardinals $\kappa$, $\lambda$, and $\mu$, we have that:
\begin{align*}
    \kappa^{\lambda \oplus \mu} &= \kappa^{\lambda} \otimes \kappa^{\mu}, \\
    (\kappa^{\lambda})^{\mu} &= \kappa^{\lambda \otimes \mu}.
\end{align*}

\subsection{Equinumerosity with Characteristic Functions}

For cardinals $\kappa$ and $\lambda$ with $\lambda \geq \omega$ and 
$2 \leq \kappa \leq \lambda$, then
$\ps{\lambda}\lambda \approx \ps{\lambda}\kappa 
\approx \ps{\lambda}2 \approx \mc{P}(\lambda)$.

\subsection{Class of Cardinals}

The class of cardinals is a proper class.

\begin{proof}
    We suppose the class of cardinals is a set, as it's the
    union of ordinals, it's an ordinal $\tau$. 
    By Cantor's theorem, $|\mc{P}(\tau)| > \tau$
    which is a cardinal not in our set of cardinals,
    a contradiction.
\end{proof}

\subsection{Unbounded Ordinals}

For any set $x$, there's an ordinal $\alpha$ with $\alpha \npreceq x$.

\begin{proof}
    We take $\alpha = |\mc{P}(x)|$ and we are done by Cantor's theorem.
\end{proof}

\subsection{The Infinite Cardinals}

For some ordinal $\alpha$, and $\lambda$ a limit ordinal, we have the infinite
cardinals: \begin{align*}
    \aleph_0 = \omega_0 &= \omega, \\
    \aleph_{\alpha + 1} = \omega_{\alpha + 1} = \omega_{\alpha}^+ 
    &= \text{the least ordinal containing } \omega_{\alpha} \\
    \aleph_{\lambda}  = \omega_{\lambda} = \sup(\{\omega_{\tau} : \tau < \beta\}).
\end{align*} We have a function $F_\aleph$ from the ordinals to
the infinite cardinals defined by: \begin{align*}
    F_\aleph(\alpha) = \omega_\alpha.
\end{align*}
