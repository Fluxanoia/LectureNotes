\section{Cardinality}

\subsection{Equinumerosity}

We say that two sets, $A$ and $B$, are equinumerous if there is a bijection 
between them, written as $A \approx B$.
\\[\baselineskip]
We have that $\approx$ is an equivalence relation with equivalence classes
as collections of all equinumerous sets of a size.

\subsection{Finite Sets}

A set is finite if it is equinumerous with a natural number. Sets that are
not finite are infinite.

\subsection{Pidgeon-hole Principle}

No natural number is equinumerous to a proper subset of itself and thus: \begin{itemize}
    \item No finite set is equinumerous to a proper subset of itself,
    \item Any set equinumerous to a proper subset of itself is infinite,
    \item Any finite set is equinumerous to a unique natural number,
    \item $\omega$ is infinite.
\end{itemize}

\begin{proof}
    We take $Z = \{n \in \omega : \forall \, f, 
    (f : n \to n \text{ and injective}) \Rightarrow (\Ran(f) = n)\}$. Trivially,
    $Z$ contains $0$. For $n$ in $Z$, we consider $f : (n + 1) \to (n + 1)$
    an injective function.

    \paragraph{Case 1} We suppose that $f \upharpoonright n : n \to n$
    is an injective function and by our inductive hypothesis,
    $\Ran(f \upharpoonright n) = n$. Thus, $\Ran(f) = n + 1$.

    \paragraph{Case 2} We suppose that $f(m) = n$ for some $m < n$.
    As $f$ is injective, for some $k < n$, $f(n) = k$. We
    define $g$ identically to $f$ except $g(m) = k$ and $g(n) = n$
    so that $g : (n + 1) \to (n + 1)$ and injective so \textbf{Case 1}
    applies to $g$. Hence, \linebreak $\Ran(g) = n + 1 = \Ran(f)$.
\end{proof}

\subsection{Cantor's Diagonal Argument}

The natural numbers are not equinumerous with the real numbers.

\begin{proof}
    We appeal to the contrary and suppose we have some injective map
    $f : \omega \to \mb{R}$: \begin{align*}
        f(0) &= 2.72938 \ldots \\
        f(1) &= 3.47000 \ldots \\
        f(2) &= 9.32789 \ldots \\
        &\vdots 
    \end{align*} We can generate some $x$ not in $\Ran(f)$ by
    setting the $i^{\text{th}}$ decimal place to the $i^{\text{th}}$
    decimal place of $f(i)$ mapped by: \begin{align*}
        k \mapsto \begin{cases}
            1 & k \text{ even} \\
            2 & k \text{ odd.} \\
        \end{cases}
    \end{align*} Thus, $x$ would differ from every element of $\Ran(f)$.
    A contradiction.
\end{proof}

\subsection{Infinite Infinities}

No set is equinumerous to its powerset.

\begin{proof}
    We appeal to the contrary and suppose $f : X \to \mc{P}(X)$ is a bijection
    for some set $X$. We set $Z = \{u \in X : u \notin f(u)\}$ and see that
    $Z \subseteq X$ so $Z$ is in $\mc{P}(X)$. As such, $Z = f(u)$ for some
    $u$, but:
    \begin{align*}
        u \in Z &\Longrightarrow u \notin f(u), \\
        u \notin Z &\Longrightarrow u \in f(u),
    \end{align*} which is a contradiction.
\end{proof}

\subsection{Cantor-Schröder-Bernstein Theorem}

For sets $X$, $Y$, $X \preceq Y$ if there's an injection from $X$ to $Y$
and $X \prec Y$ if $X \preceq Y$ and $Y \npreceq X$.
We have that $X \preceq Y$ and $Y \preceq X$ is equivalent
to $X \approx Y$.

\newpage

\begin{proof}
    ($\Rightarrow$) We have $f : X \to Y$ and $g : Y \to X$ both
    injective and want to form some $h : X \to Y$ bijective.
    We consider $C_0 = X \backslash \Ran(g)$, the values suppressing
    the surjectivity of $g$. For $n$ in $\mb{N}$, we define: \begin{align*}
        D_{n}     &= f''C_n \\
        C_{n + 1} &= g''D_n = g''(f''C_n), \\
        \\
        h(v) &= \begin{cases}
            f(v) & \text{if $v$ is in $C_n$ for some $n$} \\
            g^{-1}(v) & \text{otherwise.}
        \end{cases}
    \end{align*} To see that $h$ is injective, we consider $u$ and $v$ in
    $X$ and note that as both $f$ and $g$ are injective, it's sufficient to
    show that $h$ is injective under $u$ in some $C_n$ and $v$ not in any
    $C_n$ (without loss of generality). In this case, we take $u$ to be in $C_m$ 
    and see that: \begin{align*}
        h(u) &= f(u) \in D_m, \\
        h(v) &= g^{-1}(v) \notin D_m,
    \end{align*} as otherwise $g(g^{-1}(v)) = v \in C_{m + 1}$ which is
    a contradiction to how we selected $v$. So, $u \neq v$ implies that
    $h(u) \neq h(v)$ and as such, $h$ is injective. To see that $h$ is
    surjective, we first note that $U = \bigcup_{m \in \mb{N}} D_m \subseteq \Ran(h)$.
    We consider $u$ in $Y \backslash U$, $g(u)$ is not in $C_0 = X \backslash \Ran(g)$
    and not in any $C_{n + 1}$ either as $u$ is not in any $D_n$ and as $g$ is
    injective, there's no $v$ in $D_n$ such that $g(v) = g(u)$. As such,
    $h(g(u)) = g^{-1}(g(u)) = u$. So, $h$ is surjective and as such, bijective.
    \\[\baselineskip]
    ($\Leftarrow$) This direction follows from the definitions.
\end{proof}

\subsection{Characteristic Function}

For a set $X$, we define the the characteristic function of any $Y \subseteq X$
to be $\varchi_Y : X \to 2$ defined by: \begin{align*}
    \varchi_Y(a) = \begin{cases}
        1 & \text{if $a$ is in $Y$} \\
        0 & \text{if $a$ is in $X \backslash Y$}.
    \end{cases}
\end{align*}

\subsection{Countability}

A set $X$ is countably infinite if $X \approx \omega$ and countable if
$X \preceq \omega$. 
\\[\baselineskip]
The union of countably infinite sets is also countably infinite.
Subsets of countable sets are countable.
