\section{The Axioms}

\subsection{Axiom of Extensionality}

For two sets $a$ and $b$, we have that $a = b$ if and only if for all
$x$ we have that: \begin{align*}
    x \in a \Longleftrightarrow x \in b.
\end{align*} 
For two classes $A$ and $B$, we have that $A = B$ if and only if for all
$x$ we have that: \begin{align*}
    x \in a \Longleftrightarrow x \in b.
\end{align*}

\subsection{Axiom of Pair Sets}

For any sets $x$ and $y$, there is a set $z = \{x, y\}$. This is the
(unordered) pair set of $x$ and $y$.

\subsection{Axiom of the Powerset}

For each set $x$, there exists a set which is the collection of the
subsets of $x$, the powerset $\mathcal{P}(x)$.
We have the powerset defined as 
$\mathcal{P}(x) = \{z : z \subseteq x\}$.

\subsection{Axiom of the Empty Set}

There exists a set with no members, the empty set $\emptyset$.
We have the empty set defined as $\emptyset = \{x : x \neq x\}$.

\subsection{Axiom of Subsets}

For some set $x$, we have that $\{y \in x : \Phi(y)\}$ is a set
for some well-defined property of sets $\Phi$.

\subsection{Axiom of Infinity}

There exists an inductive set.

\subsection{Axiom of Unions (1.6)} \label{1.6}

We have the basic union of two sets $x_1$ and $x_2$: \begin{align*}
    x_1 \cup x_2 = \{y : y \in x_1 \text{ or } y \in x_2\},
\end{align*} but for when we want to unify the members of the sets in
a set $x$, we define: \begin{align*}
    \bigcup x = \{y : \exists \, z \in x, y \in z\}.
\end{align*} This axiom states that for a set $x$, $\bigcup x$ is a set. 

\subsection{Intersections (1.8)} \label{1.8}

We have the basic intersection of two sets $x_1$ and $x_2$: \begin{align*}
    x_1 \cap x_2 = \{y : y \in x_1 \text{ and } y \in x_2\},
\end{align*} but for when we want to intersect the members of 
the sets in a set $x$, we define: \begin{align*}
    \bigcup x = \{y : \forall \, z \in x, y \in z\}.
\end{align*} This is a set by the Axiom of Subsets.

\subsection{Axiom of Replacement}

For a function $F$ from $V$ to itself and a set $x$, $F''x$ is a set.

\subsection{Well-ordering Principle}

For a set $X$, there is a well-ordering $\ang{X, R}$. 

\subsection{Axiom of Choice (5.1)} \label{5.1}

For a set of non-empty sets $\mc{G}$, there is a choice function
$F$ from $\mc{G}$ to $\bigcup \mc{G}$ such that for all $X$ in $\mc{G}$,
$F(X)$ is in $X$. This is equivalent to the Well-ordering Principle.

\begin{proof}
    ($\Longrightarrow$) For an arbitrary set $Y$, it is sufficient to
    show $Y$ has a well-ordering. 
    We take $Y \neq \emptyset$ as otherwise $Y$
    is trivially well-ordered. We take 
    $\mc{G} = \{X \subseteq Y : X \neq \emptyset\}$. By the Axiom
    of Choice, we have a choice function $F_0$ for $\mc{G}$.
    We take $u$ to be any set not in $Y$ and
    define $F$ from $V$ to $V$: \begin{align*}
        F(t) &= \begin{cases}
            F_0(t) & \text{if } t \in \mc{G} \\
            u      & \text{otherwise}.
        \end{cases}
    \end{align*} By the recursion theorem, we define $H_0$ from
    the ordinals to $Y \cup \{u\}$: \begin{align*}
        H_0(\xi) &= F(Y \setminus \{H_0(\zeta) : \zeta < \xi\}).
    \end{align*} We can see that: \begin{align*}
        H_0(0) &= F_0(Y) \in Y, \\
        H_0(1) &= F_0(Y \setminus \{F_0(Y)\}) \in Y \setminus \{F_0(Y)\}, \\
        H_0(n) &= F_0(Y \setminus \{F_k(Y) : k \in [n - 1]_0\})
            \in Y \setminus \{F_k(Y) : k \in [n - 1]_0\}.
    \end{align*} So, we can select distinct elements from $Y$
    recursively via our choice function on the subsets of $Y$.
    We want to show that there's some ordinal $\beta$ such that
    $H_0(\beta) = u$. If we suppose there isn't, then $H_0$
    is injective from the ordinals to $Y$, but we know
    that $\Ran(H_0) \subseteq Y$ is a set by the Axiom of Replacement. 
    Thus, $H_0^{-1}$ is a surjection from $\Ran(H_0)$ to the ordinals, which is a 
    contradiction as the ordinals form a proper class. So,
    we take $\alpha$ to be the least ordinal such that $H_0(\alpha) = u$.
    We let $H = H_0 \upharpoonright \alpha$, $H$ is a bijection
    from $\alpha$ to $Y$, which gives us a well-ordering on $Y$ via the
    well-ordering on $\alpha$.
    \\[\baselineskip]
    ($\Longleftarrow$) For $\mc{G}$ any set of non-empty sets, we take
    $A = \bigcup \mc{G}$. By the well-ordering principle, there's a
    well-ordering $\ang{A, R}$. We can define a choice function as: \begin{align*}
        F(X) = R\text{-least element of } \ang{X, R},
    \end{align*} as required.
\end{proof}

\subsection{Chains}

Any collection $\mc{G}$ of sets is called a chain if for all $X$
and $Y$ in $\mc{G}$, $X \subseteq Y$ or $Y \subseteq X$.

\subsection{Zorn's Lemma (5.2)} \label{5.2}

For a set $\mc{F}$ such that for every chain $\mc{G} \subseteq \mc{F}$,
$\bigcup \mc{G}$ is in $\mc{F}$, we have that $\mc{F}$ contains a
maximal element $Y$ where for all $Z$ in $\mc{F}$: \begin{align*}
    Y \subseteq Z \Longrightarrow Y = Z.
\end{align*} This is equivalent to the Axiom of Choice and thus the 
Well-ordering Principle.

\begin{proof}
    (ZL $\Longrightarrow$ AC) For a collection of non-empty sets $\mc{G}$,
    we want a choice function for $\mc{G}$. We define $\mc{F}$ to be
    the set of all choice functions that exist for subsets of $\mc{G}$,
    that is, for $f$ in $\mc{F}$: \begin{align*}
        \Dom(f) \subseteq \mc{G} \text{ and } 
        \forall \, x \in \Dom(f), f(x) \in x.
    \end{align*} We know that $\mc{F}$ is non-empty as for some
    $x$ in $\mc{G}$, $x$ is non-empty so we choose any $u$ in $x$
    and thus $\{\ang{x, u}\}$ is in $\mc{F}$. For any chain $\mc{H}$ in
    $\mc{F}$, $\mc{H}$ is a chain of partial choice functions on subsets
    of $\mc{G}$. We take $h = \bigcup \mc{H}$, so $h$ is a function
    with $\Dom(h) = \bigcup \{\Dom(f) : f \in \mc{H}\} \subseteq \mc{G}$.
    Thus, $f$ is a choice function so is in $\mc{F}$.
    \\[\baselineskip]
    By Zorn's Lemma, there's a maximal $m$ in $\mc{F}$ and we want
    to show that $m$ is a choice function for $\mc{G}$. We know
    $m$ must be a partial choice function so it's sufficient to
    show that $\Dom(m) = \mc{G}$. We suppose that $\Dom(m) \neq \mc{G}$,
    and take $x$ in $\mc{G} \setminus \Dom(m)$ which must be non-empty
    as it is in $\mc{G}$. For $u$ in $x$, $m \cup \{\ang{u, x}\}$
    is a partial choice function in $\mc{F}$ with domain
    $\Dom(m) \cup \{u\}$ so $m \subset m \cup \{\ang{u, x}\}$.
    This is a contradiction of the maximality of $m$, 
    so $m$ is a choice function for $\mc{G}$.
    \\[\baselineskip]
    (WP $\Longrightarrow$ ZL) We take $\mc{F}$ to be a set such that
    for every chain $\mc{G} \subseteq \mc{F}$ we have that 
    $\bigcup \mc{G}$ is in $\mc{F}$. By the Well-ordering Principle,
    $\mc{F}$ can be well-ordered by some relation $R$, we take
    an ordinal $\alpha$ such that $\ang{\alpha, \in} \cong \ang{\mc{F}, R}$
    for some order isomorphism $k$. By recursion on the ordinals 
    $\beta < \alpha$, we define a maximal chain $\mc{H}$ of $\mc{F}$.
    We start by putting $k(0)$ into $\mc{H}$, if $k(0) \subset k(1)$
    then we add $k(1)$ too, if not, we move on, adding $k(\beta)$
    if it contains the current maximal element of $\mc{H}$. This
    clearly forms a chain, and we will show that $Y = \bigcup \mc{H}$ is
    a maximal element of $\mc{F}$. By the definition of $\mc{F}$, 
    as $\mc{H}$ is a chain, $Y$ is in $\mc{F}$. If we suppose that there
    is some $Z$ in $\mc{F}$ with $Y \subseteq Z$, then $k(\gamma)
    \subseteq Z$ for any $\gamma$ such that $k(\gamma)$ is in $\mc{H}$.
    As $Y$ is in $\mc{F}$, for some $\delta < \alpha$, $Z = k(\delta)$.
    But, by the definition of our recursion, at the stage $\gamma$, we
    decided that $Z$ should be added to $\mc{H}$ so 
    $Z \subseteq \bigcup \mc{H} = Y$ and as such, $Z = Y$
    as required.
\end{proof}

\subsection{Axiom of Foundation (6.4)} \label{6.4}

Every set $x$ is well-founded, so if $x$ is non-empty, there exists
some $y$ in $x$ such that $x \cap y = \emptyset$.
This is equivalent to saying there exists some $\alpha$ such that
$x$ is in $V_\alpha$.

\begin{proof}
    For a set $x$, we set $T = TC(X)$. If $T \subset V$, then for some $\alpha$, 
    $\rho \fran T \subseteq \alpha$ so $T \subseteq V_\alpha$. 
    Thus, we are done for this case as $x \subseteq T \subseteq V_\alpha$ 
    so $x \in V_{\alpha + 1}$.
    \\[\baselineskip]
    If we suppose that $T \setminus V \neq \emptyset$ and take $y$ in
    $T \setminus V$ such that $(T \setminus V) \cap y = \emptyset$
    by the Axiom of Foundation, then for any $z$ in $y$, 
    as $z$ must be in $T$ by the properties of $TC$.
    Also, $z$ must be in $V$ as $(T \setminus V) \cap y = \emptyset$.
    Hence, $y \subseteq V$. But, as in the first case, $\rho \fran y$ is a set
    of ordinals, with some strict upper bound $\beta$. As such, 
    $y \subseteq V_\beta$ which implies $y$ is in $V_{\beta + 1}$
    which is a contradiction of the definition of $y$.
\end{proof}
