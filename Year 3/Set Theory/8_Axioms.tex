\section{Axioms}

\subsection{Axiom of Replacement}

For a function $F$ from $V$ to itself and a set $x$, $F''x$ is a set.

\subsection{Axiom of Choice}

For a set of non-empty sets $\mc{G}$, there is a choice function
$F$ from $\mc{G}$ to $\bigcup \mc{G}$ such that for all $X$ in $\mc{G}$,
$F(X)$ is in $X$.
\\[\baselineskip]
We have that the Axiom of Choice is equivalent to the Well-ordering
Principle.

\begin{proof}
    ($\Longrightarrow$) For an arbitrary set $Y$, it is sufficient to
    show $Y$ has a well-ordering. 
    We take $Y \neq \emptyset$ as otherwise $Y$
    is trivially well-ordered. We take 
    $\mc{G} = \{X \subseteq Y : X \neq \emptyset\}$. By the Axiom
    of Choice, we have a choice function $F_0$ for $\mc{G}$.
    We take $u$ to be any set not in $Y$ ($Y \neq V$) and
    define $F$ from $V$ to $V$: \begin{align*}
        F(t) &= \begin{cases}
            F_0(t) & \text{if } t \in \mc{G} \\
            u      & \text{otherwise}.
        \end{cases}
    \end{align*} By the recursion theorem, we define $H_0$ from
    the ordinals to $Y \cup \{u\}$: \begin{align*}
        H_0(\xi) &= F(Y \setminus \{H_0(\zeta) : \zeta < \xi\}).
    \end{align*} We can see that: \begin{align*}
        H_0(0) &= F_0(Y) \in Y, \\
        H_0(1) &= F_0(Y \setminus \{F_0(Y)\}) \in Y \setminus \{F_0(Y)\}, \\
        H_0(n) &= F_0(Y \setminus \{F_k(Y) : k \in [n - 1]_0\})
            \in Y \setminus \{F_k(Y) : k \in [n - 1]_0\}.
    \end{align*} So, we can select distinct elements from $Y$
    recursively via our choice function on the subsets of $Y$.
    We want to show that there's some ordinal $\beta$ such that
    $H_0(\beta) = u$. If we suppose there isn't, then $H_0$
    is injective from the ordinals to $Y$, but we know
    that $\Ran(H_0) \subseteq Y$ is a set. Thus, $H_0^{-1}$ is a 
    surjection from $\Ran(H_0)$ to the ordinals, which is a 
    contradiction as the ordinals form a proper class. So,
    we take $\alpha$ to be the least ordinal such that $H_0(\alpha) = u$.
    We let $H = H_0 \upharpoonright \alpha$, $H$ is a bijection
    from $\alpha$ to $Y$, which gives us a well-ordering on $Y$ via the
    well-ordering on $\alpha$.
    \\[\baselineskip]
    ($\Longleftarrow$) For $\mc{G}$ any set of non-empty sets, we take
    $A = \bigcup \mc{G}$. By the well-ordering principle, there's a
    well-ordering $\ang{A, R}$. We can define a choice function as: \begin{align*}
        F(X) = R-\text{least element of } \ang{X, R},
    \end{align*} as required.
\end{proof}

\subsection{Chains}

Any collection $\mc{G}$ of sets is called a chain if for all $X$
and $Y$ in $\mc{G}$, $X \subseteq Y$ or $Y \subseteq X$.

\subsection{Zorn's Lemma}

For a set $\mc{F}$ such that for every chain $\mc{G} \subseteq \mc{F}$,
$\bigcup \mc{G}$ is in $\mc{F}$, we have that $\mc{F}$ contains a
maximal element $Y$ where for all $Z$ in $\mc{F}$: \begin{align*}
    Y \subseteq Z \Longrightarrow Y = Z.
\end{align*} We have that this lemma is equivalent to the Axiom
of Choice and the Well-ordering principle.

\begin{proof}
    (ZL $\Longrightarrow$ AC) For a collection of non-empty sets $\mc{G}$,
    we want a choice function for $\mc{G}$. We define $\mc{F}$ to be
    the set of all choice function that exist for subsets of $\mc{G}$,
    that is, for $f$ in $\mc{F}$: \begin{align*}
        \Dom(f) \subseteq \mc{G} \text{ and } 
        \forall \, x \in \Dom(f), f(x) \in x.
    \end{align*} We know that $\mc{F}$ is non-empty as for some
    $x$ in $\mc{G}$, $x$ is non-empty so we choose any $u$ in $x$
    and thus $\{\ang{x, u}\}$ is in $\mc{F}$. For any chain $\mc{H}$ in
    $\mc{F}$, $\mc{H}$ is a chain of partial choice functions on subsets
    of $\mc{G}$. We take $h = \bigcup \mc{H}$, so $h$ is a function
    with $\Dom(h) = \bigcup \{\Dom(f) : f \in \mc{H}\} \subseteq \mc{G}$.
    Thus, $f$ is a choice function so is in $\mc{F}$.
    \\[\baselineskip]
    By Zorn's Lemma, there's a maximal $m$ in $\mc{F}$ and we want
    to show that $m$ is a choice function for $\mc{G}$. We know
    $m$ must be a partial choice function so it's sufficient to
    show that $\Dom(m) = \mc{G}$. We suppose that $\Dom(m) \neq \mc{G}$,
    and take $x$ in $\mc{G} \setminus \Dom(m)$ which must be non-empty
    as it is in $\mc{G}$. For $u$ in $x$, $m \cup \{\ang{u, x}\}$
    is a partial choice function in $\mc{F}$ with domain
    $\Dom(m) \cup \{u\}$ so $m \subset m \cup \{\ang{u, x}\}$.
    This is a contradiction of the maximality of $m$, 
    so $m$ is a choice function for $\mc{G}$.
    \\[\baselineskip]
    (WP $\Longrightarrow$ ZL) We take $\mc{F}$ to be a set such that
    for every chain $\mc{G} \subseteq \mc{F}$ we have that 
    $\bigcup \mc{G}$ is in $\mc{F}$. By the Well-ordering Principle,
    $\mc{F}$ can be well-ordered by some relation $R$, we take
    an ordinal $\alpha$ such that $\ang{\alpha, \in} \cong \ang{\mc{F}, R}$
    for some order isomorphism $k$. By recursion on the ordinals 
    $\beta < \alpha$, we define a maximal chain $\mc{H}$ of $\mc{F}$.
    We start by putting $k(0)$ into $\mc{H}$, if $k(0) \subset k(1)$
    then we add $k(1)$ too, if not, we move on, adding $k(\beta)$
    if it contains the current maximal element of $\mc{H}$. This
    clearly forms a chain, and we will show that $Y = \bigcup \mc{H}$ is
    a maximal element of $\mc{F}$. By the definition of $\mc{F}$, 
    as $\mc{H}$ is a chain, $Y$ is in $\mc{F}$. If we suppose that
    is some $Z$ in $\mc{F}$ with $Y \subseteq Z$, then $k(\gamma)
    \subseteq Z$ for any $\gamma$ such that $k(\gamma)$ is in $\mc{H}$.
    As $Y$ is in $\mc{F}$, for some $\delta < \alpha$, $Z = k(\delta)$.
    But, by the definition of our recursion, at the stage $\gamma$, we
    decided that $Z$ should be added to $\mc{H}$ so 
    $Z \subseteq \bigcup \mc{H} = Y$ and as such, $Z = Y$
    as required.
\end{proof}
