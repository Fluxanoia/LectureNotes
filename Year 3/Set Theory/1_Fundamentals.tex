\section{The Fundamentals}

\subsection{Axiom of Extensionality}

For two sets $a$ and $b$, we have that $a = b$ if and only if for all
$x$ we have that: \begin{align*}
    x \in a \Longleftrightarrow x \in b.
\end{align*} 
For two classes $A$ and $B$, we have that $A = B$ if and only if for all
$x$ we have that: \begin{align*}
    x \in a \Longleftrightarrow x \in b.
\end{align*}

\subsection{Axiom of Pair Sets}

For any sets $x$ and $y$, there is a set $z = \{x, y\}$. This is the
(unordered) pair set of $x$ and $y$.

\subsection{Axiom of the Powerset}

For each set $x$, there exists a set which is the collection of the
subsets of $x$, the powerset $\mathcal{P}(x)$.
\\[\baselineskip]
For some set $x$, we have the powerset defined as follows 
$\mathcal{P}(x) = \{z \, : \, z \subseteq x\}$.

\subsection{Axiom of the Empty Set}

There exists a set with no members, the empty set $\emptyset$.
\\[\baselineskip]
We have the empty set defined as follows 
$\emptyset = \{x \, : \, x \neq x\}$.

\subsection{Axiom of Subsets}

For some set $x$, we have that $\{y \in x \, : \, \Phi(y)\}$ is a set
for some well-defined property of sets $\Phi$.

\newpage

\subsection{Axiom of Unions}

We have the basic union of two sets $x_1$ and $x_2$: \begin{align*}
    x_1 \cup x_2 = \{y \, : \, y \in x_1 \text{ or } y \in x_2\},
\end{align*} but for cases where we want to unify the members of the sets in
a set $X$, we define: \begin{align*}
    \bigcup X = \{y \, : \, \exists \, x \in X, y \in x\}.
\end{align*} This axiom states that for a set $X$, $\bigcup X$ is a set. 

\subsection{Classes}

We have that classes are collection of objects, these could also be sets.
Classes that are not sets are called proper classes.

\subsection{The Set $\omega$}

We have the set of natural numbers, $\mathbb{N} = \{0, 1, 2, \ldots\}$,
and from this, we define $\omega$: \begin{align*}
    \omega = \{0, 1, 2, \ldots\},
\end{align*} where for some $n$ in $\omega$, \begin{align*}
    n = \{0, 1, 2, \ldots, n - 1\},
\end{align*} with $0_\omega$ being the empty set. We can go beyond this
definition, defining: \begin{align*}
    \omega + 1 &= \{0, 1, 2, \ldots, \omega\}, \\
    \omega + 2 &= \{0, 1, 2, \ldots, \omega, \omega + 1\}, \\
    &\ldots \\
    \omega + n &= \{0, 1, 2, \ldots, \omega, \omega + 1, \ldots \omega + n - 1\}.
\end{align*}

\subsection{Russell's Theorem}

We have that $R = \{x \, : \, x \notin x\}$ is not a set.
\begin{proof}
    Suppose we have a set $z$ such that $z = R$, is $z$ in $R$?
    If we suppose $z$ is in $R$, we have that $z$ is not in $z$ by the definition
    of $R$ (as $z = R$) but $z$ is $R$ so $z$ is not in $R$, a contradiction.
    Thus, we have that there is no set $z$ equal to $R$, so $R$ is not a set
    but a proper class.
\end{proof}

\subsection{The Universe of Sets}

We define the universe of sets as $V = \{x \, : \, x = x\}$.
We have that $V$ is a proper class.
\begin{proof}
    If we suppose $V$ is a set, we apply the axiom of subsets with
    $\Phi(x) = x \notin x$ and reach a contradiction via Russell's
    theorem.
\end{proof}

