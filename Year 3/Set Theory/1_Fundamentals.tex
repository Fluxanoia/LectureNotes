\section{The Fundamentals}

\subsection{Axiom of Extensionality}

For two sets $a$ and $b$, we have that $a = b$ if and only if for all
$x$ we have that: \begin{align*}
    x \in a \Longleftrightarrow x \in b.
\end{align*} 
For two classes $A$ and $B$, we have that $A = B$ if and only if for all
$x$ we have that: \begin{align*}
    x \in a \Longleftrightarrow x \in b.
\end{align*}

\subsection{Axiom of Pair Sets}

For any sets $x$ and $y$, there is a set $z = \{x, y\}$. This is the
(unordered) pair set of $x$ and $y$.

\subsection{Axiom of the Powerset}

For each set $x$, there exists a set which is the collection of the
subsets of $x$, the powerset $\mathcal{P}(x)$.
We have the powerset defined as 
$\mathcal{P}(x) = \{z : z \subseteq x\}$.

\subsection{Axiom of the Empty Set}

There exists a set with no members, the empty set $\emptyset$.
We have the empty set defined as $\emptyset = \{x : x \neq x\}$.

\subsection{Axiom of Subsets}

For some set $x$, we have that $\{y \in x : \Phi(y)\}$ is a set
for some well-defined property of sets $\Phi$.

\subsection{Axiom of Infinity}

There exists an inductive set.

\subsection{Axiom of Unions (1.6)} \label{1.6}

We have the basic union of two sets $x_1$ and $x_2$: \begin{align*}
    x_1 \cup x_2 = \{y : y \in x_1 \text{ or } y \in x_2\},
\end{align*} but for when we want to unify the members of the sets in
a set $x$, we define: \begin{align*}
    \bigcup x = \{y : \exists \, z \in x, y \in z\}.
\end{align*} This axiom states that for a set $x$, $\bigcup x$ is a set. 

\subsection{Intersections (1.8)} \label{1.8}

We have the basic intersection of two sets $x_1$ and $x_2$: \begin{align*}
    x_1 \cap x_2 = \{y : y \in x_1 \text{ and } y \in x_2\},
\end{align*} but for when we want to intersect the members of 
the sets in a set $x$, we define: \begin{align*}
    \bigcup x = \{y : \forall \, z \in x, y \in z\}.
\end{align*} This is a set by the Axiom of Subsets.

\subsection{Classes}

We have that classes are collection of objects, these could also be sets.
Classes that are not sets are called proper classes.

\subsection{Russell's Theorem (1.4)} \label{1.4}

We have that $R = \{x : x \notin x\}$ is a proper class.
\begin{proof}
    Suppose we have a set $z$ such that $z = R$, we consider the membership of $z$
    in $R$. If we suppose $z$ is in $R$, by the definition of $R$, $z$ is not in
    $z = R$, a contradiction. If we suppose $z$ is not in $R$, by the definition 
    of $R$, $z$ is in $z = R$, a contradiction. Thus, $z$ cannot be a set,
    so $R$ is a proper class.
\end{proof}

\subsection{The Universe of Sets (1.5)} \label{1.5}

We define the universe of sets as $V = \{x : x = x\}$.
We have that $V$ is a proper class.
\begin{proof}
    If we suppose $V$ is a set, we apply the Axiom of Subsets with
    $\Phi(x) = x \notin x$ and reach a contradiction via (\ref{1.4}).
\end{proof}

