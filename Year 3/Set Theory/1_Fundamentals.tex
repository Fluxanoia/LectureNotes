\section{Fundamentals}

\subsection{The Set $\omega$}

We have the set of natural numbers, $\mathbb{N} = \{0, 1, 2, \ldots\}$,
and from this, we define $\omega$: \begin{align*}
    \omega = \{0, 1, 2, \ldots\},
\end{align*} where for some $n$ in $\omega$, \begin{align*}
    n = \{0, 1, 2, \ldots, n - 1\},
\end{align*} with $0_\omega$ being the empty set. We can go beyond this
definition, defining: \begin{align*}
    \omega + 1 &= \{0, 1, 2, \ldots, \omega\}, \\
    \omega + 2 &= \{0, 1, 2, \ldots, \omega, \omega + 1\}, \\
    &\ldots \\
    \omega + n &= \{0, 1, 2, \ldots, \omega, \omega + 1, \ldots \omega + n - 1\}.
\end{align*}

\subsection{Axiom of Extensionality}

For two sets $a$ and $b$, we have that $a = b$ if and only if for all
$x$ we have that: \begin{align*}
    x \in a \Longleftrightarrow x \in b.
\end{align*} 
For two classes $A$ and $B$, we have that $A = B$ if and only if for all
$x$ we have that: \begin{align*}
    x \in a \Longleftrightarrow x \in b.
\end{align*}

\subsection{Axiom of Pair Sets}

For any sets $x$ and $y$, there is a set $z = \{x, y\}$. This is the
(unordered) pair set of $x$ and $y$.

\subsection{Axiom of the Powerset}

For each set $x$, there exists a set which is the collection of the
subsets of $x$, the powerset $\mathcal{P}(x)$.
\\[\baselineskip]
For some set $x$, we have the powerset defined as follows 
$\mathcal{P}(x) = \{z \text{ such that } z \subseteq x\}$.

\subsection{Axiom of the Empty Set}

There exists a set with no members, the empty set $\emptyset$.
\\[\baselineskip]
We have the empty set defined as follows 
$\emptyset = \{x \text{ such that } x \neq x\}$.
