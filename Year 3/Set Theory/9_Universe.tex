\section{The Universe of Sets}

\subsection{The Well-founded Hierarchy of Sets}

For a limit ordinal $\lambda$, we define the function $V_\alpha$ by 
transfinite recursion: \begin{align*}
    V_0 &= \emptyset, \\
    V_{\alpha + 1} &= \mc{P}(V_\alpha), \\
    V_{\lambda} &= \bigcup_{\alpha < \lambda} \mc{P}(V_\alpha), \\
    V &= \bigcup_{\alpha \in \text{On}} \mc{P}(V_\alpha).
\end{align*}

\subsection{Transitivity of $V_\alpha$}

For any $\alpha$, we have that $V_\alpha$ is transitive and for all
$\beta < \alpha$, $V_\beta$ is in $V_\alpha$.

\begin{proof}
    We proceed by induction, for $\alpha = 0$, $V_0 = \emptyset$
    which trivially satisfies both statements. For $\alpha = \beta + 1$,
    we use the fact that if $\beta$ is transitive, then $\mc{P}(\beta)$
    is also. By the inductive hypothesis, $V_\alpha = \mc{P}(\beta)$
    is transitive. As $V_\beta$ is in $\mc{P}(V_\beta)$, we have that
    $V_\beta$ is in $V_\alpha$ and if $\beta' < \beta$ then
    by the inductive hypothesis, $V_{\beta'}$ is in $V_{\beta}$ and
    hence $V_{\beta'}$ is in $V_\alpha$ by transitivity. 
    For $\alpha$ a limit ordinal,
    $V_\alpha = \bigcup_{\beta < \alpha} V_\beta$ is transitive by
    the inductive hypothesis. For $\beta < \alpha$, it must be that
    $V_\beta$ is in $V_\alpha$ by the definition and transitivity 
    of $V_\alpha$.
\end{proof}

\newpage

\subsection{The Rank Function}

For any $x$ in $V$, $\rho(x)$ is the least 
$\tau$ such that $x \subseteq V_\tau$ (or rather, $x$ is in $V_{\tau + 1}$). 
So, for $y$ in $x$, we have that $\rho(y) < \rho(x)$.
\\[\baselineskip]
We have that: \begin{enumerate}
    \item $\rho(\{x\}) = \rho(x) + 1$,
    \item $\rho(\{x, y\}) = \max(\{\rho(x), \rho(y)\}) + 1$,
    \item $\rho(\ang{x, y}) = \max(\{\rho(x), \rho(y)\}) + 2$.
\end{enumerate}

\begin{proof}
    
\end{proof}

\subsection{Axiom of Foundation}

Every set $x$ is well-founded, so if $x$ is non-empty, there exists
some $y$ in $x$ such that $x \cap y = \emptyset$.
\\[\baselineskip]
This is equivalent to saying there exists some $\alpha$ such that
$x$ is in $V_\alpha$.

\begin{proof}
    For a set $x$, we set $T = TC(X)$. 
    If $T \subseteq V$ then for some ordinal $\alpha$,
    $T \subseteq V_\alpha$ as $\rho''T$ is a set of ordinals by
    the Axiom of Replacement. So, for some $\alpha$, $\rho''T \subseteq \alpha$
    so $TC(x) \subseteq V_\alpha$. So, we are done for this case as
    $x \in T \subseteq V_\alpha$.
    \\[\baselineskip]
    If suppose that $T \setminus V \neq \emptyset$ and take $y$ in
    $T \setminus V$ such that $(T \setminus V) \cap y = \emptyset$
    by the Axiom of Foundation, then for any $z$ in $y$, 
    as $z$ must be in $T$ by the properties of $TC$.
    Also, $z$ must be in $V$ as $(T \setminus V) \cap y = \emptyset$.
    Hence, $y \subseteq V$. But, as in the first case, $\rho''y$ is a set
    of ordinals, with some strict upper bound $\beta$. As such, 
    $y \subseteq V_\beta$ which implies $y$ is in $V_{\beta + 1}$
    which is a contradiction of the definition of $y$.
\end{proof}
