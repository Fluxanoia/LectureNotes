\section{The Universe of Sets}

\subsection{The Well-founded Hierarchy of Sets}

For a limit ordinal $\lambda$, we define the function $V_\alpha$ by 
transfinite recursion: \begin{align*}
    V_0 &= \emptyset, \\
    V_{\alpha + 1} &= \mc{P}(V_\alpha), \\
    V_{\lambda} &= \bigcup_{\alpha < \lambda} \mc{P}(V_\alpha), \\
    V &= \bigcup_{\alpha \in \text{On}} \mc{P}(V_\alpha).
\end{align*}

\subsection{Transitivity of $V_\alpha$}

For any $\alpha$, we have that $V_\alpha$ is transitive and for all
$\beta < \alpha$, $V_\beta$ is in $V_\alpha$.

\begin{proof}
    We proceed by induction, for $\alpha = 0$, $V_0 = \emptyset$
    which trivially satisfies both statements. For $\alpha = \beta + 1$,
    we use the fact that if $\beta$ is transitive, then $\mc{P}(\beta)$
    is also. By the inductive hypothesis, $V_\alpha = \mc{P}(\beta)$
    is transitive. As $V_\beta$ is in $\mc{P}(V_\beta)$, we have that
    $V_\beta$ is in $V_\alpha$ and if $\beta' < \beta$ then
    by the inductive hypothesis, $V_{\beta'}$ is in $V_{\beta}$ and
    hence $V_{\beta'}$ is in $V_\alpha$ by transitivity. 
    For $\alpha$ a limit ordinal,
    $V_\alpha = \bigcup_{\beta < \alpha} V_\beta$ is transitive by
    the inductive hypothesis. For $\beta < \alpha$, it must be that
    $V_\beta$ is in $V_\alpha$ by the definition and transitivity 
    of $V_\alpha$.
\end{proof}

\subsection{The Rank Function}

For any $x$ in $V$, $\rho(x)$ is the least 
$\tau$ such that $x \subseteq V_\tau$. So, for $y$ in $x$, we have that
$\rho(y) < \rho(x)$.
