\section{Commutators}

For $x$, $y$ in a group $G$, we define the commutator
of $x$ and $y$ as: \begin{align*}
    [x, y] = x^{-1}y^{-1}xy.
\end{align*} This can be considered as the 'cost' of commuting
$x$ and $y$: \begin{align*}
    xy = yx[x, y].
\end{align*} Note that
for a homomorphism $\varphi$ with domain $G$, we have that
$\varphi([x, y]) = [\varphi(x), \varphi(y)]$.

\subsection{Commutator Subgroups}

For a group $G$ with $H, K \leq G$, we define a subgroup $[H, K]$ by:
\begin{align*}
    [H, K] = \ang{[h, k] : h \in H, k \in K}.
\end{align*} The subgroup $[G, G]$ is called the commutator subgroup.
Furthermore, if $G$ is abelian, $[G, G] = \{e_G\}$.

\subsection{Commutator Subgroup of Characteristic Subgroups}

For a group $G$ with $H, K \csub G$, $[H, K] \csub G$. Furthermore,
$[G, G] \csub G$.

\begin{proof}
    We take $\varphi$ in $\Aut(G)$: \begin{align*}
        \varphi([H, K]) 
        &= \varphi(\ang{[h, k] : h \in H, k \in K}) \\
        &= \ang{\varphi([h, k]) : h \in H, k \in K} \\
        &= \ang{[\varphi(h), \varphi(k)] : h \in H, k \in K} \\
        &= \ang{[h, k] : h \in H, k \in K} \tag{$H, K \csub G$} \\
        &= [H, K],
    \end{align*} as required.
\end{proof}

\subsection{Abelian Quotients}

For a group $G$ with $H \nsub G$, $G/H$ is abelian if and only if 
$[G, G] \leq H$. Furthermore, this shows that a quotient of $G$
is abelian if and only if it is isomorphic to a quotient of $G/[G, G]$
(by the second isomorphism theorem).

\begin{proof}
    We take $\pi : G \to G/H$ to be the quotient homomorphism.
    \\[\baselineskip]
    ($\Rightarrow$) 
    If $G/H$ is abelian then we take $x$, $y$ arbitrary in $G$. 
    We have that $\pi([x, y]) = [\pi(x), \pi(y)] = e_GH$. 
    Thus, $[x, y]$ is in $H$. 
    Thus, as $x$, $y$ are arbitary, $[G, G] \subseteq H$.
    \\[\baselineskip]
    ($\Leftarrow$)
    If $[G, G] \subseteq H$ then for every $xH$, $yH$ in $G/H$ we have
    that: \begin{align*}
        [xH, yH] &= (x^{-1}H)(y^{-1}H)(xH)(yH) \\
        &= [x, y]H \\
        &= H.
    \end{align*} Thus, $G/H$ is abelian.
\end{proof}

\subsubsection{Quotients of Abelien Groups}

Every quotient of an abelian group is abelian.

\begin{proof}
    If $G$ is abelian then $[G, G] = \{e_G\}$. So, for each $H \csub G$
    we have $[G, G] \subseteq H$ and so $G/H$ is abelian by the above.
\end{proof}

\subsection{The Abelianisation}

For a group $G$, the abelianisation of $G$ is the quotient group 
$G/[G, G]$. This group is always abelian and is the largest possible
abelian quotient of $G$.
\\[\baselineskip]
It can be that $G/[G, G] = \{e_G\}$ ($[G, G] = G$). These groups are
called perfect. An example is non-abelian simple groups as $[G, G] \csub G$.
