\section{Automorphisms}

An automorphism is an isomorphism from a group to itself. The set of
all automorphisms on a group $G$ is denoted by $\Aut(G)$ which
is a group under composition.

\subsection{Inner Automorphisms}

For a group $G$, we have that $\varphi : G \to G$ defined for some $g$ in $G$ 
as  $x \mapsto g^{-1}xg$ is an automorphism. Any automorphism of this
form is called an inner automorphism.

\begin{proof}
    For $x$, $y$ in $G$: \begin{align*}
        \varphi(xy) &= g^{-1}xyg \\
        &= g^{-1}xe_Gyg \\
        &= g^{-1}xgg^{-1}yg \\
        &= \varphi(x)\varphi(y), \\
    \end{align*} so $\varphi$ is a homomorphism. We can see that 
    $g^{-1}xg = e_G$ implies that $x = gg^{-1} = e_G$ so 
    $\Ker(\varphi) = \{e_G\}$. Finally, we see that $x = g^{-1}(gxg^{-1})g$
    so $\varphi$ is surjective as $x$ is arbitrary in $G$. 
    Thus, $\varphi$ is an automorphism.
\end{proof}

\subsection{Conjugation}

The operation performed by inner automorphisms is called conjugation
by an element. For a group $G$ with $x$, $y$, $g$ in $G$ and $X \subseteq G$:
\begin{itemize}
    \item $g^{-1}xg$ is the conjugation of $x$ by $g$,
    \item $g^{-1}xg$ is denoted by $x^g$,
    \item $g^{-1}Xg$ is similarly denoted by $X^g$,
    \item $x$ and $y$ are said to be conjugate if there exists some $g$ in
        $G$ such that $x = y^g$.
\end{itemize}

\subsubsection{Conjugations on Subgroups}

For $G$ a group with $H \leq G$ and $g$ in $G$, $H^g$ is a subgroup of
$G$ and $H^g \cong H$.
\\[\baselineskip]
Two subgroups $H, K \leq G$ are said to be conjugate if there exists some
$g$ in $G$ with $H = K^g$. 