\section{The Fundamentals}

\subsection{Binary Operations}

A binary operation on a set $X$ is a map $X \times X \to X$.
\\[\baselineskip]
Take a binary operation $\ast$ on a set $X$, we say that $\ast$ is 
associative if for all $x, y, z$ in $X$: \begin{align*}
    x \ast (y \ast z) = (x \ast y) \ast z.
\end{align*} Furthermore, we say $e$ in $X$ is an identity element of $\ast$ if
for all $x$ in $X$: \begin{align*}
    e \ast x = x \ast e,
\end{align*} and we say that $y$ in $X$ is the inverse to $x$ if
$x \ast y$ and $y \ast x$ are both identities of $\ast$.

\subsection{Groups}

A group $(G, \ast)$ is a non-empty set $G$ combined with a binary operation
$\ast$ such that: \begin{itemize}
    \item $\ast$ is associative,
    \item $G$ contains an identity for $\ast$,
    \item for each element in $G$, there exists some inverse in $G$ 
        with respect to $\ast$.
\end{itemize}

\subsubsection{Symmetric Groups}

For a set $X$, the set of bijections $X \to X$ is a group under function
composition denoted by $\Sym(X)$. We typically write 
$\Sym(\{1, 2, \ldots, n\})$ as $S_n$.

\subsubsection{Cyclic Groups}

If we consider a regular $n$-gon $P_n$, we take rotations of
$\frac{2\pi}{n}$ radians about the centre to be $r$ and can define: \begin{align*}
    C_n = \{e, r, r^2, \ldots, r^{n - 1}\},
\end{align*} to be the group of rotational symmetries of $P_n$, the cyclic
group on $P_n$.

\subsubsection{Dihedral Groups}

If we consider again, a regular $n$-gon $P_n$ and take: \begin{align*}
    r &= \text{a rotation of } \frac{2\pi}{n} \text{ radians about the centre}, \\
    s &= \text{reflection in some fixed line of symmetry},
\end{align*} then we have that: \begin{align*}
    \Sym(P_n) = \{e, r, r^2, \ldots, r^{n - 1}, s, rs, r^2s, \ldots, r^{n - 1}s\},
\end{align*} called the dihedral group, denoted by $D_{2n}$.

\subsubsection{The Infinite Cyclic/Dihedral Group}

A map $\varphi$ from $\mathbb{Z} \to \mathbb{Z}$ is a symmetry if for some
$n$ and $m$ in $\mathbb{Z}$: \begin{align*}
    |\varphi(m) - \varphi(n)| = |m - n|.
\end{align*} Taking $r$ to be the symmetry $n \mapsto n + 1$, we can define the
infinite cyclic group: \begin{align*}
    C_\infty = \{\ldots, r^{-2}, r^{-1}, e, r, r^2, \ldots\}.
\end{align*} Taking $s$ to be the symmetry $n \mapsto -n$, we can define the
infinite dihedral group: \begin{align*}
    D_\infty = \{\ldots, r^{-2}, r^{-1}, e, r, r^2, \ldots, r^{-2}s, r^{-1}s, s, rs, r^2s\}.
\end{align*}