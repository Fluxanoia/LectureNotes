\section{The Fundamentals}

\subsection{Binary Operations}

A binary operation on a set $X$ is a map $X \times X \to X$.
\\[\baselineskip]
Take a binary operation $\ast$ on a set $X$, we say that $\ast$ is 
associative if for all $x, y, z$ in $X$: \begin{align*}
    x \ast (y \ast z) = (x \ast y) \ast z.
\end{align*} Furthermore, we say $e$ in $X$ is an identity element of $\ast$ if
for all $x$ in $X$: \begin{align*}
    e \ast x = x \ast e,
\end{align*} and we say that $y$ in $X$ is the inverse to $x$ if
$x \ast y$ and $y \ast x$ are both identities of $\ast$.

\subsection{Groups}

A group $(G, \ast)$ is a non-empty set $G$ combined with a binary operation
$\ast$ such that: \begin{itemize}
    \item $\ast$ is associative,
    \item $G$ contains an identity for $\ast$,
    \item for each element in $G$, there exists some inverse in $G$ 
        with respect to $\ast$.
\end{itemize}

\subsubsection{Symmetric Groups}

For a set $X$, the set of bijections $X \to X$ is a group under function
composition denoted by $\Sym(X)$. We typically write 
$\Sym(\{1, 2, \ldots, n\})$ as $S_n$.

\subsubsection{Cyclic Groups}

If we consider a regular $n$-gon $P_n$, we take rotations of
$\frac{2\pi}{n}$ radians about the centre to be $r$ and can define: \begin{align*}
    C_n = \{e, r, r^2, \ldots, r^{n - 1}\},
\end{align*} to be the group of rotational symmetries of $P_n$, the cyclic
group on $P_n$.

\subsubsection{Dihedral Groups}

If we consider again, a regular $n$-gon $P_n$ and take: \begin{align*}
    r &= \text{a rotation of } \frac{2\pi}{n} \text{ radians about the centre}, \\
    s &= \text{reflection in some fixed line of symmetry},
\end{align*} then we have that: \begin{align*}
    \Sym(P_n) = \{e, r, r^2, \ldots, r^{n - 1}, s, rs, r^2s, \ldots, r^{n - 1}s\},
\end{align*} called the dihedral group, denoted by $D_{2n}$.

\subsubsection{The Infinite Cyclic/Dihedral Group}

A map $\varphi$ from $\mathbb{Z} \to \mathbb{Z}$ is a symmetry if for some
$n$ and $m$ in $\mathbb{Z}$: \begin{align*}
    |\varphi(m) - \varphi(n)| = |m - n|.
\end{align*} Taking $r$ to be the symmetry $n \mapsto n + 1$, we can define the
infinite cyclic group: \begin{align*}
    C_\infty = \{\ldots, r^{-2}, r^{-1}, e, r, r^2, \ldots\}.
\end{align*} Taking $s$ to be the symmetry $n \mapsto -n$, we can define the
infinite dihedral group: \begin{align*}
    D_\infty = \{\ldots, r^{-2}, r^{-1}, e, r, r^2, \ldots, r^{-2}s, r^{-1}s, s, rs, r^2s\}.
\end{align*}

\subsection{Order}

For a group $G = (X, \ast)$, $G$ has order $|X|$. The order of an element $x$ of
$X$ is defined as follows: \begin{center}
    \begin{tabular}{ r c l l }
        $|x|$ & $=$ & $\infty$ & if $x^n \neq e_G$ for any $n$ in $\mathbb{N}$, \\
        $|x|$ & $=$ & $\min\{n \in \mathbb{N} \, | \, x^n = e_G\}$ & otherwise. 
    \end{tabular}
\end{center} Taking $x$ in $X$, if $x$ has finite order, then: \begin{enumerate}
    \item $x^n = e_G$ if and only if $|x|$ divides $n$,
    \item $x^n = x^m$ if and only if $|x|$ divides $m - n$,
\end{enumerate} and if $x$ has infinite order: \begin{enumerate}
    \item[3.] $x^n = x^m$ if and only if $n = m$.
\end{enumerate}

\begin{proof}
    For (1), we take $n = q|x| + r$ for some $q$ in $\mathbb{Z}$, 
    $r$ in $\{0, 1, \ldots, |x| - 1\}$. Thus: \begin{align*}
        x^n &= x^{q|x|}x^r, \\
        &= e_G^qx^r, \\
        &= x^r,
    \end{align*} and we can see that $x^r = e_G$ if and only if $r = 0$ as $r < |x|$
    and $|x|$ is minimal. Thus, $x^n = e_G$ if and only if $r = 0$ which occurs
    if and only if $|x|$ divides $n$.
    \\[\baselineskip]
    For (2) and (3), we take $x$ to have any order and consider: \begin{align*}
        x^n = x^m, \\
        x^{m - n} = e_G.
    \end{align*} Thus, if $|x| < \infty$ then $|x|$ divides $m - n$ by (1) and
    if $|x| = \infty$ then $m - n = 0$ by the definition of order.
\end{proof}

\subsubsection{Torsion Groups}

A group is a torsion group if every element has finite order and torsion-free
if every non-identity element has infinite order.

\subsection{$p$-groups}

For $p$ in $\mathbb{P}$, we say that a group $G$ is a $p$-group if the order
of each element of $G$ is a power of $p$.

\subsection{Isomorphisms}

For $(G, \ast)$, $(H, \circ)$ groups, an isomorphism $\varphi : G \to H$ is a
bijection such that $\varphi(x \ast y) = \varphi(x) \circ \varphi(y)$ for all
$x$, $y$ in $G$. If such a map exists, we say $G$ is isomorphic to $H$, denoted
by $G \cong H$.
\\[\baselineskip]
For $G$, $H$, and $K$ groups, $\varphi : G \to H$ and $\psi : H \to K$ isomorphisms,
we have that: \begin{itemize}
    \item $\varphi^{-1}$ is an isomorphism,
    \item $(\psi \circ \varphi)$ is an isomorphism,
\end{itemize} which means $\cong$ is an equivalence relation on any set of groups.

\subsection{Set Multiplication}

For $X$, $Y$ subsets of a group $(G, \ast)$, we define: \begin{align*}
    X \ast Y = \{x \ast y : x \in X, y \in Y\},
\end{align*} the product set of $X$ and $Y$ (which is a subset of $G$).
We have that $\ast$ is an associative binary operation on 
$\mathcal{P}(G)$. Additionally, we define: \begin{align*}
    X^{-1} = \{x^{-1} : x \in X\}.
\end{align*} However, these definitions do not define a group on
$\mathcal{P}(G)$ as an inverse does not necessarily exist for each
element, despite the existence of an identity $\{e_G\}$.
