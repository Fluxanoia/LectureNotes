\section{The Fundamentals}

\subsection{Binary Operations}

A binary operation on a set $X$ is a map $X \times X \to X$.
\\[\baselineskip]
Take a binary operation $\ast$ on a set $X$, we say that $\ast$ is 
associative if for all $x, y, z$ in $X$: \begin{align*}
    x \ast (y \ast z) = (x \ast y) \ast z.
\end{align*} Furthermore, we say $e$ in $X$ is an identity element of $\ast$ if
for all $x$ in $X$: \begin{align*}
    e \ast x = x \ast e,
\end{align*} and we say that $y$ in $X$ is the inverse to $x$ if
$x \ast y$ and $y \ast x$ are both identities of $\ast$.

\subsection{Groups}

A group $(G, \ast)$ is a non-empty set $G$ combined with a binary operation
$\ast$ such that: \begin{itemize}
    \item $\ast$ is associative,
    \item $G$ contains an identity for $\ast$,
    \item for each element in $G$, there exists some inverse in $G$ 
        with respect to $\ast$.
\end{itemize}

\subsubsection{Cyclic Groups} \label{1.13} \label{1.16}

A group $G$ is cyclic if it is generated by a single element. Elements in
$G$ that generate $G$ are called generators. Cyclic groups are
abelian, subgroups of cyclic groups are cyclic.

\subsubsection{Dihedral Groups}

The dihedral group $D_{2n}$ is the set of symmetries of
the regular $n$-gon, with a rotation $r$ by $\frac{2\pi}{n}$
radians and a reflection $s$, $D_{2n} = C_n \cup sC_n$.

\subsubsection{Torsion Groups}

A group is a torsion group if every element has finite order and 
torsion-free if every non-identity element has infinite order.

\subsubsection{$p$-groups}

For $p$ a prime, we say that a group $G$ is a $p$-group if the order
of each element of $G$ is a power of $p$.

\subsection{Set Multiplication} \label{1.5}

For $X$, $Y$ subsets of a group $(G, \ast)$, we define: \begin{align*}
    X \ast Y = \{x \ast y : x \in X, y \in Y\},
\end{align*} the product set of $X$ and $Y$ (which is a subset of $G$).
We have that $\ast$ is an associative binary operation on 
$\mathcal{P}(G)$. Additionally, we define: \begin{align*}
    X^{-1} = \{x^{-1} : x \in X\}.
\end{align*} However, these definitions do not define a group on
$\mathcal{P}(G)$ as an inverse does not necessarily exist for each
element, despite the existence of an identity $\{e\}$.

\subsection{Centre} \label{1.8}

For a group $G$, the centre of $G$ is the set of elements that commute with all
elements of $G$, denoted by $Z(G)$: \begin{align*}
    Z(G) = \{z \in G : gz = zg, \forall \, g \in G\}.
\end{align*} We have that $Z(G)$ is a subgroup of $G$.

\subsection{Properties of Sets}

For a group $(G, \ast)$ with $X \subseteq G$, we have some defined properties:
\begin{itemize}
    \item $X$ is symmetric if for each $x$ in $X$, $x^{-1}$ is also in $X$,
    \item $X$ is closed under $\ast$ if for all $x$, $y$ in $X$, $x \ast y$ is in $X$.
\end{itemize}

\newpage

\subsection{Order} \label{1.3}

For a group $G = (X, \ast)$, $G$ has order $|X|$. The order of an element $x$ of
$X$ is defined as follows: \begin{center}
    \begin{tabular}{ r c l l }
        $|x|$ & $=$ & $\infty$ & if $x^n \neq e$ for any $n$ in $\mathbb{N}$, \\
        $|x|$ & $=$ & $\min\{n \in \mathbb{N} \, | \, x^n = e\}$ & otherwise. 
    \end{tabular}
\end{center} Taking $x$ in $X$: \begin{itemize}
    \item[1.] $x^i = x^j$ if and only if $i \equiv j \bmod |x|$,
\end{itemize} if $x$ has finite order, then: \begin{itemize}
    \item[2.] $x^n = e$ if and only if $|x|$ divides $n$,
    \item[3.] $x^n = x^m$ if and only if $|x|$ divides $m - n$,
\end{itemize} and if $x$ has infinite order, then: \begin{itemize}
    \item[4.] $x^n = x^m$ if and only if $n = m$.
\end{itemize} 

\begin{proof}
    (1) This trivially holds for the identity, we consider
    $x \neq e$. If $x^i = x^j$ for some $i \not\equiv j \bmod |x|$,
    we take $i < j$ without loss of generality and see that:
    \begin{align*}
        x^i = x^j 
        &\Longleftrightarrow e \equiv x^{j - i},
    \end{align*} but this contradicts the minimality of $|x|$.
    \\[\baselineskip]
    (2) For $n$ in $\mb{N}$, we take $n = q|x| + r$ for some $q$ 
    in $\mathbb{Z}$, $r$ in $\{0, 1, \ldots, |x| - 1\}$ by the 
    division algorithm. Thus: \begin{align*}
        x^n &= x^{q|x|}x^r, \\
        &= e^qx^r, \\
        &= x^r,
    \end{align*} and we can see that $x^r = e$ if and only if $r = 0$ as $r < |x|$
    and $|x|$ is minimal. Thus, $x^n = e$ if and only if $r = 0$ which occurs
    if and only if $|x|$ divides $n$.
    \\[\baselineskip]
    ((3) and (4)), We take $x$ to have any order so: \begin{align*}
        x^n = x^m &\Longleftrightarrow x^{m - n} = e.
    \end{align*} Thus, if $|x| < \infty$ then $|x|$ divides $m - n$ by (1) and
    if $|x| = \infty$ then $m - n = 0$ by the definition of order.
\end{proof}

\subsection{Homomorphisms}

For $(G, \ast)$ and $(H, \circ)$ groups, a homomorphism $\varphi : G \to H$ is a
map such that \linebreak 
$\varphi(x \ast y) = \varphi(x) \circ \varphi(y)$ for all
$x$ and $y$ in $G$.

\subsection{Isomorphisms}

An isomorphism from $G$ to $H$ is a bijective homomorphism
from $G$ to $H$. 
If such a map exists, we say $G$ is isomorphic to $H$, denoted
by $G \cong H$.

\subsection{Subgroups}

A subset $X$ of a group $(G, \ast)$ is a subgroup if and only if $(X, \ast)$
is a group, denoted by $X \leq G$ (if $X$ is a proper subset, this
is denoted by $X < G$).

\subsubsection{The Product of Subgroups} \label{1.9}

For $H$ and $K$ subgroups of a group $G$, 
$HK$ is a subgroup of $G$ if and only if $HK = KH$.

\begin{proof}
    ($\Rightarrow$) If $HK \leq G$: \begin{align*}
        HK &= (HK)^{-1} \\
        &= K^{-1}H^{-1} \\
        &= KH.
    \end{align*} 
    ($\Leftarrow$) If $HK = KH$: \begin{align*}
        ee &= e \text{ in } HK, \\
        (HK)(HK) &= H(KH)K = H(HK)K = (HH)(KK) = HK, \\
        (HK)^{-1} &= K^{-1}H^{-1} = KH = HK,
    \end{align*} so $HK$ is a subgroup.
\end{proof}

\subsection{The Intersection of Subgroups} \label{1.11}

For a group $G$ with $\mc{X}$ a set of subgroups of $G$: \begin{align*}
    \bigcap_{X \in \mc{X}} X \leq G.
\end{align*}

\begin{proof}
    We take $A$ to be the intersection of the subgroups in 
    $\mc{X}$, $A$ must be non-empty as each subgroup
    must contain $e$. Taking $x$ and $y$ in $A$, for each $X$ in $\mc{X}$
    we know that $x$ and $y$ are in $X$. As $X$ is a subgroup, $x^{-1}$ and
    thus $x^{-1}y$ are in $X$. As $X$ is arbitrary, $x^{-1}y$ must be in $A$.
    Thus, $A$ is a subgroup of $G$ by the subgroup test.
\end{proof}

\subsection{The Subgroup Test} \label{1.10}

For $X$ a subset of a group $G$, $X$ is a subgroup if and only if 
$X \neq \emptyset$ and $x^{-1}y$ is in $X$ for each $x$, $y$ in $X$.

\begin{proof}
    ($\Rightarrow$) If $X \leq G$, then $e$ is in $X$ so 
    $X \neq \emptyset$. For $x$ and $y$ in $X$, $x^{-1}$ is 
    in $X$, so $x^{-1}y$ is also in $X$ as $X$ is closed.
    \\[\baselineskip]
    ($\Leftarrow$) Supposing the latter and taking $x$ and $y$ 
    in $X$, we have that $x^{-1}x = e$, \linebreak
    $x^{-1}e = x^{-1}$,
    $xy = (x^{-1})^{-1}y$ are all in $X$.
\end{proof}

\subsection{Generated Subgroups} \label{1.12}

For a group $G$ with $X \subseteq G$ non-empty, we define the subgroup generated by $X$ as:
\begin{align*}
    \langle X \rangle = \bigcap_{A \leq G : X \subseteq A} A,
\end{align*} the intersection of all the subgroups containing $X$.
This can also be called the smallest subgroup containing $X$.
Alternatively, we have that: \begin{align*}
    \langle X \rangle = \Gamma(X) = \{x_1 x_2 \cdots x_n : x_i \in X \cup X^{-1}, m \in \mathbb{N} \}.
\end{align*}

\begin{proof}
    We can see that $\Gamma(X) \subseteq \langle X \rangle$ as $\langle X \rangle$
    contains $X$ and is a subgroup so it contains all the finite products
    of elements of $X \cup X^{-1}$.
    \\[\baselineskip]
    If we can show that $\Gamma(X)$ is a subgroup, then that would mean
    $\langle X \rangle \subseteq \Gamma(X)$ as $\Gamma(X)$ contains $X$
    so would have been included in the intersection used to generate 
    $\langle X \rangle$. We know that $\Gamma(X)$ is non-empty as $X$ is
    non-empty. We take $x$ and $y$ in $\Gamma(X)$, and some $n$ and $m$ in $\mb{N}$
    and see that: \begin{align*}
        x &= x_1 x_2 \cdots x_n, \\
        y &= y_1 y_2 \cdots y_m, 
    \end{align*} by the definition of $\Gamma(X)$. For each $i$ in $[n]$, 
    we know that $x_i^{-1}$ is in $\Gamma(X)$ as 
    $X^{-1} \subseteq \Gamma(X)$ so: \begin{align*}
        x^{-1}y &= (x_1 x_2 \cdots x_n)^{-1}y \\
        &= x_n^{-1}x_{n - 1}^{-1} \cdots x_1^{-1} y_1 y_2 \cdots y_m,
    \end{align*} is in $\Gamma(X)$. Thus, $\Gamma(X)$
    is a subgroup as required.
\end{proof}

\subsection{Cosets} \label{1.18}

For a group $G$ with $H \leq G$ and $x$ in $G$, the subset $xH$ is a left
coset of $H$ in $G$ and similarly, $Hx$ is a right coset. 
For $x$ and $y$ in $G$: \begin{itemize}
    \item $xH = yH$ if and only if $x$ is in $yH$,
    \item either $xH = yH$ or $xH \cap yH = \emptyset$,
    \item $|xH| = |H|$.
\end{itemize}

\subsubsection{A Bijection from Left to Right Cosets} \label{1.17}

For a group $G$ with $H \leq G$, the map $xH \mapsto Hx^{-1}$ is a 
bijection from the set of left cosets to the set of right cosets.

\subsubsection{Index}

For a group $G$ with $H \leq G$, the number of distinct left cosets of $H$
in $G$ is called the index of $H$ in $G$, denoted by $[G : H]$.

\subsection{Lagrange's Theorem} \label{1.19}

For a finite group $G$ with $H \leq G$, $|G| = [G : H]|H|$.
Thus, for any subgroup $H \leq G$: \begin{itemize}
    \item $[G : H]$ and $|H|$ divide $|G|$,
    \item for $x$ in $G$, $|x|$ divides $|G|$,
    \item if $|G|$ is prime, $G$ is cyclic,
    \item for a prime $p$ and $P$ and $Q$ subgroups of $G$ with order $p$,
        $P \cap Q = \emptyset$ or $P = Q$.
\end{itemize}

\subsection{Outer Direct Product}

For $G_1, \ldots, G_n$ groups, we set: \begin{align*}
    G_1 \times \cdots \times G_n = \{(a_1, \ldots, a_n) : a_i \in G_i, i \in [n] \},
\end{align*} and define a binary operation on 
$G = G_1 \times \cdots \times G_n$ by: \begin{align*}
    (a_1, \ldots, a_n)(b_1, \ldots, b_n) = (a_1b_1, \ldots, a_nb_n).
\end{align*} $G$ is a group under this operation.

\subsection{Properties of the Outer Direct Product}

For $G_1, \ldots, G_n$ groups, with $G = \prod_{i \in [n]} G_i$: \begin{itemize}
    \item $|G| = \prod_{i \in [n]}|G_i|$,
    \item $Z(G) = \prod_{i \in [n]}Z(G_i)$,
    \item if $G$ is cyclic, for each $i$ in $[n]$, $G_i$ is cyclic,
    \item for all $\sigma$ in $S_n$, $G \cong \prod_{i \in [n]} G_{\sigma(i)}$,
    \item for the integers $1 \leq n_1 < n_1 < \cdots < n_r < n$, 
        \begin{gather*}
            G \cong 
            (G_1 \times \cdots \times G_{n_1}) \times 
            (G_{n_1 + 1} \times \cdots \times G_{n_2}) \times \cdots \times
            (G_{n_r + 1} \times \cdots \times G_n),
        \end{gather*}
    \item for $H_1, \ldots, H_n$ groups with $G_i \cong H_i$ for each $i$ in $[n]$
        $G \cong \prod_{i \in [n]} H_i$.
\end{itemize}
