\section{Direct Products}

We have already seen the outer direct product as: \begin{align*}
    G_1 \times \cdots \times G_n = \{(g_1, \ldots, g_n) : g_i \in G_i\},
\end{align*} for groups $G_1, \ldots, G_n$ which forms a group with
component-wise group operations.
\\[\baselineskip]
For a group $G$ with $H_1, \ldots, H_n \nsub G$. We say $G$ is the
inner direct product of $H_1, \ldots, H_n$ if: \begin{itemize}
    \item $G = H_1 \times \cdots \times H_n$,
    \item $H_i \cap (H_1 \times \cdots \times H_{i - 1} 
        \times H_{i + 1} \times \cdots \times H_n) = \{e_G\}$ 
        for all $i$ in $[n]$.  
\end{itemize} We have that $|G| = \prod_i H_i$.

\subsection{Component Groups}

We let $G = G_1 \times \cdots \times G_n$, for each $i$ in $[n]$, we
set: \begin{align*}
    \widehat{G_i} = \{(e, \ldots, e, g_i, e, \ldots, e) : g_i \in G_i\}.
\end{align*} We have that: \begin{enumerate}
    \item For each $i$ in $[n]$, $\widehat{G_i} \nsub G$,
    \item For each $i$ in $[n]$, $\widehat{G_i} \cong G_i$,
    \item $G$ is the inner direct product of 
        $\widehat{G_1}, \ldots, \widehat{G_n}$.
\end{enumerate}

\begin{proof}
    (1) We can see that: \begin{align*}
        \psi((g_1, \ldots, g_n)) = (g_1, \ldots, g_{i - 1}, e, g_{i + 1}, \ldots g_n),
    \end{align*} is a homomorphism with kernel $\widehat{G_i}$. Thus,
    $\widehat{G_i} \nsub G$.
    \\[\baselineskip]
    (2) We can see that: \begin{align*}
        \varphi_i((e, \ldots, e, g_i, e, \ldots, e)) = g_i,
    \end{align*} is an isomorphism. Thus, $\widehat{G_i} \cong G_i$.
    \\[\baselineskip]
    (3) We have that $G = \widehat{G_1} \cdots \widehat{G_n}$ as:
    \begin{align*}
        (g_1, \ldots, g_n)
        &= (g_1, e, \ldots)(e, g_2, e, \ldots)
        \cdots (e, \ldots, e, g_n).
    \end{align*} Furthermore, $\widehat{G_i} \cap G'_i = \{e\}$ where
    $G' = \widehat{G_1}, \ldots, \widehat{G_{i - 1}}, \widehat{G_{i + 1}}, 
    \ldots G_n$ as the elements of $G'_i$ are of the form
    $(g_1, \ldots, g_{i - 1}, e, g_{i + 1}, \ldots, g_n)$ whereas elements
    of $\widehat{G_i}$ are of the form $(e, \ldots, e, g_i, e, \ldots, e)$.
    Thus, the only element in common is $e_G$.
\end{proof}

\subsection{The Commutator of Normal Subgroups}

For a group $G$ with $H, K \nsub G$, $[H, K] \subseteq H \cap K$.

\begin{proof}
    For $h$ in $H$ and $k$ in $K$, $[h, k] = h^{-1}k^{-1}hk$.
    But: \begin{itemize}
        \item $h^{-1}k^{-1}h$ is in $h^{-1}Kh = K$,
        \item $k^{-1}hk$ is in $k^{-1}Hk = H$,
    \end{itemize} so $[h, k]$ is in $H \cap K$.
\end{proof} 

\noindent
Furthermore, if $G = H_1 \times \cdots \times H_n$
is an inner direct product, then for $i \neq j$ both in $[n]$,
we have that the elements of $H_i$ commute with the elements of
$H_j$.

\begin{proof}
    The definition of the inner direct product means that
    $H_i \cap H_j = \{e\}$. This means that $[H_i, H_j] = \{e\}$
    as required.
\end{proof}

\subsection{Isomorphism between Products}

For a group $G$ the inner direct product of subgroups $H_1, \ldots, H_n$,
$G \cong H_1 \times \cdots \times H_n$.

\begin{proof}
    We define $\varphi : H_1 \times \cdots \times H_n \to G$ by: \begin{align*}
        \varphi((h_1, \ldots, h_n)) = h_1 \cdots h_n,
    \end{align*} which is a homomorphism by the commutativity of
    $H_i$ and $H_j$ (where $i \neq j$). The definition of the
    inner direct product implies that it is surjective. We
    take $(h_1, \ldots, h_n) \in \Ker(\varphi)$: \begin{align*}
        &h_1 \cdots h_n = e \\
        \Longrightarrow \qquad& h_i^{-1} = h_1 \cdots h_{i - 1}h_{i + 1} \cdots h_n \\
        \Longrightarrow \qquad& h_i^{-1} \in H_i \cap (H_1 \cdots H_{i - 1} H_{i + 1} \cdots H_n) \\
        \Longrightarrow \qquad& h_i^{-1} = e.
    \end{align*} Thus, as $i$ was chosen arbitrarily, $(h_1, \ldots, h_n) = (e, \ldots, e)$.
    Thus, $\varphi$ is an isomorphism.
\end{proof}

\subsection{Criteria for Inner Direct Products}

\subsubsection{By Unique Compositions}

For a group $G$ with $H_1, \ldots, H_n$ normal subgroups of $G$,
$G$ is an inner direct product of $H_1, \ldots, H_n$ if and
only if for all $g$ in $G$, there exists a unique $h_i$ in each 
$H_i$ such that $g = \prod_ih_i$.

\begin{proof}
    ($\Rightarrow$)
    By the definition, we have $g = \prod_ih_i$ for some $h_i$
    in each $H_i$ so it suffices to show this product is unique. 
    We suppose that: \begin{align*}
        \prod_i k_i = g = \prod_i h_i,
    \end{align*} for some $k_i$, $h_i$ in each $H_i$.
    We fix $i$ and see that: \begin{align*}
        e &= g^{-1}g \\
        &= h_n^{-1} \cdots h_1^{-1} k_1 \cdots k_n \\
        &= h_1^{-1}k_1 \cdots h_n^{-1}k_n \\
        &= h_i^{-1}k_i h_1^{-1}k_1 \cdots 
            h_{i - 1}^{-1}k_{i - 1} h_{i + 1}^{-1}k_{i + 1} \cdots
            h_n^{-1}k_n \\
        k_i^{-1}h_i &= h_1^{-1}k_1 \cdots 
            h_{i - 1}^{-1}k_{i - 1} h_{i + 1}^{-1}k_{i + 1} \cdots
            h_n^{-1}k_n \\
        &\in H_i \cap (H_1 \cdots H_{i - 1}H_{i + 1} \cdots H_n) \\
        &= \{e\}.
    \end{align*} as $G$ is the direct product of $H_1, \ldots, H_n$
    which means elements from differing subgroups commute. Thus,
    for each $i$, $h_i = k_i$.
    \\[\baselineskip]
    ($\Leftarrow$) Clearly $G = H_1 \cdots H_n$ so it suffices to
    show that: \begin{align*}
        H_i \cap (H_1 \cdots H_{i - 1}H_{i + 1} \cdots H_n) = \{e\}
    \end{align*} for each $i$. We take $x$ in this intersection: \begin{align*}
        x = h_i 
        &= h_1 \cdots h_{i - 1} h_{i + 1} \cdots h_n \\
        e \cdots e h_i e \cdots e 
        &= h_1 \cdots h_{i - 1} e h_{i + 1} \cdots h_n,
    \end{align*} which, by the uniqueness of the composition 
    of $x$, means that $x = e$ as required.
\end{proof} 

\newpage

\subsubsection{By the Size}

For $G$ a finite group with $H_1, \ldots, H_n \nsub G$ such that
$G = H_1 \cdots H_n$. $G$ is an inner direct product if and only if
$|G| = \prod_i |H_i|$.

\begin{proof}
    ($\Rightarrow$) As $G$ is an inner direct product we have the result.
    \\[\baselineskip]
    ($\Leftarrow$) As $|G| = \prod_i |H_i|$, each $h_1 \cdots h_n$
    product of elements in $H_1 \cdots H_n$ are distinct. By the above,
    this means $G$ is an inner direct product.
\end{proof}
