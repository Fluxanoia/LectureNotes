\section{Finite Simple Groups}

\subsection{Classification of Abelian Simple Groups} \label{7.2}

For an abelian group $G$, $G$ is simple if and only if 
$G \cong \mb{Z}_p$ for some prime $p$.

\begin{proof}
    ($\Rightarrow$) Supposing the antecedent, for some
    non-identity element $x$ in $G$, $\ang{x} \nsub G$
    so $\ang{x} = G$ as $G$ is simple. As such, $G$ is cyclic.
    If $G$ is infinite, $\ang{x^2}$ is a 
    non-trivial proper normal subgroup of
    $G$, a contradiction of the simplicity of $G$. If $|G|$
    is not prime, $|G| = mn$ for some $m$ and $n$ in $\mb{N}_{>1}$.
    Then $\ang{x^m}$ is, again, a non-trivial proper normal
    subgroup of $G$. As such, $G$ is a finite cyclic group
    of prime order, so $G \cong \mb{Z}_p$ for some prime $p$.
    \\[\baselineskip]
    ($\Leftarrow$) By Lagrange's theorem, $\mb{Z}_p$ has no
    non-trivial proper subgroups.
\end{proof}

\subsection{Bound on the Order of Centres of Finite $p$-groups} \label{7.3}

For a prime $p$ and $G$ a non-trivial finite $p$-group,
$|Z(G)| \geq p$.

\begin{proof}
    By (\ref{6.3}), $|G| = p^m$ for some $m$ in $\mb{Z}$.
    For some $g$ in $G$, if the conjugacy class of $g$ contains
    more than one element, then $C_G(g) \neq g$ so 
    $[G : C_G(g)] > 1$. By Lagrange's theorem, 
    $[G : C_G(g)]$ must be a multiple of $p$.
    Since $|G|$ is also a multiple of $p$, 
    $|Z(G)|$ must be too. As $Z(G)$ contains the
    identity, $|Z(G)| \geq p$.
\end{proof}

\subsection{Existence of Non-abelian Finite Simple $p$-groups} \label{7.4}

There are no non-abelian finite simple $p$-groups.

\begin{proof}
    The centre of a finite simple $p$-group $G$ has
    size at least $p$, so for $G$ to be simple, $Z(G) = G$
    so $G$ is abelian.
\end{proof}

\subsection{Classification of Simple $p$-groups} \label{7.5}

For a prime $p$ and a finite simple $p$-group, $G$ is simple
if and only if $G \cong \mb{Z}_p$.

\begin{proof}
    By (\ref{7.4}), $G$ is abelian. We apply (\ref{7.2}) and
    we are done.
\end{proof}

\subsection{Bound on the Quantity of Sylow $p$-subgroups 
in Non-abelian Finite Simple Groups} \label{7.6}

For a non-abelian finite simple group $G$ and a prime
$p$ dividing $|G|$, $n_p(G) > 1$.

\begin{proof}
    Sylow's first theorem implies that $G$ has at least one
    non-trivial \Syls $P$. By (\ref{7.5}), there are no
    non-abelian finite simple $p$-groups so $P$ is a
    non-trivial proper subgroup of $G$.
    As $G$ is simple, $P \not\nsub G$ so there exists
    some conjugation of $P$ not equal to $P$ which would
    also be a Sylow $p$-subgroup.
    Thus, $n_p(G) > 1$. 
\end{proof}

\subsection{Simple Groups of Order 56} \label{7.7}

There are no simple groups of order 56.

\begin{proof}
    We appeal to the contrary and take $G$ to be a simple
    group of order $56 = 7 \cdot 2^3$. We know that $G$
    is not abelian by (\ref{7.2}). We know that $n_7(G) > 1$
    by (\ref{7.6}) and by Sylow's third theorem, 
    $n_7(G) \equiv 1 \bmod 7$ and $n_7(G)$ divides 8.
    Thus, $n_7(G)$ must be 8.
    \\[\baselineskip]
    By Cauchy's theorem, every Sylow 7-subgroup has size 7,
    so must be isomorphic to $C_7$. As these subgroups
    are distinct, their intersection must be $\{e\}$.
    This gives us $48 = 7 \cdot 6$ distinct elements of
    order 7 in $G$. This leaves 8 elements not of order $7$,
    which must form a Sylow 2-subgroup of order $8$ by Sylow's
    first theorem. This accounts for all 56 elements of $G$,
    there can be no other Sylow 2-subgroups, contradicting
    (\ref{7.6}).
\end{proof}
