\section{Finite Simple Groups}

\subsection{Classification of Abelian Simple Groups} \label{7.2}

For an abelian group $G$, $G$ is simple if and only if 
$G \cong \mb{Z}_p$ for some prime $p$.

\begin{proof}
    ($\Rightarrow$) Supposing the antecedent, for some
    non-identity element $x$ in $G$, $\ang{x} \nsub G$
    so $\ang{x} = G$ as $G$ is simple. As such, $G$ is cyclic.
    If $G$ is infinite, $\ang{x^2}$ is a 
    non-trivial proper normal subgroup of
    $G$, a contradiction of the simplicity of $G$. If $|G|$
    is not prime, $|G| = mn$ for some $m$ and $n$ in $\mb{N}_{>1}$.
    Then $\ang{x^m}$ is, again, a non-trivial proper normal
    subgroup of $G$. As such, $G$ is a finite cyclic group
    of prime order, so $G \cong \mb{Z}_p$ for some prime $p$.
    \\[\baselineskip]
    ($\Leftarrow$) By Lagrange's theorem, $\mb{Z}_p$ has no
    non-trivial proper subgroups.
\end{proof}

\subsection{Bound on the Order of Centres of Finite $p$-groups} \label{7.3}

For a prime $p$ and $G$ a non-trivial finite $p$-group,
$|Z(G)| \geq p$.

\begin{proof}
    By (\ref{6.3}), $|G| = p^m$ for some $m$ in $\mb{Z}$.
    For some $g$ in $G$, if the conjugacy class of $g$ contains
    more than one element, then $C_G(g) \neq g$ so 
    $[G : C_G(g)] > 1$. By Lagrange's theorem, 
    $[G : C_G(g)]$ must be a multiple of $p$.
    Since $|G|$ is also a multiple of $p$, 
    $|Z(G)|$ must be too. As $Z(G)$ contains the
    identity, $|Z(G)| \geq p$.
\end{proof}

\subsection{Existence of Non-abelian Finite Simple $p$-groups} \label{7.4}

There are no non-abelian finite simple $p$-groups.

\begin{proof}
    The centre of a finite simple $p$-group $G$ has
    size at least $p$, so for $G$ to be simple, $Z(G) = G$
    so $G$ is abelian.
\end{proof}

\subsection{Classification of Simple $p$-groups} \label{7.5}

For a prime $p$ and a finite simple $p$-group, $G$ is simple
if and only if $G \cong \mb{Z}_p$.

\begin{proof}
    By (\ref{7.4}), $G$ is abelian. We apply (\ref{7.2}) and
    we are done.
\end{proof}

\subsection{Bound on the Quantity of Sylow $p$-subgroups 
in Non-abelian Finite Simple Groups} \label{7.6}

For a non-abelian finite simple group $G$ and a prime
$p$ dividing $|G|$, $n_p(G) > 1$.

\begin{proof}
    Sylow's first theorem implies that $G$ has at least one
    non-trivial \Syls $P$. By (\ref{7.5}), there are no
    non-abelian finite simple $p$-groups so $P$ is a
    non-trivial proper subgroup of $G$.
    As $G$ is simple, $P \not\nsub G$ so there exists
    some conjugation of $P$ not equal to $P$ which would
    also be a Sylow $p$-subgroup.
    Thus, $n_p(G) > 1$. 
\end{proof}

\subsection{Simple Groups of Order 56} \label{7.7}

There are no simple groups of order 56.

\begin{proof}
    We appeal to the contrary and take $G$ to be a simple
    group of order $56 = 7 \cdot 2^3$. We know that $G$
    is not abelian by (\ref{7.2}). We know that $n_7(G) > 1$
    by (\ref{7.6}) and by Sylow's third theorem, 
    $n_7(G) \equiv 1 \bmod 7$ and $n_7(G)$ divides 8.
    Thus, $n_7(G)$ must be 8.
    \\[\baselineskip]
    By Cauchy's theorem, every Sylow 7-subgroup has size 7,
    so must be isomorphic to $C_7$. As these subgroups
    are distinct, their intersection must be $\{e\}$.
    This gives us $48 = 7 \cdot 6$ distinct elements of
    order 7 in $G$. This leaves 8 elements not of order $7$,
    which must form a Sylow 2-subgroup of order $8$ by Sylow's
    first theorem. This accounts for all 56 elements of $G$,
    there can be no other Sylow 2-subgroups, contradicting
    (\ref{7.6}).
\end{proof}

\subsection{Simple Groups of Order consisting of 2 or 3 Factors} \label{7.8}

For $p$, $q$, $r$ primes, there are no finite simple groups
of order $pq$ or $pqr$.

\begin{proof}
    We suppose that $G$ is a finite simple group, we note that
    $|G|$ is $pq$ or $pqr$, $G$ cannot be abelian by (\ref{7.2})
    \\[\baselineskip]
    \textbf{Case 1} We suppose that $|G| = pq$. By (\ref{7.5}),
    $p \neq q$. Sylow's third theorem implies that $n_p(G)$ 
    divides $q$ and $n_q(G)$ divides $p$ but this means: \begin{align*}
        n_p(G) &\in \{1, q\}, \\
        n_q(G) &\in \{1, p\}.
    \end{align*} But, by $(\ref{7.6})$, $n_p(G)$ and $n_q(G)$ must
    be greater than 1, so $n_p(G) = q$ and $n_q(G) = p$.
    Again, by Sylow's third theorem, we have that: \begin{align*}
        p \equiv 1 \bmod q, \\ 
        q \equiv 1 \bmod p.
    \end{align*} But, if we suppose that $q < p$, then $q \equiv q
    \bmod p$ and similarly for $q > p$. This is a contradiction.
    \\[\baselineskip]
    \textbf{Case 2} We suppose that $|G| = pqr$. By (\ref{7.5}),
    we have that either $pqr = p^2q$ with $p$ and $q$ distinct 
    (\textbf{2a})
    or $p$, $q$, and $r$ are all distinct (\textbf{2b}).
    \\[\baselineskip]
    \textbf{Case 2a} Sylow's third theorem implies that: \begin{align*}
        n_p(G) &\in \{1, q\}, \\
        n_q(G) &\in \{1, p, p^2\},
    \end{align*} and with (\ref{7.6}), they both must be greater than
    1. If $n_q(G) = p$, we have a contradiction by the reasoning
    in \textbf{Case 1}. Otherwise, $n_q(G) = p^2$. So, we have 
    $p^2$ distinct subgroups of order $q$ which admit $q - 1$
    unique elements of order $q$. Thus, there are 
    $p^2(q - 1) = p^2q - p^2 = |G| - p^2$ elements of order $q$ in $G$.
    This leaves $p^2$ elements not of order $q$, which must form
    a unique Sylow $p$-subgroup. But, we know that $n_p(G) > 1$,
    so this is a contradiction.
    \\[\baselineskip]
    \textbf{Case 2b} We suppose that $p < q < r$ without loss of
    generality. Sylow's third theorem implies that
    $n_r(G)$ divides $pq$ and is congruent to $1 \bmod r$
    combined with (\ref{7.6}) again, $n_r(G)$ is in
    $\{p, q, pq\}$. But, as $r > q > p$, $n_r(G)$ must be
    equal to $pq$ as otherwise: \begin{align*}
        n_r(G) = p \not\equiv 1 \bmod r, \\ 
        n_r(G) = q \not\equiv 1 \bmod r.
    \end{align*} By a similar argument, $n_q(G)$ is in $\{r, pr\}$
    and $n_p(G)$ is in $\{q, r, qr\}$. Thus, in $G$,
    there are: \begin{align*}
        pq(r - 1) &\text{ elements of order } r, \\
        \text{at least } r(q - 1) &\text{ elements of order } q, \\
        \text{at least } q(p - 1) &\text{ elements of order } p.
    \end{align*} This accounts for: \begin{align*}
        pq(r - 1) + r(q - 1) + q(p - 1)
        &= pqr - pq + rq - r + qp - q \\
        &= pqr + (rq - r - q),
    \end{align*} elements in $G$, but this is greater than $pqr = |G|$,
    a contradiction.
\end{proof}

\subsection{Simplicity of the First Alternating Groups} \label{7.10}

We have that $A_1 = A_2 = \{e\}$, $A_3 \cong C_3$ is simple, and
$A_4$ is not simple.

\begin{proof}
    We can see that $S_1 = \{e\}$ and $S_2 = \{e, (1, 2)\}$,
    so $A_1 = A_2 = \{e\}$. Also, $A_3 = \{e, (1, 2, 3), (1, 3, 2)\}
    \cong C_3$, and $A_4$ has a normal subgroup: \begin{align*}
        \{e, (1, 2)(3, 4), (1, 3)(2, 4), (1, 4)(2, 3)\}.
    \end{align*}
\end{proof}

\subsection{Conjugacy of 3-cycles in Alternating Groups} \label{7.11}

For $n \geq 5$, all the 3-cycles in $A_n$ are conjugate.

\begin{proof}
    By (\ref{4.4}), we know that for $i_1, \ldots, i_k$
    distinct in $[n]$ and $g$ in $S_n$: \begin{align*}
        g(i_1, \ldots, i_k)g^{-1} 
        = (g(i_1), \ldots, g(i_k)).
    \end{align*} For $i$, $j$, $k$ arbitrary in
    $[n]$, we have that if there's some $g$ in
    $A_n$ such that: \begin{align*}
        g(1) = i, 
        \qquad g(2) = j, 
        \qquad g(3) = k, \tag{$\ast$}
    \end{align*} then $g(1, 2, 3)g^{-1} = (i, j, k)$.
    Thus, it's sufficient to show such $g$ exists in $A_n$.
    We take $g_0$ to be the element of $S_n$ with property
    ($\ast$). We suppose that $g_0$ is not in $A_n$,
    so is a composition of an odd number of transpositions.
    As such, $(4, 5)g_0$ is in $A_n$ and adheres to the
    property ($\ast$) as required.
\end{proof}

\subsection{Simple Alternating Groups} \label{7.9}

The alternating group $A_n$ is simple for $n = 3$ and $n \geq 5$.

\begin{proof}
    For $n < 5$, we have (\ref{7.10}). We suppose $n \geq 5$
    and take $N \nsub A_n$ with $N \neq \{e\}$ and $a \neq e$ in $N$.
    We note that it is sufficient to show that $N$ contains a
    3-cycle as by (\ref{7.11}) and the properties of normal subgroups,
    $N$ would then equal $A_n$, showing there are no proper normal 
    subgroups. We write $a$ as a product of disjoint cycles $a_1, \ldots, a_t$,
    assuming each $a_i$ is not a 1-cycle: \begin{align*}
        a = a_1 \cdots a_t. \tag{$\ast$}
    \end{align*} For all $b$ in $A_n$, as $a^{-1}$ is also in
    $N$ and $N$ is normal, we see that $aba^{-1}b^{-1}$ is in $N$.
    \\[\baselineskip]
    \textbf{Case 1} We suppose $(\ast)$ contains an $r$-cycle
    with $r \geq 4$, without loss of generality, we set
    $a_1 = (i_1, \ldots, i_r)$ and then take $b = (i_1, i_2, i_3)$
    in $A_n$. We know that: \begin{align*}
        aba^{-1}b^{-1} 
        &= (a(i_1), a(i_2), a(i_3))(i_3, i_2, i_1) \\
        &= (i_2, i_3, i_4)(i_3, i_2, i_1) \\
        &= (i_2, i_4, i_2),
    \end{align*} is a 3-cycle.
    \\[\baselineskip]
    \textbf{Case 2}
    We suppose $(\ast)$ contains at least two 3-cycles,
    $(i_1, i_2, i_3)$ and $(i_4, i_5, i_6)$, we take
    $b = (i_1, i_2, i_4)$ in $A_n$. We know that: \begin{align*}
        aba^{-1}b^{-1}
        &= (a(i_1), a(i_2), a(i_4))(i_4, i_2, i_1) \\
        &= (i_2, i_3, i_5)(i_4, i_2, i_1) \\
        &= (i_1, i_4, i_3, i_5, i_2),
    \end{align*} is a 5-cycle. This induces a 3-cycle in $N$ by
    \textbf{Case 1}.
    \\[\baselineskip]
    \textbf{Case 3}
    We suppose $(\ast)$ contains exactly one 3-cycle $(i_1, i_2, i_3)$
    and at least one transposition $(i_4, i_5)$. We take 
    $b = (i_1, i_2, i_4)$ in $A_n$. We know that: \begin{align*}
        aba^{-1}b^{-1}
        &= (a(i_1), a(i_2), a(i_4))(i_4, i_2, i_1) \\
        &= (i_1, i_4, i_3, i_5, i_2),
    \end{align*} inducing a 3-cycle in $N$ by \textbf{Case 2}.
    \\[\baselineskip]
    \textbf{Case 4}
    We suppose that $(\ast)$ contains only transpositions.
    As such, $t$ must be even as at least 2, we take $(i_1, i_2)$
    and $(i_3, i_4)$ to be two of these transpositions. As $n \geq 5$,
    there's some $i_5$ in $[n] \setminus \{i_1, \ldots, i_4\}$. We
    take $b = (i_1, i_3, i_5)$. We know that: \begin{align*}
        aba^{-1}b^{-1}
        &= (a(i_1), a(i_3), a(i_5))(i_5, i_3, i_1) \\
        &= (i_2, i_4, a(i_5))(i_5, i_3, i_1) \\
        &= \begin{cases}
            (i_1, i_2, i_4, i_5, i_3) & a(i_5) = i_5 \\
            (i_1, i_2, i_6)(i_5, i_3, i_1) & \text{otherwise}.
        \end{cases}
    \end{align*} In the former case, we use \textbf{Case 2}.
    In the latter case, $i_6$ is in $[n] \setminus 
    \{i_1, \ldots, i_5\}$ (as $a$ is formed by disjoint cycles)
    so we have two disjoint 3-cycles, which we use \textbf{Case 2}
    on.
\end{proof}

\subsection{Faithful Non-trivial Actions on Simple Groups} \label{7.13}

For a simple group $G$ and a non-empty set $X$ with $\varphi$
from $G$ to $\Sym(X)$ a non-trivial action, $\varphi$ is
faithful and $G$ is isomorphic to a subgroup of $\Sym(X)$.

\begin{proof}
    We have that $\ker(\varphi) \neq G$ as the action is non-trivial,
    and as $\ker(\varphi) \nsub G$ and $G$ is simple, $\ker(\varphi)
    = \{e\}$. Thus, $\varphi$ is faithful. The homomorphism theorem
    implies that $\varphi(G)$ is isomorphic to some subgroup of 
    $\Sym(X)$.
\end{proof}

\subsection{Alternating Subgroups of Index $n$} \label{7.12}

For $n \geq 5$, if $H \leq A_n$ has index $n$, then $H \cong A_{n - 1}$.

\begin{proof}
    We take $\varphi$ from $A_n$ to $\Sym(A_n / H)$ to be the left
    multiplication action. This action is transitive, so non-trivial
    and we know that $A_n$ is simple by (\ref{7.9}) so it must be
    a faithful action by $(\ref{7.13})$. We take $\psi$ from
    $H$ to $\Sym(A_n / H)$ to be the restriction of $\varphi$, noting
    that $H$ is a fixed point for $\psi$. We define an action
    on $X = (A_n / H) \setminus \{H\}$ as $\psi'$ from $H$ to
    $\Sym(X)$ as the restriction of $\psi$.
    \\[\baselineskip]
    We want to show that $\psi'$ is faithful, we take $h$ in $H$
    with $h \neq e$ so $\psi(h)(xH) \neq xH$ for some $x$ in
    $A_n$ (as $\psi$ is faithful). But, since $\psi(h)(H) = H$
    for all $h$ in $H$, $xH \neq H$. As such, $\psi'(h)(xH) \neq xH$
    so $\psi'$ is faithful. But, $\psi'$ acts on $X$ of size $n - 1$,
    so with the homomorphism theorem, we have that 
    $H \cong \psi'(H)  \leq \Sym(X) \cong S_{n - 1}$. By Lagrange's 
    theorem: \begin{align*}
        |A_n| = |H| \cdot [A_n : H]
        & \Longleftrightarrow
        \frac{n!}{2} = n|H| \\
        & \Longleftrightarrow
        |H| = \frac{(n - 1)!}{2},
    \end{align*} and the only subgroup of $S_{n - 1}$ of index 2 is
    $A_{n - 1}$ by ($\ref{4.8}$).
\end{proof}

\subsection{Simple Groups of Order 60} \label{7.14}

All simple groups of order 60 are isomorphic to $A_5$.

\begin{proof}
    We take $G$ to be a simple group of order 60. We know that 
    $G$ is not abelian as 60 is not prime (by (\ref{7.2})). 
    We then use (\ref{7.6}) to show that $n_p(G) > 1$ for all primes 
    dividing 60. Sylow's third theorem implies that 
    $n_5(G) \equiv 1 \bmod 5$ and divides $12$ so $n_5(G) = 6$. 
    Sylow's second theorem implies that $G$ acts transitively by 
    conjugation on $\Syl_5(G)$ and by (\ref{7.13}), this action is faithful and $G$ is isomorphic
    to a subgroup of $\Sym(\Syl_5(G))$. As $n_5(G) = 6$,
    $\Sym(\Syl_5(G)) \cong S_6$ so $G$ is isomorphic to some
    subgroup $G' \leq S_6$.
    \\[\baselineskip]
    We want to show that $G' \cap A_6 = G'$, we let $\pi$ from
    $S_6$ to $S_6 / A_6$ be the quotient homomorphism. The first
    isomorphism theorem implies that $\pi(G') \cong G'/(A_6 \cap G')$
    but as $G'$ is simple, $G' / (A_6 \cap G')$ must have order 1 or
    60. However, $S_6 / A_6$ has order 2, so $|G' / A_6 \cap G'| = 1$.
    As such, $A_6 \cap G' = G'$. In particular, $G' \leq A_6$.
    As $|A_6| = 360$ and $|G'| = 60$, it must be that 
    $G' \cong A_5$ by (\ref{7.12}).
\end{proof}

\subsection{The Smallest Non-abelian Finite Simple Group}

The smallest non-abelian finite simple group has order 60.

\begin{proof}
    We take the smallest non-abelian finite simple group to be $G$.
    By (\ref{7.14}), $|G| \leq 60$ as $A_5$ is a non-abelian finite
    simple group.
    By (\ref{7.5}), $|G|$ can't be a prime power, by (\ref{7.7}),
    $|G| \neq 56$, and by (\ref{7.8}), $|G|$ can't be the product
    of two or three primes. Thus: \begin{align*}
        |G| \in \{24, 36, 40, 48, 54, 60\}.
    \end{align*} We reason on a case-by-case basis: \begin{itemize}
        \item By Sylow's third theorem and (\ref{7.6}),
            if $|G| = 24$ or $48$ then $n_2(G) = 3$. However,
            Sylow's second theorem and (\ref{7.13}) implies that
            $G$ is isomorphic to a subgroup of $S_3$ which is
            impossible as $|S_3| = 6$,
        \item By Sylow's third theorem and (\ref{7.6}),
            if $|G| = 36$ then $n_3(G) = 4$. However,
            Sylow's second theorem and (\ref{7.13}) implies that
            $G$ is isomorphic to a subgroup of $S_4$ which is
            impossible as $|S_4| = 24$,
        \item By Sylow's third theorem, if $|G| = 40$ then 
            $n_5(G) = 1$, contradicting (\ref{7.13}),
        \item By Sylow's third theorem, if $|G| = 54$ then 
            $n_3(G) = 1$, contradicting (\ref{7.13}).
    \end{itemize} Thus, $|G| = 60$.
\end{proof}
