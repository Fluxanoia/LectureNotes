\section{The Homomorphism Theorem}

For $G$, $H$ groups with $\varphi : G \to H$ a homomorphism, we let 
$\pi : G \to G / \Ker(\varphi)$ be the quotient homomorphism.
There exists an isomorphism $\psi : G / \Ker(\varphi) \to \Ima(\varphi)$
such that $\varphi = \psi \circ \pi$.
\\[\baselineskip]
If $\varphi$ is injective, this shows that $G \cong \Ima(\varphi)$.

\begin{proof}
    We set $I = \Ima(\varphi)$ and $K = \Ker(\varphi)$, and define 
    $\psi : G / K \to I$ by $gK \mapsto \varphi(g)$. We then consider:
    \begin{align*}
        (gK = hK) &\Longleftrightarrow (g^{-1}h \in K) \\
        &\Longleftrightarrow (\varphi(g^{-1}h) = e_H) \\
        &\Longleftrightarrow (\varphi(g)^{-1}\varphi(h) = e_H) \\
        &\Longleftrightarrow (\varphi(g) = \varphi(h)).
    \end{align*} So, the map is well-defined and injective.
    Furthermore, $\psi(\pi(g)) = \psi(gK) = \varphi(g)$.
    Consider: \begin{align*}
        \psi(ghK) &= \varphi(gh) \\
        &= \varphi(g)\varphi(h) \\
        &= \psi(gK)\psi(hK),
    \end{align*} so $\psi$ is a homomorphism and is 
    trivially surjective as required.
\end{proof}

\section{The First Isomorphism Theorem}

For a group $G$ with $N \nsub G$, $\pi : G \to G/N$ the
quotient homomorphism, and $H \leq G$: \begin{enumerate}
    \item $H \cap N \nsub H$,
    \item $\pi(H) \cong H/(H \cap N)$.
\end{enumerate}

\begin{proof}
    We write $\pi|_H$ for the restriction of $\pi$ to $H$.
    Note that $\pi|_H : H \to G/N$ is a homomorphism.
    Furthermore: \begin{align*}
        \Ima(\pi|_H) &= \pi(H), \\
        \Ker(\pi|_H) &= H \cap \Ker(\pi) = H \cap N.
    \end{align*} As the kernel of a homomorphism
    is a normal subgroup in the domain, $H \cap N \nsub H$.
    The homomorphism says that $\pi(H) \cong H / H \cap N$.
\end{proof}

\newpage
\noindent
Additionally, we have that $HN \leq G$ and $\pi(H) = HN/N$.
\begin{proof}
    We know that $HN \leq G$ if and only if $HN = NH$
    which is implied by the normality of $N$.
    We consider the group: \begin{align*}
        HN / N &=\Bigl(\{hnN : h \in H, n \in N\}, \times \Bigr), \\
        &=\Bigl(\{hN : h \in H\}, \times \Bigr), \tag{$N$ is a subgroup} \\
        &= \pi(H).
    \end{align*} As required.
\end{proof} 

\subsection{The Order of the Product}\label{2.24}

Let $G$ be a group with $N \nsub G$, and $H \leq G$.
If $HN$ is finite, then: \begin{align*}
    |HN| = \frac{|H||N|}{|H \cap N|}.
\end{align*}

\begin{proof}
    We can see that: \begin{align*}
        \frac{|HN|}{|N|} 
        &= [HN : N] \tag{By Lagrange's Theorem} \\
        &= |\pi(H)| \tag{By the above} \\
        &= [H : H \cap N] \tag{By the First Isomorphism Theorem} \\
        &= \frac{|H|}{|H \cap N|}, \tag{By Lagrange's Theorem}
    \end{align*} as required.
\end{proof}

\newpage

\section{The Second Isomorphism Theorem}

For a group $G$ with $N \leq H \leq G$, and $N, H \nsub G$, we have
that $H/N \nsub G/N$ and $(G / N)/(H / N) \cong G/H$.

\begin{proof}
    We let $\varphi : G/N \to G/H$ be defined by $gN \mapsto gH$.
    We have that: \begin{align*}
        aN = bN \Rightarrow ab^{-1} \in N \subseteq H \Rightarrow aH = bH,
    \end{align*} so $\varphi$ is well-defined. It is a homomorphism
    because: \begin{align*}
        \varphi(aNbN) &= \varphi(abN) \\
        &= abH \\
        &= aHbH \\
        &= \varphi(aN)\varphi(bN),
    \end{align*} and is trivially surjective. Considering: \begin{align*}
        \Ker(\varphi) 
        &= \{gN : gH = eH\} \\
        &= \{gN : g \in H\} \\
        &= H/N,
    \end{align*} we have that $H / N \nsub G / N$ as it is the kernel
    of a homomorphism and that \linebreak 
    $(G / N)/(H / N) \cong G / H$ by the
    homomorphism theorem.
\end{proof}

\section{The Correspondence Theorem}

For a group $G$ with $N \nsub G$ and $\pi : G \to G/N$ the quotient 
homomorphism. We have that: \begin{enumerate}
    \item If $K \subseteq G/N$ then: \begin{enumerate}
        \item $K \leq G/N$ if and only if $K = H/N$ for some
            $H \leq G$ containing $N$,
        \item $K \nsub G/N$ if and only if $K = H/N$ for some
            $H \nsub G$ containing $N$,
    \end{enumerate}
    \item If $N \subseteq H \subseteq G$ then: \begin{enumerate}
        \item $H \leq G$ if and only if $H = \pi^{-1}(K)$
            for some $K \leq G/N$,
        \item $H \nsub G$ if and only if $H = \pi^{-1}(K)$
            for some $K \nsub G/N$.
    \end{enumerate}
\end{enumerate}

\begin{proof}
    We have already proved the ($\Leftarrow$) direction in 
    (\ref{normal_prop}).
    \\[\baselineskip]
    (1)(a)
    Note that $K = \pi(\pi^{-1}(K))$. By the
    ($\Rightarrow$) direction of (2)(a), we know that
    $\pi^{-1}(K)$ is a subgroup of $G$ and contains
    $N$ as it's a subgroup. So, \linebreak
    $\pi(\pi^{-1}(K)) = \pi^{-1}(K) / N$.
    Taking $H = \pi^{-1}(K)$ proves the ($\Rightarrow$) direction of (1)(a).
    \\[\baselineskip]
    (1)(b) 
    To prove the ($\Rightarrow$) direction of (1)(b), we just need to
    prove that $K \nsub G/N$ implies that $\pi^{-1}(K) \nsub G$ which
    we proved in the ($\Leftarrow$) direction of (2)(b).
    \\[\baselineskip]
    (2)
    We know that $H$ is a union of left cosets of $N$ as it's a subgroup, 
    this means that $H = \pi^{-1}(\pi(H))$. We apply (\ref{normal_prop}) 
    again with $\phi = \pi$ and get the $(\Rightarrow)$ direction of
    (2).
\end{proof}
