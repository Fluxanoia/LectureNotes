\section{Soluble and Nilpotent Groups}

\subsection{Normal and Subnormal Series}

For a group $G$, a subnormal series of $G$ is a finite sequence:
\begin{align*}
    \{e\} = G_0 \nsub G_1 \nsub \cdots \nsub G_n = G.
\end{align*} If each $G_i$ is such that $G_i \nsub G$ then this is a
normal series. The length of the series is $n$ and we call
each $G_{i + 1} / G_i$ a factor of the series.

\subsection{Soluble Groups}

A group is soluble if it has a subnormal series in which every
factor is abelian. The length of the shortest such series is the
derived length.

\subsection{Insolubility of Non-abelian Simple Groups} \label{8.1}

For a non-abelian simple group $G$, $G$ is not soluble.

\begin{proof}
    If we have a subnormal series $G_0, G_1, \ldots, G_n$ for $G$,
    we take $m$ to be maximal such that $G_m \neq G$ and see that
    $G_m \nsub G_{m + 1} = G$ so $G_m = \{e\}$ as $G$ is simple.
    As such, $G / \{e\} \cong G$ is a non-abelian factor of $G$,
    contradicting the solubility of $G$.
\end{proof}

\subsection{Derived Series}

For a group $G$, the derived series of $G$ is: \begin{align*}
    G \geq G^{(0)} \geq G^{(1)} \geq \cdots
\end{align*} where for each $i$ in $[n]$, we define:
\begin{align*}
    G^{(0)} &= G, \\
    G^{(i)} &= [G^{(i - 1)}, G^{(i - 1)}].
\end{align*} We say that $G^{(m)}$ is the $m^{\text{th}}$ derived
subgroup of $G$.

\subsection{Derived and Subnormal Series} \label{8.3}

For a group $G$, $G$ is soluble of derived length at most $n$ if
and only if $G^{(n)} = \{e\}$.

\begin{proof}
    ($\Longrightarrow$) We have
    $\{e\} = G_0 \nsub G_1 \nsub \cdots \nsub G_n = G$ a
    subnormal series for $G$ with abelian factors. We want to
    show that $G^{(i)} \leq G_{n - i}$ for each $i$ in $[n]$.
    This shows that $G^{(n)} \leq G_0 = \{e\}$ and hence 
    $G^{(n)} = \{e\}$ as required. If $i = 0$, then this is
    true by definition. We proceed by induction with $i > 0$,
    by our hypothesis, we have that $G^{(i - 1)} \leq G_{n - i + 1}$.
    Since $G_{n - i + 1}/G^{n - i}$ is abelian by assumption, (\ref{2.28})
    implies that $[G_{n - i + 1}, G_{n - i + 1}] \subseteq G_{n - i}$
    so we have that
    $G^{(i)} = [G^{(i - 1)}, G^{(i - 1)}] \leq G_{n - i}$.
    \\[\baselineskip]
    ($\Longleftarrow$) We know that 
    $G^{(n)} \nsub G^{(n - 1)} \nsub \cdots \nsub G^{(0)} = G$
    is a subnormal series for $G$ by (\ref{2.27}) and the factors
    are abelian by (\ref{2.28}). Thus, $G$ is soluble of derived length
    at most $n$.
\end{proof}

\subsection{Derived Groups under Homomorphisms} \label{8.4}

For $G$ and $H$ groups with a homomorphism $\varphi$ from $G$ to $H$,
we have that $\varphi(G^{(k)}) = \varphi(G)^{(k)}$ for all $k$ in
$\mb{Z}_{\geq 0}$. 

\begin{proof}
    We have the $k = 0$ case by definition, we proceed by induction
    assuming $k > 0$: \begin{align*}
        \varphi(G)^{(k)} 
        &= [\varphi(G)^{(k - 1)}, \varphi(G)^{(k - 1)}] \\
        &= \ang{[x, y] : x, y \in \varphi(G)^{(k - 1)}} \\
        &= \ang{[x, y] : x, y \in \varphi(G^{(k - 1)})} \tag{IH} \\
        &= \ang{[\varphi(g), \varphi(h)] : g, h \in G^{(k - 1)}} \\
        &= \ang{\varphi([g, h]) : g, h \in G^{(k - 1)}} \tag{\ref{2.3}} \\
        &= \varphi(\ang{[g, h] : g, h \in G^{(k - 1)}}) \tag{\ref{2.7}} \\
        &= \varphi(G^{(k)}),
    \end{align*} as required.
\end{proof}

\subsection{Solubility of Subgroups and Quotients} \label{8.5}

For a soluble group $G$ of derived length at most $n$, every
subgroup and quotient of $G$ is soluble of derived length at
most $n$.

\begin{proof}
    By (\ref{8.3}), $G^(n) = \{e\}$ so for $H \leq G$ then
    $H^(n) \leq G^(n) = \{e\}$ so $H$ is soluble of derived
    length at most $n$ by (\ref{8.3}). For $N \nsub G$
    with $\pi$ from $G$ to $G / N$ the quotient homomorphism,
    (\ref{8.4}) implies that: \begin{align*}
        (G / N)^{(n)} = \varphi(G)^{(n)} = \pi(G^(n)) 
        = \pi(\{e\}) = N. \tag{8.4}
    \end{align*} Thus, by (\ref{8.3}), $G / N$ is soluble 
    of derived length at most $n$.
\end{proof}
