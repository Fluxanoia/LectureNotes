\section{Normal and Characteristic Subgroups}

For a group $G$, a subgroup $H$ of $G$ is normal if for each $g$ in $G$,
$gH = Hg$. This is denoted by $H \nsub G$.
\\[\baselineskip]
We say $H$ is a characteristic subgroup if for every $\varphi$ in $\Aut(G)$,
$\varphi(H) = H$ (denoted by $H \csub G$). We know characteristic subgroups
are normal as $\Aut(G)$ contains inner automorphisms.

\subsection{Properties of Normal Subgroups}

We have that for a group $G$, the set of normal subgroups on $G$
is closed under set multiplication and intersection. For $G$, $H$.
groups with $\varphi : G \to H$ a homomorphism, we have that: \begin{enumerate}
    \item If $K \leq G$ then $\varphi(K) \leq H$,
    \item If $K \nsub G$ then $\varphi(K) \nsub \varphi(G)$,
    \item If $K \leq H$ then $\varphi^{-1}(K) \leq G$,
    \item If $K \nsub H$ then $\varphi^{-1}(K) \nsub G$.
\end{enumerate} Using $K = \{e_H\}$ in (4), we can see that $\Ker(\varphi) \nsub G$.


\subsection{A Test for Normal and Characteristic Subgroups}

Let $G$ be a group with $H \leq G$: \begin{enumerate}
    \item If for every $g$ in $G$, $H^g \subseteq H$ then $H \nsub G$,
    \item If for every $\varphi$ in $\Aut(G)$, $\varphi(H) \subseteq H$ 
        then $H \csub G$.
\end{enumerate}

\begin{proof}
    (2) Suppose that $\varphi(H) \subseteq H$ for each $\varphi$ in $\Aut(G)$.
    We take $\varphi$ in $\Aut(G)$, $\varphi^{-1}$ is also an isomorphism
    so is also in $\Aut(G)$. We have that $\varphi^{-1}(H) \subseteq H$
    by our assumption, applying $\varphi$ to both sides, we see that
    $H \subseteq \varphi(H)$ so combined with our assumptions, 
    $H = \varphi(H)$ as required.
    \\[\baselineskip]
    (1) We can perform the same argument as (2) by using the fact that the
    inverse of an inner automorphism is also an inner automorphism.
\end{proof}

\subsection{Normal Subgroups of Index $2$}

For a group $G$ with $H \leq G$ and $[G : H] = 2$, $H \nsub G$.

\begin{proof}
    Taking $x$ in $G$, suppose $x$ is in $H$, then $xH = H = Hx$.
    \\[\baselineskip]
    Suppose $x$ is not in $H$, then $xH \neq H$ as $x$ is in $xH$.
    Thus, $xH$ and $H$ are disjoint cosets of $H$ and as $[G : H] = 2$, 
    $G = H \cup xH$ the disjoint union of these cosets. So, 
    $xH = G \backslash H$. We can apply this reasoning to the right coset
    and deduce that $xH = Hx$ as required.
\end{proof}

\subsection{Properties of the Centre}

For a group $G$, $Z(G)$ is a characteristic subgroup of $G$ and every
subgroup of $Z(G)$ is normal.

\begin{proof}
    We know that $Z(G) \leq G$. We take $\varphi$ in $\Aut(G)$ and take $z$
    in $Z(G)$. We take an arbitrary $g$ in $G$, as $z$ is in $Z(G)$, 
    $zg = gz$, thus $\varphi(z)\varphi(g) = \varphi(g)\varphi(z)$ as 
    $\varphi$ is a homomorphism. Furthermore, $\varphi(z)h = h\varphi(z)$
    for every $h$ in $G$ as $\varphi$ is surjective. Thus, $\varphi(z)$
    is in $Z(G)$ as required.
    \\[\baselineskip]
    Taking $H \leq Z(G)$, we know that for all $g$ in $G$, $h$ in $H$,
    $gh = hg$ as $h$ is in $Z(G)$. Thus, $gH = Hg$ for all $g$ in $G$.
\end{proof}

\subsection{Simple Groups}

A non-trivial group is simple if its only normal subgroups are itself 
and the trivial subgroup.
