\section{Soluble and Nilpotent Groups}

\subsection{Normal and Subnormal Series}

For a group $G$, a subnormal series of $G$ is a finite sequence:
\begin{align*}
    \{e\} = G_0 \nsub G_1 \nsub \cdots \nsub G_n = G.
\end{align*} If each $G_i$ is such that $G_i \nsub G$ then this is a
normal series. The length of the series is $n$ and we call
each $G_{i + 1} / G_i$ a factor of the series.

\subsection{Soluble Groups}

A group is soluble if it has a subnormal series in which every
factor is abelian. The length of the shortest such series is the
derived length.

\subsection{Insolubility of Non-abelian Simple Groups (8.1)} \label{8.1}

For a non-abelian simple group $G$, $G$ is not soluble.

\begin{proof}
    If we suppose $G$ is soluble and take a subnormal series 
    $G_0, G_1, \ldots, G_n$ for $G$ and $m$ to be maximal such that 
    $G_m \neq G$, we see that $G_m \nsub G_{m + 1} = G$ so $G_m = \{e\}$ 
    as $G$ is simple. As such, $G / \{e\} \cong G$ is a non-abelian factor 
    of $G$, contradicting the solubility of $G$.
\end{proof}

\subsection{Derived Series}

For a group $G$, the derived series of $G$ is: \begin{align*}
    \cdots \leq G^{(1)} \leq G^{(0)} = G,
\end{align*} where for each $i$ in $\mb{N}$, we define:
\begin{align*}
    G^{(0)} &= G, \\
    G^{(i + 1)} &= [G^{(i)}, G^{(i)}].
\end{align*} We say that $G^{(n)}$ is the $n^{\text{th}}$ derived
subgroup of $G$.

\subsection{Derived and Subnormal Series (8.3)} \label{8.3}

For a group $G$, $G$ is soluble of derived length at most $n$ if
and only if $G^{(n)} = \{e\}$.

\begin{proof}
    ($\Longrightarrow$) We have
    $\{e\} = G_0 \nsub G_1 \nsub \cdots \nsub G_n = G$ a
    subnormal series for $G$ with abelian factors. We want to
    show that $G^{(i)} \leq G_{n - i}$ for each $i$ in $[n]$.
    This shows that $G^{(n)} \leq G_0 = \{e\}$ and hence 
    $G^{(n)} = \{e\}$ as required. If $i = 0$, then this is
    true by definition. We proceed by induction with $i > 0$,
    by our hypothesis, we have that $G^{(i - 1)} \leq G_{n - i + 1}$.
    Since $G_{n - i + 1}/G^{n - i}$ is abelian by assumption, (\ref{2.28})
    implies that $[G_{n - i + 1}, G_{n - i + 1}] \subseteq G_{n - i}$
    so we have that
    $G^{(i)} = [G^{(i - 1)}, G^{(i - 1)}] \leq G_{n - i}$.
    \\[\baselineskip]
    ($\Longleftarrow$) We know that 
    $\{e\} = G^{(n)} \nsub G^{(n - 1)} \nsub \cdots \nsub G^{(0)} = G$
    is a subnormal series for $G$ by (\ref{2.27}) and the factors
    are abelian by (\ref{2.28}). Thus, $G$ is soluble of derived length
    at most $n$.
\end{proof}

\subsection{Derived Groups under Homomorphisms (8.4)} \label{8.4}

For $G$ and $H$ groups with a homomorphism $\varphi$ from $G$ to $H$,
we have that $\varphi(G^{(k)}) = \varphi(G)^{(k)}$ for all $k$ in
$\mb{Z}_{\geq 0}$. 

\begin{proof}
    We have the $k = 0$ case by definition, we proceed by induction
    assuming $k > 0$: \begin{align*}
        \varphi(G)^{(k)} 
        &= [\varphi(G)^{(k - 1)}, \varphi(G)^{(k - 1)}] \\
        &= \ang{[x, y] : x, y \in \varphi(G)^{(k - 1)}} \\
        &= \ang{[x, y] : x, y \in \varphi(G^{(k - 1)})} \tag{IH} \\
        &= \ang{[\varphi(g), \varphi(h)] : g, h \in G^{(k - 1)}} \\
        &= \ang{\varphi([g, h]) : g, h \in G^{(k - 1)}} \tag{\ref{eq2.3}} \\
        &= \varphi(\ang{[g, h] : g, h \in G^{(k - 1)}}) \tag{\ref{2.7}} \\
        &= \varphi(G^{(k)}),
    \end{align*} as required.
\end{proof}

\subsection{Solubility of Subgroups and Quotients (8.5)} \label{8.5}

For a soluble group $G$ of derived length at most $n$, every
subgroup and quotient of $G$ is soluble of derived length at
most $n$.

\begin{proof}
    By (\ref{8.3}), $G^{(n)} = \{e\}$ so for $H \leq G$ then
    $H^{(n)} \leq G^{(n)} = \{e\}$ so $H$ is soluble of derived
    length at most $n$ by (\ref{8.3}). For $N \nsub G$
    with $\pi$ from $G$ to $G / N$ the quotient homomorphism,
    (\ref{8.4}) implies that: \begin{align*}
        (G / N)^{(n)} = \varphi(G)^{(n)} = \pi(G^{(n)}) 
        = \pi(\{e\}) = N. \tag{8.4}
    \end{align*} Thus, by (\ref{8.3}), $G / N$ is soluble 
    of derived length at most $n$.
\end{proof}

\subsection{Commutator of Symmetric Groups (8.7)} \label{8.7}

For $n$ in $\mb{N}$, $[S_n, S_n] = A_n$.

\begin{proof}
    The cases for $n < 3$ are trivial as $A_n = \{e\}$, 
    so we consider $n \geq 3$.
    As $S_n / A_n \cong C_2$ we have $[S_n, S_n] \leq A_n$ by (\ref{2.28}). 
    Thus, it is sufficient to show that $A_n \leq [S_n, S_n]$. For
    a 3-cycle $(x, y, z)$, we know that: \begin{align*}
        (x, y, z) = [(x, y), (y, z)] \in [S_n, S_n],
    \end{align*} and as the 3-cycles generate $A_n$ and
    $(x, y, z)$ was arbitrary, we are done.
\end{proof}

\subsection{Insolubility of Symmetric Groups (8.6)} \label{8.6}

For $n$ in $\mb{N}$, $S_n$ is soluble if and only if $n \leq 4$.

\begin{proof}
    As $S_1 \leq S_2 \leq S_3 \leq S_4$, (\ref{8.5}) shows that the
    cases for $n \leq 4$ all follow from the case $n = 4$ where:
    \begin{align*}
        \{e\} \nsub \{e, (1, 2)(3, 4), (1, 3)(2, 4), (1, 4)(2, 3)\}
        \nsub A_4 \nsub S_4,
    \end{align*} has abelian factors so $S_4$ is soluble.
    For $n > 4$, $A_n$ is a non-abelian simple group
    so by (\ref{8.1}), it is not soluble. So, by (\ref{8.5}),
    $S_n$ is not soluble.
\end{proof}

\subsection{Central Series}

For a group $G$, we define a finite series 
$\{e\} = Z_0 \leq \cdots \leq Z_k = G$ of subgroups of $G$ to be 
central if for every $i$ in $[k]$, we have that 
$[G, Z_{i + 1}] \leq Z_{i}$. The length of the series is $k$.

\subsection{Nilpotent Groups}

A group that admits a central series is nilpotent. The
length of the shortest such series is called the step of $G$.

\subsection{Lower Central Series}

For a group $G$, the lower central series of $G$ is: \begin{align*}
     \cdots \leq G_2 \leq G_1 = G,
\end{align*} where for each $i$ in $\mb{N}$, we define:
\begin{align*}
    G_1 &= G, \\
    G_{i + 1} &= [G, G_i].
\end{align*}

\subsection{Lower Central and Central Series (8.8)} \label{8.8}

For a group $G$ with a lower central series 
$\cdots \leq G_2 \leq G_1 = G$, $G$ is nilpotent of step at most
$s$ if and only if $G_{s + 1} = \{e\}$.

\begin{proof}
    ($\Longrightarrow$) As $G$ is nilpotent of step at most $s$,
    it has a central series: \begin{align*}
        \{e\} = Z_0 \leq \cdots \leq Z_s = G.
    \end{align*} We want to show that $G_i \leq Z_{s + 1 - i}$
    for all $i$ in $[s]$ as this shows that $G_{s + 1} = \{e\}$. 
    If $i = 1$, then this is true by definition.
    We proceed by induction with $i > 0$: \begin{align*}
        G_i 
        &= [G, G_{i - 1}] \\
        &\leq [G, Z_{s + 2 - i}] \tag{IH} \\
        &\leq Z_{s + 1 - i}.
    \end{align*}
    ($\Longleftarrow$) We have that: \begin{align*}
        \{e\} = G_{s + 1} \leq \cdots \leq G_2 \leq G_1 = G,
    \end{align*} is a central series of length $s$ as
    for all $k$ in $[s]$, $[G, G_k] = [G, G_{k + 1}]$
    by definition. 
\end{proof}

\subsection{The Normaliser Condition (8.9)} \label{8.9}

For a nilpotent group $G$ with $H < G$, we have that $H \neq N_G(H)$.

\begin{proof}
    We take $\{e\} = Z_0 \leq \cdots \leq Z_k = G$ to be a central
    series for $G$ and set $n = \max(\{m \in [k]_0 : Z_m \leq H\})$.
    It must be that $n < k$ as $H \neq G$, so we consider some
    $z$ in $Z_{n + 1}$ and $h$ in $H$. By definition, 
    $h^{-1}z^{-1}hz = [h, z]$ is in $Z_n$ so 
    $z^{-1}hz$ is in $hZ_n$. But, as $Z_n \leq H$, $hZ_n \subseteq H$.
    Since $h$ was arbitrary, $z^{-1}Hz = H$ so $z$ is in $N_G(H)$
    and thus, as $z$ was arbitrary, $Z_{n + 1} \leq N_G(H)$. 
    By the maximality of $n$, $H \nleq Z_{n + 1} \leq N_G(H)$
    so $N_G(H) \neq H$.
\end{proof}
