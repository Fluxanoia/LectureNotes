\section{Normal, Characteristic, and Quotient Groups}

For a group $G$, a subgroup $H$ of $G$ is normal if for each $g$ in $G$,
$gH = Hg$. This is denoted by $H \nsub G$.
\\[\baselineskip]
We say $H$ is a characteristic subgroup if for every $\varphi$ in $\Aut(G)$,
$\varphi(H) = H$ (denoted by $H \csub G$). We know characteristic subgroups
are normal as $\Aut(G)$ contains inner automorphisms.

\subsection{Properties of Normal Subgroups (2.14-17)}
\label{2.14} \label{2.15} \label{2.16} \label{2.17}

For a group $G$, the set of normal subgroups on $G$
is closed under set multiplication and intersection. For $G$ and $H$
groups with $\varphi$ from $G$ to $H$ a homomorphism, we have that: \begin{enumerate}
    \item if $K \leq G$ then $\varphi(K) \leq H$,
    \item if $K \nsub G$ then $\varphi(K) \nsub \varphi(G)$,
    \item if $K \leq H$ then $\varphi^{-1}(K) \leq G$,
    \item if $K \nsub H$ then $\varphi^{-1}(K) \nsub G$.
\end{enumerate} Using $K = \{e\}$ in (4), we can see that $\Ker(\varphi) \nsub G$.

\begin{proof}
    For $P$ and $Q$ normal subgroups of $G$, $PQ = QP$ by normality so
    $PQ \leq G$ by (\ref{1.9}). We know that $PQ$ is normal as for all $g$ in $G$,
    $g^{-1}PQg = g^{-1}Pgg^{-1}Qg = HK$ by the normality of $P$ and $Q$.
    Then, we know that the intersection of subgroups is a subgroup by (\ref{1.11}),
    for a set of normal subgroups of $\mc{A} \subseteq \mc{P}(G)$ and $g$ in $G$: \begin{align*}
        \left( \bigcap_{A \in \mc{A}}A \right)^g
        = \bigcap_{A \in \mc{A}} A^g
        = \bigcap_{A \in \mc{A}} A,
    \end{align*} so this intersection is normal.
    \\[\baselineskip]
    (1) For $\varphi(x)$ and $\varphi(y)$ in $\varphi(K)$, 
    $\varphi(x)^{-1}\varphi(y) = \varphi(x^{-1}y)$ which is in $\varphi(K)$ as $K$ is
    a subgroup. The result follows from the subgroup test.
    \\[\baselineskip]
    (2) For every $g$ in $G$, we have that 
    $\varphi(g)^{-1}\varphi(K)\varphi(g) = \varphi(K^g) = \varphi(K)$.
    \\[\baselineskip]
    (3) For $x$ and $y$ in $\varphi^{-1}(K)$, $\varphi(x^{-1}y) = \varphi(x)^{-1}\varphi(y)$
    is in $K$ so $x^{-1}y$ is in $\varphi^{-1}(K)$. The result follows from the subgroup test.
    \\[\baselineskip]
    (4) For $g$ in $G$, $\varphi(g^{-1}\varphi^{-1}(K)g) = \varphi(g)^{-1}K\varphi(g) = K$
    so $g^{-1}\varphi^{-1}(K)g \subseteq \varphi^{-1}(K)$. The result follows from (\ref{2.12}).
\end{proof}

\subsection{A Test for Normal and Characteristic Subgroups (2.12)} \label{2.12}

Let $G$ be a group with $H \leq G$: \begin{enumerate}
    \item if for every $g$ in $G$, $H^g \subseteq H$ then $H \nsub G$,
    \item if for every $\varphi$ in $\Aut(G)$, $\varphi(H) \subseteq H$ 
        then $H \csub G$.
\end{enumerate}

\begin{proof}
    (2) We suppose that $\varphi(H) \subseteq H$ for each $\varphi$ in $\Aut(G)$.
    For $\varphi$ in $\Aut(G)$, $\varphi^{-1}$ is also in $\Aut(G)$. 
    We have that $\varphi^{-1}(H) \subseteq H$ by our assumption, 
    applying $\varphi$ to both sides, we see that
    $H \subseteq \varphi(H)$ so $H = \varphi(H)$ as required.
    \\[\baselineskip]
    (1) We can perform the same argument as (2) as conjugation is an inner
    automorphism.
\end{proof}

\subsection{Normal Subgroups of Index 2 (2.11)} \label{2.11}

For a group $G$ with $H \leq G$ such that $[G : H] = 2$, $H \nsub G$.

\begin{proof}
    For $x$ in $H$, $xH = H = Hx$.
    For $x$ in $G \setminus H$, $xH \neq H$ by (\ref{1.18}).
    Thus, $xH$ and $H$ are disjoint cosets of $H$ and as $[G : H] = 2$, 
    $G = H \cup xH$ is a disjoint union. We can apply the same argument 
    to the right coset and deduce that $xH = Hx$ as required.
\end{proof}

\subsection{Properties of the Centre (2.13)} \label{2.13}

For a group $G$, $Z(G)$ is a characteristic subgroup of $G$ and every
subgroup of $Z(G)$ is normal.

\keyin{We have that $\varphi(Z(G))$ commutes with $\Ima(\varphi)$ but this is just $G$
as $\varphi$ is an isomorphism. Then, we just use (\ref{2.12}).}

\begin{proof}
    We note that $Z(G) \leq G$ by (\ref{1.8}). For $\varphi$ in $\Aut(G)$ and $z$
    in $Z(G)$, we take some $g$ in $G$, so $zg = gz$ and thus 
    $\varphi(z)\varphi(g) = \varphi(g)\varphi(z)$ as 
    $\varphi$ is a homomorphism. Thus, as $g$ was arbitrary and $\varphi$ is surjective, 
    $\varphi(z)$ must be in $Z(G)$. Since $z$ was arbitrary, $Z(G)$
    is a characteristic subgroup by (\ref{2.12}).
    \\[\baselineskip]
    Every subgroup of $Z(G)$ contains only elements that commute with
    all elements of $G$, so must be normal.
\end{proof}

\subsection{Quotient Groups (2.18)} \label{2.18}

For a group $G$ with $H \nsub G$, the quotient of $G$ by $H$, 
$G / H$, is a group under
set multiplication and for every $a$ and $b$ in $G$ satisfies
$(aH)(bH) = (ab)H$. Furthermore, the map $\pi$ from $G$ to $G / H$
defined by $g \mapsto gH$ is a surjective homomorphism with
kernel $H$, called the quotient homomorphism.

\begin{proof}
    We know set multiplication is associative so for 
    $a$ and $b$ in $G$, we see that: \begin{align*}
        (aH)(bH) &= aHbH \\
        &= (ab)(HH) \tag{$H \nsub G$} \\
        &= (ab)H.
    \end{align*} Thus, $\pi$ is a homomorphism, $G / H$ is closed under this operation,
    $eH$ is the identity, and for $g$ in $G$, the inverse of $gH$ is $g^{-1}H$. 
    So, $G/H$ is a group under set multiplication.
    We have that $\pi$ is trivially surjective and for $g$ in $\Ker(\pi)$,
    $\varphi(g) = gH = H$ which means that $g$ is in $H$ by (\ref{1.18}).
\end{proof} 
