\section{Finitely Generated Abelian Groups}

We will write $\mb{Z}^n = \{(m_1, \ldots, m_n) : m_1, \ldots, m_n \in \mb{Z}\}$
and $e_i = (0, \ldots, 1, \ldots, 0) \in \mb{Z}^n$ with $1$ in the 
$i^{\text{th}}$ entry. These are the standard generators for $\mb{Z}^n$.
\\[\baselineskip]
For some $n$ in $\mb{N}$, we write $\mb{Z}_n$ to be the integers
modulo $n$ which is a group under addition. Additionally, $n\mb{Z}$
is a subgroup of $\mb{Z}$ and $\mb{Z}_n \cong \mb{Z}/n\mb{Z}$.

\subsection{Classification of Cyclic Groups}

For a cyclic group $G$, if $|G| = n$ finite, we have that $G \cong \mb{Z}_n$.
Otherwise, $G \cong \mb{Z}$.

\begin{proof}
    We choose $x$ as a generator of $G$. We take $\varphi : \mb{Z} \to G$
    to be defined as $\varphi(m) = x^m$. We can see that $\varphi$ is a
    surjective homomorphism. If $|x| = \infty$ then $\Ker(\varphi) = \{0\}$,
    otherwise, $\Ker(\varphi) = |x|\mb{Z}$. By the homomorphism theorem:
    \begin{align*}
        G = \Ima(\varphi) \cong \mb{Z}/\Ker(\varphi).
    \end{align*} The result follows as $\mb{Z}/\Ker(\varphi) = \mb{Z}$
    if $|x| = \infty$ and $\mb{Z}_{|x|}$ otherwise.
\end{proof}

\subsection{The Torsion Subgroup}

For an abelian group $G$ with $T \subseteq G$ the set of
elements in $G$ of finite order and a prime $p$ with
$G_p \subseteq T$ the set of elements in $T$ of with order
equal to a power of $p$. We have that $G_p \leq T \leq G$
and $G/T$ is torsion-free
with $T$ called the torsion subgroup of $G$ and $G_p$ called
the $p$-primary component of $G$.

\begin{proof}
    We suppose $x$ and $y$ are in $T$ with $|x| = k$, $|y| = m$.
    We know that \linebreak $km(x - y) = 0$ so $|x - y| \leq km < \infty$,
    thus $(x - y)$ is in $T$ so $T$ is a subgroup of $G$ by the subgroup test. 
    \\[\baselineskip]
    Furthermore,
    $|x - y|$ must divide $km$ so if $x$ and $y$ are in $G_p$
    then $km$ is a power of $p$. Thus, $|x - y|$ is a
    power of $p$ so $(x - y)$ is in $G_p$. Again, by the subgroup test,
    $G_p$ is a subgroup of $T$.
    \\[\baselineskip]
    Suppose $z + T$ has finite order for some $z$ in $G$. If so,
    there exists some $m$ in $\mb{N}$ with $mx + T = T$, in particular 
    $mx$ is in $T$. By the definition of $T$, there exists some $n$
    in $\mb{N}$ with $nmx = 0$. But, this would mean $x$ has finite order
    so $x$ is in $T$. Thus, $G / T$ is torsion-free.
\end{proof}

\subsection{The Primary Decomposition Theorem}

For a finite abelian group $G$, we take $p_1, \ldots, p_k$ to be the
prime factors of $|G|$. We have that $G = G_{p_1} \oplus \cdots \oplus
G_{p_k}$.

\begin{proof}
    We take $x$ in $G$, by Lagrange's Theorem, we have that
    $x = p_1^{l_1} \cdots p_k^{l_k}$ for some $l_1, \ldots, l_k$ in $\mb{N}_0$.
    For each $i$ in $[k]$, we set: \begin{align*}
        n_i = \prod_{j \in [k]\backslash\{i\}} p_j^{l_j},
    \end{align*} and note that $|n_ix| = p_i^{l_i}$ so $n_ix$ is in $G_{p_i}$.
    Clearly $\gcd(n_1, \ldots, n_k) = 1$, so by the Euclidean algorithm there
    exists $m_1, \ldots, m_k$ such that $m_1n_1 + \cdots + m_kn_k = 1$. Thus:
    \begin{align*}
        x 
        &= \left(\sum_{i = 1}^k m_in_i\right) \cdot x \\
        &= \sum_{i = 1}^k m_i(n_ix) \\
        &\in \sum_{i = 1}^k G_{p_i}. 
    \end{align*} Thus, $G = G_{p_1} + \cdots + G_{p_k}$.
    \\[\baselineskip]
    We now consider $x_i$, $x_i'$ in $G_{p_i}$ for each $i$ in $[k]$ such
    that $\sum_{i \in [k]} x_i = \sum_{i \in [k]} x_i'$. We write
    $y_i = x_i - x_i'$ so that $\sum_{i \in [k]} y_i = 0$. Furthermore,
    we say $|y_i| = p_i^{d_i}$ and set: \begin{align*}
        r_i 
        = \prod_{j \in [k] \backslash \{i\}} |y_i|
        = \prod_{j \in [k] \backslash \{i\}} p_j^{d_j}.
    \end{align*} As $|y_i|$ divides $|r_j|$ for all 
    $j \in [k] \backslash \{i\}$, we know that $r_iy_j = 0$.
    This implies that $r_iy_i = 0$ as $\sum_{i = 1}^k y_i = 0$.
    \\[\baselineskip]
    Moreover, as $r_i$ and $p_i$ are coprime by definition, the
    Euclidean algorithm implies that there exists $a$, $b$ in $\mb{Z}$
    such that: \begin{align*}
        &ar_i + bp_i^{d_i} = 1, \\
        \Longrightarrow \qquad & y_i = (ar_i + bp_i^{d_i})y_i,  \\
        \Longrightarrow \qquad & y_i = ar_iy_i + bp_i^{d_i}y_i,  \\
        \Longrightarrow \qquad & y_i = 0 + 0 = 0, 
    \end{align*} so $x_i = x_i'$ for each $i$ in $[k]$. Thus, our
    compositions are unique as required.
\end{proof}

\subsection{Finitely Generated Abelian Torsion Groups}

A finitely generated torsion group is finite.

\begin{proof}
    We take $x_1, \ldots, x_n$ to be the finite generating set for
    an abelian torsion group $G$ so: \begin{align*}
        G = \{l_1x_1 + \cdots + l_nx_n : 0 \leq l_i < |x_i|\},
    \end{align*} which is finite since $|x_i| < \infty$ for all $i$ in $[n]$.
\end{proof}

\subsection{Fundamental Theorem of Finitely Generated Abelian Groups}

Suppose $G$ is a finitely generated abelian group, there exists
non-negative integers $n$ and $k$, primes $p_1, \ldots, p_k$, and
natural numbers $n_1, \ldots, n_k$ such that: \begin{align*}
    G \cong \mb{Z}_{p_1^{m_1}} \oplus \cdots 
    \oplus \mb{Z}_{p_k^{m_k}} \oplus \mb{Z}^n
\end{align*}
