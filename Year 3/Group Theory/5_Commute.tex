\section{Commutators} 

For $x$ and $y$ in a group $G$, we define the commutator
of $x$ and $y$ as $[x, y] = x^{-1}y^{-1}xy$.
This can be interpreted as the 'cost' of commuting
$x$ and $y$ as $xy = yx[x, y]$.

\subsection{Commutators under Homomorphisms} \label{eq2.3}

For $x$ and $y$ in a group $G$ with a homomorphism $\varphi$ from $G$ to $H$, 
we have that: \begin{align*}
    \varphi([x, y]) = [\varphi(x), \varphi(y)].
\end{align*}

\begin{proof}
    Trivial from the definitions.
\end{proof}

\subsection{Commutator Subgroups}

For a group $G$ with $H$ and $K \leq G$, we define the subgroup $[H, K]$ by:
\begin{align*}
    [H, K] = \ang{[h, k] : h \in H, k \in K}.
\end{align*} The subgroup $[G, G]$ is the commutator subgroup of $G$.
If $G$ is abelian, $[G, G] = \{e\}$.

\subsection{Commutator of Normal Subgroups (2.32)} \label{2.32}

For a group $G$ with $H$ and $K \nsub G$, $[H, K] \subseteq (H \cap K)$.

\begin{proof}
    For $h$ in $H$ and $k$ in $K$, $[h, k] = h^{-1}k^{-1}hk$ so: \begin{itemize}
        \item $h^{-1}k^{-1}h$ is in $h^{-1}Kh = K$,
        \item $k^{-1}hk$ is in $k^{-1}Hk = H$.
    \end{itemize} Hence, $[h, k]$ is in $(H \cap K)$.
\end{proof} 

\subsection{Commutators of Characteristic Subgroups (2.27)} \label{2.27}

For a group $G$ with $H$ and $K \csub G$, $[H, K] \csub G$. Thus, $[G, G] \csub G$.

\begin{proof}
    For $\varphi$ in $\Aut(G)$: \begin{align*}
        \varphi([H, K]) 
        &= \varphi(\ang{[h, k] : h \in H, k \in K}) \\
        &= \ang{\varphi([h, k]) : h \in H, k \in K} \tag{\ref{2.7}} \\
        &= \ang{[\varphi(h), \varphi(k)] : h \in H, k \in K} \tag{\ref{eq2.3}} \\
        &= \ang{[h, k] : h \in H, k \in K} \tag{$H$ and $K \csub G$} \\
        &= [H, K],
    \end{align*} as required.
\end{proof}

\subsection{Abelian Quotients (2.28)} \label{2.28}

For a group $G$ with $H \nsub G$, $G / H$ is abelian if and only if 
$[G, G] \leq H$. Furthermore, a quotient of $G$
is abelian if and only if it is isomorphic to $G / [G, G]$.

\begin{proof}
    We take $\pi$ from $G$ to $G / H$ to be the quotient homomorphism.
    \\[\baselineskip]
    ($\Longrightarrow$) 
    We take $x$ and $y$ in $G$, we have that $\pi([x, y]) = [\pi(x), \pi(y)] = eH$,
    thus $[x, y]$ is in $H$. Thus, as $x$ and $y$ are arbitary, $[G, G] \subseteq H$.
    \\[\baselineskip]
    ($\Longleftarrow$)
    For every $xH$ and $yH$ in $G / H$, we have that: \begin{align*}
        [xH, yH] 
        &= (x^{-1}H)(y^{-1}H)(xH)(yH) \\
        &= [x, y]H \\
        &= H.
    \end{align*} Thus, $G / H$ is abelian. So, $G / H$ is abelian if
    and only if $[G, G] \leq H$ which is true if and only if $G / H$ is isomorphic
    to a quotient to $G / [G, G]$ as this implies: \begin{align*}
        (G / [G, G]) / (H / [G, G]) \cong G / H,
    \end{align*} by the Second Isomorphism Theorem.
\end{proof}

\subsubsection{Quotients of Abelian Groups (2.29)} \label{2.29}

Every quotient of an abelian group is abelian.

\begin{proof}
    If $G$ is abelian then $[G, G] = \{e\}$. So, for each $H \csub G$
    we have $[G, G] \leq H$ and so $G / H$ is abelian by (\ref{2.28}).
\end{proof}

\subsection{The Abelianisation}

For a group $G$, the abelianisation of $G$ is the quotient group 
$G/[G, G]$. This group is always abelian and is the largest possible
abelian quotient of $G$.
