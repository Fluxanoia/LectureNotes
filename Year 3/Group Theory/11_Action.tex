\section{Group Actions}

For a group $G$ and a non-empty set $X$, an action of $G$ on $X$ is
a homomorphism $\varphi : G \to \Sym(X)$. We say that: \begin{itemize}
    \item the action is faithful if $\varphi$ is injective,
    \item the action is transitive if for all $x$, $y$ in $X$,
        there exists $g$ in $G$ such that $\varphi(g)(x) = y$.
\end{itemize} We will abbreviate $\varphi(g)(x)$ to $g \cdot x$.

\subsection{The Orbit and Stabiliser}

For a group $G$ acting on a set $X$, for each $x$ in $X$: \begin{align*}
    \Orb_G(x)  &= G \cdot x = \{g \cdot x : g \in G\}, \\
    \Stab_G(x) &= G_x = \{g \in G : g \cdot x = x\},
\end{align*} are the orbit and stabiliser of $x$, respectively.

\subsection{The Orbit-Stabiliser Theorem}

For a group $G$ acting on a set $X$ with $x$ in $X$, 
$\Stab_G(x)$ is a subgroup of $G$ and there is a
well-defined bijection $\varphi$ from $\Orb_G(x)$ to $G / \Stab_G(x)$
defined by: \begin{align*}
    \varphi(g \cdot x) = g\Stab_G(x).
\end{align*} If $G$ is finite, $|G| = |\!\Orb_G(x)| \cdot |\!\Stab_G(x)|$.

\begin{proof}
    We want to show that $\Stab_G(x) \leq G$.
    As the action is a homomorphism, $e \cdot x = x$, so $e$ is
    in $\Stab_G(x)$. For $g$, $h$ in $\Stab_G(x)$, then: \begin{align*}
        (gh) \cdot x 
        &= g \cdot (h \cdot x) \tag{action is homomorphic} \\
        &= g \cdot (x) \\
        &= x.
    \end{align*} For $g$ in $\Stab_G(x)$: \begin{align*}
        g^{-1} \cdot (g \cdot x) = x
        & \Longleftrightarrow g^{-1} \cdot x = x \\
        & \Longleftrightarrow g^{-1} \in \Stab_G(x),
    \end{align*} so $\Stab_G(x) \leq G$. We know that
    $\varphi$ is well-defined and injective as: \begin{align*}
        \bigl[ g \cdot x = h \cdot x \bigr]
        & \Longleftrightarrow h^{-1}g \cdot x = x \\
        & \Longleftrightarrow h^{-1}g \in \Stab_G(x) \\
        & \Longleftrightarrow g \in h\Stab_G(x) \\
        & \Longleftrightarrow g\Stab_G(x) = h\Stab_G(x).
    \end{align*} As $\varphi$ is trivially surjective, it is a
    well-defined bijection as required.
\end{proof}

\subsection{Relation via the Orbit}

For a group $G$ acting on a set $X$, we define an equivalence relation
on $X$ by $x \sim y$ if $y$ is in $\Orb_G(x)$. The orbits of elements
$x$ in $G$ are the equivalence classes of this relation, so they 
partition $X$.

\begin{proof}
    \textbf{Reflexivity} For all $x$ in $X$, we have that
    $e \cdot x = x$ so $x \sim x$.
    \\[\baselineskip]
    \textbf{Symmetry} If $g \cdot x = y$ then $g^{-1} \cdot y = x$.
    \\[\baselineskip]
    \textbf{Transitivity} If $x \sim y \sim z$ then there exists
    $g$ such that $y = g \cdot x$ and $h$ such that $z = h \cdot y$
    so $z = (hg) \cdot x$ so $x \sim z$.
\end{proof}

\subsection{Fixed Points}

For a group $G$ acting on a set $X$, $x$ in $X$ is a fixed point
for this action if $\Orb_G(x) = \{x\}$.
We write $\Fix_G(X)$ for the set of fixed points of this action. 
\\[\baselineskip]
For $X$ finite, we write $\mc{O}_G(X)$
for the set of orbits of $X$ under this action. For each orbit
$O$ in $\mc{O}_G(X)$, we pick can arbitrary element $x_O \in O$
and see that: \begin{align*}
    |X| = |\!\Fix_G(X)| + \sum_{O \in \mc{O}_G(X), \, |O| > 1} [G : \Stab_G(x_O)].
\end{align*}

\begin{proof}
    We have that: \begin{align*}
        |X| &= \sum_{O \in \mc{O}_G(X)} |O| \tag{Relation via the Orbit} \\
        &= |\!\Fix_G(X)| + \sum_{O \in \mc{O}_G(X), \, |O| > 1} |O| \\
        &= |\!\Fix_G(X)| + \sum_{O \in \mc{O}_G(X), \, |O| > 1} [G : \Stab_G(x_O)]. \tag{Orbit-Stabiliser}
    \end{align*}
\end{proof}

\subsection{The Conjugation Action}

For a group $G$, acting on itself via the conjugacy action 
($g \cdot x = gxg^{-1}$), we take $x$ in $G$. The conjugacy class 
of $x$, denoted by $x^G$, is defined by: \begin{align*}
    x^G = \{gxg^{-1} : g \in G\} = \Orb_G(x).
\end{align*} The centraliser of $x$, denoted by $C_G(x)$, 
is defined by: \begin{align*}
    C_G(x) = \{g \in G : gxg^{-1} = x\} = \Stab_G(x).
\end{align*} For $H \leq G$, the normaliser of $H$ in $G$,
denoted by $N_G(H)$, is defined by: \begin{align*}
    N_G(H) = \{g \in G : gHg^{-1} = H\}.
\end{align*} We note that this is the stabiliser of $H$ under the
conjugation action of $G$ onto the set of subgroups of $G$.

\subsection{Partitioning on Conjugacy Classes}

For a group $G$, the conjugacy classes of $G$ partition $G$.

\subsection{The Orbit-Stabiliser Theorem for Conjugation}

For a group $G$ with $x$ in $G$, $C_G(x) \leq G$ and there exists
a well-defined bijection $\varphi$ from $x^G$ to $G / C_G(x)$ 
defined by: \begin{align*}
    \varphi(gxg^{-1}) = gC_G(x).
\end{align*} If $G$ is finite, $|G| = |x^G||C_G(x)|$. If we apply
this to the conjugation action of $G$ onto the set of its subgroups,
we get that: \begin{align*}
    |\{K \leq G : K \text{ is conjugate to } H\}|
    = [G : N_G(H)].
\end{align*}

\subsection{The Class Equation}

For a finite group $G$, we write $\mc{C}$ for the set of conjugacy classes
of $G$, for each conjugacy class $C$, we can pick an arbitrary element
$g_C$ and see that: \begin{align*}
    |G| = |Z(G)| + \sum_{C \in \mc{C}(G), \, |C| > 1} [G : C_G(g_C)].
\end{align*}


