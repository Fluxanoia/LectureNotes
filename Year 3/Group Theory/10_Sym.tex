\section{Symmetric Groups}

For a set $X$, a permutation of $X$ is a bijection from $X$ to $X$,
the set of all permutations of $X$ forms a group under composition
denoted by $\Sym(X)$. For $n$ in $\mb{N}$, we write $\Sym([n])$ as $S_n$.
Note that $|\Sym(X)| = |X|!$.

\begin{proof}
    We prove this by considering the number of bijections between sets
    $X$, $Y$ of size $n$. For $n = 1$, $f = \{(x, y) : x \in X, y \in Y\}$
    has only one pair. For $n > 1$: \begin{align*}
        m 
        &= \sum_{y \in Y} |\{\text{bijections from } X \text{ to } Y 
            : x \mapsto y\}| \\
        &= \sum_{y \in Y} |\{\text{bijections from } X \backslash \{x\} 
            \text{ to } Y \backslash \{y\}\}| \\
        &= \sum_{y \in Y} (n - 1)! \\
        &= n \cdot (n - 1)! \\
        &= n!,
    \end{align*} as required.
\end{proof}

\subsection{Cycles}

For $k$ in $\mb{N}$, a permutation $f$ in $S_n$ is called $k$-cycle if
there are $k$ distinct members $i_1, \ldots, i_k$ in $[n]$ such that:
\begin{align*}
    f(i_j) = \begin{cases}
        i_{j + 1} & j \in [k - 1] \\
        i_1 & j = k,
    \end{cases}
\end{align*} in which case, we write $f = (i_1, \ldots, i_k)$.
A $2$-cycle is called a transposition and cycles 
$(i_1, \ldots, i_k)$, $(j_1, \ldots, j_l)$
are disjoint if $\{i_1, \ldots, i_k\} \cap \{j_1, \ldots, j_l\} = \emptyset$.
Furthermore: \begin{itemize}
    \item A $k$-cycle has order $k$,
    \item $(i_1, \ldots, i_k) = (i_2, \ldots, i_k, i_1)$,
    \item $(i_1, \ldots, i_k)^{-1} = (i_k, \ldots,v i_1)$,
    \item $(i_1, \ldots, i_k) = (i_1, i_2)(i_2, i_3) \cdots (i_{k - 1}, i_k)$,
    \item Disjoint cycles commute.
\end{itemize} 

\subsection{Permutations as Disjoint Cycles} \label{4.1}

For $n$ in $\mb{N}$, each element of $S_n$ can be written as a product of
disjoint cycles with lengths summing to $n$ which is unique up to reordering.
Every element can also be written as a product of transpositions.
\\[\baselineskip]
From this, we can see that $S_n$ is generated by the set of transpositions on
$1$, $\{(1, 2), (1, 3), \ldots, (1, n)\}$ as $(1, i)(1, j)(1, i) = (i, j)$.

\subsection{Cycle Type} \label{4.2}

For $f$ in $S_n$ written as a product of disjoint cycles with lengths summing
to $n$, we take $l_1, \ldots, l_k$ be the lengths of these cycles in
descending order. The $k$-tuple $(l_1, \ldots, l_k)$ is the cycle type of $f$.
\\[\baselineskip]
From this, we can see that $|f| = \Lcm(l_1, \ldots, l_k)$.

\subsection{Conjugacy in $S_n$} \label{4.4}

For all $g$ in $S_n$ with $i_1, \ldots, i_k$ distinct elements of $[n]$: 
\begin{align*}
    g(i_1, \ldots, i_k)g^{-1} = (g(i_1), \ldots, g(i_k)).
\end{align*}

\begin{proof}
    For $k = 1$, $(i_1) = e$ so $g(i_1)g^{-1} = gg^{-1} = e = (g(i_1))$.
    For $k = 2$ and $x$ in $S_n \backslash \{g(i_1), g(i_2)\}$: \begin{align*}
        g(i_1, i_2)g^{-1}(g(i_1)) 
        &= g(i_2), \\
        g(i_1, i_2)g^{-1}(g(i_2)) 
        &= g(i_1), \\
        g(i_1, i_2)g^{-1}(x) 
        &= gg^{-1}(x) \\
        &= x,
    \end{align*} so $g(i_1, i_2)g^{-1} = (g(i_1), g(i_2))$. For $k > 2$,
    $(i_1, \ldots, i_k) = (i_1, i_2)\cdots(i_{k - 1}, i_k)$ so: \begin{align*}
        g(i_1, \ldots, i_k)g^{-1} 
        &= g(i_1, i_2)\cdots(i_{k - 1}, i_k)g^{-1}\\
        &= g(i_1, i_2)g^{-1}g \cdots g^{-1}g(i_{k - 1}, i_k)g^{-1}\\
        &= (g(i_1), g(i_2)) \cdots (g(i_{k - 1}), g(i_k)) \\
        &= (g(i_1), \ldots, g(i_k)),
    \end{align*} as required.
\end{proof}

\subsection{Conjugacy and Cycle Type} \label{4.3}

For $x$, $y$ in $S_n$, $x$ and $y$ are conjugate if and only if
they have the same cycle type.

\begin{proof}
    We take the cycle type of $x$ be $(l_1, \ldots, l_k)$ so: \begin{align*}
        x = (a_1^{(1)}, \ldots, a_{l_1}^{(1)}) \cdots (a_1^{(k)}, \ldots, a_{l_k}^{(k)}),
    \end{align*} with each value in $[n]$ corresponding to some $a_j^{(r)}$.
    For $g$ in $S_n$: \begin{align*}
        gxg^{-1} 
        &= g(a_1^{(1)}, \ldots, a_{l_1}^{(1)})g^{-1}g \cdots g^{-1}g(a_1^{(k)}, \ldots, a_{l_k}^{(k)})g^{-1} \\
        &= (g(a_1^{(1)}), \ldots, g(a_{l_1}^{(1)})) \cdots (g(a_1^{(k)}), \ldots, g(a_{l_k}^{(k)})),
    \end{align*} with the disjoint property of the cycles preserved as the
    contrary would contradict the disjoint property of the cycles of $x$.
    Thus, all conjugates of $x$ have the same cycle type as $x$.
    \\[\baselineskip]
    For $y$ in $S_n$ with cycle type equal to $(l_1, \ldots, l_k)$: \begin{align*}
        y = (b_1^{(1)}, \ldots, b_{l_1}^{(1)}) \cdots (b_1^{(k)}, \ldots, b_{l_k}^{(k)}),
    \end{align*} with each value in $[n]$ corresponding to some $b_j^{(r)}$.
    We define $g$ in $S_n$ by $g(a_i^{(j)}) = b_i^{(j)}$ and see that: \begin{align*}
        gxg^{-1} = y,
    \end{align*} using the previous result.
\end{proof}

\subsection{Parity of Transposition Representations}

For $x$ in $S_n$ with $x = t_1 \cdots t_r = s_1 \cdots s_k$ and
each $t_i$, $s_i$ a transposition, $r \equiv k \bmod 2$.

\subsection{Signature}

For $x$ in $S_n$ with $x = t_1 \cdots t_r$ and each $t_i$ a transposition,
the signature of $x$ is defined as: \begin{align*}
    \varepsilon(x) &= \begin{cases}
        1 & r \equiv 0 \bmod 2 \\
        -1 & \text{otherwise}.
    \end{cases}
\end{align*}

\subsection{The Signature Homomorphism}

For $n$ in $\mb{N}$, $\varepsilon : S_n \to (\{-1, 1\}, \times)$ is a
homomorphism.

\begin{proof}
    For $x$, $y$ in $S_n$ with $x = x_1 \cdots x_r$ and $y = y_1 \cdots y_s$
    where each $x_i$ and $y_j$ is a transposition: \begin{align*}
        \varepsilon(xy) 
        &= \varepsilon(x_1 \cdots x_ry_1 \cdots y_s) \\
        &= (-1)^{r + s} \\
        &= (-1)^r(-1)^s \\
        &= \varepsilon(x)\varepsilon(y).
    \end{align*}
\end{proof}

\subsection{Alternating Groups}

We define the alternating group $A_n$ to be the set of even permutations
in $S_n$. Thus, $A_n \nsub S_n$.

\begin{proof}
    $A_n = \Ker(\varepsilon)$.
\end{proof}

\subsection{Subgroups of Index 2 in $S_n$}

For $n > 1$, $H \leq S_n$ has index 2 if and only if $H = A_n$.

\begin{proof}
    ($\Rightarrow$) We know that $H \nsub S_n$ so we consider $S_n / H$ 
    which must have order 2 as $H$ has index 2.
    Thus, $S_n / H \cong C_2 \cong (\{-1, 1\}, \times)$ so there's a 
    surjective homomorphism $\pi$ from $S_n$ to $(\{-1, 1\}, \times)$
    with kernel $H$. For $t_1$, $t_2$ transpositions, 
    there exists $g$ such that $t_1 = g^{-1}t_2g$ so: \begin{align*}
        \pi(t_1) 
        &= \pi(g)^{-1}\pi(t_2)\pi(g) \\
        &= \pi(t_2)\pi(g)^{-1}\pi(g) \tag{$(\{-1, 1\}, \times)$ is abelian}\\
        &= \pi(t_2),
    \end{align*} meaning $\pi$ takes the same value $k$ on all transpositions.
    The set of transpositions $T$ generates $S_n$ so $\pi(T)$ generates
    $(\{-1, 1\}, \times)$ but $\pi(T) = \{k\}$ so $k = -1$. Thus,
    for $x = x_1 \cdots x_r$ a product of transpositions, 
    $\pi(x) = (-1)^r = \varepsilon(x)$ so $\pi = \varepsilon$.
    As such, $H = \Ker(\pi) = \Ker(\varepsilon) = A_n$.
    \\[\baselineskip]
    ($\Leftarrow$) By the homomorphism theorem, 
    $\Ima(\varepsilon) \cong S_n / \Ker(\varepsilon) = S_n / A_n$.
    So, \linebreak $[S_n : A_n] = |\{-1, 1\}| = 2$.
\end{proof}

\subsection{Alternating Groups generated by $3$-Cycles}

For $n$ in $\mb{N}$, $A_n$ is generated by its subset of $3$-cycles.

\begin{proof}
    Each element of $A_n$ is a product of an even number of transpositions,
    so a product of permutations of the form $(i, j)(k, l)$. It suffices
    to show that these permutations must be $3$-cycles.

    \paragraph{Case 1} If $\{i, j\} = \{k, l\}$, as $(i, j) = (j, i)$, 
    $(i, j)(k, l) = e$, a product of zero $3$-cycles.

    \paragraph{Case 2} If $|\{i, j\} \cap \{k, l\}| = 1$, we take $j = k$
    without loss of generality so: \begin{align*}
        (i, j)(k, l)
        &= (i, j)(j, l) \\
        &= (i, j, l),
    \end{align*} a $3$-cycle.

    \paragraph{Case 3} If $i$, $j$, $k$, and $l$ are all distinct then: \begin{align*}
        (i, j)(k, l)
        &= (i, j)(j, k)(j, k)(k, l) \\
        &= (i, j, k)(j, k, l),
    \end{align*} a product of two $3$-cycles.

\end{proof}
