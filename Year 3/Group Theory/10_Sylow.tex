\section{Sylow's Theorems}

\subsection{Cauchy's Theorem (6.1-2)} \label{6.1} \label{6.2}

For a finite group $G$ and a prime $p$ such that $p$ divides $|G|$,
$G$ contains an element of order $p$.

\begin{proof}
    We first prove the theorem for abelian groups, then for all groups.
    \\[\baselineskip]
    \textbf{Abelian Case} 
    Suppose $G$ is abelian.
    If $|G| = p$, then $G$ is cyclic with a generator of order $p$.
    So, we consider $|G| > p$ and proceed by induction on $|G|$. 
    We take $g$ in $G \setminus \{e\}$, if $p$ divides $|g|$, 
    we take $g^{\frac{|g|}{p}}$ and we are done.
    Otherwise, by Lagrange's theorem, $|G| = |g| \cdot [G : \ang{g}]$
    so $p$ divides $[G : \ang{g}]$. We have that $G / \ang{g}$ is an abelian
    group by (\ref{2.29}) and has order strictly less than $|G|$, so by induction, 
    it contains an element of order $p$, $h\ang{g}$.
    We write $n = |h|$, we have that: \begin{align*}
        (h\ang{g})^n = h^n\ang{g} = e\ang{g} = \ang{g},
    \end{align*} so $p$ divides $n$. Thus, $h^{\frac{n}{p}}$ has order
    $p$ in $G$ as required.
    \\[\baselineskip]
    \textbf{General Case} We remove our supposition that $G$ is abelian.
    As before, if $|G| = p$, then $G$ is cyclic with a generator of 
    order $p$. So, we consider $|G| > p$ and proceed by induction on $|G|$.
    If $p$ divides $|Z(G)|$, as $Z(G)$ is abelian, we are done by the first case. 
    Otherwise, we consider the class equation: \begin{align*}
        |G| = |Z(G)| + \sum_{C \in \mc{C}(G), \, |C| > 1} [G : C_G(g_C)].
    \end{align*} As $p$ divides $|G|$ but not $|Z(G)|$, there is some
    term of the summation that is not divisible by $p$. Thus, there exists
    $g$ in $G$ such that $g \in C$ where $C$ is a conjugacy class
    of size at least $2$ and $[G : C_G(g)]$ is not divisible by $p$.
    We have that Lagrange's Theorem implies that $|C_G(g)|$ is divisible
    by $p$ as: \begin{align*}
        |G| = |C_G(g)|[G : C_G(g)].
    \end{align*} Since $|C| \geq 2$, $g$ is not central in $G$, so
    $C_G(g) \neq G$. By induction, $C_G(g)$ contains an element of
    order $p$. Hence, $G$ does.
\end{proof}

\subsection{Order of $p$-groups (6.3)} \label{6.3}

For a prime $p$ and a finite group $G$, $G$ is a $p$-group if
and only if $|G| = p^m$ for some $m$ in $\mb{N}$.

\begin{proof}
    If $|G| = p^m$ for some $m$ in $\mb{N}$ then every element has
    order dividing $p^m$ by Lagrange's Theorem. As such, $G$ is a
    $p$-group. Conversely, if $|G|$ is divisible by some prime
    $q \neq p$, then Cauchy's Theorem implies that $G$ has
    an element of order $q$, which is not a power of $p$.
\end{proof}

\subsection{Sylow's First Theorem (6.4)} \label{6.4}

We consider a prime $p$ and a finite group $G$ with $|G| = p^rm$ 
for some $r$ in $\mb{N}_0$ and some $m$ in $\mb{N}$ such that $p$ 
does not divide $m$. We have that for every $k$ in $\mb{N}_0$,
there exists a subgroup of $G$ of order $p^k$ if and only if
$k \leq r$.

\begin{proof}
    We note that when $k > r$, $p^k$ cannot divide $p^rm$
    so, by Lagrange's Theorem, there's no subgroup
    of size $p^k$. Thus, we consider $k$ in $[r]_0$.
    The theorem is trivial when $G = \{e\}$, so we assume $|G| > 1$
    and proceed by induction on $|G|$.
    \\[\baselineskip]
    \textbf{Case 1} We suppose that $p$ divides $|Z(G)|$.
    Cauchy's Theorem implies that there is a central element
    $x$ of order $p$, so $\ang{x} \nsub Z(G) \csub G$ (by (\ref{2.13})). We consider
    $G / \ang{x}$ which has size $p^{r - 1}m$ so by induction
    has subgroups of order $p^k$ for $k$ in $[r - 1]_0$
    denoted by $H_0, \ldots, H_{r - 1}$ with each $i$
    in $[r - 1]_0$ yielding $|H_i| = p^i$.
    \\[\baselineskip]
    We take $\pi$ from $G$ to $G / \ang{x}$ to be the
    quotient homomorphism and note that by the Correspondence 
    Theorem, for all $i$ in $[r - 1]_0$, $\pi^{-1}(H_i) \leq G$ and so:
    \begin{align*}
        |\pi^{-1}(H_i)| = |\ang{x}| \cdot |H_i| = p|H_i| = p^{i + 1}.
    \end{align*} Thus, since we have the trivial subgroup of order
    $1$, we have subgroups of order $1, p, \ldots, p^r$ as
    required.
    \\[\baselineskip]
    \textbf{Case 2} We suppose that $p$ does not divide $|Z(G)|$.
    We take $\mc{C}$ to be the set of conjugacy classes in $G$
    and for each $c$ in $\mc{C}$, we pick an element $g_C$ in $C$
    and use the class equation: \begin{align*}
        |G| = |Z(G)| + \sum_{C \in \mc{C}, |C| \geq 2} [G : C_G(g_C)].
    \end{align*}
    Thus, there must be some $g$ not in $Z(G)$ such that
    $[G : C_G(g)]$ is not divisible by $p$ so: \begin{align*}
        \frac{|G|}{|C_G(g)|},
    \end{align*} is not divisible by $p$. However, since
    $|G|$ is divisible by $p^r$, $|C_G(g)|$ must be also.
    As $g$ is not in $Z(G)$, $C_G(g) \neq G$ so 
    $|C_G(g)| < |G|$. By induction, $C_G(g)$ contains
    subgroups of order $1, p, \ldots, p^r$ which are
    also subgroups of $G$.
\end{proof}

\subsection{Sylow Subgroups}

For a prime $p$ and a group $G$, a $p$-subgroup $H \leq G$
is a Sylow $p$-subgroup if it is not a subgroup of any other
$p$-subgroup of $G$. We write $\Syl_p(G)$ for the set of
these subgroups and $n_p(G)$ for the quantity of them.

\subsection{Closure of $p$-groups under Conjugacy (6.5)}  \label{6.5}

For a prime $p$ and a group $G$ with $H \leq G$ a $p$-group,
for every $g$ in $G$, $H^g$ is also a $p$-group.
If $H$ is a Sylow $p$-group, so is $H^g$. 

\begin{proof}
    As conjugacy is an automorphism, $H^g$ is another $p$-group.
    If $H$ is a Sylow $p$-subgroup and $H^g$ is not, then $H^g$ 
    must be a proper subgroup of some other $p$-group $K \leq G$.
    However, $K^g$ should be another $p$-subgroup but: \begin{align*}
        H = g^{-1}(gHg^{-1})g < g^{-1}Kg, \tag{\ref{2.9}}
    \end{align*} which contradicts the fact that $H$ is a
    Sylow $p$-group.
\end{proof}

\subsection{Sylow's Second Theorem (6.6)}  \label{6.6}

For a prime $p$ and a finite group $G$, the Sylow $p$-groups
of $G$ are all conjugate to each other.  Thus, 
the conjugation action of $G$ on the Sylow $p$-subgroups 
gives us that $|\!\Orb_G(P)| = n_p(G)$. So, by the Orbit-Stabiliser
Theorem: \begin{align*}
    |\!\Stab_G(P)| = \frac{|G|}{n_p(G)}.
\end{align*}

\begin{proof}
    We write $|G| = p^rm$ with $p$ not dividing $m$.
    By Sylow's First Theorem, we have that there exists
    a Sylow $p$-subgroup $P \leq G$ with $|P| = p^r$.
    We will show $P$ is conjugate to an arbitrary Sylow
    $p$-subgroup $H$.
    We take $H$ to act on $G / P$ by $\varphi(h)(gP) = (hg)P$
    and $\mc{O}$ to be the set of orbits of this action.
    By (\ref{5.2}), the orbits partition $G / P$ so 
    $m = |G / P| = [G : P] = \sum_{O \in \mc{O}} |O|$. But, $m$
    cannot be divisible by $p$ so there must be some orbit
    $O$ with $|O|$ not divisible by $p$.
    The Orbit-Stabiliser Theorem gives us that: \begin{align*}
        |H| = |O| \cdot |\!\Stab_H(x)|,
    \end{align*} for some $x$ in $G / P$, so $|O|$ divides
    $|H|$. Since $H$ is a $p$-group, $|O|$ must be a power
    of $p$. Thus, $|O| = 1$ and as such, the action of $H$ on 
    $G / P$ has a fixed point, for some $g$ in $G$ and for 
    all $h$ in $H$: \begin{align*}
        HgP = gP
        & \Longleftrightarrow g^{-1}HgP = P \\
        & \Longleftrightarrow g^{-1}Hg \subseteq P.
    \end{align*} By (\ref{6.5}), $g^{-1}Hg = P$.
\end{proof}

\subsection{Order of Sylow Subgroups (6.7)}  \label{6.7}

For a prime $p$ and a finite group $G$ with $|G| = p^rm$
where $r$ is in $\mb{N}_0$, $m$ is in $\mb{N}$, and 
$p$ doesn't divide $m$, every Sylow $p$-subgroup of $G$
has order $p^r$. 
\begin{proof}
    This is direct from Sylow's First and Second Theorem.
\end{proof}

\subsection{The Quantity of Sylow Subgroups (6.8)} \label{6.8}

For a finite group $G$ and $P \leq G$ a Sylow $p$-subgroup,
$n_p(G) = [G : N_G(P)]$. In particular, $P \nsub G$ if and
only if $P$ is the unique Sylow $p$-subgroup of $G$.

\begin{proof}
    By Sylow's Second Theorem: \begin{align*}
        n_p(G) &= |\{H \leq G : H \text{ is conjugate to } P\}| \\
        &= [G : N_G(P)]. \tag{\ref{5.5}}
    \end{align*} If $n_p(G) = 1$, then $P^g = P$ for all $g$ in $G$
    by Sylow's Second Theorem.
\end{proof}

\subsection{Sylow Subgroups of Abelian Groups (6.9)} \label{6.9}

For a finite abelian group $G$, $n_p(G) = 1$ for all primes $p$.

\begin{proof}
    As $G$ is abelian, $N_G(P) = G$ so we have that $n_p(G) = [G : N_G(P)] = 1$.
\end{proof}

\subsection{Fixed Point of Conjugation on Sylow Subgroups (6.11)} \label{6.11}

For a finite group $G$ and a Sylow $p$-subgroup $P$
where $P$ acts on $\Syl_p(G)$ by conjugation via $\varphi$,
we have that $\Fix_P(\Syl_p(G)) = \{P\}$.

\begin{proof}
    We know that $P$ is in $\Fix_P(\Syl_p(G))$ as $gPg^{-1} = P$
    for any $g$ in $P$. For $Q$ in $\Fix_P(\Syl_p(G))$,
    by definition, $gQg^{-1} = Q$ for all $g$ in $P$ so,
    $P \subseteq N_G(Q)$. As we know $Q \nsub N_G(Q)$, (\ref{2.23})
    shows that $PQ \leq G$, and by (\ref{2.24}), $|PQ|$ divides $|P||Q|$.
    But, as $P$ and $Q$ are $p$-groups, they must have an order
    that is a power of $p$. Thus, $|PQ|$ is also a power of $p$
    so $PQ$ is a $p$-group. However, since $P$ and $Q$ are both
    Sylow $p$-subgroups in $PQ$, $P = PQ = Q$, as required.
\end{proof}

\subsection{Sylow's Third Theorem (6.10)} \label{6.10}

For a prime $p$ and a finite group $G$ with
$|G| = p^rm$ for some prime $p$ that doesn't divide $m$, 
$n_p(G)$ divides $m$ and $n_p(G) \equiv 1 \bmod p$.

\begin{proof}
    We take $P$ to be a \Syls with $P$ acting on $\Syl_p(G)$
    by conjugation. By Lagrange's Theorem, $p$ divides $[P : \Stab_P(Q)]$
    for every $Q$ in $\Syl_p(G)$ that is not in $\Fix_P(G)$. So, we have that: \begin{align*}
        n_p(G) 
        &\equiv |\!\Fix_P(\Syl_p(G))| \bmod p \tag{\ref{5.3}} \\
        &\equiv 1. \tag{\ref{6.11}}
    \end{align*} By (\ref{6.8}) and Lagrange's Theorem, $n_p(G)$ must divide $|G|$
    and as $n_p(G) \equiv 1 \bmod p$, $n_p(G)$ is not divisible by $p$. Thus, 
    $n_p(G)$ divides $m$.
\end{proof}
