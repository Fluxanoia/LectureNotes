\section{The Type System}

\subsection{Type Assignment (8.1)}

A type assignment is a pair of a term $M$ and a type scheme $A$
written $M:A$. The term part is called the subject and the type
part the predicate.

\subsection{Type Environments (8.2)}

A type environment written $\Gamma$ is a finite set of
type assignments of the form $x : \forall \, \bar{a}.A$
which are consistent in the sense that multiple 
type assignments of the same subject will agree.
The subjects of $\Gamma$ is the set $\Dom(\Gamma)$.

\subsection{Type Judgement}

A type judgement is a triple consisting of a type environment $\Gamma$,
a term $M$, and a monotype $A$ written as $\Gamma \vdash M : A$.

\subsection{The Type System (8.3)}

The type system is defined by the following rules.
For $x : \forall \, \bar{a}.A$ in $\Gamma$ we have the type variable
rule: \begin{align*}
    \frac{}{\Gamma \vdash x : A[\bar{B}/\bar{a}]},
\end{align*} the rule of type application: \begin{align*}
    \frac{
        \Gamma \vdash M : B \to A \qquad \Gamma \vdash N : B
    }{
        \Gamma \vdash MN : A
    },
\end{align*} and for $x$ not in $\Dom(\Gamma)$, we have the rule
of type abstraction: \begin{align*}
    \frac{
        \Gamma \cup \{x : B\} \vdash M : A
    }{
        \Gamma \vdash \lambda x.M : B \to A
    }.
\end{align*} A proof tree justifying a type judgement is called a type
derivation.

\subsection{Typability (8.4, Lemma 8.1)}

We say a closed term $M$ is typable if there is some type $A$ such that
$\vdash M : A$. We have that $\lambda x.xx$ is untypable.

\subsection{The Inversion Theorem (Thm. 8.1)}

Suppose $\Gamma \vdash M : A$ is derivable, we have that: \begin{itemize}
    \item if $M$ is a variable $x$ then there is a type scheme
        $\forall \, \bar{a}.B$ in $\Gamma$ with $A = B[\bar{C}/\bar{a}]$ 
        for some monotypes $\bar{C}$,
    \item if $M$ is an application $PQ$ then there is a type $B$ such
        that $\Gamma \vdash P : B \to A$ and $\Gamma \vdash Q : B$,
    \item if $M$ is an abstraction $\lambda x.P$ then there are types
        $B$ and $C$ such that $A = B \to C$ and 
        $\Gamma \cup \{x : B\} \vdash P : C$.
\end{itemize}

\subsection{Properties of the System (Lemma 9.1-2, Thm. 9.1-2)}

For a term $M$, environment $\Gamma$ and type $S$, we
have the following properties.

\paragraph{Subterm Closure}
\leavevmode\newline
If $\Gamma \vdash M : S$ is derivable and $N$ is a subterm
of $M$, there is some $\Gamma'$ containing $\Gamma$ and
some $S'$ such that $\Gamma' \vdash N : S'$. 

\paragraph{Relevance ($1^{\text{st}}$)}
\leavevmode\newline
If $\Gamma \vdash M : S$ then $FV(M) \subseteq \Dom(\Gamma)$.

\paragraph{Relevance ($2^{\text{nd}}$)}
\leavevmode\newline
If $\Gamma \vdash M : S$ then $\{x : A \text{ where }
x : A \in \Gamma \text{ and } x \in FV(M) \} \vdash M : S$.

\paragraph{Weakening}
\leavevmode\newline
If $\Gamma \vdash M : S$ and $\Gamma \subseteq \Gamma'$ then
$\Gamma' \vdash M : S$.

\paragraph{Subject Reduction}
\leavevmode\newline
If $\Gamma \vdash M : A$ and 
$M \to_\beta N$ then $\Gamma \vdash N : A$.

\paragraph{Preservation under Substitution}
\leavevmode\newline
If $\Gamma \cup \{x : B\} \vdash M : A$ and $\Gamma \vdash N : B$
then $\Gamma \vdash M[N / x] : A$.

\paragraph{Subformula}
\leavevmode\newline
We suppose $\Gamma$ has predicates solely consisting of monotypes,
for some term $M$ in $\beta$-normal form with $\Gamma \vdash M : A$
we have that the derivation of this judgement is unique and all
of the types mentioned in the derivation are substrings of types 
mentioned in the conclusion.
