\section{Recursion}

\subsection{Fixed Points (5.2)}

We say that a term $N$ is a fixed point of another term $M$ if
$MN =_\beta N$.

\subsection{The First Recursion Theorem (Thm. 5.1)}

Every term posesses a fixed point.

\begin{proof}
    Let $M$ be a term. We define: \begin{align*}
        \mbf{Y} = \lambda f.(\lambda x.f(xx))(\lambda x.f(xx)),
    \end{align*} called Curry's Paradoxical Combinator.
    We will show that $M(\mbf{Y}M) =_\beta \mbf{Y}M$ (so
    $\mbf{Y}M$ is a fixed point of $M$): \begin{align*}
        \mbf{Y}M &\to_\beta (\lambda x.M(xx))(\lambda x.M(xx)) \\
        &\to_\beta M((\lambda x.M(xx))(\lambda x.M(xx))) \\
        \\
        M(\mbf{Y}M) &\to_\beta M((\lambda x.M(xx))(\lambda x.M(xx))) \\
        \\
        M(\mbf{Y}M) &=_\beta \mbf{Y}M,
    \end{align*} as required.
\end{proof}

\subsubsection{Solving Recursive Definitions}

We want to solve for some term $M$: \begin{align*}
    M x_1 \cdots x_n &=_\beta N[M / y] \\
    \Longleftrightarrow M &=_\beta \lambda x_1 \cdots x_n.N[M / y] \\
    \Longleftrightarrow M &=_\beta (\lambda y.
        \lambda x_1 \cdots x_n.N)M
\end{align*} so if we find a fixed point of $(\lambda y.
\lambda x_1 \cdots x_n.N)$ then we have an $M$ which satisfies
the equation. But, by the First Recursion Theorem, we can always
find such $M$.
Thus, we can just set $M = \mbf{Y}(\lambda y. \lambda x_1 
\cdots x_n.N)$.