\section{Induction}

We let $\Phi$ be some property of terms. We have that
if the following conditions are met then it follows that
$\Phi$ holds for all terms: \begin{enumerate}
    \item For all variables $x$, $\Phi(x)$ holds,
    \item For all terms $P$ and $Q$, if $\Phi(P)$ and
        $\Phi(Q)$ hold then $\Phi(PQ)$ holds,
    \item For all terms $P$ and variables $x$, if
        $\Phi(P)$ holds then $\Phi(\lambda x.P)$ holds.
\end{enumerate}

\subsection{Free Variables of Substitutions of Unbound Variables 
    (Lemma 6.1)}

For all $\lambda$-terms $M$ and $N$, if $x$ is not in $FV(M)$
then $FV(M[N / x]) = FV(M)$. This can be proven by induction
on the structure of $\lambda$-terms.

\subsection{The Induction Metaprinciple (Prin. 6.1)}

For some set $S$ with an inductive definition defined by
rules $R_1, \ldots, R_k$. The induction principle for proving
that for all $s$ in $S$, $\Phi(s)$ holds has $k$ clauses
corresponding to the rules of $S$.
\\[\baselineskip]
If a rule $R_i$ has $m$ premises and a side condition $\psi$:
\begin{align*}
    \psi \, \dfrac{s_1 \in S \cdots s_m \in S}{s \in S}, \, (R_i)
\end{align*} then the corresponding clause in the induction
principle requires that if 
\linebreak $\Phi(s_1), \ldots, \Phi(s_m), \Phi(\psi)$
hold then $\Phi(s)$ holds.
