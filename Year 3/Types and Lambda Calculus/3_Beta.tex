\section{Beta}

A term of the form $(\lambda x.M)N$ is called a $\beta$-redex
and $M[N/x]$ is the contraction of the redex.

\subsection{Beta Reductions}

The one-step $\beta$-reduction relation, denoted by 
$\to_\beta$, is inductively defined by the redex rule: 
\begin{align*}
    \dfrac{}{
        (\lambda x.M)N \to_\beta M[N/x]
    },
\end{align*} the left and right application rules: \begin{align*}
    \dfrac{M \to_\beta M'}{MN \to_\beta M'N}
    \qquad
    \dfrac{N \to_\beta N'}{MN \to_\beta MN'},
\end{align*} and the abstraction rule:
\begin{align*}
    \dfrac{M \to_\beta N}{\lambda x.M \to_\beta \lambda x.N}.
\end{align*} A term $M$ is said to be in $\beta$-normal form 
if there is no term $N$ such that $M \to_\beta N$.
\\[\baselineskip]
In general, $\beta$-reductions are sequences of consecutive
one-step $\beta$-reductions: \begin{align*}
    M_0 \to_\beta M_1 \to_\beta \cdots \to_\beta M_k,
\end{align*} for some $k$ in $\mb{N}_0$. We say that
$M_0$ $\beta$-reduces to $M_k$, denoted by 
$M_0 \bred M_k$.
\\[\baselineskip]
If $M \bred N$, we say that $N$ is a
reduct of $M$, and is a proper reduct if $N \neq M$.
If we can choose some $\beta$-normal $N$ such that
$M \bred N$ then $M$ is normalisable.
If a term admits no infinite $\beta$-reductions then we say
that it is strongly normalisable. 

\subsection{Standard Combinators}

We have some interesting programs described below:
\begin{center}
    \begin{tabular}{ c c l p{1cm} c }
        $\mathbf{I}$ & $=$ & $\lambda x.x$ 
        && $\mathbf{I} M \bred M$ \\
        $\mathbf{K}$ & $=$ & $\lambda xy.x$ 
        && $\mathbf{K} MN \bred M$ \\
        $\mathbf{S}$ & $=$ & $\lambda xyz.xz(yz)$ 
        && $\mathbf{S} MNP \bred MP(NP)$ \\
        $\omega$ & $=$ & $\lambda x.xx$ 
        && $\omega M \bred MM$ \\
        $\Omega$ & $=$ & $\omega\omega$ 
        && $\Omega \to_\beta \Omega$ \\
        $\Theta$ & $=$ & $(\lambda xy.y(xxy))(\lambda xy.y(xxy))$ 
        && $\Theta M \bred M(\Theta M)$
    \end{tabular}
\end{center}

\subsection{Confluence of Beta}

For a term $M$, if $M \bred P$ and
$M \bred Q$ then there exists some
term $N$ such that $P \bred N$ and
$Q \bred N$.   

\subsection{Beta Convertibility}

For $M$, $N$ terms, if $M$ and $N$ have a common reduct then
we say that $M$ and $N$ are $\beta$-convertible, denoted by
$M =_\beta N$.