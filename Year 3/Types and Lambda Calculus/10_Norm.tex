\section{Normalisation}

\subsection{Strong Normalisation}

If $\Gamma \vdash M : A$ then $M$ is strongly normalising.

\subsection{The Inhabitation Problem (13.1)}

The inhabitation problem is the problem of, given a type $A$, determining
whether there is a closed term $M$ such that $\vdash M : A$.

\subsection{Curry-Howard Theorem (Thm. 13.1)}

For proofs in the implicational fragment of propositional
logic, the following is true: \begin{itemize}
    \item From a constructive proof of $A$ from the initial assumptions
        $\Gamma$, we can extract a term $M$ such that $\Gamma \vdash M : A$,
    \item From each term $M$ with $\Gamma \vdash M : A$, we can extract
        a constructive proof that $A$ follows from the assumptions $\Gamma$.
\end{itemize}

\subsection{Proof from Truth}

A proof of $A$ from the starting assumptions $\Gamma$ is a certificate
guaranteeing that $A$ is true if $\Gamma$ is true.

\subsection{Truth from Proof}

A proof of $A$ from the starting assumptions $\Gamma$ is a method
of constructing evidence of $A$ from evidence of $\Gamma$.

\subsection{The BHK Interpretation of Logic}

The BHK interpretation of logic consists of the following rules:
\begin{itemize}
    \item There can be no evidence for the truth from false,
    \item No evidence is needed for the truth of true,
    \item Evidence for $A$ and $B$ is a pair consisting of evidence
        for $A$ and evidence for $B$,
    \item Evidence for $A$ or $B$ is evidence of $A$ or evidence of $B$,
    \item Evidence of $A \Rightarrow B$ is a procedure for transforming
        evidence of $A$ to evidence of $B$,
    \item Evidence of $\neg A$ is a procedure for transforming evidence
        of $A$ to false,
    \item Evidence of $\forall \, x \in X, A$ is a procedure for transforming
        any $y$ and evidence of $y$ being in $X$ into evidence of $A[y / x]$,
    \item Evidence of $\exists \, x \in X, A$ is a triple consisting of a
        $y$, evidence of $y$ being in $X$, and evidence for $A[y / x]$.
\end{itemize}
