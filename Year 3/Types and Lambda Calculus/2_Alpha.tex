\section{Alpha Equivalence}

\subsection{Free Variables}

We define the function $FV : \Lambda \to \mathcal{P}(\mathbb{V})$,
which returns the set of variables contained within a term $M$ that
are not bound. We define it recursively on the structure of terms: 
\begin{align*}
    FV(x) &= \{x\}, \\
    FV(MN) &= FV(M) \cup FV(N), \\
    FV(\lambda x.M) &= FV(M) \backslash \{x\}.
\end{align*} If a term has no free variables we say it is closed,
and if a term has at least one free variable then we say it is open.
The set of all closed terms is denoted by $\Lambda^0$.

\subsection{Substitution}

We define 'capture-avoiding' substitution of a term $M$ for a variable $x$
recursively on the structure of terms: \begin{center}
    \begin{tabular}{ r c l l }
        $y[M/x]$             & $=$ & $y$                        & if $y \neq x$, \\
        $y[M/x]$             & $=$ & $M$                        & if $y = x$, \\
        $(PQ)[M/x]$          & $=$ & $P[M/x] \, \cup \, Q[M/x]$, & \\
        $(\lambda y.P)[M/x]$ & $=$ & $\lambda y.P$              & if $y = x$, \\
        $(\lambda y.P)[M/x]$ & $=$ & $\lambda y.P[M/x]$         & if $y \neq x$ and $y \notin FV(M)$.
    \end{tabular} 
\end{center} On the final case, we stipulate that $y$ cannot be a free variable of
$M$ because otherwise free variables in the substitution could be captured by
the lambda.

\subsection{Alpha Equivalence}

Suppose we have a term $\lambda x.M$ and $y$ in $\mathbb{V} \, \backslash \, FV(M)$,
then substituting $y$ for $x$ is a change of bound variable name. If two terms
can be made identical through changes of bound variable name, they are
$\alpha$-equivalent. The set of $\lambda$-terms is the set $\Lambda$ under
$\alpha$-equivalence.
\\[\baselineskip]
This equivalence is much more useful to us than string comparision, so for the
remainder of the notes we will always be referring to $\lambda$-terms as terms.

\subsection{The Variable Convention}

For $M_1, \ldots, M_k$ terms occuring in the same scope, we assume each term
has distinct bound variables. We can make this assumption as otherwise, we
can use changes of bound variable names to make it so.
