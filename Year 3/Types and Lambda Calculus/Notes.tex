\documentclass[a4paper, 12pt, twoside]{article}
\usepackage[left = 3cm, right = 3cm]{geometry}
\usepackage[english]{babel}
\usepackage[utf8]{inputenc}
\usepackage{mathtools}
\usepackage{amssymb}
\usepackage{amsmath}
\usepackage{amsthm}
\usepackage{multicol}
\usepackage{multirow}
\usepackage{pgfplots}
\usepackage{pgfplotstable}
\usepackage{listings}
\usepackage{xcolor}
\usepackage{color}
\usepackage{graphicx}
\usepackage{arydshln}
\usepackage{pifont}

\makeatletter
\newcommand\Pimathsymbol[3][\mathord]{%
  #1{\@Pimathsymbol{#2}{#3}}}
\def\@Pimathsymbol#1#2{\mathchoice
  {\@Pim@thsymbol{#1}{#2}\tf@size}
  {\@Pim@thsymbol{#1}{#2}\tf@size}
  {\@Pim@thsymbol{#1}{#2}\sf@size}
  {\@Pim@thsymbol{#1}{#2}\ssf@size}}
\def\@Pim@thsymbol#1#2#3{%
  \mbox{\fontsize{#3}{#3}\Pisymbol{#1}{#2}}}
\makeatother
\input{utxmia.fd}
\DeclareFontShape{U}{txmia}{m}{n}{<->ssub * txmia/m/it}{}
\DeclareFontShape{U}{txmia}{bx}{n}{<->ssub * txmia/bx/it}{}
\newcommand{\pilambdaup}{\Pimathsymbol[\mathord]{txmia}{21}}
\let\oldlambda\lambda
\let\lambda\pilambdaup


\pgfplotsset{compat=1.5.1}

\lstset{frame=none,
  language=Haskell,
  aboveskip=3mm,
  belowskip=3mm,
  showstringspaces=false,
  columns=flexible,
  basicstyle={\small\ttfamily},
  numbers=none,
  numberstyle=\tiny\color{gray},
  keywordstyle=\color{blue},
  commentstyle=\color{gray},
  stringstyle=\color{orange},
  breaklines=true,
  breakatwhitespace=true,
  tabsize=2
}

\newcommand\doubleplus{+\kern-1.3ex+\kern0.8ex}
\newcommand\mdoubleplus{\ensuremath{\mathbin{+\mkern-10mu+}}}

\newcommand*{\bfrac}[2]{\genfrac{}{}{0pt}{}{#1}{#2}}

\begin{document}

\title{Types and Lambda Calculus Notes}
\date{}
\author{by Tyler Wright \\
  \\
  github.com/Fluxanoia $\qquad$ fluxanoia.co.uk
}
\maketitle

\vfill

\textit{These notes are not necessarily correct,
consistent, representative of the course as it stands today or, 
rigorous. Any result of the above is not the author's fault.}

% \addtocounter{section}{-1}
% \section{Notation}

We commonly deal with the following concepts in 
Language Engineering
which I will abbreviate as follows for brevity:
\begin{center}
    \begin{tabular}{ | r | c | }
        \hline
        Term & Notation \\
        \hline \hline
        \hline
    \end{tabular}
\end{center}

\newpage

\tableofcontents

\section{Terms (1.1)}

We suppose that we have a countably infinite set of variables $\mathbb{V}$
(we usually refer to elements of this set as $x$, $y$, $z$, etc.),
from this we define the alphabet of lambda calculus 
$\mathbb{V} + \{\lambda, ., (, )\}$. 
The set of terms of lambda calculus $\Lambda$ is defined inductively 
for some $x$ in $\mathbb{V}$ by the variable axiom: \begin{align*}
    \dfrac{}{x \in \Lambda},
\end{align*} the application axiom: \begin{align*}
    \dfrac{M \in \Lambda \qquad N \in \Lambda}{(MN) \in \Lambda},
\end{align*} and the abstraction axiom: \begin{align*}
    \dfrac{M \in \Lambda}{(\lambda x.M) \in \Lambda}.
\end{align*}

\subsection{Subterms}

Subterms of a term $M$ are substrings of $M$ that are themselves terms and
not captured by a $\lambda$ (directly preceeded by).

\subsection{Syntactical Conventions}

Parentheses allow our lambda calculus to be unambigious, but for the sake
of simplicity, we will construct conventions that will allow us to retain
unique meaning with less parentheses: \begin{itemize}
    \item Omit outermost parentheses,
    \item Terms associate to the left, $(MNP)$ parses as $((MN)P)$,
    \item Bodies of abstractions end at parentheses, $(\lambda x.MN)$
        parses as $(\lambda x.(MN))$,
    \item Group repeated abstractions, $(\lambda xy.M)$ 
        parses as $(\lambda x.(\lambda y.M))$.
\end{itemize}

\end{document}