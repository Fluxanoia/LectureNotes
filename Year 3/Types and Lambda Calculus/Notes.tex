\documentclass[a4paper, 12pt, twoside]{article}
\usepackage[left = 3cm, right = 3cm]{geometry}
\usepackage[english]{babel}
\usepackage[utf8]{inputenc}
\usepackage{ragged2e}
\usepackage{mathtools}
\usepackage{amssymb}
\usepackage{amsmath}
\usepackage{amsthm}
\usepackage{multicol}
\usepackage{multirow}
\usepackage{array}
\usepackage{listings}
\usepackage{xcolor}
\usepackage{color}
\usepackage{graphicx}
\usepackage{arydshln}
\usepackage{pifont}
\usepackage{fancyhdr}
\usepackage{hhline}

% Math operators

\DeclareMathOperator{\ID}{id}
\DeclareMathOperator{\Max}{max}
\DeclareMathOperator{\Min}{min}
\DeclareMathOperator{\sgn}{sgn}
\DeclareMathOperator{\Deg}{deg}
\DeclareMathOperator{\Char}{char}
\DeclareMathOperator{\Span}{span}
\DeclareMathOperator{\Dim}{dim}
\DeclareMathOperator{\Ker}{Ker}
\DeclareMathOperator{\Ima}{Im}
\DeclareMathOperator{\Rank}{rank}
\DeclareMathOperator{\Null}{nullity}
\DeclareMathOperator{\End}{End}
\DeclareMathOperator{\Sym}{Sym}

% Empty Set Symbol

\let\oldemptyset\emptyset
\let\emptyset\varnothing

% Nicer Lambda

\makeatletter
\newcommand\Pimathsymbol[3][\mathord]{%
  #1{\@Pimathsymbol{#2}{#3}}}
\def\@Pimathsymbol#1#2{\mathchoice
  {\@Pim@thsymbol{#1}{#2}\tf@size}
  {\@Pim@thsymbol{#1}{#2}\tf@size}
  {\@Pim@thsymbol{#1}{#2}\sf@size}
  {\@Pim@thsymbol{#1}{#2}\ssf@size}}
\def\@Pim@thsymbol#1#2#3{%
  \mbox{\fontsize{#3}{#3}\Pisymbol{#1}{#2}}}
\makeatother
\input{utxmia.fd}
\DeclareFontShape{U}{txmia}{m}{n}{<->ssub * txmia/m/it}{}
\DeclareFontShape{U}{txmia}{bx}{n}{<->ssub * txmia/bx/it}{}
\newcommand{\pilambdaup}{\Pimathsymbol[\mathord]{txmia}{21}}
\let\oldlambda\lambda
\let\lambda\pilambdaup

% lstlisting config

\lstset{frame=none,
  language=Haskell,
  aboveskip=3mm,
  belowskip=3mm,
  showstringspaces=false,
  columns=flexible,
  basicstyle={\small\ttfamily},
  numbers=none,
  numberstyle=\tiny\color{gray},
  keywordstyle=\color{blue},
  commentstyle=\color{gray},
  stringstyle=\color{orange},
  breaklines=true,
  breakatwhitespace=true,
  tabsize=2
}

% Double plus symbol

\newcommand\doubleplus{+\kern-1.3ex+\kern0.8ex}
\newcommand\mdoubleplus{\ensuremath{\mathbin{+\mkern-10mu+}}}

% Fraction without line

\newcommand*{\bfrac}[2]{\genfrac{}{}{0pt}{}{#1}{#2}}

\begin{document}

\title{Types and Lambda Calculus Notes}
\date{}
\author{by Tyler Wright \\
  \\
  github.com/Fluxanoia $\qquad$ fluxanoia.co.uk
}
\maketitle

\vfill

\textit{These notes are not necessarily correct,
consistent, representative of the course as it stands today or, 
rigorous. Any result of the above is not the author's fault.}

% \addtocounter{section}{-1}
% \section{Notation}

We commonly deal with the following concepts in 
Language Engineering
which I will abbreviate as follows for brevity:
\begin{center}
    \begin{tabular}{ | r | c | }
        \hline
        Term & Notation \\
        \hline \hline
        \hline
    \end{tabular}
\end{center}

\newpage

\tableofcontents

\section{Terms (1.1)}

We suppose that we have a countably infinite set of variables $\mathbb{V}$
(we usually refer to elements of this set as $x$, $y$, $z$, etc.),
from this we define the alphabet of lambda calculus 
$\mathbb{V} + \{\lambda, ., (, )\}$. 
The set of terms of lambda calculus $\Lambda$ is defined inductively 
for some $x$ in $\mathbb{V}$ by the variable axiom: \begin{align*}
    \dfrac{}{x \in \Lambda},
\end{align*} the application axiom: \begin{align*}
    \dfrac{M \in \Lambda \qquad N \in \Lambda}{(MN) \in \Lambda},
\end{align*} and the abstraction axiom: \begin{align*}
    \dfrac{M \in \Lambda}{(\lambda x.M) \in \Lambda}.
\end{align*}

\subsection{Subterms}

Subterms of a term $M$ are substrings of $M$ that are themselves terms and
not captured by a $\lambda$ (directly preceeded by).

\subsection{Syntactical Conventions}

Parentheses allow our lambda calculus to be unambigious, but for the sake
of simplicity, we will construct conventions that will allow us to retain
unique meaning with less parentheses: \begin{itemize}
    \item Omit outermost parentheses,
    \item Terms associate to the left, $(MNP)$ parses as $((MN)P)$,
    \item Bodies of abstractions end at parentheses, $(\lambda x.MN)$
        parses as $(\lambda x.(MN))$,
    \item Group repeated abstractions, $(\lambda xy.M)$ 
        parses as $(\lambda x.(\lambda y.M))$.
\end{itemize}
\section{Alpha}

\subsection{Free Variables (2.1)}

We define the function $FV : \Lambda \to \mathcal{P}(\mathbb{V})$,
which returns the set of variables contained within a term $M$ that
are not bound. We define it recursively on the structure of terms: 
\begin{align*}
    FV(x) &= \{x\}, \\
    FV(MN) &= FV(M) \cup FV(N), \\
    FV(\lambda x.M) &= FV(M) \setminus \{x\}.
\end{align*} If a term has no free variables we say it is closed,
and if a term has at least one free variable then we say it is open.
The set of all closed terms is denoted by $\Lambda^0$.

\subsection{Substitution (2.2)}

We define 'capture-avoiding' substitution of a term $M$ for a variable $x$
recursively on the structure of terms: \begin{center}
    \begin{tabular}{ r c l l }
        $y[M/x]$             & $=$ & $y$                        & if $y \neq x$, \\
        $y[M/x]$             & $=$ & $M$                        & if $y = x$, \\
        $(PQ)[M/x]$          & $=$ & $P[M/x] Q[M/x]$, & \\
        $(\lambda y.P)[M/x]$ & $=$ & $\lambda y.P$              & if $y = x$, \\
        $(\lambda y.P)[M/x]$ & $=$ & $\lambda y.P[M/x]$         & if $y \neq x$ and $y \notin FV(M)$.
    \end{tabular} 
\end{center} On the final case, we stipulate that $y$ cannot be a free variable of
$M$ because otherwise free variables in the substitution could be captured by
the lambda.

\subsection{Alpha Equivalence (2.3)}

Suppose we have a term $\lambda x.M$ and $y$ in $\mathbb{V} \, \setminus \, FV(M)$,
then substituting $y$ for $x$ is a change of bound variable name. If two terms
can be made identical through changes of bound variable name, they are
$\alpha$-equivalent. The set of $\lambda$-terms is the set $\Lambda$ under
$\alpha$-equivalence.
\\[\baselineskip]
This equivalence is much more useful to us than string comparision, so for the
remainder of the notes we will always be referring to $\lambda$-terms as terms.

\subsection{The Variable Convention}

For $M_1, \ldots, M_k$ terms occuring in the same scope, we assume each term
has distinct bound variables. We can make this assumption as otherwise, we
can use changes of bound variable names to make it so.


\end{document}