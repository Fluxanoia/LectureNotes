\section{The Axioms of Lambda Calculus}

We suppose that we have a countably infinite set of variables $\mathbb{V}$
(we usually refer to elements of this set as $x$, $y$, $z$, etc.),
from this we define the alphabet of lambda calculus 
$\mathbb{V} + \{\lambda, ., (, )\}$. 
\\[\baselineskip]
The set of terms of lambda calculus $\Lambda$ is defined inductively 
by the axioms. A string is a term if and only if there is a proof tree using 
the following axioms that concludes it is a term.

\subsection{Axiom of Variables}

For some $x$ in $\mathbb{V}$, we have the axiom: \begin{align*}
    \dfrac{}{x \in \Lambda},
\end{align*} which says that each variable is a term.

\subsection{Axiom of Application}

We have the axiom: \begin{align*}
    \dfrac{M \in \Lambda \qquad N \in \Lambda}{(MN) \in \Lambda},
\end{align*} which says that for some terms $M$ and $N$, $(MN)$ is also
a term.

\subsection{Axiom of Abstraction}

For some $x$ in $\mathbb{V}$, we have the axiom: \begin{align*}
    \dfrac{M \in \Lambda}{(\lambda x.M) \in \Lambda},
\end{align*} which says that for some term $M$, $(\lambda x.M)$ is also
a term.

\subsection{Subterms}

Subterms of a term $M$ are substrings of $M$ that are themselves terms and
not captured by a $\lambda$ (directly preceeded by).

\subsection{Syntactical Conventions}

Parentheses allow our lambda calculus to be unambigious, but for the sake
of simplicity, we will construct conventions that will allow us to retain
unique meaning with less parentheses: \begin{itemize}
    \item Omit outermost parentheses,
    \item Terms associate to the left, $(MNP)$ parses as $((MN)P)$,
    \item Bodies of abstractions end at parentheses, $(\lambda x.MN)$
        parses as $(\lambda x.(MN))$,
    \item Group repeated abstractions, $(\lambda xy.M)$ 
        parses as $(\lambda x.(\lambda y.M))$.
\end{itemize}