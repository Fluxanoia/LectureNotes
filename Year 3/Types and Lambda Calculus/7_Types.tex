\section{Types}

\subsection{Monotypes}

We assume a countable set of type variables $\mb{A}$ (usually
denoted by $a$, $b$, $c$, etc.). The monotypes $\mb{T}$ are a
set of strings defined inductively by the type variable rule:
\begin{align*}
    \frac{}{a \in \mb{T}},
\end{align*} for some $a \in \mb{A}$ and the arrow rule: \begin{align*}
    \frac{A \in T \qquad B \in T}{(A \to B) \in \mb{T}}.
\end{align*}

\subsection{Type Schemes}

Types schemes are pairs consisting of a finite set of type
variables $a_1, \ldots, a_m$ and a monotype $A$ which we
write as $\forall \, a_1 \cdots a_m.A$.

\subsection{Free Type Variables}

We define the set of free type variables for a type scheme
$\forall \, \bar{a}.A$, denoted by
\linebreak $FTV(\forall \, \bar{a}.A)$,
recursively by the following rules: \begin{align*}
    FTV(a) &= \{a\}, \\
    FTV(A \to B) &= FTV(A) \cup FTV(B), \\
    FTV(\forall \, a_1 \cdots a_m.A) &= FTV(A) \backslash \{a_1, \ldots, a_m\}.
\end{align*} We consider type schemes that only differ by choice
of bound variable names to be equivalent.

\subsection{Type Substitution}

A type substitution is a total map $\sigma, \tau, \theta$ 
from $\mb{A}$ to 
$\mb{T}$ with the property that $\sigma(a) \neq a$ for only
finitely many $a \in \mb{A}$. We define the map as follows:
\begin{align*}
    a\sigma &= \sigma(a) \\
    (A \to B)\sigma &= A\sigma \to B\sigma.
\end{align*}

\subsubsection{Composition of Substitutions}

We write $\sigma_1\sigma_2$ for the substitution obtained by
composing $\sigma_2$ after $\sigma_1$, defined as by
$(\sigma_1\sigma_2)(a) = (\sigma_1(a))\sigma_2$.