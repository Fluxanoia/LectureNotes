\section{Type Constraints (10.1)}

A type constraint is a pair of monotypes $(A, B)$ written
suggestively as $A \teq B$.

\subsection{Constraint Generation}

We create a constraint generating function, which takes in an
environment and a term and returns a pair consisting of a type
variable and a set of constraints: \begin{align*}
    &\text{CGen}(\Gamma, x) = \\
    &\qquad \text{let } a \text{ be fresh} \\
    &\qquad \text{let } \forall \, a_1, \ldots, a_k.A = \Gamma(x) \\
    &\qquad \text{let } b_1, \ldots, b_k \text{ be fresh} \\
    &\qquad (a, \{a \teq A[b_1/a_1, \ldots, b_k/a_k]\}) \\
    \\
    &\text{CGen}(\Gamma, MN) = \\
    &\qquad \text{let } a \text{ be fresh} \\
    &\qquad \text{let } (b, \mc{C}_1) = \text{CGen}(\Gamma, M) \\
    &\qquad \text{let } (c, \mc{C}_2) = \text{CGen}(\Gamma, N) \\
    &\qquad (a, \mc{C}_1 \cup \mc{C}_2 \cup \{b \teq c \to a\}) \\
    \\
    &\text{CGen}(\Gamma, \lambda x.M) = \\
    &\qquad \text{let } a \text{ be fresh} \\
    &\qquad \text{let } b \text{ be fresh} \\
    &\qquad \text{let } (c, \mc{C}) = \text{CGen}(\Gamma \cup \{x : b\}, M) \\
    &\qquad (a, \mc{C} \cup \{a \teq b \to c\})
\end{align*}

\subsection{Unifier (11.1, Lemma 11.1)}

A unifier to a finite set of type constraints
$\{A_1 \teq B_1, \ldots, A_n \teq B_n\}$ is a type substitution $\sigma$
such that for all $i$ in $[n]$, $A_i\sigma = B_i\sigma$.
If CGen$(\Gamma, M) = (a, \mc{C})$ then 
$\sigma$ is a unifier for $\mc{C}$ if and only if 
$\Gamma\sigma \vdash M : \sigma(a)$ is derivable.

\subsubsection{Most General Unifier (11.4)}

For a set of type constraints $\mc{C}$, a unifier of $\mc{C}$, $\sigma$, 
is the most general unifier if every unifier $\sigma^*$ of $\mc{C}$ is 
of the form $\sigma\sigma'$ for some $\sigma'$.

\subsection{Constraint Solved Form (11.2)}

A set of type constraints $\mc{C} = \{a_1 \teq A_1, \ldots, a_m \teq A_m\}$
is in solved form if each of the $a_i$ are distinct and do not occur
in any $A_j$. If $\mc{C}$ is in solved form, we define 
$[[\mc{C}]] = [A_1/a_1, \ldots, A_m/a_m]$.

\subsection{Robinson's Algorithm (11.3, Thm. 11.2, Cor. 11.1)}

Robinson'a algorithm is the repeated and exhaustive application of the 
following rules, returning the most general unifier from
a set of solvable constraints: \begin{align*}
    \{A \teq A\} \uplus \mc{C} 
    &\mapsto \mc{C}, 
    \\
    \{A_1 \to A_2 \teq B_1 \to B_2\} \uplus \mc{C} 
    &\mapsto \{A_1 \teq B_1, A_2 \teq B_2\} \uplus \mc{C}, 
    \\
    \{A \teq a\} \uplus \mc{C} 
    &\mapsto \{a \teq A\} \uplus \mc{C}, \tag{$A \notin \mb{A}$} 
    \\
    \{a \teq A\} \uplus \mc{C} 
    &\mapsto \{a \teq A\} \uplus \mc{C}[A / a], \tag{$a \notin FTV(A)$} 
\end{align*} Thus, every solvable set of constraints has a most general solution,
as this algorithm will return it.

\subsection{Hindley-Milner Type Inference (11.5)}

For an input, closed term $M$, we perform the following steps: \begin{enumerate}
    \item Generate the constraints $\mc{C}$ and type variable $a$ using
        CGen$(\emptyset, M)$,
    \item Solve $\mc{C}$ using Robinson's algorithm to obtain the most general
        unifier or deduce insolvability,
    \item If $\mc{C}$ has no solution, $M$ is untypable. Otherwise, we
        return $\sigma(a)$.
\end{enumerate}

\subsection{Principle Type Scheme Theorem (Thm. 11.3)}

If a closed term $M$ is typable, then Hindley-Milner type inference
returns a type $A$ that is principal meaning:
\begin{itemize}
    \item $\vdash M : A$ is derivable,
    \item If $\vdash M : B$ is derivable, there exists monotypes 
        $C_1, \ldots, C_k$ such that $B = A[C_1/a_1, \ldots, C_k/a_k]$.
\end{itemize}
