\section{Lighting}

\subsection{Proximity Lighting}

This is simply reducing the intensity of light proportionally to the
square of the distance of the surface from the light source. This gives
realistic brightness, as this is in accordance with the laws of physics.
However, different scales may be used to achieve certain stylistic goals,
which may or may not agree with the laws, but the effect is similar.

\subsection{Angle of Incidence Lighting}

This is calculating the angle between the surface normal and the vector
to the light source and varying the light intensity accordingly. If the ray
to light is parallel with the normal, light intensity should be maximal, but
if it is perpendicular with the normal, or through the surface, the
light intensity should be minimal.
\\[\baselineskip]
The normal can be calculated via a cross-product of the triangle edges
(but take care of the order you calculate the product), and
the cosine of the angle can be calculated with the well-known dot product
formula: \begin{align*}
    \cos(\theta) = \frac{a \cdot b}{|a| \cdot |b|}.
\end{align*}

\subsection{Specular Lighting}

We calculate the angle of reflection of light off of a surface and, similarly
to angle of incidence lighting, vary the light intensity based on the angle
between the reflected ray and the vector to the camera. If the light is reflecting
directly into the camera, the intensity should be maximal, and minimal if the
reflected ray is pointing away from the camera. We can raise the cosine of the
angle between these vectors to a power called the 'specular exponent'. The greater
the value, the shinier the surface appears.
\\[\baselineskip]
The reflected ray $r$ can be calculated with the incident ray $i$ and the
normal $n$ via: \begin{align*}
    r = i - 2(i \cdot n)n.
\end{align*}

\newpage

\subsection{Ambient Lighting}

Rays of light in real life reflect multiple times, yet we are only simulating up
to one reflection at the moment. This causes our scene to appear dimmer in
areas which light may take multiple reflections to reach. Since this would
be very computationally expensive to calculate, we emulate the effect with 
ambient lighting. We either set a minimum threshold for light intensity or
add a flat amount of brightness when we calculate the brightness of a pixel.

\subsection{Gouraud Shading}

Previously, we would use the vertex normal of the whole face to calculate the 
lighting effects on the whole face. This can lead to blocky lighting
as lighting effects do not smoothly transition across the face.
Gouraud shading is the process of, instead, generating normals for the vertices
of the face (by taking the mean of the normals of the faces the vertex is
contained in) and using those to generate the lighting effects at the vertices.
These lighting effects can then be interpolated to the point being rendered. 

\subsection{Phong Shading}

Phong shading is similar to Gouraud shading, except we instead interpolate the
vertex normals, so we can calculate the vertex normal at any point on a face.
We then use this to calculate lighting effects directly, giving a more
convincing result than Gouraud shading (at the expense of computational cost).
