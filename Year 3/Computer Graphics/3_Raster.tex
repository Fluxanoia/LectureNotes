\section{Rasterising}

Rasterising is about converting world-space images into screen-space images to draw
objects to the screen.

\subsection{Interpolation}

We can interpolate in one dimension between two floating point values
$x$ and $y$ in $s \geq 2$ steps by defining $v_1, \ldots, v_s$: \begin{align*}
    v_i &= x + (i - 1) \left(\frac{y - z}{s - 1}\right),
\end{align*} we will write the sequence $(v_i)_{i \in [s]}$ as $I(x, y, s)$ as
the interpolation from $x$ to $y$ in $s$ steps.
\\[\baselineskip]
We can extrapolate this to higher dimensions. For $d > 1$, we have that the interpolation of
$d$-dimensional vectors $x$ and $y$ in $s$ steps is $v_1, \ldots, v_s$ defined as: \begin{align*}
    x &= (x_1, \,\cdots, x_d), \\
    y &= (y_1, \,\cdots, y_d), \\
    w_{ab} &= \text{the $b^\text{th}$ element of } I(x_a, y_b, s), \\
    v_i &= (w_{1i}, \,\cdots, w_{di}).
\end{align*}

\subsection{Lines}

We can draw basic straight lines to the screen by using two one-dimensional
interpolations. For $(x_1, y_1)$ and $(x_2, y_2)$ the points we want to draw
between, we interpolate from $x_1$ to $x_2$ and $y_1$ to $y_2$ in $s$ steps
where: \begin{align*}
    s = \max(\{ 2, |x_2 - x_1|, |y_2 - y_1| \}),
\end{align*} and fill in the corresponding pixel for each $i$ in $[s]$.

\subsection{Triangles}

Triangles are important building blocks of computer graphics as we can use them to build
any polygon. Stroking triangles is simple, but filling them is a more complex task, which
is where we rasterise the triangle.
\newpage
\noindent
We want to fill the triangle in 'rakes', top to bottom, left to right. We first take
the vertices and order them ascending based on their $y$ coordinate on the screen
to get $v_1$, $v_2$, and $v_3$ with $v_i = (x_i, y_i)$ for each $i$ in $[3]$. 
We calculate the point $v'$ along $(v_1, v_3)$ with $y$ coordinate equal to that 
of $v_2$ by setting: \begin{align*}
    v' &= v_1 + \left( \frac{y_2 - y_1}{y_3 - y_1} \right)(v_3 - v_1).
\end{align*} We can then calculate $I(x_1, x', y_2 - y_1)$ and $I(x_1, x_2, y_2 - y_1)$,
and draw the horizontal lines between them at each $i$ in $[y_2 - y_1]$. We similarly
calculate $I(x', x_3, y_3 - y_2)$ and $I(x_2, x_3, y_3 - y_2)$ to fill the bottom
section of the triangle.

\subsubsection{Texture Mapping}

Instead of filling triangles with solid colour, we can also fill them with texture.
By having corresponding texture-space coordinates associated with each point on a
triangle, we can perform the same interpolation as in the solid colour case
simultaneously to find the texture-space points along the edge of the triangle,
then instead of drawing each rake normally, we interpolate between the texture
points on either side to sample colour from our texture.
\\[\baselineskip]
It is worth noting that this approach ignores perspective so may give odd results
when viewing from certain angles with certain textures. The solution to this is
out of the scope of this unit.

\subsection{Projecting}

We can project the world-space coordinates of an object to screen coordinates
by considering the screen plane to be some distance $f$ in front of the
camera so that where $(u, v)$ is our desired screen-space coordinate and $(x, y, z)$ is
our world-space coordinate, we have similar right-angled triangles formed by: 
\begin{itemize}
    \item $(0, 0, f)$, $(u, v, 0)$, and $(0, 0, 0)$,
    \item $(0, 0, f)$, $(x, y, z)$, and $(0, 0, z)$,
\end{itemize} not accounting for camera position.
Using this, we can calculate the desired screen-space coordinate: 
\begin{align*}
    (u, v, f) = f \cdot \frac{(x, y, f + |z|)}{f + |z|}.
\end{align*} We can then convert $(u, v)$ to pixel-space coordinates 
by inverting the $y$-axis
and translating by half the dimensions of our screen.

\subsubsection{Depth}

We can calculate the depth of a world-space point $(x, y, z)$ by taking the reciprocal
of $|z|$. Thus, a smaller depth value means the point is farther away from the 
camera, with zero representing a point infinitely far from the camera.
This can be used to allow proper layering when rasterising.
We note that the positive $z$ direction points into the camera and negative
$z$ points to where the camera is facing, which is why we take the reciprocal
of the absolute value.

\subsection{Culling}

Considering how we project our points, things may become strange if we consider
points behind the camera. Thus, it makes sense to 'cull' points that have negative
depth, to not consider them.