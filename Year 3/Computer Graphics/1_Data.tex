\section{Data Types}

\subsection{Objects}

Objects are stored in \texttt{.obj} files with texture and colour information
stored in \texttt{.mtl} files. Object files define sets of points and, from these
points, elements consisting of triangular faces. Similarly, texture points
can also be defined and attached to the faces. Elements can also be assigned
materials defined in an inherited \texttt{.mtl} which should define the material,
giving it a texture path or a colour.

\subsubsection{Object-space to World-space Coordinates}

We convert object-space coordinates to world-space coordinates by translating,
scaling, and rotating our object. This can be done with matrices.

\subsection{Colour}

Colours can be represented in multiple ways, in \texttt{.mtl} files they are
three floats ranging from \texttt{0.0} to \texttt{1.0} to represent red, green,
and blue. It's also useful to have the same representation but from \texttt{0}
to \texttt{255} instead with an additional alpha value for transparency.

\subsubsection{Packing}

We can pack a colour represented by four 8-bit values 
(\texttt{r}, \texttt{g}, \texttt{b}, \texttt{a}) into a 32-bit 
integer \texttt{c} in \texttt{C++} with: \begin{lstlisting}
    c = (a << 24) + (r << 16) + (g << 8) + b;
\end{lstlisting} This representation is used to set the colour of pixels on
the screen.