\section{Camera}

\subsection{Rotation Matrices}

We can perform an anti-clockwise 3D rotation of $\theta$ degrees 
using the 3D rotation matrices $R_x$, $R_y$, and $R_z$ corresponding 
to rotations around the $x$, $y$, and $z$ axes respectively: \begin{align*}
    R_x = \begin{pmatrix}
        1 & 0 & 0 \\
        0 & \cos(\theta) & -\sin(\theta) \\
        0 & \sin(\theta) & \cos(\theta) \\
    \end{pmatrix}, \\
    R_y = \begin{pmatrix}
        \cos(\theta) & 0 & \sin(\theta) \\
        0 & 1 & 0 \\
        -\sin(\theta) & 0 & \cos(\theta) \\
    \end{pmatrix}, \\
    R_z = \begin{pmatrix}
        \cos(\theta) & -\sin(\theta) & 0 \\
        \sin(\theta) & \cos(\theta) & 0 \\
        0 & 0 & 1 \\
    \end{pmatrix}.
\end{align*} The 2D anti-clockwise rotation is represented by the matrix $R$: \begin{align*}
    R = \begin{pmatrix}
        \cos(\theta) & -\sin(\theta) \\
        \sin(\theta) & \cos(\theta) \\
    \end{pmatrix}.
\end{align*}


\subsection{Camera Position and Orientation}

We can simply represent the position of the camera with a 3D vector,
which we can translate quite easily.
The orientation, however, is more complex and is represented with a 3D matrix $C$, 
consisting of vectors $r$, $u$, and $f$, corresponding to the direction
'right' from the camera, 'up' from the camera, and 'forward'
into the camera: \begin{align*}
    C = \begin{pmatrix}
        \uparrow & \uparrow & \uparrow \\
        r & u & f \\
        \downarrow & \downarrow & \downarrow \\
    \end{pmatrix}.
\end{align*} This matrix can be rotated using the standard rotation matrices.

\subsubsection{Look At}

We can set our camera orientation matrix to look at a point by setting the
vectors inside it appropriately. The 'forward' vector is the vector from the
desired point to the camera. The 'right' vector is the cross-product of
$(0, 1, 0)$ and the 'forward' vector. The 'up' vector is the cross-product
of 'forward' and 'right'.

\subsection{World-space to Camera-space Coordinates}

We can apply the translation of our camera by translating vertices by the
camera position multiplied by $-1$. The orientation of the camera can
be applied by taking the vector from camera to the vertex and right-multiplying
it by $C$.

\subsection{Homogenous Coordinates}

The homogenous coordinate system adds a dimension to the camera orientation
matrix in order to embed translation information. This allows for all camera
information to be stored in a single matrix and the composition of translations
and rotations. We can map to homogenous coordinates as follows: \begin{align*}
    \begin{pmatrix}
        x \\ y \\ z \\
    \end{pmatrix} &\mapsto \begin{pmatrix}
        x \\ y \\ z \\ 1 \\
    \end{pmatrix}, \\
    \begin{pmatrix}
        \uparrow & \uparrow & \uparrow \\
        r & u & f \\
        \downarrow & \downarrow & \downarrow \\
    \end{pmatrix} &\mapsto \begin{pmatrix}
        \uparrow & \uparrow & \uparrow & x \\
        r & u & f & y \\
        \downarrow & \downarrow & \downarrow & z \\
        0 & 0 & 0 & 1 \\
    \end{pmatrix},
\end{align*} where $x$, $y$, and $z$ store the translation information of the
matrix, by default, this would be all zeroes (in rotation matrices for example).
We can still perform in-place rotations by simply translating the matrix to
the origin before the rotation and then translating it back afterwards. 

