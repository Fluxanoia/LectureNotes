\section{The Fundamentals}

\subsection{Rings (1.1)} \label{1.1}

A ring is a set with two binary operations, addition and multiplication,
such that they are both commutative, associative, and addition is distributive over
multiplication, so for $a$, $b$, and $c$ in some ring: \begin{align*}
    (a + b)c = ac + bc.
\end{align*} We also have that rings must contain 'zero' and 'one' elements, the additive and
multiplicative identities, and every element of the ring has an additive inverse.

\subsection{Properties of Rings (1.3)} \label{1.3}

For a ring $R$ with $a$, $b$, and $c$ in $R$: \begin{itemize}
    \item if $a + b = b$ then $a = 0$, $0$ is unique,
    \item if $a \cdot x = x$ for all $x$ in $R$, then $a = 1$, $1$ is unique,
    \item if $a + b = 0 = a + c$ then $b = c$, $-a$ is unique, 
    \item we have $0 \cdot a = 0$,
    \item we have $-1 \cdot a = -a$,
    \item we have $0 = 1$ if and only if $R = \{0\}$.
\end{itemize}

\subsection{Units (1.6-7)} \label{1.6} \label{1.7}

For a ring $R$, with $r$ in $R$, if there exists some $s$ such that $rs = 1$ then $r$
is a unit and $s = r^{-1}$ is the multiplicative inverse of $r$. We write $R^\times$ to
be the set of all units in $R$, which is an abelian group under multiplication.

\subsection{Fields (1.9)} \label{1.9}

A non-zero ring $R$ is a field if $R \setminus \{0\} = R^\times$.

\subsection{Subrings (1.14-15)} \label{1.14} \label{1.15}

For a ring $R$, $S \subseteq R$ is a subring of $R$ if it is a ring and contains zero and one.
This is equivalent to saying $S$ is closed under addition, multiplication, and additive inverses,
and contains 1.

\subsection{The Gaussian Integers (1.17, 1.19)} \label{1.17} \label{1.19}

We define the Gaussian integers as: \begin{align*}
    \mb{Z}[i] = \{a + bi : a, b \in \mb{Z}\},
\end{align*} which is the smallest subring of $\mb{C}$ containing $i$.
Generally, for $\alpha$ in $\mb{C}$, $\mb{Z}[\alpha]$ is the smallest subring
containing $\alpha$ and for a ring $R$ with a subring $S$, for some $\beta$ in
$R$, we have $S[\beta]$ is the smallest subring of $R$ containing $S$ and $\beta$.

\subsection{Product Rings (1.20)} \label{1.20}

For $R$ and $S$ rings, we have that $R \times S$ is a ring under component-wise
addition and multiplication.

\subsection{Distributivity of Taking Units (1.22)} \label{1.22}

For rings $R$ and $S$, $(R \times S)^\times = R^\times \times S^\times$.

\begin{proof}
    We consider: \begin{align*}
        (r, s) \in (R \times S)^\times
        &\Longleftrightarrow (r, s)(p, q) = (1, 1) \text{ for some } (p, q) \in R \times S \\
        &\Longleftrightarrow rp = 1 \text{ and } sq = 1 \text{ for some } 
            p \in R \text{ and } q \in S \\
        &\Longleftrightarrow r \in R^\times \text{ and } s \in S^\times,
    \end{align*} as required.
\end{proof}

\subsection{Polynomials (1.23)} \label{1.23}

For a ring $R$ and a symbol $x$, we have that the following is a ring: \begin{align*}
    R[x] = \{a_0 + a_1x + \cdots + a_nx^n : n \in \mb{Z}_{\geq 0}, (a_i)_{i \in [n]} \in R^n\}.
\end{align*} 

\subsection{Ring Homomorphisms (2.7, 2.12)} \label{2.7} \label{2.12}

For $R$ and $S$ rings, a map $\varphi$ from $R$ to $S$ is a ring
homomorphism if it preserves addition and multiplication.
This implies that $0$ and $1$ are fixed points of $\varphi$
and taking additive inverses is preserved by $\varphi$.

\newpage
\noindent
We have some properties of ring homomorphisms: \begin{itemize}
    \item $\varphi(0) = 0$,
    \item $\varphi(-a) = -\varphi(a)$,
    \item the image of $\varphi$ is a subring of $S$,
    \item homomorphisms are preserved under composition.
\end{itemize}

\subsection{Ring Isomorphisms (2.1)} \label{2.1}

A ring isomorphism is a bijective ring homomorphism.

\subsection{The Kernel (2.13, 2.18)} \label{2.13} \label{2.18}

The kernel of a homomorphism is the set of values it maps to $0$.
This is not necessarily a ring. The kernel is $\{0\}$ if and only
if the homomorphism is injective.

\subsection{Ideals (2.15-16)} \label{2.15} \label{2.16}

For a ring $R$ with $I \subseteq R$, $I$ is an ideal if it is an additive
subgroup of $R$ and for all $r$ in $R$ and $i$ in $I$, $ri$ is in $I$.
The kernel of homomorphisms are ideals.

\subsection{Preservation of Satisfaction (2.20)} \label{2.20}

For a ring $R$ with $r$ in $R$, if for some $n$ in $\mb{Z}_{\geq 0}$ we have 
$(a_i)_{i \in [n]}$ in $\mb{Z}^n$ such that: \begin{align*}
    a_nr^n + \cdots + a_1r + a_0 = 0,
\end{align*} then for any homomorphism $\varphi$ on $R$ to some other ring $S$,
we have that: \begin{align*}
    \varphi(a_nr^n + \cdots + a_1r + a_0) = 0.
\end{align*}

\subsection{Cosets (2.22)} \label{2.22}

For a ring $R$ with $r$ in $R$ and an ideal $I$ of $R$, the coset of $r$ modulo $I$ is
the set: \begin{align*}
    r + I = \{r + i : i \in I\}.
\end{align*} For each $r$ and $s$ in $R$, we define a relation by: \begin{align*}
    r \sim s \Longleftrightarrow r - s \in I,
\end{align*} which is an equivalence relation, with equivalence classes the cosets of
$R$ modulo $I$. Thus, cosets are either identical or disjoint.
