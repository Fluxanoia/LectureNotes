\section{Principal Ideal, Euclidean, and Unique Factorisation Domains}

\subsection{Noetherian Rings (5.1)} \label{5.1}

A ring $R$ is Noetherian if every increasing chain of ideals in $R$
is finite.

\subsection{Finitely Generated Ideals (5.3)} \label{5.3}

For a ring $R$ with $I \subseteq R$ an ideal, $I$ is generated by
$i_1$, $\ldots$, $i_n$ in $I$ for some $n$ in $\mb{Z}_{>0}$ if:
\begin{align*}
    I = i_1R + \cdots + i_nR = (i_1, \ldots, i_n).
\end{align*}

\subsection{Finite Generated Ideals in Noetherian Rings (5.4)} \label{5.4}

A ring is Noetherian if and only if every ideal of $R$ is finitely
generated.

\begin{proof}
    ($\Longrightarrow$) For an ideal $I \subseteq R$, we consider
    $i_1$ in $I$ and take $I_1 = (i_1)$. If $I_1 = I$ then we are
    done, otherwise we consider $i_2$ in $I \setminus I_1$ and
    take $I_2 = (i_1, i_2)$. Following this process, we get
    $I_1 \subset I_2 \subset \cdots \subset I_n$, for some $n$ in
    $\mb{Z}_{>0}$ as $R$ is Noetherian. Thus, we get a finite 
    generating set $(i_1, \ldots, i_n)$ for $I$.
    \bs
    ($\Longleftarrow$) If we have a chain of ideals
    $I_1 \subset I_2 \subset \cdots$, then 
    $I = \bigcup_{k \in \mb{Z}_{>0}} I_k$ is finitely generated
    by some $(i_1, \ldots, i_n)$ by assumption. For each $k$ in
    $[n]$, $i_k$ must be in $I_{m_k}$ for some $m_k$, so
    taking $m = \max(m_1, \ldots, m_n)$, $I_m = I$ as required.
\end{proof}

\subsection{Preservation of Noetherian Rings (5.5-6)} \label{5.5} \label{5.6}

All quotients of, products of, and polynomials with coefficients
in a Noetherian ring are Noetherian rings.

\begin{proof}[Proof of Quotients]
    For a Noetherian ring $R$ with $I$ an ideal of $R$, we consider
    $\varphi$ from $R$ to $R / I$ mapping $r \mapsto r + I$.
    By (\ref{3.15}), we have a bijection from ideals
    in $R$ containing $I$ and ideals of $R / I$ via $\varphi$
    which preserves the Noetherian property.
\end{proof}

\subsection{Divisibility (5.10-12)} \label{5.10} \label{5.11} \label{5.12}

For an integral domain $R$, we say that $b$ in $R$ divides $a$ in $R$
if there exists $c$ in $R$ such that $a = bc$. Thus, $a$ is in $(b)$
and similarly $(a) \subseteq (b)$. We note that such a $c$ is unique.
\bs
If $a$ and $b$ both divide each other ($(a) = (b)$, or $a = b\varepsilon$
for some unit $\varepsilon$ in $R$), we say they are associates.
For some $p \neq 0$ in $R$ where $p$ is not a unit, have that: \begin{align*}
    p \text{ irreducible} &\Longleftrightarrow 
    \bigl[p = ab \Longrightarrow a \in R^\times \text{ or } b \in R^\times
    \bigr], \\
    p \text{ prime} &\Longleftrightarrow 
    \bigl[p \, | \, ab \Longrightarrow p \, | \, a \text{ or } p \, | \, b
    \bigr]
    \Longleftrightarrow (p) \text{ is a non-zero prime ideal}.
\end{align*}

\subsection{Irreducible Primes (5.14)} \label{5.14}

For an integral domain $R$, primes of $R$ are irreducible.

\begin{proof}
    For a prime $p$ in $R$ with $b$ and $c$ in $R$ such that $p = bc$,
    so $b$ and $c$ both divide $p$. By the definition of primes, we have 
    that $p$ divides $bc$ so either $p$ divides $b$ or $p$ divides
    $c$. As such, if $p$ divides $b$ then $p$ and $b$ are associate
    so $c$ is a unit (and similarly for $p$ dividing $c$).
\end{proof}

\subsection{ (5.16)} \label{5.16} 
