\section{Principal Ideal, Euclidean, and Unique Factorisation Domains}

\subsection{Noetherian Rings (5.1)} \label{5.1}

A ring $R$ is Noetherian if every increasing chain of ideals in $R$
is finite.

\subsection{Finitely Generated Ideals (5.3)} \label{5.3}

For a ring $R$ with $I \subseteq R$ an ideal, $I$ is generated by
$i_1$, $\ldots$, $i_n$ in $I$ for some $n$ in $\mb{Z}_{>0}$ if:
\begin{align*}
    I = i_1R + \cdots + i_nR = (i_1, \ldots, i_n).
\end{align*}

\subsection{Finitely Generated Ideals in Noetherian Rings (5.4)} \label{5.4}

A ring is Noetherian if and only if every ideal of $R$ is finitely
generated.

\begin{proof}
    ($\Longrightarrow$) For an ideal $I \subseteq R$, we consider
    $i_1$ in $I$ and take $I_1 = (i_1)$. If $I_1 = I$ then we are
    done, otherwise we consider $i_2$ in $I \setminus I_1$ and
    take $I_2 = (i_1, i_2)$. Following this process, we get
    $I_1 \subset I_2 \subset \cdots \subset I_n$, for some $n$ in
    $\mb{Z}_{>0}$ as $R$ is Noetherian. Thus, we get a finite 
    generating set $(i_1, \ldots, i_n)$ for $I$.
    \bs
    ($\Longleftarrow$) If we have a chain of ideals
    $I_1 \subset I_2 \subset \cdots$, then 
    $I = \bigcup_{k \in \mb{Z}_{>0}} I_k$ is finitely generated
    by some $(i_1, \ldots, i_n)$ by assumption. For each $k$ in
    $[n]$, $i_k$ must be in $I_{m_k}$ for some $m_k$, so
    taking $m = \max(m_1, \ldots, m_n)$, $I_m = I$ as required.
\end{proof}

\subsection{Preservation of Noetherian Rings (5.5-6)} \label{5.5} \label{5.6}

All quotients of, products of, and polynomials with coefficients
in a Noetherian ring are Noetherian rings.

\begin{proof}[Proof of Quotients]
    For a Noetherian ring $R$ with $I$ an ideal of $R$, we consider
    $\varphi$ from $R$ to $R / I$ mapping $r \mapsto r + I$.
    By (\ref{3.15}), we have a bijection from ideals
    in $R$ containing $I$ and ideals of $R / I$ via $\varphi$
    which preserves the Noetherian property.
\end{proof}

\subsection{Divisibility (5.10-12)} \label{5.10} \label{5.11} \label{5.12}

For an integral domain $R$, we say that $b$ in $R$ divides $a$ in $R$
if there exists $c$ in $R$ such that $a = bc$. Thus, $a$ is in $(b)$
and similarly $(a) \subseteq (b)$. We note that such a $c$ is unique.
\bs
If $a$ and $b$ both divide each other ($(a) = (b)$, or $a = b\varepsilon$
for some unit $\varepsilon$ in $R$), we say they are associates.
For some $p \neq 0$ in $R$ where $p$ is not a unit, have that: \begin{align*}
    p \text{ irreducible} &\Longleftrightarrow 
    \bigl[p = ab \Longrightarrow a \in R^\times \text{ or } b \in R^\times
    \bigr], \\
    p \text{ prime} &\Longleftrightarrow 
    \bigl[p \, | \, ab \Longrightarrow p \, | \, a \text{ or } p \, | \, b
    \bigr]
    \Longleftrightarrow (p) \text{ is a non-zero prime ideal}.
\end{align*}

\subsection{Irreducible Primes (5.14)} \label{5.14}

For an integral domain $R$, primes of $R$ are irreducible.

\begin{proof}
    For a prime $p$ in $R$ with $b$ and $c$ in $R$ such that $p = bc$,
    so $b$ and $c$ both divide $p$. By the definition of primes, we have 
    that $p$ divides $bc$ so either $p$ divides $b$ or $p$ divides
    $c$. As such, if $p$ divides $b$ then $p$ and $b$ are associate
    so $c$ is a unit (and similarly for $p$ dividing $c$).
\end{proof}

\subsection{Factorisation (5.16)} \label{5.16}

For a Noetherian integral domain $R$, every $r \neq 0$ in $R$ can be factored as:
\begin{align*}
    r = \varepsilon q_1q_2 \cdots q_n,
\end{align*} for some unit $\varepsilon$ in $R$ and $q_1, \ldots, q_n$ irreducible in $R$
for some $n$ in $\mb{Z}_{\geq 0}$.

\begin{proof}
    If $r$ is a unit, then we are done with $\varepsilon = r$ and $n = 0$. If $r$ is
    not a unit, we first want to show that $r = q_1s$ for some irreducible $q_1$ and
    $s$ in $R$.
    \bs
    If $r$ is irreducible, we can take $q_1 = r$ and $s = 1$. If $r$ is not irreducible,
    then $r = b_1s_1$ for some non-units $b_1$ and $s_1$ in $R$. We continue, if $b_1$ 
    is irreducible we are done, otherwise, we write $b_1 = b_2s_2$ so $r = b_2s_1s_2$
    for non-units $b_2$ and $s_2$. Applying this process repeatedly, until we have
    $r = b_ns_1 \cdots s_n$ with $b_n$ irreducible. This process terminates
    as $b_{k + 1}$ divides $b_k$ for all $k$ in $[n - 1]$ and this implies that: \begin{align*}
        (r) \subset (b_1) \subset (b_2) \subset \cdots, \tag{$\ast$}
    \end{align*} which must terminate as $R$ is Noetherian.
    Using this fact, we can write $r = q_1r_1$ for some irreducible $q_1$ and $r_1$ in $R$.
    If $r_1$ is irreducible we are done, otherwise we apply the process repeatedly,
    yielding $r = q_1 \cdots q_nr_n$ which terminates by similar reasoning to $(\ast)$.
\end{proof}

\subsection{Unique Factorisation Domains (UFD) (5.17)} \label{5.17}

For an integral domain $R$, we say that $R$ is a UFD if every $r \neq 0$ in $R$ 
is a product of finitely many irreducible elements and a unit of $R$, unique up to
reordering and units.

\subsection{Confluence of Primality and Irreducibility (5.23)} \label{5.23}

For a UFD $R$, $p$ in $R$ is prime if and only if $p$ is irreducible.

\begin{proof}
    ($\Longrightarrow$) Proved in (\ref{5.14}).
    \bs
    ($\Longleftarrow$) We know that $p$ is non-zero and not a unit as it is irreducible.
    We suppose that $p$ divides some $ab$ for $a$ and $b$ in $R$, so $ab = pc$ for some
    $c$ in $R$. Using the properties of UFDs, we factor $a$, $b$, and $c$ into
    irreducibles which, by uniqueness, must contain $p$ (as $ab = pc$). Thus,
    $p$ divides $a$ or $b$ so $p$ is prime.
\end{proof}

\subsection{Highest Common Factor (5.24)} \label{5.24}

For a UFD $R$, with $a$ and $b$ non-zero in $R$, the highest common factor of $a$ and $b$
is the product of the common irreducibles in their unique factorisation (which is well-defined
up to units). Taking $h = \hcf(a, b)$: \begin{itemize}
    \item $h$ divides $a$ and $b$,
    \item $a = hx$ and $a = hy$ for some coprime $x$ and $y$ in $R$,
    \item $\hcf(ac, bc) = c \cdot \hcf(a, b)$ for any $c \neq 0$ in $R$,
    \item if $a$ and $b$ are coprime and $a$ divides $bc$ for some $c$ in $R$, then
        $a$ divides $c$.
\end{itemize}

\subsection{Coprimaility (5.25)} \label{5.25}

For a UFD $R$, with $a$ and $b$ non-zero in $R$, $a$ and $b$ are coprime if $\hcf(a, b)$
is a unit.

\subsection{Principle Ideal Domains (PID) (5.28)} \label{5.28}

For an integral domain $R$, we have that $R$ is a principle ideal domain if every ideal
of $R$ is principle (generated by a single element).

\subsection{Irreducibility, Primality, and Ideal Maximality in PIDs (5.31)} \label{5.31}

For a PID $R$, with $p$ non-zero and not a unit in $R$: \begin{align*}
    p \text{ irreducible} 
    \Longleftrightarrow p \text{ prime}
    \Longleftrightarrow (p) \text{ prime}
    \Longleftrightarrow (p) \text{ maximal}.
\end{align*}

\begin{proof}
    As $R$ is an integral domain we already have: \begin{align*}
        (p) \text{ maximal}
        \Longrightarrow (p) \text{ prime}
        \Longleftrightarrow p \text{ prime}
        \Longrightarrow p \text{ irreducible}.
    \end{align*} So, it is sufficient to show that: \begin{align*}
        p \text{ irreducible} \Longrightarrow (p) \text{ maximal}.
    \end{align*} We suppose $p$ is irreducible and $(p) \subseteq I \subseteq R$
    for some ideal $I$. As $R$ is a PID, we know that $I = (a)$ for some $a$ in
    $R$ so: \begin{align*}
        (p) \subseteq (a) 
        &\Longrightarrow a \text{ divides } p \\
        &\Longrightarrow p = ab \text{ for some } b \in R \\
        &\Longrightarrow a \in R^\times \text{ or } b \in R^\times \tag{$p$ irreducible} \\
        &\Longrightarrow I = R \text{ or } (p) = I,
    \end{align*} as required.
\end{proof}
