\section{Finite Fields}

\subsection{Main Theorem of Finite Fields, Part (a) (9.1, 9.6)} \label{9.1a} \label{9.6}

Every finite field $K$ has $p^n$ elements for some prime $p$ and $n$ in $\mb{Z}_{> 0}$.

\begin{proof}
    By (\ref{8.11}) and as $K$ is finite, we know that $K$ contains $\mb{F}_p$ for some 
    prime $p$. Thus, $K$ is a $\mb{F}_p$-vector space with finite dimension $n$ (as $K$ is
    finite) giving us $|K| = p^n$ and $n = [K : \mb{F}_p]$.
\end{proof}

\subsection{Upper Bound on Distinct Roots of Polynomials (9.7)} \label{9.7}

For a field $K$ and a polynomial $f$ in $K[x]$ with degree $n > 0$, we have that
$f$ has at most $n$ distinct roots in $K$.

\begin{proof}
    If $f$ has no roots, we are done. Otherwise, we take $\alpha_1, \ldots, \alpha_m$ 
    to be the distinct roots of $f$ (for some $m$ in $\mb{Z}_{>0}$) so
    $x - \alpha_1, \ldots, x - \alpha_m$ divide $f$ in $K[x]$ by (\ref{7.8}).
    As these linear polynomials are irreducible in $K[x]$, they are non-associate,
    prime elements of $K[x]$, which is a UFD, so their product also divides $f$.
    Thus, $m \leq n$.
\end{proof}

\subsection{Structure of Finite Abelian Groups (9.8)} \label{9.8}

Every finite abelian group $G$ is isomorphic to a product of cyclic groups: \begin{align*}
    G \cong C_{m_1} \times \cdots \times C_{m_k},
\end{align*} with $m_i$ dividing $m_{i - 1}$ for each $i$ in $\{k, \ldots, 1\}$.

\subsection{Cyclic Finite Subgroups of Units (9.9-10)} \label{9.9} \label{9.10} 

For a field $K$ and a finite subgroup $U \subseteq K^\times$, $U$ is cyclic.
Thus, $K^\times$ is a cyclic subgroup of $K$ if $K$ is finite.

\begin{proof}
    As $U$ is finite and abelian, by (\ref{9.8}) we have: \begin{align*}
        U \cong C_{m_1} \times \cdots \times C_{m_k},
    \end{align*} and $m_i$ dividing $m_{i - 1}$ for each $i$ in $\{k, \ldots, 1\}$.
    If $U$ is not cyclic, we consider that $k > 1$ and $g^{m_1} = e$ for every $g$ in $U$
    since $m_k$, $\ldots$, $m_2$ divide $m_1$. Furthermore, $|U| = m_1\cdots m_k > m_1$
    so $x^{m_1} - 1$ has more than $m_1$ roots in $K$ but this is impossible by (\ref{9.7}).
    If $K$ is finite, $K^\times$ is a finite group so is cyclic.
\end{proof}

\subsection{Main Theorem of Finite Fields, Part (b) (9.1)} \label{9.1b}

For a finite field $K$, if $|K| = p^n$ for some prime $p$ and $n$ in $\mb{Z}_{> 0}$ then
$K^\times$ is cyclic with order $p^n - 1$.

\begin{proof}
    Follows from (\ref{9.10}).
\end{proof}

\subsection{Roots of Finite Fields of Prime Power Order (9.15)} \label{9.15}

For a finite field $K$ with $|K| = p^n$: \begin{enumerate}
    \item every element $\alpha$ of $K$ is a root of $x^{p^n} - x$, so $\alpha^{p^n} = \alpha$,
    \item we have $x^{p^n} - x = \prod_{\alpha \in K} (x - \alpha)$.
\end{enumerate}

\begin{proof}
    (1) We know by (\ref{9.1b}) that $K^\times$ has order $p^n - 1$ so every $\alpha$ in
    $K^\times$ satisfies $\alpha^{p^n - 1} = 1$ by Lagrange's Theorem. As such,
    $\alpha^{p^n} = \alpha$ which means $\alpha$ is a root of $x^{p^n} - x$. Since zero
    is also a root of this, we are done.
    \bs
    (2) Since $x^{p^n} - x$ has degree $p^n$ and has at precisely $p^n$ roots by (1) and
    (\ref{9.7}), it must be $\prod_{\alpha \in K}(x - \alpha)$ by comparing degrees and
    leading coefficients.

\end{proof}

\subsection{Wilson's Theorem (9.17)} \label{9.17}

We have that $(p - 1)! \equiv -1 \bmod p$ for all primes $p$.

\begin{proof}
    We consider the following polynomial $x^p - x$ in $\mb{F}_p[x]$ using
    (\ref{9.15}): \begin{align*}
        x^p - x = x(x - 1)\cdots(x - (p - 1))
        &\Longrightarrow x^{p - 1} - 1 = (x - 1)\cdots(x - (p - 1)) \\
        &\Longrightarrow 0^{p - 1} - 1 = (0 - 1)\cdots(0 - (p - 1)) \\
        &\Longrightarrow -1 = (-1)\cdots(-(p - 1)) = (-1)^{p - 1}(p - 1)!,
    \end{align*} which means $-1 = (p - 1)!$ since $p$ is either
    odd or $2$ (and $-1 = 1 \bmod 2$).
\end{proof}

\newpage

\subsection{Prime Characteristic and Subfields with Prime Power Order (9.18)} \label{9.18}

For a field $F$ with characteristic $p$ (so $\mb{F}_p \subseteq F$), we have that:
\begin{enumerate}
    \item if $F$ contains a subfield with $p^n$ elements for some $n$ in $\mb{Z}_{>0}$,
        then $x^{p^n} - x$ has $p^n$ roots in $F$,
    \item conversely, if $x^{p^n} - x$ has $p^n$ roots in $F$ for some $n$ in $\mb{Z}_{>0}$,
        then these roots form a field with $p^n$ elements.
\end{enumerate}

\begin{proof}
    (1) This follows directly from the assumption and (\ref{9.15}).
    \bs
    (2) We take $K$ to be the roots of $x^{p^n} - x$ in $F$, and note that it contains
    $-1$, $0$, and $1$. We will show that $K$ is closed under addition and multiplication
    so that $K$ is a subring of $F$, this means $K$ is a finite integral domain and thus,
    a field. Firstly, we show that $\varphi$ from $F$ to $F$ defined by $x \mapsto x^p$
    is a ring homomorphism. We know that $0^p = 0$, $1^p = 1$, and $(-1)^p = -1$, and that
    for $x$ and $y$ in $F$ we have $(xy)^p = x^py^p$ and:
    \begin{align*}
        (x + y)^p = x^p + \binom{p}{1}x^{p - 1}y + \cdots + \binom{p}{p - 1}xy^{p - 1} + y^p.
    \end{align*} So, as $\binom{p}{k} = \frac{p!}{k!(p-k)!}$, $p! \equiv 0 \bmod p$, and
    $k!(p-k)! \not\equiv 0 \bmod p$, we have that
    \linebreak $(x + y)^p = x^p + y^p$. As such,
    $\varphi$ is a ring homomorphism and by iteration, $\varphi^n$ is also
    ($\varphi^n(x) = x^{p^n}$). This means that $(a + b)^{p^n} = a^{p^n} + b^{p^n} = a + b$
    and $(ab)^{p^n} = a^{p^n}b^{p^n} = ab$ for all $a$ and $b$ in $K$ so $K$ is closed
    under addition and multiplication as required.
\end{proof}

\subsection{Splitting Fields (9.19)} \label{9.19}

For a field $K$ with a non-constant polynomial $f$ in $K[x]$, there exists a finite extension
$F / K$ in which $f(x)$ factors into a product of linear factors. The smallest such
extension is a splitting field of $f$ over $K$.

\begin{proof}
    We proceed iteratively. Firstly, if $f$ consists of only linear factors, we are done.
    Otherwise, we choose one of these irreducible factors $g(x)$ of $f(x)$ of degree at
    least $2$. Then, we replace $K$ with $K[x] / g(x)$, adding the root. We repeat this process
    until it terminates, as we are dealing with polynomials of finite degree.
\end{proof}

\newpage

\subsection{Main Theorem of Finite Fields, Part (c) (9.1, 9.21-22)}
\label{9.1c} \label{9.21} \label{9.22}

For every prime $p$ and $n$ in $\mb{Z}_{> 0}$, there is a unique field up to isomorphism
with $p^n$ elements. It is denoted by $\mb{F}_{p^n}$ and contains $\mb{F}_p$ with
$[\mb{F}_{p^n} : \mb{F}_p] = n$.

\begin{proof}
    (Existence) We take $F$ to be a finite extension of $\mb{F}_p$ which contains
    all the roots of $x^{p^n} - x$ by (\ref{9.19}). These are distinct
    (proven as an \textbf{exercise}) so form a field of order $x^{p^n}$ by (\ref{9.18}).
    \bs
    (Uniqueness) We take $K$ and $K'$ to be fields of order $p^n$. We know that
    they both contain $\mb{F}_p$ by (\ref{9.18}). We take a generator $\alpha$ in
    $K^\times$ so that $\mb{F}_p(\alpha) = K$ (the smallest subfield containing
    $\mb{F}_p$ and $\alpha$ contains all the powers of $\alpha$, so the entirety of
    $K^\times$, thus $K$). Furthemore, we know that
    $K = \mb{F}_p(\alpha) \cong \mb{F}_p[x] / f(x)$
    where $f$ is the minimal polynomial of $\alpha$ over $\mb{F}_p$.
    \bs 
    By Part (a), we have $[K : \mb{F}_p] = n$ so $\deg(f) = n$ meaning $f$ is an
    irreducible monic polynomial. Since $f$ divides every polynomial which
    has $\alpha$ as a root, it divides $x^{p^n} - x$, so all of its roots
    are elements of $K$.
    We also know that $K'$ contains all the roots of $x^{p^n} - x$ so $f$ has a
    root $\beta$ in $K'$. As such, $K = \mb{F}_p[x]/f(x)$ is injective into
    $K'$ via the map $\alpha \mapsto \beta$ as required. 
\end{proof}

\subsection{Finite Fields of Prime Power Order (9.23)} \label{9.23}

We write $\mb{F}_{p^n}$ for a prime $p$ and $n$ in $\mb{Z}_{>0}$ as the unique
field of order $p^n$.

\subsection{Monic Irreducible Polynomials of Degree $n$ (9.24)} \label{9.24}

For every prime $p$ and $n$ in $\mb{Z}_{>0}$, there exists monic irreducible polynomials
of degree $n$ in $\mb{F}_p[x]$. All such polynomial $f$ divides $x^{p^n} - x$ and:
\begin{align*}
    \mb{F}_p[x]/f(x) \cong \mb{F}_{p^n}.
\end{align*}
