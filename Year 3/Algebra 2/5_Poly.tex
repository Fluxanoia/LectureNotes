\section{Gauss' Lemma and Polynomial Reducibility}

\subsection{Content of Polynomials (6.3)} \label{6.3}

For a UFD $R$ and $f$ a non-zero polynomial in $R[x]$, the highest common factor of
the coefficients of $f$ is the content of $f$ denoted by $c_f$.

\subsection{Primitive Polynomials (6.2)} \label{6.2}

For a UFD $R$, a polynomial $f$ in $R[x]$ is primitive if $c_f = 1$.
Any polynomial $f$ in $R[x]$ can be written as $c_f \cdot f^\ast$
where $f^\ast$ is a primitive polynomial in $R[x]$.

\subsection{Gauss' Lemma (6.6)} \label{6.6}

The product of primitive polynomials is primitive.

\begin{proof}
    For a UFD $R$, we take $f$ and $g$ primitive in $R$ such that: \begin{align*}
        f &= a_nx^n + \cdots + a_1x + a_0, \\
        g &= b_mx^m + \cdots + b_1x + b_0, \\
        fg &= c_{n + m}x^{n + m} + \cdots + c_1x + c_0,
    \end{align*} for $n$ and $m$ in $\mb{Z}_{\geq 0}$, $a_i$, $b_j$, and $c_k$ in
    $R$ for $i$ in $[n]$, $j$ in $[m]$, and $k$ in $[n + m]$. For any irreducible $q$
    in $R$, $q$ does not divide all of $a_0, \ldots, a_n$ as $f$ is primitive, we take
    $i$ to be maximal such that $q$ does not divide $a_i$. Similarly, we take $j$
    maximal such that $q$ does not divided $b_j$. We consider $c_{i + j}$: \begin{align*}
        c_{i + j} = 
        \underbrace{a_{i + j}b_0 + \cdots + a_{i + 1}b_{j - 1}}_{\text{all divisible by } q}
        + a_ib_j
        + \underbrace{a_{i - 1}b_{j + 1} + \cdots + a_0b_{i + j}}_{\text{all divisible by } q}.
    \end{align*} Since $R$ is a UFD, as $q$ doesn't divide $a_i$ and $b_j$, $q$ doesn't divide
    $a_ib_j$ so $c_{i + j}$ is not divisible by $q$.
\end{proof}

\subsection{Content under Multiplication (6.7)} \label{6.7}

For a field of fractions $F$ of a UFD $R$, with $f$ and $g$ in $F[x]$, we have that
$c_{fg} = c_fc_g$.

\begin{proof}
    Application of (\ref{6.2}) and (\ref{6.6}).
\end{proof}

\subsection{Properties of UFDs and their Polynomials (6.8)} \label{6.8}

For a UFD $R$: \begin{enumerate}
    \item for a unit $u$ in $R$, $u$ is a unit in $R[x]$,
    \item for a prime $p$ in $R$, $p$ is a prime in $R[x]$,
    \item taking $F$ to be the field of fractions of $R$, for $f$ in $R[x]$
        with positive degree, $f$ is prime in $R[x]$ if and only if $f$ is
        primitive in $R[x]$ and irreducible in $F[x]$.
\end{enumerate}

\begin{proof}
    (1) If $uv = 1$ for some $v$ in $R$, the same holds in $R[x]$ as $R \subseteq R[x]$.
    \bs
    (2) Similar to the proof of (\ref{6.6}), if $p$ doesn't divide $f$ or $g$ in $R[x]$,
    we show that it doesn't divide $fg$.
    \bs
    (3) ($\Longleftarrow$) We suppose that for some $g$ and $h$ in $R[x]$, $f$ divides $gh$.
    As such, $f$ divides $gh$ in $F[x]$ and since $f$ is irreducible and prime in $F[x]$,
    we have that $f$ divides $g$ or $h$. We suppose (without loss of generality) that $f$
    divides $g$ so $g = k \cdot f$ for some $k$ in $F[x]$. We write $f$, $g$, and $k$
    as: \begin{align*}
        f = c_f \cdot f^\ast, \qquad
        g = c_g \cdot g^\ast, \qquad
        k = c_k \cdot k^\ast,
    \end{align*} where $c_f$ is in $R^\times$ as $f$ is primitive, $c_g$ is in
    $R$ as $g$ is in $R[x]$, $c_k$ is in $F^\times$, and $f^\ast$, $g^\ast$, and 
    $k^\ast$ are primitive polynomials in $R[x]$. Since $g = k \cdot f$,
    we can deduce that: \begin{align*}
        \frac{c_g}{c_fc_k} \cdot g^\ast = f^\ast \cdot k^\ast.
    \end{align*} We know that $u = \frac{c_g}{c_fc_k}$ is in $R^\times$ as
    $f^\ast \cdot k^\ast$ must be primitive and contents are unique up to units, so we
    write: \begin{align*}
        g = \frac{c_g}{uc_f} \cdot k^\ast \cdot f.
    \end{align*} Since $c_g$ is in $R$ and $uc_f$ is in $R^\times$, $f$ divides $g$ in
    $R[x]$ as required.
    \bs
    ($\Longrightarrow$) By contrapositive, we first consider if $f$ is not primitive,
    in which case $f = c_f \cdot f^\ast$ where $f^\ast$ is primitive in $R[x]$ is a non-trivial
    factorisation of $f$ in $R[x]$ so $f$ is not irreducible, and thus, not prime.
    If $f$ is reducible in $F[x]$, $f = gh$ for some non-constant $g$ and $h$ in $F[x]$.
    Then, as in the previous direction, $f = c_f g^\ast h^\ast$ is reducible in $R[x]$
    so $f$ is not prime.
\end{proof}

\subsection{UFDs and their Polynomials (6.10)} \label{6.10}

For a UFD $R$, $R[x]$ is a UFD. Furthermore, the primes of $R[x]$
are the primes of $R$ and primitive irreducible polynomials of
positive degree, and $R[x]^\times = R^\times$. We can recursively apply this,
so $R[x_1, \ldots, x_n]$ is a UFD for any $n$ in $\mb{Z}_{\geq 0}$.

\begin{proof}
    (\textbf{Units}) By (\ref{6.8}(1)), $R^\times \subseteq R[x]^\times$, so we consider
    $f$ in $R[x]^\times$. As such, there's some $g$ in $R[x]$ with $fg = 1$.
    Thus: \begin{align*}
        \deg(f) + \deg(g) = \deg(fg) = 1,
    \end{align*} so $f$ and $g$ must be constant polynomials, meaning
    they are in $R$. Hence, $f$ is in $R^\times$ so 
    $R[x]^\times = R^\times$.
    \bs
    (\textbf{Primes}) By (\ref{6.8}(2)), we know the primes of $R$ are in $R[x]$. Also,
    by (\ref{6.8}(3)), we know that primitive irreducible polynomials of positive degree
    are prime.
    \bs
    (\textbf{Factorisation})
    For $f$ non-zero in $R[x]$, with $F$ the field of fractions of $R$, since $F[x]$ is a UFD
    (as it is a field) we can write: \begin{align*}
        f = c \cdot f_1 \cdots f_n,
    \end{align*} for $c$ in $F^\times$, $n$ in $\mb{Z}_{\geq 0}$, and 
    $f_1, \ldots, f_n$ irreducible in $F[x]$. We then write: \begin{align*}
        c^\ast &= c \cdot c_{f_1} \cdots c_{f_n} \in F^\times, \\
        f &= c^\ast \cdot f_1^\ast \cdots f_n^\ast.
    \end{align*} But, $c^\ast = c_f$ as $f_1^\ast \cdots f_n^\ast$ is primitive by
    Gauss' Lemma, so is in $R$. As such, we can write $c^\ast = u \cdot q_1 \cdots q_m$
    for $u$ in $R^\times$, $m$ in $\mb{Z}_{\geq 0}$, and
    $q_1, \ldots, q_n$ prime in $F[x]$. Thus, we can write: \begin{align*}
        f = u \cdot \underbrace{q_1 \cdots q_m \cdot f_1^\ast \cdots f_n^\ast}_{
            \text{prime in } R[x]}.
    \end{align*}
    As in the proof of (\ref{5.34}), since we showed that every $f$ in $R[x]$ can be factored
    into primes, not just irreducibles, the factorisation is automatically unique.
\end{proof}

\newpage

\subsection{Eisenstein's Criterion (7.1)} \label{7.1}

For a UFD $R$, $f$ primitive in $R[x]$ with positive degree so $f = a_nx_n + \cdots + a_0$
for some $n$ in $\mb{Z}_{>0}$ and $a_n, \ldots, a_0$ in $R$. If there is a prime $p$ in
$R$ such that: \begin{itemize}
    \item $p$ doesn't divide $a_n$,
    \item $p$ divides $a_0, \ldots, a_{n - 1}$,
    \item $p^2$ doesn't divide $a_0$,
\end{itemize} then $f$ is irreducible in $R[x]$ and also $F[x]$ where $F$ is the field of
fractions of $R$. Polynomials satisfying this criterion are called Eisenstein polynomials
(at $p$).

\begin{proof}
    We suppose $f = gh$ where $g$ and $h$ are in $R[x]$ and have positive degree. We take:
    \begin{align*}
        g &= b_mx^m + \cdots + b_0, \\
        h &= c_{n - m}x^{n - m} + \cdots + c_0,
    \end{align*} for some $m$ in $\mb{Z}_{> 0}$ and $b_0, \ldots, b_m, c_0, \ldots, c_{n - m}$
    in $R$. We know that $p$ doesn't divide $a_n = b_mc_{n - m}$, so $p$ doesn't divide
    $b_m$ or $c_{n - m}$. We take $i$ to be minimal such that $p$ doesn't divide $b_i$,
    and $j$ to be minimal such that $p$ doesn't divide $c_j$. This implies that $p$ doesn't
    divide $a_{i + j}$ so $i + j = n$, thus $i = m$ and $j = n - m$. As such,
    $p$ divides $b_0$ and $c_0$ so $p^2$ divides $a_0$, a contradiction.
\end{proof}

\subsection{Irreducibility and Linear Substitution (7.5)} \label{7.5}

For a field $K$ with $f$ in $K[x]$, $a$ and $b$ in $K$ with $a \neq 0$ then
$f(x)$ is irreducible if and only if $f(ax + b)$ is irreducible. 

\subsection{Roots and Divisibility (7.8)} \label{7.8}

For a field $K$ with $f$ in $K[x]$ and $\alpha$ in $K$: \begin{align*}
    f(\alpha) = 0 \Longleftrightarrow x - \alpha \text{ divides } f(x).
\end{align*}

\begin{proof}
    ($\Longrightarrow$) We divide $f(x)$ with remainder by $(x - \alpha)$
    so $f(x) = g(x)(x - \alpha) + r$ for some $g$ in $K[x]$ and $r$ in $K$ (since
    $\deg(r) < \deg(x - \alpha) = 1$). So, we have
    $f(\alpha) = g(\alpha)(x - \alpha) + r = r = 0$, thus $f(x) = g(x)(x - \alpha)$
    as required.
    \bs
    ($\Longleftarrow$) We have $f(x) = g(x)(x - a)$ for some $g$ in $K[x]$,
    so $f(\alpha) = 0$.
\end{proof}

\subsection{The Second Criterion (7.9)} \label{7.9}

For a field $K$ with $f$ in $K[x]$ with degree equal to $2$ or $3$: \begin{align*}
    f \text{ is irreducible} \Longleftrightarrow f \text{ has no roots in } K.
\end{align*}

\begin{proof}
    We consider the following equivalences: \begin{align*}
        f \text{ reducible}
        &\Longleftrightarrow f \text{ has a factor of degree } 1 \\
        &\Longleftrightarrow ax + b \text{ divides $f(x)$ for some $a \neq 0$ and $b$ in $K$} \\
        &\Longleftrightarrow x + \frac{b}{a} \text{ divides } f(x) \\
        &\Longleftrightarrow f\left(-\frac{b}{a}\right) = 0, \tag{\ref{7.8}}
    \end{align*} as required.
\end{proof}

\subsection{Finding Roots (7.10)} \label{7.10}

For a UFD $R$ with $K$ its field of fractions, we consider $f$ in $R[x]$ with
degree $n \geq 0$ and coefficients $a_n, \ldots, a_0$ in $R$. For some $\alpha$
in $K$, written in its simplest form as $\alpha = \frac{r}{s}$ ($\hcf(r, s) = 1$),
if $f(\alpha) = 0$ then $r$ divides $a_0$ and $s$ divides $a_n$.

\begin{proof}
    By (\ref{7.8}), we know that $x - \alpha$ divides $f$ in $K[x]$ so $sx - r$ divides
    $f^\ast$ in $K[x]$ since $f^\ast$ is the same as $f$ up to units in $K$. We note that
    $sx - r$ is primitive since $\hcf(r, s) = 1$ and irreducible as it has degree $1$.
    As such, $sx - r$ is prime in $R[x]$. From an \textbf{exercise} (see (\ref{6.10})), 
    we deduce that $sx - r$ divides $f^\ast$ in $R[x]$ also.
    \bs
    So, $f = c_f(sx-r)g$ for some polynomial $g$ in $R[x]$ of degree $n - 1$ with coefficients
    $b_{n - 1}, \ldots, b_0$ in $R$. This implies that $a_0 = -c_frb_0$ is divisible by $r$
    and $a_n = c_fsb_{n - 1}$ is divisible by $s$.
\end{proof}

\subsection{Monic Polynomials (7.14)} \label{7.14}

A polynomial is monic if its leading coefficient is $1$.

\subsection{Reflected Irreducibility from Monic Images on Induced Maps (7.15)} \label{7.15}

For a ring homomorphism on integral domains $\varphi$ from $R$ to $S$ with $\varphi$ also
acting as the induced map on $R[x]$ to $S[x]$, we have that if $f$ in $R[x]$ is monic
and $\varphi(f)$ is irreducible then $f$ is irreducible.

\begin{proof}
    We suppose that $f$ is reducible so $f = gh$ for some $g$ and $h$ in $R[x]$ with
    degrees $m$ and $k$ and coefficients $b_m, \ldots, b_0$ and $c_k, \ldots, c_0$
    respectively. As $f$ is monic, we have $b_mc_k = 1$ so we can simply rewrite
    $g \mapsto \frac{1}{b_m}g$ and $h \mapsto \frac{1}{c_k}h$, so we will just assume
    $g$ and $h$ are monic, $b_m = c_k = 1$, without loss of generality. As such:
    \begin{align*}
        f(x) = (x^m + b_{m - 1}x^{m - 1} + \cdots + b_0)
            (x^k + c_{k - 1}x^{k - 1} + \cdots + c_0).
    \end{align*} We have $m$ and $k$ strictly greater than zero as otherwise our
    factorisation $f = gh$ was not proper, so: \begin{align*}
        \varphi(f) = (x^m + \varphi(b_{m - 1})x^{m - 1} + \cdots + \varphi(b_0))
            (x^k + \varphi(c_{k - 1})x^{k - 1} + \cdots + \varphi(c_0)),
    \end{align*} is a proper factorisation of $\varphi(f)$, a contradiction.
\end{proof}

\subsection{The Third Criterion (7.18)} \label{7.18}

For an integral domain $R$ with $f$ in $R[x]$ monic, if $f \bmod (p)$
is irreducible in $R / (p)$ for some prime ideal $(p)$ in $R$, $f$ is irreducible.

\begin{proof}
    By (\ref{7.15}) on the quotient homomorphism.
\end{proof}

\subsection{Irreducibility in the Rationals (7.21)} \label{7.21}

For $f$ in $\mb{Z}[x]$ with leading coefficient $\alpha$, 
$p$ prime in $\mb{Z}$, if $\alpha \not\equiv 0 \bmod p$ and
$f \bmod (p)$ is irreducible then $f$ is irreducible in $\mb{Q}[x]$. 

\begin{proof}
    Similar to (\ref{7.18}).
\end{proof}
