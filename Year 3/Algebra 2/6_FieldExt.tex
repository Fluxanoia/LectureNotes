\section{Field Extensions}

\subsection{Polynomials in Extended Fields (7.24)} \label{7.24}

For rings $R$ and $S$ with $R \subseteq S$ and $f$ in $R[x]$, we refer
to the image of $f$ in $S[x]$ as '$f$ over $S$'.

\subsection{Finite Fields and Integers Modulo Primes (8.2)} \label{8.2}

For a prime $p$ in $\mb{Z}$, we have that the finite field of
size $p$, denoted by $\mb{F}_p$, is $\mb{Z}/p\mb{Z}$.

\subsection{The Field of Rational Functions (8.3)} \label{8.3}

We define the field of rational functions over $\mb{R}$ as the field of
fractions of $\mb{R}[x]$, denoted by $\mb{R}(x)$.

\subsection{Subfields and Extensions (8.5, 8.19)} \label{8.5} \label{8.19}

For fields $K$ and $L$ with $K \subseteq L$, we say that $K$ is a
subfield of $L$ and $L$ is an extension of $K$, which may be
denoted by $L / K$.

\subsection{Conditions for Subfields (8.9)} \label{8.9}

For a field $K$ with $U \subseteq K$, $U$ is a subfield if and only if: 
\begin{itemize}
    \item $\{0, 1\} \subseteq U$,
    \item $U$ is closed under addition and additive inverses,
    \item $U$ is closed under multiplication and multiplicative inverses. 
\end{itemize}

\begin{proof}
    Follows from (\ref{1.14}).
\end{proof}

\subsection{Prime Subfields (8.11)} \label{8.11}

For a field $K$, $K$ either contains $\mb{F}_p$ for some unique prime $p$ in $\mb{Z}$
or $\mb{Q}$. This is the prime subfield of $K$.

\begin{proof}
    By (\ref{3.6}), there is a unique homomorphism from $\mb{Z}$ to $K$ so: \begin{align*}
        \mb{Z}/\Ker(\varphi) \cong \Ima(\varphi) \subseteq K.
    \end{align*} As $K$ is a field, $\Ima(\varphi)$ must be an integral domain so
    $\Ker(\varphi)$ is a prime ideal. As such, $\Ker(\varphi)$ is either $\{0\}$ or
    $\mb{F}_p$ for some prime $p$ in $\mb{Z}$. In the former case, $\mb{Z}$ is a
    subring of $K$ so its field of fractions $\mb{Q}$ is a subfield of $K$.
    In the latter case, $\mb{F}_p$ is a subfield of $K$. By the uniqueness of $\varphi$,
    such a subfield is unique.
\end{proof}

\subsection{The Field Characteristic (8.12)} \label{8.12}

We define the characteristic function on fields: \begin{align*}
    \Char(K) = \begin{cases}
        0 & \text{if } \mb{Q} \subseteq K \\
        p & \text{if } \mb{F}_p \subseteq K.
    \end{cases}
\end{align*}

\subsection{Vector Spaces (8.15)} \label{8.15}

For fields $K$ and $L$ with $K \subseteq L$, $L$ is a vector space
over $K$.

\begin{proof}
    Follows from the field axioms.
\end{proof}

\subsection{Degree of Field Extensions (8.20)} \label{8.20}

For a field extension $L / K$, the degree of the extension is: \begin{align*}
    [L : K] = \Dim(L \text{ as a $K$-vector space}).
\end{align*} We say $L / K$ is a finite extension if $[L : K]$ is finite
and an infinite extension otherwise.

\subsection{Tower Law (8.24)} \label{8.24}

For $K$, $L$, and $M$ fields with $K \subseteq L \subseteq M$, we have
that $[M : K] = [M : L][L : K]$.

\begin{proof}
    We take $\{v_i\}_{i \in I}$ to be a basis for $L / K$ indexed by $I$
    and $\{w_j\}_{j \in J}$ to be a basis for $M / L$ indexed by $J$.
    We want to show that $\{v_i \cdot w_j\}_{i \in I, j \in J}$ is a
    basis for $M / K$, implying the theorem. For $\alpha$ in $M$,
    we have: \begin{align*}
        \alpha = \sum_{t \in T} a_tw_t,
    \end{align*} for some finite indexing set $T \subseteq J$ and 
    $\{a_t\}_{t \in T} \subseteq L$. Similarly: \begin{align*}
        a_t = \sum_{u_t \in U_t} b_{u_t}v_{u_t},
    \end{align*} where for each $t$ in $T$, $U_t \subseteq I$ is a finite indexing
    set with $\{b_{u_t}\}_{u_t \in U_t} \subseteq K$. Combining these: \begin{align*}
        \alpha = \sum_{t \in T} a_tw_t = \sum_{t \in T} \sum_{u_t \in U_t} b_{u_t}v_{u_t}w_t.
    \end{align*} If we suppose $\alpha = 0$ then since $\{w_j\}_{j \in J}$ is a basis,
    each coefficient must be zero by linear independence. Then since $\{v_i\}_{i \in I}$
    is a basis, each $b_{u_t}$ is also zero. Thus, $\{v_i \cdot w_j\}_{i \in I, j \in J}$ is a
    basis for $M / K$.
\end{proof}

\subsection{Fundamental Field Facts (8.27)} \label{8.27}

For a field $K$ with $f(x) = a_n x^n + \cdots + a_0$ irreducible in $K[x]$ with
$n$ in $\mb{Z}_{> 0}$ and taking $\alpha$ to be the image of $x$ under the quotient
homomorphism $\pi$ from $K[x]$ to $K[x] / (f)$: \begin{enumerate}
    \item $L = K[x] / (f)$ is a field,
    \item $f(\alpha) = 0$,
    \item $[L : K] = \deg(f) = n$ and $\{1, \alpha, \ldots, \alpha^{n - 1}\}$
        is a basis for $L / K$,
    \item if $F / K$ is a field extension and $f(\beta) = 0$ for some $\beta$ in $F$
        then the injection from $K$ to $F$ extends to a unique homomorphism
        $\psi$ which is injective from $L$ to $F$ such that $\psi(\alpha) = \beta$ and
        $\psi(L) = K(\beta)$.
\end{enumerate} 

\begin{proof}
    (1) Since $K[x]$ is a PID, as $f$ is irreducible, $(f)$ is a maximal ideal so
    $K[x] / (f)$ is a field by (\ref{4.12}).
    \bs
    (2) We know that $\pi(x) = \alpha$ so $\pi(f(x)) = f(\alpha)$ as $\pi$ is
    a homomorphism. Since $\Ker(\pi) = (f)$, we have $\pi(f(x)) = 0 = f(\alpha)$.
    \bs
    (3) We take some $g + (f)$ in $L$ and by (\ref{5.41}) we can write: \begin{align*}
        g = q \cdot f + r,
    \end{align*} for some $q$ and $r$ in $K[x]$ with $\deg(r) < n$. We write
    $r = \sum_{i = 0}^{n - 1} b_ix^i$ for some coefficients $b_0, \ldots, b_{n - 1}$ in
    $K$. As $\pi(f) = 0$, we have: \begin{align*}
        g + (f) = \pi(g) = \pi(r) = \sum_{i = 0}^{n - 1} b_i \alpha^i,
    \end{align*} so $1, \alpha, \ldots, \alpha^{n - 1}$ spans $L$. We take
    $c_0, \ldots, c_{n - 1}$ in $K$ such that: \begin{align*}
        c_0 + c_1\alpha + \cdots + c_{n - 1} \alpha^{n - 1} = 0,
    \end{align*} but this means $\sum_{i = 0}^{n - 1} c_ix^i$ is in $\ker(\pi) = (f)$.
    As such, $f$ divides it, but this cannot be true unless it is the zero polynomial
    (as it has degree strictly less than $n = \deg(f)$). Thus,
    $1, \alpha, \ldots, \alpha^{n - 1}$ are linearly independent, and thus form a basis
    for $L / K$.
    \newpage \noindent
    (4) We take $\psi$ to be the map from $K[x]$ to $F$ by $\psi(g) = g(\beta)$
    (an extension of the injective map from $K$ to $F$).
    As $f$ is in $\Ker(\psi)$ and $(f)$ is maximal, $\Ker(\psi) = (f)$ so we can
    extend $\psi$ again to a well-defined homomorphism from $K[x] / (f)$ to $F$ 
    defined by: \begin{align*}
        \psi(g + (f)) = g(\beta),
    \end{align*} so $\psi(\alpha) = \psi(x + (f)) = \beta$. This is unique since
    every element of $L$ is represented by a polynomial in $\alpha$ with coefficients
    in $K$ and $\psi$ is defined by the image of $K$ and $\beta$. 
    The image $\psi(L)$ is generated by $K$ and $\beta$ so is
    the smallest subfield of $F$ containing $K$ and $\beta$.
\end{proof}

\subsection{Extension to (\ref{8.27}(4)) (8.32)} \label{8.32}

For a field $K$ with $f$ non-constant and irreducible in $K[x]$, we take $L = K[x] / (f)$
and $\alpha = x + (f)$. For a field extension $\varphi$ from $K$ to $F$, there is a
bijection: \begin{align*}
    \{\text{injective homomorphisms from $L$ to $F$ extending $\varphi$}\}
    \to
    \{\text{roots of $f$ in $F$}\},
\end{align*} defined by the mapping $\psi \mapsto \psi(\alpha)$.

\begin{proof}
    We suppose $\psi$ is an injective homomorphism from $L$ to $F$ extending $\varphi$
    and take $\beta = \psi(\alpha)$.
    By (\ref{8.27}(2)), $f(\alpha) = 0$ implies that $f(\beta) = 0$ in $F$ since $\psi$
    is a homomorphism. Thus, $\psi(\alpha)$ is a root of $f$ in $F$.
    Also, for any root $\beta$ of $f$ in $F$, there is a unique such $\psi$ by (\ref{8.27}(4))
\end{proof}

\subsection{Algebraic and Transcendental Elements (8.35)} \label{8.35}

For a field extension $F / K$, for any $\beta$ in $F$ we either have: \begin{itemize}
    \item a unique monic irreducible polynomial $f$ in $K[x]$ with $f(\beta) = 0$
        (the minimal polynomial of $\beta$ over $K$) so that: \begin{align*}
            K[\beta] = K(\beta) \cong K[x] / (f),
        \end{align*} where $K(\beta)$ is the smallest subfield of $L$ containing $\beta$,
        it follows that \linebreak
        $[K(\beta) : K] = n < \infty$ where $1, \beta, \ldots, \beta^{n - 1}$
        is a basis of $K(\beta)$ and $n = \deg(f)$; we say $\beta$ is algebraic over $K$,
    \item there are no polynomials $f \neq 0$ in $K[x]$ with $f(\beta) = 0$, so: \begin{align*}
            K[\beta] \cong K[x], \qquad K(\beta) \cong K(x), \qquad [K(\beta) : K] = \infty,
        \end{align*} we say $\beta$ is transcendental over $K$.
\end{itemize}

\begin{proof}
    We take $I$ to be the ideal of the map $\varphi$ from $K[x]$ to $F$ defined by
    $\varphi(g) = g(\beta)$, so $I = \{g \in K[x] : g(\beta) = 0\} \subset K[x]$.
    As $K[x]$ is a PID, either $I = (0)$ or $I = (f)$ where $f$ is unique up to units
    (so we can assume monic) in $K$.
    \bs
    If $I = (0)$ then $g(\beta) \neq 0$ for all $g \neq 0$ in $K[x]$ so 
    $\Ima(\varphi) = K[\beta] \cong K[x]$. By taking the field of fractions we also
    have $K(\beta) \cong K(x)$.
    \bs
    If $I = (f)$ then by the Homomorphism Theorem $\Ima(f) \cong K[x] / (f)$ 
    but since $\Ima(f)$ is an integral domain as it is a subring of a field,
    $(f)$ is prime and $f$ is irreducible by (\ref{4.12} and \ref{5.14}).
    By (\ref{8.27}(4)): \begin{align*}
        K[\beta] \cong \Ima(\varphi) \cong \frac{K[x]}{\Ker(\varphi)} = \frac{K[x]}{f},
    \end{align*} which is a field by (\ref{8.27}(1)). Thus, $K(\beta) = K[\beta]$
    with degree $n = \deg(f)$ over $K$ and a basis $1, \beta, \ldots, \beta^{n - 1}$
    where $\beta = \varphi(x)$ as proven in (\ref{8.27}).
\end{proof}

\subsection{Complex Algebraicity and Transcendentality (8.36)} \label{8.36}

For $\alpha$ in $\mb{C}$, $\alpha$ is algebraic if it is algebraic over $\mb{Q}$
and transcendental otherwise.

\subsection{Algebraicity of Combinations of Algebraic Numbers (8.43)} \label{8.43}

For a field extension $F / K$ with $\alpha$ and $\beta$ in $F$ such that they are algebraic
over $K$, we have that: $\alpha + \beta$, $\alpha - \beta$, $\alpha\beta$, and
(for non-zero $\beta$) $\alpha / \beta$ are algebraic over $K$.

\begin{proof}
    We have that all of these combinations exist in $K(\alpha, \beta)$ and that: \begin{align*}
        n = [K(\alpha, \beta) : K] = [K(\alpha, \beta) : K(\alpha)][K(\alpha) : K] < \infty. 
        \tag{\ref{8.24}}
    \end{align*} Thus, since $[K(\alpha + \beta) : K]$, $[K(\alpha - \beta) : K]$,
    $[K(\alpha\beta) : K]$, and (if applicable) $[K(\alpha / \beta) : K]$ are all less than
    or equal to $n$, they must be finite which implies algebraicity via (\ref{8.35}).
\end{proof}

\subsection{Algebraic Closure (8.44-45)} \label{8.44} \label{8.45}

We write the set of algebraic numbers in $\mb{C}$ as $\overline{\mb{Q}}$, the algebraic closure
of $\mb{Q}$. This is a field.

\subsection{Algebraic Field Extensions (8.47)} \label{8.47}

A field extension $F / K$ is algebraic if every element of $F$ is algebraic over $K$.
