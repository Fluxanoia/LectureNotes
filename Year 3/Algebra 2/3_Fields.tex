\section{Integral Domains and Fields}

\subsection{Integral Domains (4.1)} \label{4.1}

For a ring $R$, $a \neq 0$ in $R$ is a zero divisor if for some 
$b \neq 0$ in $R$, $ab = 0$. We say $R$ is an integral domain if it has
no zero divisors.

\subsection{Preservation of Isomorphism}

Ring isomorphisms preserve units and zero divisors, 
so the domain is a field/integral domain if and only 
if the codomain is a field/integral domain.

\subsection{Relating Integral Domains and Fields (4.3)} \label{4.3}

We have that: \begin{enumerate}
    \item all fields are integral domains,
    \item $R$ is an integral domain if and only if for all
        $a \neq 0$ in $R$, the map $x \mapsto ax$ is injective,
    \item every finite integral domain is a field. 
\end{enumerate}

\begin{proof}
    (1) Suppose we have $a$ and $b$ in some field, such that
    $a \neq 0$ and $ab = 0$. Thus, $a^{-1}ab = 0$, so
    $b = 0$. 
    \bs
    (2) ($\Longleftarrow$) We have that $ax = 0$ if and only if
    $x = 0$ as $R$ has no zero divisors, so the map is 
    injective by (\ref{2.18}). \newline
    ($\Longrightarrow$) We appeal to the contrary and suppose
    $ax = 0$ for some non-zero $a$ and $x$ in $R$. As such,
    the mapping via $a$ has a non-zero kernel, a contradiction
    by (\ref{2.18}).
    \bs
    (3) If a integral domain $R$ is finite, then the mapping 
    in (2) is surjective, so for any $a$ in $R$, there is
    some $x$ in $R$ such that $ax = 1$.
\end{proof}

\subsection{Subrings of Integral Domains (4.4)} \label{4.4}

Every subring of an integral domain is an integral domain.

\newpage

\subsection{Field of Fractions (4.6)} \label{4.6}

For an integral domain $R$, we can consider fractions: \begin{align*}
    \left\{\frac{a}{b} : a \in R, b \in R, b \neq 0\right\},
\end{align*} and define an equivalence relation: \begin{align*}
    (a, b) \sim (c, d) \Longleftrightarrow ad = bc.
\end{align*} with the set of equivalence classes $K$, forming
a field under the ring operations: \begin{align*}
    \frac{a}{b} + \frac{c}{d} &= \frac{ad + bc}{bd}, \\[2mm]
    \frac{a}{b} \cdot \frac{c}{d} &= \frac{ac}{bd},
\end{align*} along with the expected additive and multiplicative
inverses and identites. This is the field of fractions of $R$,
denoted $f.f.(R)$. We have that $R$ is isomorphic to a subring
of $K$, and if there is an injective homomorphism between
two integral domains, there is an induced injection between
their respective fields of fractions.

\subsection{Maximal Ideals (4.8)} \label{4.8}

For a ring $R$, an ideal $I \subset R$ is maximal if there is no
ideal $J$ such that $I \subset J \subset R$.

\subsection{Prime Ideals (4.9)} \label{4.9}

For a ring $R$, an ideal $I \subset R$ is prime if for all $ab$
in $I$, either $a$ or $b$ is in $I$.

\subsection{Maximal and Prime Ideals and their Quotients (4.12-13)}
\label{4.12} \label{4.13}

For a ring $R$ with $I \subset R$ an ideal: \begin{enumerate}
    \item $I$ is maximal if and only if $R / I$ is a field,
    \item $I$ is prime if and only if $R / I$ is an integral domain.
\end{enumerate} Thus, every maximal ideal is prime since all
fields are integral domains.

\begin{proof}
    (1) By (\ref{3.15}), there's a bijection from ideals of $R$
    containing $I$ and ideals of $R / I$. Thus, the ideals of $R / I$
    are $0$ and $R / I$ if and only if the ideals of $R$ containing
    $I$ are $I$ and $R$, which is true if and only if $I$ is maximal.
    \bs
    (2) We consider $a$ and $b$ in $R$ and consider $\bar{a} = a + I$
    and $\bar{b} = b + I$: \begin{align*}
        a \in I &\Longleftrightarrow \bar{a} = I, \\
        b \in I &\Longleftrightarrow \bar{b} = I, \\
        ab \in I &\Longleftrightarrow \bar{a}\bar{b} = I.
    \end{align*} Thus, if $I$ is prime and $\bar{a}\bar{b} = I$ then
    either $a$ or $b$ is in $I$. Also, if $R / I$ is an integral domain
    and we have $ab$ in $I$, then either $\bar{a}$ or $\bar{b}$ is in
    $I$ so either $a$ or $b$ is in $I$.
\end{proof}

\subsection{Existence of Maximal Ideals (4.16)} \label{4.16}

Every ring $R \neq \{0\}$ has a maximal ideal.

\subsection{Ideals within Maximal Ideals (4.17)} \label{4.17}

For a ring $R$, every ideal $I \subset R$ is contained in some
maximal ideal.
