\section{Fundamental Results}

\subsection{The Homomorphism Theorem (3.1)} \label{3.1}

For a homomorphism $\varphi$ from $R$ to $S$, taking $I = \Ker(\varphi)$, we have
that $R / I \cong \varphi(R)$, via the map $r + I \mapsto \varphi(r)$.

\begin{proof}
    We consider the proposed map and name it $\psi$.
    We can see that $\psi$ is well defined as for some $r$ in $R$, for any $r'$ in $r + I$, 
    $r' = r + i$ for some $i$ in $I$ so: \begin{align*}
        \varphi(r') = \varphi(r) + \varphi(i) = \varphi(r).
    \end{align*} Additionally, $\psi$ is trivially a homomorphism, and is surjective
    by the definition of the image, so we consider injectivity. 
    If for some $r$ in $R$, we have $\psi(r + I) = 0$ then: \begin{align*}
        \varphi(r) = 0 \Longrightarrow r \in I \Longrightarrow r + I = I,
    \end{align*} so $\psi$ is an isomorphism.
\end{proof}
